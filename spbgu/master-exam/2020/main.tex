\documentclass[a4paper, 12pt]{article}
% math symbols
\usepackage{amssymb}
\usepackage{amsmath}
\usepackage{mathrsfs}
\usepackage{mathseries}


\usepackage[margin = 2cm]{geometry}

\tolerance = 1000
\emergencystretch = 0.74cm



\pagestyle{empty}
\parindent = 0mm

\renewcommand{\coursetitle}{\textsc{InT}}
\setcounter{curtask}{1}

\setmathstyle{Дата}{Экзамен в магистратуру}{ФМКН СПбГУ}


\begin{document}

\section*{Теоретическая информатика}

\renewcommand{\coursetitle}{\textsc{CS}}

\task{
    Рассмотрим задачу <<Определить по данному недетерминированному конечному автомату, верно ли, что он
    принимает все строки>>. Постройте алгоритм для данной задачи, использующий $s^k$ памяти, где $s$~---
    число состояний автомата, а $k$~--- фиксированная константа. Покажите, что для данной задачи не
    существует алгоритма, который бы использовал $\mathcal{O}(\log^2 s)$ памяти. 
}

\task{
    Рассмотрим задачу <<Определить по данному недетерминированному конечному автомату, верно ли, что он
    принимает не все строки>>. Постройте алгоритм для данной задачи, использующий $s^k$ памяти, где $s$~---
    число состояний автомата, а $k$~--- фиксированная константа. Покажите, что для данной задачи не
    существует алгоритма, который бы использовал $\mathcal{O}(\log^3 s)$ памяти. 
}

\section*{Мат. логика}

\renewcommand{\coursetitle}{\textsc{ML}}

\task{
    Какова мощность множества всех автоморфизмов (сохраняющих порядок биекций) упорядоченного множества
    вещественных чисел, т.е. $(\mathbb{R}, <)$? Ответ необходимо подробно обосновать.
}

\task{
    Пусть $A$~--- множество всех подмножеств натуральных чисел, а $<$~--- лексикографический порядок на
    данном множестве (т.е. $x < y$ тогда и только тогда, когда минимальный элемент $i$ из симметрической
    разности $x$ и $y$ принадлежит $y$). Какова мощность множества всех автоморфизмов (сохраняющих
    порядок биекций) множества $A$? Ответ необходимо подробно обосновать.
}



\end{document}
