Since the founding work of the 1970s, an influential line of research has zoomed in on $\NP$-complete
problems, with the satisfiability problem for Boolean logic formulas ($\SAT$) at its head, which turned
out to be exactly the right notion to capture literally thousands of important applied problems in
different fields. Based on the assumption of the hardness of $\NP$, whose validity is one of the famous
Millennium Prize Problems, a rich mathematical theory has been developed for establishing conditional
results that state that all these problems are infeasible to solve, in the worst case. 

The trouble is that real problems are not worst-case. The last two decades have seen the development of
exceedingly efficient algorithms for many of these problems, perhaps most impressively in the form of
so-called $\SAT$ solvers for logic formulas. Traditional complexity analysis claiming exponential lower
bounds is arguably not very relevant in this setting, but this also means that we lack tools to
understand how these algorithms can perform so well and why they sometimes spectacularly fail.

In this course we bring together leading theoreticians and practitioners that work on SAT and its
generalizations.
