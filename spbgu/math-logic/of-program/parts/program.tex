\begin{enumerate}
    \item[Лекция 1.]
        \begin{itemize}
            \item Пропозициональные формулы, КНФ, ДНФ. Интерпретации, выполнимые пропозициональные
                формулы, тавтологии. Эффективный алгоритм сведения задачи выполнимости формулы к задаче
                выполнимости формулы в КНФ.
            \item Деревья решений как способ доказательства невыполнимости. Метод резолюций, корректность
                и полнота.
            \item Алгоритм проверки выполнимости формулы в 2-КНФ с помощью резолюций.
        \end{itemize}
    \item[Лекция 2.]
        \begin{itemize}
            \item Теорема о компактности для пропозициональных формул и ее следствия. 
            \item Секвенции. Формульная интерпретация секвенций. Аксиомы и правила вывода исчисления
                секвенций для пропозициональной логики. Корректность исчисления секвенций. 
            \item Теорема о полноте для исчисления секвенций в слабой форме. Теорема о полноте в сильной
                форме. 
        \end{itemize}
    \item[Лекция 3.]
        \begin{itemize}
            \item Формулы исчисления предикатов. Термы, атомарные формулы, свободные и связанные
                вхождения переменных, интерпретация и оценка, вычисление значение формулы. Общезначимые
                формулы. Выразимость в арифметике: примеры. 
            \item Кодирование конечных множеств в арифметике.  Выразимость предиката <<быть степенью
                шестерки>>.
        \end{itemize}
    \item[Лекция 4.]
        \begin{itemize}
            \item Доказательство невыразимости предикатов с помощью автоморфизмов. Примеры.
            \item Элиминация кванторов в $(\mathbb{Z}, s, 0, =)$, невыразимость предиката $x < y$.
            \item Элиминация кванторов в $(\mathbb{Q}, =, <)$. Теорема о разрезании квадрата на меньшие
                квадраты.
        \end{itemize}
    \item[Лекция 5.]
        \begin{itemize}
            \item Элиминация кванторов в элементарной теории вещественных чисел. Теорема
                Тарского--Зайденберга.
            \item Элиминация кванторов в теории алгебраически замкнутых полей. Расширение полей, теорема
                Гильберта о нулях. 
        \end{itemize}
    \item[Лекция 6.]
        \begin{itemize}
            \item Теорема о компактности для предикатных формул.
            \item Секвенциальное исчисление предикатов. Корректность.
            \item Теорема о полноте секвенциального исчисления в сильной и слабой формах. 
        \end{itemize}
    \item[Лекция 7.]
        \begin{itemize}
            \item Теорема Левенгейма--Сколема о счетной модели. Аксиомы равенства. Нормальные
                модели. Теорема о компактности для нормальных моделей. Нестандартная модель арифметики.
            \item Аксиоматизируемость теорий. $\Theory(\mathbb{Z}, s, 0, =)$ не имеет конечной
                аксиоматизации.
            \item Разрешимость полной конечно аксиоматизируемой непротиворечивой теории. Теория плотного
                линейного порядка без первого и последнего элемента.
        \end{itemize}
    \item[Лекция 8.]
        \begin{itemize}
            \item Предваренная нормальная форма. Сколемизация. 
            \item Общезначимость $\Pi_1$-формул, теорема Эрбрана об общезначимости $\Sigma_1$-формул.
        \end{itemize}
\end{enumerate}
