\begin{enumerate}
    \item[Лекция 1.] Воспоминания о былом. Языки и предикаты. Интерпретации. Модели. Теорема о полноте.
    \item[Лекция 2.] 
        \begin{itemize}
            \item Элиминация кванторов и эл. эквивалентность.
            \item Эл. эквивалентность. Изоморфизмы: $(\mathbb{Q}, <)$ и $(\mathbb{Q} + \mathbb{Q}, <)$.
            \item Игры Эренфойхта--Фраиссе. 
        \end{itemize}
    \item[Лекция 3.]
        \begin{itemize}
            \item Игры Эренфойхта--Фраиссе. Примеры.
            \item Элементарные диаграммы и подмодели. Напоминание: теорема Левенгейма--Сколема.
            \item Парадокс Сколема.
        \end{itemize}
    \item[Лекция 4.]
        \begin{itemize}
            \item Теорема о полноте. Явная конструкция модели на термах.
            \item Элементарные диаграммы и подмодели. Напоминание: теорема Левенгейма--Сколема.
            \item Парадокс Сколема.
        \end{itemize}
    \item[Лекция 5.]
        \begin{itemize}
            \item Теорема о полноте.
            \item Явная конструкция модели на термах.
        \end{itemize}        
    \item[Лекция 6.]
        \begin{itemize}
            \item Разрешимые и неразрешимые теории.
            \item Разрешимость теории равенства.
            \item Теория полугрупп. Моделирование машины Тьюринга.
        \end{itemize}
    \item[Лекция 7.]
        \begin{itemize}
            \item Разрешимые и неразрешимые теории.
            \item Разрешимость теории равенства.
            \item Теория полугрупп. Моделирование машины Тьюринга.
        \end{itemize}
    \item[Лекция 8.]
        \begin{itemize}
            \item Теорема Черча.
            \item Арифметическая иерархия. 
            \item $\Sigma_{n} \subsetneq \Sigma_{n + 1}$
        \end{itemize}
    \item[Лекция 8.]
        \begin{itemize}
            \item Машина со счетчиками. Эквивалентность машине Тьюринга.
            \item $\beta$-функция Геделя.
        \end{itemize}
    \item[Лекция 9.]
        \begin{itemize}
            \item Арифмечическая иерархия и выразимость в арифметике.
            \item Теоремы Тарского и Геделя, как следствие строгости иерархии.
            \item Теоремы Тарского и Геделя, явная диагонализация.
        \end{itemize}
    \item[Лекция 9.]
        \begin{itemize}
            \item Минимальная арифметика и арифметика Пеано.
            \item $\Sigma_1$-полнота минимальной арифметики.
            \item Теорема о неполноте для $\Sigma_1$-корректных теорий.
        \end{itemize}
    \item[Лекция 10.]
        \begin{itemize}
            \item Теорема о неполноте. Версия Россера.
        \end{itemize}
    \item[Лекция 11.]
        \begin{itemize}
            \item Типы интепретаций и типы теорий.
            \item Изолированные и неизолированные типы. Примеры.
            \item Реализация типов.
            \item Эквивалентность существования неизолированного типа и бесконечного числа типов.
        \end{itemize}
    \item[Лекция 12.]
        \begin{itemize}
            \item На пути к Omitting Type Theorem.
            \item Предтипы и их реализация.
            \item Эквивалентность существования неизолированного типа и бесконечного числа типов.
        \end{itemize}
    \item[Лекция 13.]
        \begin{itemize}
            \item Насыщеные и ненасыщеные модели.
            \item Насыщеность и изоморфизм.
            \item Omitting Type Theorem.
        \end{itemize}
    \item[Лекция 14.]
        \begin{itemize}
            \item Критерий $\aleph_0$-категоричности теории.
        \end{itemize}
\end{enumerate}