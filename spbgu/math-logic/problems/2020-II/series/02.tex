\documentclass[a4paper, 12pt]{article}
% math symbols
\usepackage{amssymb}
\usepackage{amsmath}
\usepackage{mathrsfs}
\usepackage{mathseries}


\usepackage[margin = 2cm]{geometry}

\tolerance = 1000
\emergencystretch = 0.74cm



\pagestyle{empty}
\parindent = 0mm

\renewcommand{\coursetitle}{DM/ML}
\setcounter{curtask}{1}

\setmathstyle{}{Мат. логика}{3 курс}


\begin{document}

\section*{Решетки}

\begin{definition*}
    Рассмотрим сигнатуру
    $$
        \sigma \coloneqq \avg{\land, \lor; =}.
    $$

    Обозначим за $\PrSys{Lat}$ множество, состоящее из универс. замыканий следующих $\sigma$-формул:
    \begin{itemize}
        \item $x \land y = y \land x$;
        \item $x \lor y = y \lor x$;
        \item $x \land (y \land z) = (x \land y) \land z$;
        \item $x \lor (y \land z) = (x \land y) \land z$;
        \item $x \lor (x \land y) = x$ \hfill (закон поглощения);
        \item $x \land (x \lor y) = x$ \hfill (закон поглощения).
    \end{itemize}

    Под \deftext{решётками} будем понимать модели $\PrSys{Lat}$.
\end{definition*}


\libproblem{math-logic}{lattice-idempotency}

\begin{definition*}
    Введём обозначения:
    \begin{align*}
      x \leq y &\coloneqq x \land y = x;\\
      \funcmath{Inf}(x, y, z) &\coloneqq z \leq x \wedge z \leq y \wedge \forall u\, (u \leq x \wedge u
                                \leq y \rightarrow u \leq z);\\
      \funcmath{Sup}(x, y, z) &\coloneqq x \leq z \wedge x \leq z \wedge \forall u\, (x \leq u \wedge y
                                \leq u \rightarrow z \leq u).
    \end{align*}
    Интуитивно: если $\leq$ определяет частичный порядок, то $\funcmath{Inf}$ определяет инфимум
    относительно $\leq$, а $\funcmath{Sup}$~--- супремум.
\end{definition*}

\task{
    Выведите в $\PrSys{Lat}$: $x \leq y \leftrightarrow x \lor y = y$.
}

\task{
    Выведите в $\PrSys{Lat}$:
    \begin{enumcyr}
        \item $x \leq x$;
        \item $x \leq y \rightarrow (y \leq z \rightarrow x \leq z)$;
        \item $x \leq y \rightarrow (y \leq x \rightarrow x = y)$.
    \end{enumcyr}
}


\task{
    Выведите в $\PrSys{Lat}$:
    \begin{enumcyr}
        \item $x \wedge y = z \leftrightarrow \funcmath{Inf}(x, y, z)$;
        \item $x \vee y = z \leftrightarrow \funcmath{Sup}(x, y, z)$.
    \end{enumcyr}
}

\vspace{0.5cm}

Для каждой решётки $\mathfrak{L}$ возьмём
$$
    \leq_{\mathfrak{L}} \coloneqq \{(a, b) \in L^2 \mid \mathfrak{L} \Vdash {a \leq b}\}
$$
и обозначим за $\mathrm{O}(\mathfrak{L})$ ч.у.м. $\avg{L; \leq_{\mathfrak{L}}}$. Мы будем называть
ч.у.м. $\avg{A; \leq_A}$ \deftext{решёточным}, если у всякого конечного подмножества $A$ есть инфимум и
супремум относительно $\leq_A$.\footnote{Достаточно
потребовать, чтобы у всякого $\{a, b\} \subseteq A$ были инфимум и супремум относительно $\leq_A$.}

\task{
    Покажите, что:
    \begin{enumcyr}
        \item для любых решёток $\mathfrak{L}_1$ и $\mathfrak{L}_2$,
            $$
                \mathfrak{L}_1 = \mathfrak{L}_2 \quad \Longleftrightarrow \quad
                \mathrm{O}(\mathfrak{L}_1) = \mathrm{O}(\mathfrak{L}_2);
            $$
        \item для каждого решёточного ч.у.м. $\mathfrak{A}$ найдётся такая решётка $\mathfrak{L}$, что
            $\mathrm{O}(\mathfrak{L}) = \mathfrak{A}$.
    \end{enumcyr}
}


\begin{definition*}
    Будем называть решётку:
    \begin{itemize}
        \item \deftext{дистрибутивной}, если в ней истинны
            \begin{itemize}
                \item $\forall x\, \forall y\, \forall z\, (x \land (y \lor z) = (x \land y) \lor (x
                    \land z))$;
                \item $\forall x\, \forall y\, \forall z\, (x \lor (y \land z) = (x \lor y) \land (x \lor
                    z))$;
            \end{itemize}
        \item \deftext{ограниченной}, если в ней истинны
            \begin{itemize}
                \item $\exists u\, \forall x\, (u \land x = u)$; \hfill $(\exists u\, \forall x\, (u \leq
                    x))$
                \item $\exists u\, \forall x\, (x \vee u = u)$. \hfill $(\exists u\, \forall x\, (x \leq
                    u))$
            \end{itemize}
    \end{itemize}

    Пусть $\mathfrak{L}$~--- ограниченная дистрибутивная решётка. Возьмём
    \begin{align*}
      0 \coloneqq \inf_{\leq_{\mathfrak{L}}} L \quad \text{и} \quad 1 \coloneqq
      \sup_{\leq_{\mathfrak{L}}} L.
    \end{align*}

    Под \deftext{дополнением $a \in L$ в $\mathfrak{L}$} будем понимать такой $b \in L$, что ${a \land b}
    = 0$ и ${a \lor b} = 1$. Дополнение может и не существовать.
\end{definition*}


\task{
    Пусть $\mathfrak{L}$~--- ограниченная дистрибутивная решётка. Покажите, что для любых $a, b, c \in
    L$,
    $$
        a \land b = a \land c = 0 \quad \text{и} \quad a \lor b = a \lor c = 1 \quad \Longrightarrow
        \quad b = c.
    $$
    Здесь и далее мы пишем $\land$ и $\lor$ вместо $\land^{\mathfrak{L}}$ и $\lor^{\mathfrak{L}}$ во
    избежание загромождения текста.
}


\begin{definition*}
    Будем говорить, что ч.у.м. $\avg{A; \leqslant_A}$ \deftext{решёточно полон}, если у всякого
    подмножества $A$ есть инфимум и супремум относительно $\leq_A$.
\end{definition*}

\task{
    Пусть $\mathfrak{A}$~--- решёточно полный ч.у.м., а $f$~--- гомоморфизм из $\mathfrak{A}$ в
    $\mathfrak{A}$. Возьмём $F \coloneqq \{a \in A \mid f(a) = a\}$. Покажите, что индуцированный ч.у.м.\
    $\mathfrak{F}$ с носителем $F$ также решёточно полон.
}


\begin{remark*}
    Соответственно решётку $\mathfrak{L}$ называют \deftext{полной}, если у каждого подмножества $L$ есть
    инфимум и супремум относительно $\leq_{\mathfrak{L}}$. Это упражнение можно переформулировать в
    терминах решёток. Этот результат играет важную роль в теоретической информатике.
\end{remark*}


\section*{Булевы алгебры}

Рассмотрим расширенную сигнатуру:
$$
    \sigma' \coloneqq \avg{0, 1; \land, \lor, \neg; =}.
$$

Обозначим за $\PrSys{BA}$ множество, состоящее из элементов $\PrSys{Lat}$, а также универсальных
замыканий следующих $\sigma'$-формул:
\begin{itemize}
    \item $x \land (y \lor z) = (x \land y) \lor (x \land z)$;
    \item $x \lor (y \land z) = (x \lor y) \land (x \lor z)$;
    \item $0 \land x = 0$;
    \item $x \lor 1 = 1$;
    \item $x \land \neg x = 0$; \hfill \hinttext{аксиома дополнения}
    \item $x \lor \neg x = 1$. \hfill \hinttext{аксиома дополнения}
\end{itemize}
Под \deftext{булевыми алгебрами} понимаются модели $\PrSys{BA}$. Значит, $\sigma'$-структура
$\mathfrak{B}$ является булевой алгеброй тогда и только тогда, когда $\sigma$-обеднение
$\mathfrak{B}$~--- ограниченная дистрибутивная решётка, в которой у каждого элемента есть дополнение,
причём: 
\begin{itemize}
    \item $0^{\mathfrak{B}}$~--- наименьший элемент, а $1^{\mathfrak{B}}$~--- наибольший элемент;
    \item $\neg^{\mathfrak{B}}$~--- функция, сопоставляющая всякому элементу $B$ его
        дополнение.\footnote{Как мы знаем, это дополнение единственно.}
\end{itemize}
Разумеется, так как в теории решёток $x \land y = x$ равносильно $x \lor y = y$, аксиомы для наименьшего
и наибольшего элементов можно переформулировать следующим образом:
\begin{itemize}
    \item $0 \lor x = x$;
    \item $x \land 1 = x$.
\end{itemize}
Они в некотором смысле чуть сильнее, чем исходные аксиомы.


\task{
    Обозначим за $\PrSys{BA}^{\star}$ множество, состоящее из универс. замыканий следующих
    $\sigma'$-формул:
    \begin{itemtask}
        \item законы коммутативности;
        \item законы дистрибутивности;
        \item $0 \lor x = x$;
        \item $x \land 1 = x$;
        \item аксиомы дополнения.
    \end{itemtask}
    Покажите, что $[\PrSys{BA}^{\star}] = [\PrSys{BA}]$, т.е. $\PrSys{BA}^{\star}$ дедуктивно
    эквивалентно $\PrSys{BA}$. Таким образом, в $\PrSys{BA}^{\star}$ выводимы законы ассоциативности и
    законы поглощения.
}

\task{
    Выведите в $\PrSys{BA}$: $\neg 0 = 1$ и $\neg 1 = 0$. \hinttext{Комментарий:} $\neg$~---
    функциональный символ.
}

\task{
    Выведите в $\PrSys{BA}$: $\neg\neg x = x$. \hinttext{Комментарий:} $\neg$~---
    функциональный символ.
}

\task{
    Выведите в $\PrSys{BA}$ \emph{законы де Моргана}:
    \begin{enumcyr}
        \item $\neg (x \land y) = \neg x \lor \neg y$;
        \item $\neg (x \lor y) = \neg x \land \neg y$.
    \end{enumcyr}
}


\end{document}



%%% Local Variables:
%%% mode: latex
%%% TeX-master: t
%%% End:
