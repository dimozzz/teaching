\documentclass[a4paper, 12pt]{article}
% math symbols
\usepackage{amssymb}
\usepackage{amsmath}
\usepackage{mathrsfs}
\usepackage{mathseries}


\usepackage[margin = 2cm]{geometry}

\tolerance = 1000
\emergencystretch = 0.74cm



\pagestyle{empty}
\parindent = 0mm

\renewcommand{\coursetitle}{DM/ML}
\setcounter{curtask}{1}

\setmathstyle{}{Теория множеств}{1 курс}


\begin{document}

\begin{definition*}
    Определим \deftext{упорядоченную пару} $X_1$ и $X_2$ как
    $$
        \left( X_1, X_2 \right) \coloneqq \{\{ X_1 \}, \{ X_1, X_2 \}\}.
    $$
        
    Заметим, что такие пары обладают следующим свойством:
    \begin{equation*}
        (X_1, X_2) = (Y_1, Y_2) \quad \text{тогда и только тогда, когда} \quad
        X_1 = Y_1~\, \text{и}~\, X_2 = Y_2.
    \end{equation*}

    Далее, для любых $X$ и $Y$ выражение
    \begin{align*}
      {X \times Y}\
      :=&\ {\left\{ \left( x, y \right) \mid {x \in X \wedge y \in Y} \right\}}\\
      =&\ {\left\{ u \mid {\exists x}\, {\exists y}\, {\left( u = \left( x, y \right) \wedge x \in X
         \wedge y \in Y \right)} \right\}}\\
      =&\ {\left\{ u \in \mathcal{P} \left( \mathcal{P} \left( X \cup Y \right) \right) \mid {\exists
         x}\, {\exists y}\, {\left( u = \left( x, y \right)
         \wedge x \in X \wedge y \in Y \right)} \right\}}
    \end{align*}
    задаёт множество, которое мы будем называть \deftext{прямым} (или \deftext{декартовым})
    \deftext{произведением} $X$ и $Y$. Под \deftext{бинарными отношениями на} $X$ будем понимать
    произвольные подможества $X \times X$. Бинарное отношение $R \subseteq X \times X$ на $X$ будем
    называть:
    \begin{itemize}
        \item \deftext{рефлексивным}, если ${\forall x}\, {\left( {x \in X} \rightarrow {xRx} \right)}$;
        \item \deftext{иррефлексивным}, если $\neg {\exists x}\, {\left( {x \in X} \wedge {xRx}
            \right)}$;\
        \item \deftext{транзитивным}, если ${\forall x}\, {\forall y}\, {\forall z}\, {\left( \left( {x R
            y} \wedge {y R z} \right) \rightarrow {x R z} \right)}$;
        \item \deftext{симметричным}, если ${\forall x}\, {\forall y}\, {\left( {x R y} \rightarrow {y R
            x} \right)}$;
        \item \deftext{антисимметричным}, если ${\forall x}\, {\forall y}\,
            {\left( \left( {x R y} \wedge {y R x} \right) \rightarrow {x = y} \right)}$.
    \end{itemize}
    Здесь для удобства мы пишем $xRy$ вместо $\left( x, y \right) \in R$.

    Будем говорить, что отношение $R$ является:
    \begin{itemize}
        \item \deftext{предпорядком} на $X$, если $R$ рефлексивно и транзитивно;
        \item \deftext{строгим частичным порядком} на $X$, если $R$ иррефлексивно и транзитивно;
        \item \deftext{частичным порядком} на $X$, если $R$ рефлексивно, антисимметрично и транзитивно;
        \item \deftext{эквивалентностью} на $X$, если $R$ рефлексивно, симметрично и транзитивно.
    \end{itemize}
    Данные базовые понятия играют весьма важную роль в математике.
\end{definition*}

\begin{definition*}
    Пусть $\approx$~--- эквивалентность на $X$. Для каждого $x \in X$ под \deftext{классом
        эквивалентности $x$ по $\approx$} понимается множество
    $$
    {\left[ x \right]}_{\approx} \coloneqq {\left\{ u \in X \mid {x \approx u} \right\}}.
    $$

    \deftext{Фактор-множество $X$ по $\approx$} будем определять как
    $$
        X_{/\approx} \coloneqq {\left\{ {\left[ x \right]}_{\approx} \mid x \in X \right\}},
    $$
    т.е.\ как множество всех классов эквивалентности элементов $X$ по $\approx$.
\end{definition*}


Для отношений эквивалентности имеется простая характеризация. Чтобы её сформулировать, давайте введём
пару вспомогательных понятий.
\begin{definition*}
    Будем называть $Y$ (\deftext{взаимно}, или \deftext{попарно}) \deftext{дизъюнктным}, если оно
    удовлетворяет условию
    \[
        {\forall u}\, {\forall v}\, {\left( \left( {u \in Y} \wedge {v \in Y} \wedge {u \ne v} \right)
              \rightarrow {u \cap v = \varnothing} \right)} .
    \]
    Будем говорить, что $Y$ является \deftext{разбиением} $X$, если $Y$ дизъюнктно, $\varnothing \not \in Y$
    и $\bigcup Y = X$. Очевидно, элементы всякого разбиения $X$ обязаны быть подмножествами $X$.
\end{definition*}


\libproblem{set-theory}{basic-operations}

\end{document}



%%% Local Variables:
%%% mode: latex
%%% TeX-master: t
%%% End:
