\documentclass[12pt, fleqn, a4paper]{article}


\usepackage{amsmath}
\usepackage{amssymb}
\usepackage{amsfonts}
\usepackage{textcomp}
\usepackage{amsthm}
\usepackage{mathtools}
\usepackage{xspace}
\usepackage[classfont = bold]{complexity}
\usepackage{fullpage}
\usepackage[russian, english]{babel}
\usepackage[utf8]{inputenc}
\usepackage[
    sorting = ydnt,
    style = alphabetic,
    maxbibnames = 99,
    backend = biber
    ]{biblatex}
\addbibresource{main.bib}



\newtheorem{conjecture}{Conjecture}[section]
\theoremstyle{definition}
\newtheorem{theorem}{Theorem}[section]
\newtheorem*{theorem*}{Theorem}
\newtheorem{lemma}{Lemma}[section]
\newtheorem{corollary}{Corollary}[section]
\newtheorem{proposition}{Proposition}[section]
\newtheorem{fact}{Fact}[section]
\newtheorem{problem}{Problem}[section]
\newtheorem{exercise}{Exercise}[section]
\newtheorem{example}{Example}[section]
\newtheorem{definition}{Definition}[section]
\newtheorem{remark}{Remark}[section]
\newtheorem{algorithm}{Algorithm}[section]


\newcommand{\class}[1]{\mathbf{#1}}
\newcommand{\co}{\mathrm{co}}
\newcommand{\alg}[1]{\mathit{#1}}
\newcommand{\lang}[1]{\mathtt{#1}}


% classes (1)

\newcommand{\DTime}{\class{DTime}}
\newcommand{\RTime}{\class{RTime}}
\newcommand{\UTime}{\class{UTime}}
\newcommand{\NTime}{\class{NTime}}
\newcommand{\BPTime}{\class{BPTime}}


\renewcommand{\P}{\class{P}}
\newcommand{\ZPP}{\class{ZPP}}
\newcommand{\RP}{\class{RP}}
\newcommand{\coRP}{\co\class{RP}}
\newcommand{\UP}{\class{UP}}
\newcommand{\coUP}{\co\class{UP}}
\newcommand{\NP}{\class{NP}}
\newcommand{\coNP}{{\co}\class{NP}}
\newcommand{\BPP}{\class{BPP}}
\newcommand{\SigmaP}[1]{\Sigma^{#1}\class{P}}
\newcommand{\PH}{\class{PH}}
\newcommand{\PP}{\class{PP}}
\newcommand{\IP}{\class{IP}}
\newcommand{\OP}{\class{\oplus P}}


\newcommand{\EXP}{\class{EXP}}
\newcommand{\MIP}{\class{MIP}}
\newcommand{\NEXP}{\class{NEXP}}
\newcommand{\coNEXP}{{\co}\class{NEXP}}
\newcommand{\MAEXP}{\class{MA}_\class{EXP}}


% classes (2)

\newcommand{\Ppoly}{\class{P}/\class{poly}}
\newcommand{\NC}{\class{NC}}


\newcommand{\DSpace}{\class{DSpace}}
\newcommand{\NSpace}{\class{NSpace}}
\newcommand{\PSPACE}{\class{PSPACE}}

\newcommand{\EXPSPACE}{\class{EXPSPACE}}


% algorithms and proof systems


\newcommand{\DPLL}{\alg{DPLL}}
\newcommand{\OBDD}{\alg{OBDD}}
\newcommand{\pOBDD}{\pi\text{-}\alg{OBDD}}
\newcommand{\DPLLL}{\alg{DPLL}_{lin}}
\newcommand{\ResL}{\alg{Res}_{lin}}
\newcommand{\SemL}{\alg{Sem}_{lin}}



% languages


\newcommand{\SAT}{\lang{SAT}}
\newcommand{\GNI}{\lang{GNI}}
\newcommand{\MAJSAT}{\lang{MAJ}\text{-}\lang{SAT}}
\newcommand{\QBF}{\lang{QBF}}



% other

\newcommand{\poly}{\mathrm{poly}}
\newcommand{\Nat}{\mathbb{N}}
\newcommand{\bool}{\{0, 1\}}

\newcommand{\Img}{\mathop{\mathrm{Im}}}

\DeclareMathOperator*{\supp}{supp}
\DeclareMathOperator*{\Exp}{E}
\DeclareMathOperator*{\rk}{rk}




%%% Local Variables:
%%% mode: latex
%%% TeX-master: t
%%% End:


\begin{document}

	\setcounter{curtask}{9}

\mytitle{2 (на 3.10)}

\begin{task}
    Докажите, что множество всех рациональных чисел меньших $\pi$ разрешимо.
\end{task}

\begin{task}
    Существует ли алгоритм, проверяющий, работает ли данная программа
    полиномиальное время?
\end{task}

\begin{task}
    Приведите пример двух непересекающихся неперечислимых множеств.
\end{task}

\begin{task}
    Докажите, что для каждой вычислимой функции $f$ найдется
    псевдообратная вычислимая функция $g$. А именно, $g$ определена на
    множестве значений $f$, и для всех $x$ из области определения $f$
    выполняется $f(g(f(x))) = f(x)$.
\end{task}

\begin{task}
    Приведите пример неразрешимого множества $A \subseteq \Nat \times \Nat$,
    такого, что все его горизонтальные и вертикальные сечения
    разрешимы (т.е. для любого $x$ разрешимы $A \cap \{\{x\} \times \Nat\}$
    и $A \cap \{\Nat \times \{x\}\}$)
\end{task}

\begin{task}
    Докажите, что существует язык, который можно распознать с памятью $2^n$ ($n$~---
    длина слова), но нельзя с памятью $n$. (подсказка: диагонализация)
\end{task}

\end{document}



\setmathstyle{}{Мат. логика}{3 курс}


\begin{document}

Введём обозначения:
$$
    x \oplus y \coloneqq (x \land \neg y) \lor (y \land \neg x) \quad \text{и} \quad
    x \odot y \coloneqq x \land y.
$$
На булевы алгебры можно смотреть как на особого рода кольца:

\task{
    Выведите в $\PrSys{BA}$:
    \begin{enumcyr}
        \item $x \oplus (y \oplus z) = (x \oplus y) \oplus z$;
        \item $x \oplus y = y \oplus x$;
        \item $x \oplus 0 = x$;
        \item $x \oplus x = 0$;
        \item $x \odot (y \odot z) = (x \odot y) \odot z$;
        \item $x \odot y = y \odot x$;
        \item $x \odot 1 = x$;
        \item $x \odot (y \oplus z) = (x \odot y) \oplus (x \odot z)$;
        \item $x \odot x = x$.
    \end{enumcyr}
}

\task{
    Выведите в $\PrSys{BA}$:
    \begin{enumcyr}
        \item $x \wedge y = x \odot y$;
        \item ${x \lor y} = x \oplus y \oplus (x \odot y)$;
        \item $\neg x = x \oplus 1$.
    \end{enumcyr}
}

\begin{definition*}
    Обозначим через $\PrSys{BR}$ множество, состоящее из:
    \begin{itemize}
        \item аксиом теории колец (не обязательно коммутативных, но с единицей);
        \item $\forall x\, (x \cdot x = x)$.
    \end{itemize}
    Под \deftext{булевыми}~--- или \deftext{идемпотентными}~--- \deftext{кольцами} будем понимать модели
    $\PrSys{BR}$.
\end{definition*}


\task{
    Выведите в $\PrSys{BR}$:
    \begin{enumcyr}
        \item $x + x = 0$;
        \item $x \cdot y = y \cdot x$.
    \end{enumcyr}
}

\task{
    Пусть $\mathfrak{A}$~--- булево кольцо и $\abs{A} \geq 3$. Покажите, что в $\mathfrak{A}$ есть
    ненулевые делители нуля.
}

\task{
    Будем называть группу \deftext{булевой}, если в ней истинно $\forall x\, (x \circ x = e)$, т.е. все
    её элементы имеют порядок $2$. Покажите, что всякая булева группа коммутативна.
}



\vspace{0.3cm}

Для каждой булевой алгебры $\mathfrak{B}$ обозначим через $\mathrm{R}(\mathfrak{B})$ булево кольцо с
носителем $B$, в котором:
\begin{align*}
  0^{\mathrm{R}(\mathfrak{B})} &\coloneqq 0,\\
  1^{\mathrm{R}(\mathfrak{B})} &\coloneqq 1,\\
  +^{\mathrm{R}(\mathfrak{B})} &\coloneqq \lambda a.\lambda b. [a \oplus b],\\
  \cdot^{\mathrm{R}(\mathfrak{B})} &\coloneqq \lambda a.\lambda b. [a \odot b],\\
  -^{\mathrm{R}(\mathfrak{B})} &\coloneqq \lambda a. [a],
\end{align*}
где $0$, $1$, $\oplus$ и $\odot$ в правых частях интерпретируются как в $\mathfrak{B}$.

\vspace{0.3cm}


\task{
    Покажите, что:
    \begin{enumcyr}
        \item для любых булевых алгебр $\mathfrak{B}_1$ и $\mathfrak{B}_2$,
            $$
            \mathfrak{B}_1\ =\ \mathfrak{B}_2 \Longleftrightarrow \mathrm{R}(\mathfrak{B}_1) =
                \mathrm{R}(\mathfrak{B}_2);
            $$
        \item для каждого булева кольца $\mathfrak{A}$ найдётся такая булева алгебра $\mathfrak{B}$, что
            $\mathrm{R}(\mathfrak{B}) = \mathfrak{A}$.
    \end{enumcyr}
}


\libdefinition{real-closed-field}

\task{
    Выведите в теории линейно упорядоченных колец:
    $$
        x \ne 0 \longrightarrow 0 < x \cdot x.
    $$
}

\task{
    Выведите в теории линейно упорядоченных полей:
    $$
        x < y \longrightarrow \exists u\, x < u < y.
    $$
}

\task{
    Покажите, что для любого бескванторного $\sigma$-предложения $\Phi$:
    \begin{enumcyr}
        \item если $\mathfrak{R} \Vdash \Phi$, то в теории линейно упорядоченных полей выводимо $\Phi$;
        \item если $\mathfrak{R} \nVdash \Phi$, то в теории линейно упорядоченных полей выводимо $\neg
            \Phi$.
    \end{enumcyr}
}


\task{
    Покажите, что для любого линейно упорядоченного поля $\mathfrak{F}$ следующие условия эквивалентны:
    \begin{enumcyr}
        \item $\mathfrak{F}$ вещественно замкнуто;
        \item в $\mathfrak{F}$ истинно $\forall x\, (0 < x \rightarrow \exists u\, x = u \cdot u)$, а
            также все $\sigma$-формулы вида:
            $$
                \forall \overline{u}\, \forall x\, \forall y\,
                (\mathrm{p}_n(x; \overline{u}) \cdot \mathrm{p}_n(y; \overline{u}) < 0 \rightarrow
                \exists z\, (x < z < y \wedge \mathrm{p}_n(z; \overline{u}) = 0)),
            $$
            т.е.\ для $\mathfrak{F}$ верна <<полиномиальная версия>> теоремы о среднем значении.
    \end{enumcyr}
}

\dzcomment{
    Можно пользоваться задачей 11.
}


\begin{remark*}
    Отсюда получается альтернативная аксиоматизация класса всех вещественно замкнутых полей. Нетрудно
    убедиться, что дедуктивное замыкание каждой из эти аксиоматизаций совпадает с
    $\Theory(\mathfrak{R})$, в силу элиминации кванторов.
\end{remark*}

\breakline

Данная задача является бонусной и разбираться не будет.

\task{
    Покажите, что для любого линейно упорядоченного поля $\mathfrak{F}$ следующие условия эквивалентны:
    \begin{enumcyr}
        \item $\mathfrak{F}$ вещественно замкнуто;
        \item $F$ как поле не является алгебраически замкнутым, но его расширение $F[\sqrt{-1}]$
            алгебраически замкнуто.
    \end{enumcyr}
}

\end{document}



%%% Local Variables:
%%% mode: latex
%%% TeX-master: t
%%% End:
