\documentclass[12pt, fleqn, a4paper]{article}


\usepackage{amsmath}
\usepackage{amssymb}
\usepackage{amsfonts}
\usepackage{textcomp}
\usepackage{amsthm}
\usepackage{mathtools}
\usepackage{xspace}
\usepackage[classfont = bold]{complexity}
\usepackage{fullpage}
\usepackage[russian, english]{babel}
\usepackage[utf8]{inputenc}
\usepackage[
    sorting = ydnt,
    style = alphabetic,
    maxbibnames = 99,
    backend = biber
    ]{biblatex}
\addbibresource{main.bib}



\newtheorem{conjecture}{Conjecture}[section]
\theoremstyle{definition}
\newtheorem{theorem}{Theorem}[section]
\newtheorem*{theorem*}{Theorem}
\newtheorem{lemma}{Lemma}[section]
\newtheorem{corollary}{Corollary}[section]
\newtheorem{proposition}{Proposition}[section]
\newtheorem{fact}{Fact}[section]
\newtheorem{problem}{Problem}[section]
\newtheorem{exercise}{Exercise}[section]
\newtheorem{example}{Example}[section]
\newtheorem{definition}{Definition}[section]
\newtheorem{remark}{Remark}[section]
\newtheorem{algorithm}{Algorithm}[section]


\newcommand{\class}[1]{\mathbf{#1}}
\newcommand{\co}{\mathrm{co}}
\newcommand{\alg}[1]{\mathit{#1}}
\newcommand{\lang}[1]{\mathtt{#1}}


% classes (1)

\newcommand{\DTime}{\class{DTime}}
\newcommand{\RTime}{\class{RTime}}
\newcommand{\UTime}{\class{UTime}}
\newcommand{\NTime}{\class{NTime}}
\newcommand{\BPTime}{\class{BPTime}}


\renewcommand{\P}{\class{P}}
\newcommand{\ZPP}{\class{ZPP}}
\newcommand{\RP}{\class{RP}}
\newcommand{\coRP}{\co\class{RP}}
\newcommand{\UP}{\class{UP}}
\newcommand{\coUP}{\co\class{UP}}
\newcommand{\NP}{\class{NP}}
\newcommand{\coNP}{{\co}\class{NP}}
\newcommand{\BPP}{\class{BPP}}
\newcommand{\SigmaP}[1]{\Sigma^{#1}\class{P}}
\newcommand{\PH}{\class{PH}}
\newcommand{\PP}{\class{PP}}
\newcommand{\IP}{\class{IP}}
\newcommand{\OP}{\class{\oplus P}}


\newcommand{\EXP}{\class{EXP}}
\newcommand{\MIP}{\class{MIP}}
\newcommand{\NEXP}{\class{NEXP}}
\newcommand{\coNEXP}{{\co}\class{NEXP}}
\newcommand{\MAEXP}{\class{MA}_\class{EXP}}


% classes (2)

\newcommand{\Ppoly}{\class{P}/\class{poly}}
\newcommand{\NC}{\class{NC}}


\newcommand{\DSpace}{\class{DSpace}}
\newcommand{\NSpace}{\class{NSpace}}
\newcommand{\PSPACE}{\class{PSPACE}}

\newcommand{\EXPSPACE}{\class{EXPSPACE}}


% algorithms and proof systems


\newcommand{\DPLL}{\alg{DPLL}}
\newcommand{\OBDD}{\alg{OBDD}}
\newcommand{\pOBDD}{\pi\text{-}\alg{OBDD}}
\newcommand{\DPLLL}{\alg{DPLL}_{lin}}
\newcommand{\ResL}{\alg{Res}_{lin}}
\newcommand{\SemL}{\alg{Sem}_{lin}}



% languages


\newcommand{\SAT}{\lang{SAT}}
\newcommand{\GNI}{\lang{GNI}}
\newcommand{\MAJSAT}{\lang{MAJ}\text{-}\lang{SAT}}
\newcommand{\QBF}{\lang{QBF}}



% other

\newcommand{\poly}{\mathrm{poly}}
\newcommand{\Nat}{\mathbb{N}}
\newcommand{\bool}{\{0, 1\}}

\newcommand{\Img}{\mathop{\mathrm{Im}}}

\DeclareMathOperator*{\supp}{supp}
\DeclareMathOperator*{\Exp}{E}
\DeclareMathOperator*{\rk}{rk}




%%% Local Variables:
%%% mode: latex
%%% TeX-master: t
%%% End:


\begin{document}

	\setcounter{curtask}{9}

\mytitle{2 (на 3.10)}

\begin{task}
    Докажите, что множество всех рациональных чисел меньших $\pi$ разрешимо.
\end{task}

\begin{task}
    Существует ли алгоритм, проверяющий, работает ли данная программа
    полиномиальное время?
\end{task}

\begin{task}
    Приведите пример двух непересекающихся неперечислимых множеств.
\end{task}

\begin{task}
    Докажите, что для каждой вычислимой функции $f$ найдется
    псевдообратная вычислимая функция $g$. А именно, $g$ определена на
    множестве значений $f$, и для всех $x$ из области определения $f$
    выполняется $f(g(f(x))) = f(x)$.
\end{task}

\begin{task}
    Приведите пример неразрешимого множества $A \subseteq \Nat \times \Nat$,
    такого, что все его горизонтальные и вертикальные сечения
    разрешимы (т.е. для любого $x$ разрешимы $A \cap \{\{x\} \times \Nat\}$
    и $A \cap \{\Nat \times \{x\}\}$)
\end{task}

\begin{task}
    Докажите, что существует язык, который можно распознать с памятью $2^n$ ($n$~---
    длина слова), но нельзя с памятью $n$. (подсказка: диагонализация)
\end{task}

\end{document}



\setmathstyle{\includegraphics[scale = 0.05]{../pics/utia-rest.png}}{Мат. логика}{3 курс}


\begin{document}

\task{
    Покажите, что для любого натурального числа $n$ множество всех истинных замкнутых арифметических
    формул, содержащих не более $n$ кванторов, арифметично.
}

\begin{definition*}
    \deftext{Сечением} нестандатной модели арифметики Пеано $M$ будем называть непустое подмножество $C
    \subseteq M$, обладающее следующими свойствами:
    \begin{itemize}
        \item $x < y, y \in C \Rightarrow x \in C$;
        \item $x \in C \Rightarrow S(x) \in C$.
    \end{itemize}
    Соответственно \deftext{собственным сечением} будем называть сечение, в котором $С$ является
    собственным подмножеством $M$.
\end{definition*}

\task{
    Пусть $M$~--- нестандартная модель $\PrSys{PA}$ и $C$~--- собственное сечение $M$. Пусть
    $\bar{a}$~--- кортеж элементов из $M$ и $\varphi(x, \bar{a})$~--- такая формула в сигнатуре
    $\PrSys{PA}$, что $M \models \varphi(b, \bar{a})$ для всех $b \in C$. Покажите, что существует такой
    элемент $c \in M$, что:
    \begin{itemize}
        \item $c$ больше всех элементов из $C$;
        \item $M \models \varphi(c, \bar{a})$.
    \end{itemize}
}

\dzcomment{
    В следующих задачах можно пользоваться ассоциативностью и коммутативностью операций в $\PrSys{PA}$.
}

\task{
    Выведите в $\PrSys{PA}$: $z \ne 0 \rightarrow (x < y \leftrightarrow x \cdot z < y \cdot z)$.
}

\task{
    Выведите в $\PrSys{PA}$: $y \ne 0 \rightarrow x \leq x \cdot y$.
}


\task{
    Под \deftext{принципом возвратной индукции} будем понимать совокупность всех формул вида:
    $$
        \forall x\, ((\forall u < )\, \Phi(x / u) \rightarrow \Phi(x / x)) \rightarrow \forall x\, \Phi.
    $$
    Покажите, что все такие формулы выводимы в $\PrSys{PA}$.
}

\task{
    Под \deftext{принципом минимального элемента} будем понимать совокупность всех формул вида:
    $$
        \exists x\, \Phi(x / x) \rightarrow \exists x\, (\Phi(x / x) \wedge \neg (\exists u < x)\,
        \Phi(x / u)).
    $$
    Покажите, что все такие формулы выводимы в $\mathsf{PA}$.
}


\task{
    Пусть $\Gamma$ состоит из универсальных замыканий следующих формул:
    \begin{enumcyr}
        \item $x < \mathsf{s}(x)$;
        \item $x \ne 0 \rightarrow \exists y\, x = \mathsf{s}(y)$;
        \item всех формул из принципа минимального элемента.
    \end{enumcyr}
    Покажите, что схему аксиом (обычной) индукции можно вывести в $\Gamma$.
}


\task{
    Выведите в $\PrSys{PA}$:
    ${\forall x}\, {\forall y}\, (y \ne 0 \rightarrow (\exists! u)\, (\exists! v < y)\, x = y \cdot u +
    v)$.
}


\begin{definition*}
    Введём следующие обозначения:
    \begin{align*}
      x \mid y &\coloneqq \exists u\, x \cdot u = y;\\
      \funccplx{Irred}(x) &\coloneqq x \ne 0 \wedge x \ne \underline{1} \wedge
                            \forall u\, (u \mid x \rightarrow u = \underline{1} \vee u = x);\\
      \funccplx{Prime}(x) &\coloneqq x \ne 0 \wedge x \ne \underline{1} \wedge
                            \forall u\, \forall v\, (x \mid u \cdot v \rightarrow x \mid u \vee x \mid
                            v).
    \end{align*}
    Здесь используется кольцевая терминология (от англ.\ <<irreducible>> и <<prime>>).
\end{definition*}


\task{
    Выведите в $\PrSys{PA}$:
    $\funccplx{Irred}(x) \leftrightarrow \funccplx{Prime}(x)$.
}

\begin{definition}
    \deftext{Частичным $n$-типом} (или просто $n$-типом) $\sigma$-теории $T$ будем называть любое
    множество $\sigma$-формул $\Gamma$ со свободными переменными $v_1, v_2, \dots, v_n$, что с новыми
    константами $c_1, \dots, c_n$ теория $T \cup \{\varphi(c_1, c_2, \dots, c_n) \mid \varphi(v_1, v_2,
    \dots, v_n) \in \Gamma\}$ совместна.
    
    $\Gamma$ будем называть \deftext{полным $n$-типом} теории $T$, если $\Gamma$ является $n$-типом и для
    любой формулы $\varphi(v_1, \dots, v_n)$ либо $\varphi \in \Gamma$, либо $\neg \varphi \in \Gamma$.

    $n$-тип $\Gamma$ теории $T$ \deftext{реализуется} в модели $M$, если найдется такая
    последовательность $a_1, \dots, a_n \in M$:
    $$
        \varphi(v_1, \dots, v_n) \in \Gamma \Rightarrow M \models \varphi(a_1, a_2, \dots, a_n).
    $$
\end{definition}

\task{
    Рассмотрим теорию плотных линейных порядков без концов и $M$~--- модель данной теории на множестве
    $\mathbb{Q}$. Пусть $p \coloneqq \{\varphi(v) \mid M \models
    \varphi\left(\frac{1}{2}\right)\}$. Реализуется ли тип $p$ в модели $M$, если да, то конечно ли число
    точек реализации?
}


\end{document}



%%% Local Variables:
%%% mode: latex
%%% TeX-master: t
%%% End: