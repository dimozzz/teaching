Expander graphs is one the powerful tool of theoretical computer science and discrete mathematics. These
graphs was defined in 1970s and widely used in difference areas during the last 40 years. In particular,
expanders are used in derandomization constructions, error-correcting codes, proof complexity. One of the
most famous application is a proof of the $\PCP$ Theorem.

In this course we will study a connection between combination and spectrum properties of expander
graphs, some explicit constructions of it, and applications (codes, pseudo-random generator, etc.).

In the second part we will concentrated on the error-correcting codes and its applications.