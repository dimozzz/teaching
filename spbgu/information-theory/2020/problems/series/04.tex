\documentclass[a4paper, 12pt]{article}
% math symbols
\usepackage{amssymb}
\usepackage{amsmath}
\usepackage{mathrsfs}
\usepackage{mathseries}


\usepackage[margin = 2cm]{geometry}

\tolerance = 1000
\emergencystretch = 0.74cm



\pagestyle{empty}
\parindent = 0mm

\renewcommand{\coursetitle}{DM/ML}
\setcounter{curtask}{1}

\setmathstyle{Апрель 30}{Теория информации}{2 курс}


\begin{document}

\libproblem{inf-theory}{mutual-information}

\begin{definition*}
    \deftext{Идеальная схема разделения секрета}~--- это совершенная схема разделения секрета с
    дополнительным требованием <<экономности>>.
    $$
        \forall i \in \{1, 2, \dots, n\},\ h(S_i) \le h(S_0).
    $$
\end{definition*}

\libproblem{inf-theory}{secret-ideal}
\libproblem{inf-theory}{secret-not-ideal}
\libproblem{cc}{sum}

\breakline

\libproblem[3][23.04]{inf-theory}{shannon-fano-opt}
\libproblem[4][23.04]{inf-theory}{information-markov}
\libproblem[5][23.04]{inf-theory}{stone-lift}

\dzcomment{
    Условие изменено.
}


\end{document}



%%% Local Variables:
%%% mode: latex
%%% TeX-master: t
%%% End:
