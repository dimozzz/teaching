\documentclass[a4paper, 12pt]{article}
% math symbols
\usepackage{amssymb}
\usepackage{amsmath}
\usepackage{mathrsfs}
\usepackage{mathseries}


\usepackage[margin = 2cm]{geometry}

\tolerance = 1000
\emergencystretch = 0.74cm



\pagestyle{empty}
\parindent = 0mm

\renewcommand{\coursetitle}{DM/ML}
\setcounter{curtask}{1}

\setmathstyle{Дедлайн: 05.05}{Теория информации}{2 курс}


\begin{document}

\libproblem{inf-theory}{axiom-entropy}
\libproblem{inf-theory}{information-zero-ex}
\libproblem{inf-theory}{fano-inequality}
\libproblem{inf-theory}{shearers-lemma}
\libproblem{inf-theory}{triangles-count}

\dzcomment{
    В этой задаче мы рассматриваем настоящие треугольники, а не те, что были на лекции.
}

\libproblem{inf-theory}{find-n-hard}
\libproblem{inf-theory}{inf-inequality-hard}
\libproblem{cc}{inner-product}
\libproblem{cc}{median-easy}





\end{document}



%%% Local Variables:
%%% mode: latex
%%% TeX-master: t
%%% End:
