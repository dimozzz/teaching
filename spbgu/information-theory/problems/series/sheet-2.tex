\documentclass[a4paper, 12pt]{article}
% math symbols
\usepackage{amssymb}
\usepackage{amsmath}
\usepackage{mathrsfs}
\usepackage{mathseries}


\usepackage[margin = 2cm]{geometry}

\tolerance = 1000
\emergencystretch = 0.74cm



\pagestyle{empty}
\parindent = 0mm

\renewcommand{\coursetitle}{DM/ML}
\setcounter{curtask}{1}

\setmathstyle{Дедлайн: 19.05}{Теория информации}{2 курс}


\begin{document}

\libproblem{inf-theory}{secret-not-ideal-2}
\libproblem{cc}{clique-ind}

\begin{definition*}
    Пусть $f: X \times Y \to Z$ и $\mu$~--- распределение на $X \times Y$. Заметим, что для любого
    коммуникационного протокола $\Pi$ для функции $f$ распределение $\mu$ индуцирует распределение на
    листьях данного протокола естественным образом. \deftext{Внутренней информационной стоимостью} (или
    \deftext{внутрениим информационным разглашением}) протокола $\Pi$ по распределению $\mu$ будем называть
    величину:
    $$\ICint[\mu](\Pi) \coloneqq I(\Pi(X, Y) : X \mid Y) + I(\Pi(X, Y) : Y \mid X).$$
    Также определим внешнюю информационную сложность самой функции
    $\ICint[\mu](f) \coloneqq \min\limits_{\Pi} \ICint[\mu](\Pi)$.
\end{definition*}

\libproblem{cc}{inner-inf-upper}
\libproblem{cc}{fooling-limits}


\libproblem{circuit-complexity}{or-composition}
\libproblem{circuit-complexity}{maj-lower-formula}
\libproblem{inf-theory}{find-n-distribution}
\libproblem{cc}{one-way-avg}


\end{document}



%%% Local Variables:
%%% mode: latex
%%% TeX-master: t
%%% End:
