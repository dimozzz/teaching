\documentclass[a4paper, 12pt]{article}
% math symbols
\usepackage{amssymb}
\usepackage{amsmath}
\usepackage{mathrsfs}
\usepackage{mathseries}


\usepackage[margin = 2cm]{geometry}

\tolerance = 1000
\emergencystretch = 0.74cm



\pagestyle{empty}
\parindent = 0mm

\renewcommand{\coursetitle}{DM/ML}
\setcounter{curtask}{1}

\setmathstyle{Май 21}{Теория информации}{2 курс}


\begin{document}

\libproblem{cc}{one-way-inv}

\begin{definition*}
    Пусть $F\colon \{0, 1\}^* \to \{0, 1\}^*$~--- вычислимая функция. \deftext{Сложность описания} $x$
    относительно $F$ определим следующим образом: $K_F(x) \coloneqq \min\{|p| \mid F(p) = x\}$.

    Будем говорить, что способ описания $F$ \textit{не хуже} $G$, обозначим $F \prec G$, если существует
    такая константа $c_G$, что для $\forall x \in \{0, 1\}^*, K_F(x) \le K_G(x) +
    c_G$.

    \deftext{Оптимальным} будем называть такой способ описания $U$, который не хуже любого
    другого. \deftext{Колмогоровской сложностью} $x$ будем называть значение $K(x) \coloneqq K_U(x)$.
\end{definition*}

\libproblem{kolmogorov}{opt-exists}
\libproblem{kolmogorov}{series-divergent}
\libproblem{kolmogorov}{composition}
\libproblem{kolmogorov}{parity}
\libproblem{kolmogorov}{single-bit}

\breakline

\libproblem*{cc}{chrom-number}

\dzcomment{
    Только в группе 13:40.
}


\libproblem{cc}{IC-ext}
\libproblem{inf-theory}{ind-var-eq}
\libproblem{cc}{rand-gt}

\dzcomment{
    Условие изменено.
}



\end{document}



%%% Local Variables:
%%% mode: latex
%%% TeX-master: t
%%% End:
