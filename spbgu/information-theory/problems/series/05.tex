\documentclass[a4paper, 12pt]{article}
% math symbols
\usepackage{amssymb}
\usepackage{amsmath}
\usepackage{mathrsfs}
\usepackage{mathseries}


\usepackage[margin = 2cm]{geometry}

\tolerance = 1000
\emergencystretch = 0.74cm



\pagestyle{empty}
\parindent = 0mm

\renewcommand{\coursetitle}{DM/ML}
\setcounter{curtask}{1}

\setmathstyle{Май 6}{Теория информации}{2 курс}


\begin{document}

\libproblem*{cc}{chrom-number}
\libproblem{cc}{KW-hard-exists}
\dzcomment{
    \textit{Подсказка:} формульная сложность такой функции будет $2^{\Omega(n)}$.
}

\libproblem{cc}{rand-eq}
\libproblem{cc}{balance}

\begin{definition*}
    Пусть $f: X \times Y \to Z$ и $\mu$~--- распределение на $X \times Y$. Заметим, что для любого
    коммуникационного протокола $\Pi$ для функции $f$ распределение $\mu$ индуцирует распределение на
    листьях данного протокола естественным образом. \deftext{Внешней информационной стоимостью} (или
    \deftext{внешним информационным разглашением}) протокола $\Pi$ по распределению $\mu$ будем называть
    величину:
    $$\ICext[\mu](\Pi) \coloneqq I(\Pi(X, Y) : X, Y).$$
    Также определим внешнюю информационную сложность самой функции
    $\ICext[\mu](f) \coloneqq \min\limits_{\Pi} \ICext[\mu](\Pi)$.
\end{definition*}

\libproblem{cc}{IC-ext}
\libproblem{cc}{rand-gt}

\libproblem{inf-theory}{ind-var-eq}


\breakline

\libproblem[2][30.04]{inf-theory}{secret-ideal}

\dzcomment{
    Давайте будем считать в данной задаче, что $S_0$ распределено равномерно на некотором множестве.
}

\end{document}



%%% Local Variables:
%%% mode: latex
%%% TeX-master: t
%%% End:
