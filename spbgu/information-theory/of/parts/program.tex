\begin{enumerate}
    \item[Лекция 1.] Информация по Хартли. Биты. Условная информация. Применения информации (задачи про
        взвешивания монет, задача про поиск числа, дерево решений со стоимостями).
    \item[Практика 1.] Информация в тексте с заданными частотами.
    \item[Практика 2.] Неравенство Крафта--Макмиллана.
    \item[Лекция 2.] Энтропия и ее свойства. Применения энтропии (нижние оценки на количество
        взвешиваний, неравенство $\sum\limits_{i = 0}^{k} \binom{n}{i} \le 2^{h(k / n) n}$, треугольники
        и углы в графах \cite{KR11}).
    \item[Практика 3.] Взаимная информация и неравенства для нее.
    \item[Лекция 3.] Энтропия и кодирование (теорема Шеннона, код Шеннона--Фано, арифметическое
        кодирование, код Хаффмана). Нижние и верхние оценки. Код Хемминга.
    \item[Практика 4.] Оценки на код Шеннона--Фано. Коммуникационные протоколы и метод рангов.
    \item[Лекция 4.] Кодирование <<почти всех>> слов кодами длиной $h(p)$. Применения в
        криптографии. Разделение секрета. Схема Шамира.
    \item[Практика 5.] Структуры доступа без экономных схем разделения секрета.
    \item[Лекция 5.] Коммуникационная сложность. Прямоугольники и ранги. <<Fooling set>>. Игра
        Карчмера--Вигдерсона. Информационное разглашение. Оценка Храпченко.
    \item[Практика 5.] Поиск сложных объектов методом подсчета.
    \item[Лекция 6.] Колмогоровская сложность и ее свойства. Оптимальные описания,
        невычислимость. Условная сложность. Применения (задачи про перенос информации на ленте машины
        Тьюринга).
    \item[Лекция 7.] Префиксная сложность, случайные по Мартин-Лёфу. Закон больших чисел. Язык не
        распознаваемых при помощи автомата с $k$ головками.
\end{enumerate}