\begin{enumerate}
    \item Лекция 1. Информация по Хартли. Биты. Условная информация.
        Применения информации (задачи про взвешивания монет, задача про поиск числа, дерево решений со
        стоимостями).
    \item Практика 1. Информация в тексте с заданными частотами.
    \item Практика 2. Неравенство Крафта--Макмиллана.
    \item Лекция 2. Энтропия и ее свойства. Применения энтропии (нижние оценки на количество взвешиваний,
        неравенство $\sum\limits_{i = 0}^{k} \binom{n}{i} \le 2^{h(k / n) n}$, треугольники и углы в
        графах (Kopparty--Rossman)).
        Практика 3. Взаимная информация и неравентсва для нее.
    \item Лекция 3. Энтропия и кодирование (теорема Шеннона, код Шеннона--Фано, арифметическое
        кодирование, код Хаффмана). Нижние и верхние оценки.
    \item Практика 4. В частности, задачи на ``энтропийные профили''.
    \item Лекция 4. Применения в криптографии. Разделение секрета.
    \item Лекция 5. Коммуникационная сложность. Прямоугольники и ранги. Игра
        Карчмера--Вигдерсона. Информационное разглашение. Оценка Храпченко.
    \item Лекция 6. Колмогоровская сложность и ее свойства. Оптимальные описания,
        невычислимость. Условная сложность. Применения (задачи про перенос информации на ленте машины Тьюринга).
    \item Лекция 7. Префиксная сложность, случайные по Мартин-Лёфу.
\end{enumerate}