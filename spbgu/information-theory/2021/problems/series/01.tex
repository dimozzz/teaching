\documentclass[a4paper, 12pt]{article}
% math symbols
\usepackage{amssymb}
\usepackage{amsmath}
\usepackage{mathrsfs}
\usepackage{mathseries}


\usepackage[margin = 2cm]{geometry}

\tolerance = 1000
\emergencystretch = 0.74cm



\pagestyle{empty}
\parindent = 0mm

\renewcommand{\coursetitle}{DM/ML}
\setcounter{curtask}{1}

\setmathstyle{01.04.2021}{Теория информации}{2 курс}


\begin{document}

\libproblem{inf-theory}{find-n}
\libproblem{inf-theory}{volume-3-dim}
\libproblem{inf-theory}{volume-4-dim}
\libproblem{inf-theory}{i-dont-know}
\libproblem{inf-theory}{entropy-infinite}

\begin{definition*}
    Будем называть \deftext{кодом} функцию $C\colon \{a_1, \dots, a_n\} \to \{0, 1\}^*$, сопоставляющую
    буквам некоторого алфавита \deftext{кодовые слова}. Если любое сообщение, которое получено
    применением кода $C$, декодируется однозначно (т.е. только единственным образом разрезается на образы
    $C$), то такой код называется \deftext{однозначно декодируемым}.

    Код называется \deftext{префиксным (беспрефиксным, prefix-free)}, если никакое кодовое слово не
    является префиксом другого кодового слова.
\end{definition*}

\libproblem{inf-theory}{prefix-codes}

\end{document}



%%% Local Variables:
%%% mode: latex
%%% TeX-master: t
%%% End:
