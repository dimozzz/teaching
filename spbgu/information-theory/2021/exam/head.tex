\documentclass[a4paper, 12pt]{article}
\usepackage{amssymb}
\usepackage{dsfont}
\usepackage{mathrsfs}
\usepackage{amsfonts}
\usepackage{mathseries}

\usepackage[margin = 2cm]{geometry}

\tolerance = 1000
\emergencystretch = 0.74cm

\pagestyle{empty}
\parindent = 0mm

\setcounter{curtask}{1}

\newcommand{\supp}{\mathrm{supp}}


\begin{document}

\setmathstyle{2021}{Экзамен}{Теория Информации}

\renewcommand{\coursetitle}{\textsc{Ex}}


\subsection*{Правила экзамена}

\begin{itemize}
    \item Экзамен заканчивается в 14:00, решения должны в этот момент оказаться на почте
        sokolov.dmt@gmail.com (допускается погрешность в 5 минут).
    \item Разрешается пользоваться материалами, но нельзя использовать других людей.
    \item Все ответы на вопросы должны быть обоснованы. В случае отсутствия обоснования вопрос не
        засчитавается. Без доказательства можно использовать теоремы, доказанные на лекциях. Любые
        неточности трактуются по желанию проверяющего.
    \item Решения, которые трудно читать, проверяться не будут и будет засчитан провал экзамена.
    \item \textbf{Пожалуйста, не разглашайте варианты заданий.}
\end{itemize}

\vspace{1cm}