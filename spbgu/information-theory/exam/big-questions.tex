\documentclass[a4paper, 12pt]{article}
% math symbols
\usepackage{amssymb}
\usepackage{amsmath}
\usepackage{mathrsfs}
\usepackage{mathseries}


\usepackage[margin = 2cm]{geometry}

\tolerance = 1000
\emergencystretch = 0.74cm



\pagestyle{empty}
\parindent = 0mm

\renewcommand{\coursetitle}{DM/ML}
\setcounter{curtask}{1}

\newcommand{\norm}[1]{\left\lVert#1\right\rVert}
\newlang{\UPATH}{UPATH}

\begin{document}

\setmathstyle{2020}{Теория информации}{Вопросы к экзамену}

\begin{enumerate}
    \item Информация по Хартли. Условная информация. Связь объема и площади сечений: $2 \chi(A) \le
        \chi(A_{12}) + \chi(A_{13}) + \chi(A_{23})$. Поиск загаданного числа: симметричный вариант и
        вариант с разными ценами.
    \item Информация по Шеннону. Аксиоматическое задание. Базовые свойства. Условная информация.
        Оценка на биномиальные коэффициенты.
    \item Задача о поиске фальшивой монетки в куче из $14$ монеток. <<Треугольники>> и <<углы>> в
        графах.
    \item Однозначно декодируемые и беспрефиксные коды. Неравенство Крафта--Макмиллана.
        Построение беспрефиксного кода по однозначно декодируемому. Теорема Шеннона.
    \item Существование <<почти оптимального>> кода. Код Шеннона--Фано, код Хаффмана, арифметическое
        кодирование (конструкция, оценка на среднюю длину). Оптимальность кода Хаффмана.
    \item Кодирование с ошибками. Теорема Шеннона.
    \item Задача о разделении секрета. Теорема Шеннона (о совершенной схеме разделения секрета). Схема Шамира.
    \item Задача о разделении секрета. Теорема Шеннона (о совершенной схеме разделения секрета). Пример
        структуры доступа, для которой не существует идеальной схемы разделения секрета.
    \item Коммуникационные протоколы. Метод ранга. Доказательство нижних оценок для функций $\EQ$ и
        $\GT$. Балансировка протоколов.
    \item Коммуникационные протоколы. Прямоугольники и <<fooling set>>. Теорема Карчмера--Вигдерсона.
    \item Метод подсчета. Существование сложной функции. Существование сложной монотонной функции.
    \item Вероятностные коммуникационные протоколы. Эффективные протоколы для функций $\EQ$ и
        $\GT$.
    \item Подсчет функции $\Ind$ при помощи односторонних протоколов.
    \item Внутреннее и внешнее информационное разглашение. Связь с коммуникационными протоколами.
        Теорема Храпченко.
    \item Колмогоровская сложность. Существования оптимального способа описания. Базовые
        свойства. Невычислимость нижних оценок на колмогоровскую сложность. Теорема Колмогорова--Левина.
    \item Колмогоровская сложность. Невычислимость нижних оценок на колмогоровскую
        сложность. Существование языка, не распознающегося автоматом с $k$ головками.
    \item Префиксная сложность. Существования оптимального беспрефиксного способа описания.
        Случайные по Мартин-Лёфу. Почти любая последовательность случайна. Закон больших чисел.
\end{enumerate}



\end{document}

