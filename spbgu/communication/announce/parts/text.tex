Коммуникационная сложность изучает коммуникацию, как вычислительный ресурс. Базовая модель
коммуникационной сложности рассматривает двух участников: Алису и Боба. Их задача посчитать значение
известной функции на некотором входе, распределенном между ними. Для этого участники общаются по заранее
обговоренному протоколу, а мы измеряем число переданных бит между участниками во время подсчета
функции. Эта задача, как и ее вариации часто встречается как на практике, так и на разных уровнях
математической абстракции вычислений --- в сетевых протоколах, алгоритмах для NP-трудных задач, нижних
оценках на булевы схемы.

В этом курсе мы обсудим классические результаты коммуникационной сложности, а также затронем часть
последних исследований и открытых вопросов. Мы исследуем различные коммуникационные модели ---
детерминированную, недетерминированную, вероятностную и протоколы с несколькими участниками. Основной
нашей задачей будет изучение техник для доказательства нижних оценок, но мы также рассмотрим и некоторые
применения.