\section{Верхняя оценка на \texorpdfstring{$\Disj^{\le k}_n$}{Disj(n, <=k)}}

\begin{theorem}[H{\aa}stad--Wigderson, \cite{HW07-disj}]
    $\Rpub(\Disj^{\le k}_n) = \bigO{k}$.
\end{theorem}

\paragraph{Простая оценка.}
Сначала покажем, что $\Rpub(\Disj^{\le k}_n) = \bigO{2^{2k}}$. Заметим, что нет зависимости от
$n$. Случайный сепаратор будет разделять дизъюнктные $(x, y)$ с вероятностью не меньше
$\frac{1}{2^{2k}}$. Выберем $c2^{2k}$ случайных сепараторов, где $c$~--- некоторая константа. Хотя бы
один из них будет разделять $(x, y)$ с вероятностью не меньше $1 - (1 - \frac{1}{2^{2k}})^{c 2^{2k}} \ge
1 - \frac{1}{e^c}$. Для каждого сепаратора Алиса и Боб проверяют, правда ли, что он их разделяет.

\paragraph{<<Настоящая>> оценка.} Пусть $(S_1, \dots, S_{\ell})$~--- случайные множества. Начнем с
некоторого описания интуиции:
\begin{itemize}
    \item пусть $x \subseteq S_j$, и пусть $y \cap S_j$ не пусто (иначе уже всё хорошо);
    \item если $S_j$ маленькое, тогда пересечение с $y$ тоже маленькое;
    \item Боб выкидывает $y \setminus S_j$ и они продолжают.
\end{itemize}

Протокол будет повторять следующие итерации.
\begin{enumerate}
    \item Если $|x| + |y| \le c$ для достаточно большой константы $c$, запустим <<простой протокол>> за
        $\bigO{2^{2c}} = \bigO{1}$.
    \item Мы знаем, что $|x| + |y| \le 2k$. Сравним $|x|$ и $|y|$ за $\bigO{\log k}$ битов коммуникации,
        пусть $|x| \le |y|$ (иначе меняются ролями). Пусть $(S_1, \dots, S_{\ell})$~--- случайные
        множества (параметр $\ell$ зависит от размеров $x$ и $y$, его мы выберем позднее).
        Алиса находит такой первый $j \in [\ell]$, что $x \subseteq S_j$, и посылает число $j$ Бобу~---
        $\bigO{\log \ell}$ бит.
    \item Боб проверяет, пересекается ли $S_j$ с $y$:
        \begin{enumerate}
            \item если не пересекается, то $x$ и $y$ не пересекаются, победа;
            \item если $|S_j \cap y| \ge \frac{3}{4}|y|$, то отвечаем <<пересекаются>>;
            \item иначе Боб выкидывает все элементы $y \setminus S_j$, то есть его множество уменьшилось
                хотя бы на четверть.
        \end{enumerate}
\end{enumerate}


\paragraph{Количество итераций.} Величина $s = (|x| + |y|)$ каждый раз умножается на $\le \frac{7}{8}$,
так как вычитаем четверть от $|y|$, и $|y| \ge |x|$, следжовательно за $\bigO{\log 2k}$ итераций уменьшим
сумму размеров до $\le c$.

\paragraph{Оценка ошибки.} Ошибка может появиться в нескольких местах.
\begin{enumerate}
    \item Среди случайных множеств может не оказаться такого $S_j$, что $S_j \supseteq x$. Поскольку
        суммарная ошибка должна быть ограничена, но на каждой итрации алгоритма ошибка должна не
        превосходить $\bigO{\varepsilon \frac{1}{\log k}}$.

        $S$ содержит $x$ с вероятностью $\frac{1}{2^{|x|}}$. Пусть $\ell_i$~--- количество случайных
        множеств на шаге $i$. Положим $\ell_i = 2^{10|x|}$ (оба игрока знают $|x|$). Тогда
        $$
            \Pr[\forall j \in [\ell_i],  x \nsubseteq S_j] = (1 - \frac{1}{2^{|x|}})^{\ell} \le
            \exp\left(-2^{9|x|}\right)
        $$.
        %\mycomment{\textcolor{red}{Здесь мы должны потребовать, чтобы $|x|\ge \log\log\log k$. Однако,
        %непонятно, как быть в пункте 0, если $|x|$ очень маленькое, а $|y|$ большое.}}
    \item $|S_j \cap y| \ge \frac{3}{4}|y|$, но $x$ и $y$ не пересекаются. Вероятность конкретного
        пересечения $S_j \cap y$ не больше $2^{-(3/4)|y|}$. По union bound получаем вероятность ошибки на
        этом шаге не более $\binom{|y|}{(3 / 4)|y|}\cdot 2^{-(3 / 4)|y|} \le 2^{-\frac{|y|}{20}}$.
        %\mycomment{\textcolor{red}{На лекции последнее неравенство было получено без пояснений. Из
        %энтропийной оценки оно вроде бы не получается: $h(3/4) > 0.8 > 3/4$. Кажется, если бы вместо
        %$3/4$ выбрали $4/5$, получилось бы.}}

        Заметим, что $-\frac{|y|}{20} \le -\frac{|x| + |y|}{40}$. Тогда суммарная ошибка этого типа по
        всем шагам $i$ не превышает:
        $$
            \sum_{i = 1}^{\#\text{it}} 2^{-\frac{|x| + |y|}{40}} = \sum_{i = 1}^{\#\text{it}}
            2^{-\frac{2k}{40} \left(\frac{7}{8}\right)^i} = \sum_{i = \#\text{it}}^{1}
            2^{-\frac{2k}{40}\left(\frac{7}{8}\right)^i} \le \sum_{i' = 1}^{\#\text{it}}
            2^{-c\left(\frac{8}{7}\right)^{i'}}, 
        $$
        в последнем переходе использовали, что перед последней итерацией сумма размеров хотя бы $c$.
        %\mycomment{\textcolor{red}{В первом переходе мы знаем $|x|+|y|\leqslant 2k(\frac{7}{8})^i$, но
        %это неравенство не в ту сторону. Кажется, просто не надо переходить к $2k$.}}

        Это прогрессия убывает быстрее, чем геометрическая, поэтому её сумма оценивается первым членом (с
        точностью до константы), поэтому вся сумма меньше константы, которую можно сделать меньшей
        единицы, выбрав $c$ достаточно большим.
\end{enumerate}



\paragraph{Сложность протокола.} Общее количество переданных бит на всех шагах:
$\bigO{\sum\limits_i (\log k +  \log \ell_i)} + \bigO{2^{2c}}$. Итого $\bigO{\log^2 k}$, плюс
$$
    \sum_i \log l_i \le \sum_i 10|x_i| \le \sum_i 10\frac{|x_i|+|y_i|}{2} \le \sum_i
    10\frac{2k}{2}\left(\frac{7}{8}\right)^i \le \bigO{10\frac{2k}{2}} = \bigO{k}.
$$
Последнее неравенство выполнено, так как это геометрическая прогрессия, она ограничивается первым членом
с точностью до константы.

На этом оценка протокола закончена.


А что в случае большого $k$?
\begin{theorem}
    $\Rpub(\Disj_n) = \Omega(n)$. Более того, $\Rpub(\UDisj_n) = \Omega(n)$.
\end{theorem}


$\UDisj_n(x, y)$~--- промис версия задачи $\Disj(x, y)$, где предполагается, что $|x \cap y| \le 1$.
Докажем эту теорему позже. Давайте сначала поймём, почему вообще $\Disj$ так важна.

\section{Critical Block Sensitivity}

Будем следовать статье \cite{GP18-crit}.

\begin{theorem}
    Существует такое семейство КНФ-формул $\{\varphi_n(x_1, \ldots, x_n)\}$ от $n$ переменных, что
    $\Rpub(\Search_{\varphi_n}) \ge \Rpub(\UDisj_{\Omega{\sqrt{n}}})$ .
\end{theorem}

В статье даётся оценка $\ge \Rpub\left(\UDisj_{\frac{n}{\log n}}\right)$, идея та же, но доказывать
свойства формулы сложнее.

Применения:
\begin{enumerate}
    \item нижняя оценка на время работы алгоритмов расщепления для SAT ($\DPLL$), см. теорему~\ref{DPLL
        and Search}.
    \item Оценка на размер монотонных формул. Используем $\KWm$, которое по теореме~\ref{KW is a complete
        relation} является полным отношением в смысле сведений: $\mathrm{L}(f) = \DCC(\KWm[f]) \ge
        \DCC(\Search_{\varphi}) \ge \RCC(\Search_{\varphi})$. 
\end{enumerate}

Для обоих последствий было бы достаточно нижней оценки на $\DCC(\Search_{\varphi})$. Зачем нужна оценка
на $\RCC(\Search_{\varphi})$? Во-первых, есть другие последствия, которые сложнее описать.
%\mycomment{Кажется, одно мы уже знаем по теореме~\ref{DPLL-oplus and Search}: нижняя оценка на
%    $\DPLL(\oplus)$.} Во-вторых, сведение будет вероятностным, мы не умеем сводить детерминировано! 

%Само доказательство можно прочитать по-русски \href{https://www.dropbox.com/s/qfdeeaodzqofzpq/pc.pdf?dl=0#}{здесь}, разделы 8.1--8.3.

\begin{comment}
    $f\colon \{0, 1\}^n \to A$. Для $z \in \{0, 1\}^n$ выделяем непересекающиеся блоки $\{B_i\}$ переменных
    $B_i\subseteq [n]$ со свойством: $f(z) \neq f(z^{B_i})$, где $z^{B_i}$ означает, что мы инвертируем
    все биты $z$ в позициях $B_i$. 

Block sensitivity: $bs_z(f)$ "--- максимальное количество непересекающихся блоков с таким свойством.

Пусть теперь $S$~--- отношение: $S \subseteq \{0, 1\}^n \times C$ 
Критическая точка $S$~--- $x$ такой, что $\exists! c\colon (x, c) \in S$.

Пример. Пусть 3-КНФ формула невыполнима, рассмотрим случайную подстановку. В среднем у неё будет
$\frac{1}{8}$ доля невыполненных клозов. То есть ответов для $\Search_{\varphi}$ на случайном входе будет
много, то есть случайный вход~--- не критическая точка.

Будем говорить, что функция $f \subseteq S$, если для каждого входа $x$ выбран какой-то один из
подходящих ответов. (Видимо, мы считаем, что отношение тотальное.)
(???)

Critical block sensitivity отношения $S$: $cbs(S) = \min_{f \subseteq S}\max\limits_{x\text{ ---
        критический для }S} bs_x(f)$

Мы докажем, что существуют $\varphi$ такие, что
$\RCC(\Search_{\varphi}) \ge \Omega(\RCC(\UDisj_{cbs(\Search_{\varphi}))}) \ge
\Omega(cbs(\Search_{\varphi}))$

Это не первая такая статья, были уже про функции и $bs$. $cbs$ было определено в \cite{HN12-crit}. Мы
рассмотрим доказательство через сведение, более простое.

Сначала построим формулы, для которых $cbs$ большое. Цейтинские формулы.

\begin{definition}
    Граф $k$-маршрутизируемый~--- есть $2k$ вершин таких, что для любого их разбиения на $k$ пар можно
    построить $k$ рёберно-непересекающихся путей, которые будут соединять концы этих пар.
\end{definition}

\begin{theorem}
    Если связный граф $k$-маршрутизируемый, и $Ts_{G, c}$ невыполнимая, то $cbs(\Search_{Ts_{G, c}}) \ge
    \Omega(k)$.
\end{theorem}

\begin{proposition}
    Решётка $2k \times 2k$ является $k$-маршрутизируемой. (То есть $\Omega(\sqrt{n})$ для $n = |V|$.)
\end{proposition}

\begin{proof}
    Возьмём один столбец.
\end{proof}

\end{comment}