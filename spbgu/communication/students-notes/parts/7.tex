\section{Верхняя оценка на \texorpdfstring{$\DISJ^{\leqslant k}_n$}{DISJ(n, <=k)}}

\begin{theorem}[Håstad, Wigderson; 2006]
$\R^{pub}(\DISJ^{\leqslant k}_n) = \O(k)$.
\end{theorem}

\begin{proof}~
\paragraph{Простой алгоритм.}
Сначала покажем, что $\R^{pub}(\DISJ^{\leqslant k}_n) = \O(2^{2k})$. Заметим, что нет зависимости от $n$.

Случайный сепаратор будет разделять дизъюнктные $(x, y)$ с вероятностью не меньше $\frac{1}{2^{2k}}$.

Выберем $c2^{2k}$ случайных сепараторов. Хотя бы один из них будет разделять $(x, y)$ с вероятностью не меньше $1-(1-\frac{1}{2^{2k}})^{c2^{2k}}\geqslant 1-\frac{1}{e^c}$.

Для каждого сепаратора Алиса и Боб проверяют, правда ли, что он их разделяет.

\paragraph{Усложнённый алгоритм.}
Пусть $(S_1, \ldots, S_{l})$ "--- случайные множества.

Интуиция:
пусть $x\subseteq S_j$, и пусть $y\cap S_j$ не пусто (иначе уже всё хорошо).
Пусть $S_j$ маленькое, тогда пересечение с $y$ тоже маленькое.
Боб выкидывает $y\setminus S_j$ и они продолжают.

Итеративный алгоритм (повторяем пункты 0--2, пока не остановимся):

0. Если $|x|+|y|\leqslant c$ для достаточно большой константы $c$, запустим <<простой алгоритм>> за $\O(2^{2c}) = \O(1)$.

1. $|x|+|y|\leqslant 2k$ "--- дано.
Сравним $|x|$ и $|y|$ за $\O(\log k)$ битов коммуникации, пусть $|x|\leqslant |y|$ (иначе меняются ролями).
Пусть $(S_1, \ldots, S_{l})$ "--- случайные множества (как выбирается $l$, определим ниже; оно будет разным на разных шагах).
Алиса находит первое $j$ такое, что $x\subseteq S_j$, и посылает число $j$ Бобу "--- $\O(\log l)$ бит.

2. Боб проверяет, пересекается ли $S_j$ с $y$.
\begin{enumerate}[---]
    \item Если не пересекается, то $x$ и $y$ не пересекаются, победа.
    \item Если $|S_j\cap y| \geqslant \frac{3}{4}|y|$, то отвечаем <<пересекаются>>.
    \item Иначе Боб выкидывает все элементы $y\setminus S_j$, то есть его множество уменьшилось хотя бы на четверть. Далее он посылает новый размер своего множества. Это нужно для поддержания инварианта $|x|\leqslant |y|$. Если он нарушился, они меняются ролями.
\end{enumerate}


\paragraph{Количество итераций.}
Величина $s = (|x|+|y|)$ каждый раз умножается на $\leqslant \frac{7}{8}$, так как вычитаем четверть от $|y|$, и $|y|\geqslant |x|$. Так что за $\O(\log 2k)$ итераций уменьшим сумму размеров до $\leqslant c$.

\paragraph{Оценка ошибки.}
1). Среди случайных множеств может не оказаться $S_j$ такого, что $S_j\supseteq x$.
Мы хотим ошибку порядка $\frac{1}{\log k}$ на каждом шаге, чтобы за $\O(\log k)$ шагов накопленная ошибка была ограничена константой.

$S$ содержит $x$ с вероятностью $\frac{1}{2^{|x|}}$.
Пусть $l_i$ "--- количество случайных множеств на шаге $i$. Положим $l_i = 2^{10|x|}$ (оба игрока знают $|x|$).
Тогда $\Pr[\forall j\in [l_i]\ x\not\subseteq S_j] = (1-\frac{1}{2^{|x|}})^l \leqslant e^{-2^{9|x|}}$. \mycomment{\textcolor{red}{Здесь мы должны потребовать, чтобы $|x|\ge \log\log\log k$. Однако, непонятно, как быть в пункте 0, если $|x|$ очень маленькое, а $|y|$ большое.}}

2). $|S_j\cap y|\geqslant \frac{3}{4}|y|$, но $x$ и $y$ не пересекаются. Вероятность конкретного пересечения $S_j\cap y$ не больше $2^{-(3/4)|y|}$. По union bound получаем вероятность ошибки на этом шаге не более $\binom{|y|}{(3/4)|y|}\cdot 2^{-(3/4)|y|}\leqslant 2^{-\frac{|y|}{20}}$. \mycomment{\textcolor{red}{На лекции последнее неравенство было получено без пояснений. Из энтропийной оценки оно вроде бы не получается: $h(3/4) > 0.8 > 3/4$. Кажется, если бы вместо $3/4$ выбрали $4/5$, получилось бы.}}

Заметим, что $-\frac{|y|}{20} \leqslant -\frac{|x| + |y|}{40}$.
Тогда суммарная ошибка этого типа по всем шагам $i$ не превышает:
$$\sum_{i=1}^{\#\text{it}} 2^{-\frac{|x| + |y|}{40}} =
\sum_{i=1}^{\#\text{it}} 2^{-\frac{2k}{40}\left(\frac{7}{8}\right)^i} =
\sum_{i=\#\text{it}}^{1} 2^{-\frac{2k}{40}\left(\frac{7}{8}\right)^i} \leqslant
\sum_{i'=1}^{\#\text{it}} 2^{-c\left(\frac{8}{7}\right)^{i'}},
$$
в последнем переходе использовали, что перед последней итерацией сумма размеров хотя бы $c$.
\mycomment{\textcolor{red}{В первом переходе мы знаем $|x|+|y|\leqslant 2k(\frac{7}{8})^i$, но это неравенство не в ту сторону. Кажется, просто не надо переходить к $2k$.}}

Это прогрессия убывает быстрее, чем геометрическая, поэтому её сумма оценивается первым членом (с точностью до константы), поэтому вся сумма меньше константы, которую можно сделать меньшей единицы, выбрав $c$ достаточно большим.

\paragraph{Сложность протокола.}
Общее количество переданных бит за все шаги: $\O\left(\sum_i (\log k + \log l_i) + \O(2^{2c})\right)$, первое слагаемое "--- передача размеров множеств, последнее слагаемое "--- коммуникация в пункте 0. Итого $\O(\log^2 k)$, плюс
$$\sum_i \log l_i \leqslant
\sum_i 10|x_i| \leqslant
\sum_i 10\frac{|x_i|+|y_i|}{2} \leqslant
\sum_i 10\frac{2k}{2}\left(\frac{7}{8}\right)^i
\leqslant \O(10\frac{2k}{2}) =
\O(k).$$
Последнее неравенство выполнено, так как это геометрическая прогрессия, она ограничивается первым членом с точностью до константы.
\end{proof}

А что в случае большого $k$?
\begin{theorem}
$\R(\DISJ_n)\geqslant \Omega(n)$.
Более того, $\R(\UDISJ_n)\geqslant \Omega(n)$.
\end{theorem}

$\UDISJ_n(x, y) = \DISJ(x, y)$, но определён только на $(x, y)\colon |x\cap y| \leqslant 1$.
Это действительно promise-задача (а не как $\DISJ^{\leqslant k}_n)$, то есть множество допустимых $y$ зависит от $x$.

Докажем эту теорему позже. Давайте сначала поймём, почему вообще $\DISJ$ так важна.

\section{Критическая блочная чувствительность}
(Следуем статье [Göös, Pitassi; 2014 "--- Critical Block Sensitivity].)

\begin{theorem}
Существует семейство КНФ-формул $\{\phi_n(x_1, \ldots, x_n)\}$ такое, что $\R(\Search_{\phi_n})\geqslant \R(\UDISJ_{\sqrt{n}})$ .
\end{theorem}

В статье даётся оценка $\geqslant \R\left(\UDISJ_\frac{n}{\log n}\right)$, идея та же, но доказывать свойства формулы сложнее.

Что это даёт?
\begin{enumerate}
    \item Нижняя оценка на время работы алгоритмов расщепления для SAT ($\DPLL$), см. теорему~\ref{DPLL and Search}.
    \item Оценка на размер монотонных формул. Используем $\KW^m$, которое по теореме~\ref{KW is a complete relation} является полным отношением в смысле сведений: $\mathrm{L}(f) = \D(\KW^m_f) \geqslant \D(\Search_\phi) \geqslant \R(\Search_\phi)$.
\end{enumerate}

Для обоих последствий было бы достаточно нижней оценки на $\D(\Search_\phi)$. Зачем нужна оценка на $\R(\Search_\phi)$? Во-первых, есть другие последствия, которые сложнее описать. \mycomment{Кажется, одно мы уже знаем по теореме~\ref{DPLL-oplus and Search}: нижняя оценка на $\DPLL(\oplus)$.} Во-вторых, сведение будет вероятностным, мы не умеем сводить детерминировано!

Само доказательство можно прочитать по-русски \href{https://www.dropbox.com/s/qfdeeaodzqofzpq/pc.pdf?dl=0#}{здесь}, разделы 8.1--8.3.

\begin{comment}
$f\colon \{0, 1\}^n\to A$
Для $z\in \{0, 1\}^n$ выделяем непересекающиеся блоки $\{B_i\}$ переменных $B_i\subseteq [n]$ со свойством: $f(z) \neq f(z^{B_i})$, где $z^{B_i}$ означает, что мы инвертируем все биты $z$ в позициях $B_i$.

Block sensitivity: $bs_z(f)$ "--- максимальное количество непересекающихся блоков с таким свойством.

Пусть теперь $S$ "--- отношение: $S\subseteq \{0, 1\}^n\times C$
Критическая точка $S$ "--- $x$ такой, что $\exists! c\colon (x,c)\in S$.

Пример. Пусть 3-КНФ формула невыполнима, рассмотрим случайную подстановку. В среднем у неё будет $\frac{1}{8}$ доля невыполненных клозов. То есть ответов для $\Search_\phi$ на случайном входе будет много, то есть случайный вход "--- не критическая точка.

Будем говорить, что функция $f\subseteq S$, если для каждого входа $x$ выбран какой-то один из подходящих ответов.
(Видимо, мы считаем, что отношение тотальное.)
(???)

Critical block sensitivity отношения $S$: $cbs(S) = \min_{f\subseteq S}\max_{x\text{ --- критический для }S} bs_x(f)$

Мы докажем, что существуют $\phi$ такие, что
$\R(\Search_\phi)\geqslant \Omega(\R(\UDISJ_{cbs(\Search_\phi))})\geqslant \Omega(cbs(\Search_\phi))$

Это не первая такая статья, были уже про функции и $bs$.
$cbs$ было определено в в [Huynh, Nordström; 2012], но они доказали напрямую без $\UDISJ$, доказательство через информационную сложность.
Мы рассмотрим доказательство через сведение, более простое.

Сначала построим формулы, для которых $cbs$ большое.
Цейтинские формулы.

\begin{definition}
Граф $k$-маршрутизируемый "--- есть $2k$ вершин таких, что для любого их разбиения на $k$ пар можно построить $k$ рёберно-непересекающихся путей, которые будут соединять концы этих пар.
\end{definition}
\begin{theorem}
Если связный граф $k$-маршрутизируемый, и $Ts_{G, c}$ невыполнимая, то $cbs(\Search_{Ts_{G, c}})\geqslant \Omega(k)$.
\end{theorem}
\begin{proposition}
Решётка $2k\times 2k$ является $k$-маршрутизируемой. (То есть $\Omega(\sqrt{n})$ для $n=|V|$.)
\end{proposition}
\begin{proof}
Возьмём один столбец.
\end{proof}
\end{comment}