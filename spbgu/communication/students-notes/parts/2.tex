\section{Детерминированная коммуникация}

\subsection{Определения}

Для доказательства нижних оценок нам нужны формальные определения. $f\colon X\times Y\to Z$.

Детерминированная коммуникация: бинарное ориентированное от корня дерево (исходящая степень всех
внутренних вершин равна 2), каждая внутренняя вершина помечена A или B и булевой функцией на $X$ или на
$Y$, в листьях написаны элементы $Z$.  
Коммуникационная сложность~--- высота дерева в рёбрах.

Коммуникационная матрица $M_f$~--- матрица $X \times Y$, $M_{x, y} = f(x, y)$.

Изначально текущее множество~--- это вся матрица $M_f$. Затем на каждом шаге мы либо делим
<<горизонтальной линией>> (с точностью до перестановки элементов $X$), если ходит Алиса, или вертикальной
линией, если Боб.

Таким образом, каждой вершине дерева соответствует комбинаторный прямоугольник матрицы $M_f$. А для
каждого листа его прямоугольник должен быть одноцветным, во всех клетках написано одно и то же значение
$z$.

Таким образом, мы получаем:

\begin{theorem}
    \label{th:first}
    Пусть $T$~--- коммуникационное дерево для $f$ (или $R$) с $\ell$ листьями. Тогда $M_f$ ($M_R$)
    разбивается на $l$ одноцветных комбинаторных прямоугольников.
\end{theorem}

Применим для $R = \Search_{\varphi}$. Вся $M_{\Search_{\varphi}}$ покрыта прямоугольниками, каждый из
которых~--- множество $(x, y)$, на которых клоз опровергается. Получили $m$ прямоугольников. Но это не
разбиение, а покрытие. Между ними большое различие!


То есть протокол задаёт разбиение на прямоугольники. В обратную сторону вообще говоря неверно.
Две проблемы:
\begin{enumerate}
    \item \label{itm:problem partition-to-protocol} не всякое разбиение соответствует протоколу,
        см. \href{https://youtu.be/OoUD-zF_5ZE?t=1755}{рисунок} "--- мы не можем сделать первый же
        разрез;
    \item \label{itm:problem leaves-to-cc} прямоугольники разбиения соответствуют листьям, а что с
        высотой дерева?
\end{enumerate}

Решим проблему~\ref{itm:problem leaves-to-cc}:
\begin{theorem}[Балансирование]
    Если существует протокол с $l$ листьями, то $\DCC(f) \le 2 \log_{3 / 2} \ell$.
\end{theorem}

\begin{proof}
    Берём центроид, проверяем, дойдут ли Алиса и Боб до этой вершины (тратим два бита), спускаемся в один
    из двух кусков.

    В общем, полностью аналогично сведению коммуникации к $\DPLL$.
\end{proof}

Таким образом, если есть нижняя оценка на коммуникационную сложность, то есть нижняя оценка на размер
дерева.

\begin{remark}
    $\DCC(f) \ge \log_2 \ell$, если в любом протоколе для $f$ хотя бы $\ell$ листьев.
\end{remark}


\subsection{Небалансирующиеся протоколы}
Другая коммуникация: A и B посылают судье по вещественному числу $a$ и $b$, проверяем, правда ли, что $a
> b$. Идём либо налево, либо направо.

$\GT$ становится тривиальной.

Эта модель слабее вероятностной коммуникации (в вероятностной можно симулировать $\GT$).
Однако, у нас нет нижних оценок на размер (не высоту) протокола!

Не балансируется: есть примеры, когда протокол~--- дерево маленького размера, но высота нужна большая.

\subsection{Нижняя оценка на \texorpdfstring{$\DCC(\EQ)$}{D(EQ)} и \texorpdfstring{$\DCC(\GT)$}{D(GT)}}

\begin{theorem}
    $\DCC(\EQ) = n + 1$.
\end{theorem}

\begin{proof}
    Свойство комбинаторного прямоугольника: $(x, y), (x', y') \in R \implies\allowbreak (x', y), (x,
    y')\in R$.

    $M_{\EQ}$~--- единичная матрица. Две единицы на диагонали не могут быть в одном
    прямоугольнике. Значит, хотя бы $2^n + 1$ прямоугольников. Значит, $\DCC \ge n + 1$.
\end{proof}


\begin{theorem}
    $\DCC(\GT) = n + 1$.
\end{theorem}

\begin{proof}
    Аналогично доказательству для $\EQ$.
\end{proof}


\subsection{Связь протоколов и разбиений/покрытий}

\begin{definition}
    $\chi_0(f)$~--- количество нулевых прямоугольников в минимальном разбиении $M_f$.

    $\chi_1(f)$~--- количество единичных прямоугольников в минимальном разбиении $M_f$.

    $\chi(f) \coloneqq  \chi_0(f) + \chi_1(f)$.
\end{definition}

\begin{definition}
    $C^0(f)$~--- количество нулевых прямоугольников в минимальном покрытии $M_f$.

    $C^1(f)$~--- количество единичных прямоугольников в минимальном покрытии $M_f$.
\end{definition}

Частичное решение проблемы~\ref{itm:problem partition-to-protocol}:
\begin{theorem}
    \label{D < log C0 * log C1}

    Пусть $f$~--- функция (не отношение).

    Тогда для разбиений выполняется:
    $\log \chi \le \DCC(f) \le \bigO{\log \chi_0 \log \chi_1} = \bigO{\log^2 \chi}$.

    И для покрытий выполняется:
    $\log (C^0 + C^1) \le \DCC(f) \le \bigO{\log C^0 \log C^1}$.

    Таким образом, $\log \chi \le \DCC(f) \le \O(\log C^0 \log C^1)$.
\end{theorem}


\begin{proof}
    Оценки снизу уже доказывали. Докажем оценки сверху.

    Рассмотрим два любых прямоугольника разбиения: $R' = X' \times Y'$ и $R'' = X'' \times Y''$. Либо
    иксы, либо игреки не пересекаются~--- иначе пересекаются $R'$ и $R''$.

    Алиса и Боб пытаются доказать, что ответ равен 1. Изначально у нас есть все нулевые прямоугольники.

    \begin{enumerate}
        \item Алиса ищет такой 1-прямоугольник, который содержит её $x$, и при этом запрещающий хотя бы
            половину нулевых прямоугольников (то есть не пересекающийся с ними по координате $X$). Если
            такой есть, посылает Бобу его номер.
        \item Иначе Боб делает то же самое, пытаясь уменьшить хотя бы вдвое количество потенциальных
            нулевых прямоугольников.
        \item Если нулевые прямоугольники закончились, то ответ единица. Если не закончились, то
            утверждаем, что ответ ноль.
    \end{enumerate}

    От противного, посмотрим на 1-прямоугольник, в котором лежит $(x, y)$. Этот прямоугольник
    пересекается по $X$ более чем с половиной 0-прямоугольников, и по $Y$ более чем с половиной
    0-прямоугольников. Но тогда есть 0-прямоугольник, с которым он пересекается и по $X$, и по $Y$,
    значит, у них есть общая точка. Противоречие.

    Раундов $\ceil{\log \chi_0}$, в каждом пересылаем $\ceil{\log \chi_1}$ бит.

    Отметим, что мы не пользовались тем, что у нас разбиение, а не покрытие. Но пользовались тем, что
    дана функция, а не отношение, то есть ответ однозначен.
\end{proof}