
Хотим усилить утверждение.

\begin{theorem}
    $\DCC(f) \le \min(\log^2\chi_0(f), \log^2\chi_1(f))$ (только для разбиений).
\end{theorem}

\begin{proof}
    Сведём к $\CIS$. Не умаляя общности, докажем, что $\le \log^2 \chi_1$.

    Строим граф $G$. Вершины соответствуют 1-прямоугольникам, то есть $\chi_1$ вершин. $(u, v) \in E \iff
    R_u$ и $R_v$ пересекаются по $X$.

    Алиса: $x \to S_C = \{R_u \mid R_u \text{ единичный и пересекается строкой $x$}\}$. Это клика в
    графе.

    Боб: $y \to S_I = \{R_v \mid R_v \text{ единичный и пересекается столбцом $y$}\}$. Это независимое
    множество в графе: иначе два прямоугольника пересекаются и по $X$, и по $Y$, а значит, имеют общую
    точку.

    $S_C$ пересекается с $S_I$ $\iff$ есть 1-прямоугольник, содержащий $(x,y)$, то есть ответ 1.
\end{proof}

А что с покрытиями?
\begin{proposition}
    Теорема не выполняется для покрытий.
\end{proposition}

\begin{proof}
    Рассмотрим $\EQ$. $\chi_1 = C^1 = 2^n$, а вот $C^0 \le 2n$. Предъявим $2n$ прямоугольников,
    покрывающих все нули:
    $$
    S^{a,b}_i = \{(x, y) \mid x_i = a \land y_i = b\},
    $$
    для $i \in [n]$, $a, b \in \{0, 1\}$, $a \neq b$.
\end{proof}

\subsection{Связь протоколов и формул: теорема Карчмера --- Вигдерсона}

\begin{definition}
    \deftext{Формульной сложностью} $L(f)$ формулы $f$ будем называть минимальное возможное число листьев
    дерева, вычисляющего эту формулу (в вершинах дерева стоят булевы операции $\vee, \wedge$, а на
    некоторых рёбрах стоят $\neg $).
\end{definition}


Рассмотрим функцию $f\colon \{0, 1\}^n \to \{0, 1\}$. Для $f$ можно рассмотреть следующую
коммуникационную задачу $\KW[f]$: Алиса получает число $x \in f^{-1}(1)$, Боб~--- $y \in f^{-1}(0)$. Их
цель~--- найти хотя бы одну позицию $i$, в которой $x_i \ne y_i$. В случае, если таких битов несколько,
то подойдет любой.

Аналогично определим монотоннту версию задачи $\KW$. Пусть $f$ монотонна (то есть если $x \le y$
покоординатно, то $f(x) \le f(y)$). Определим задачу $\KWm[f]$: Алиса получает число $x \in f^{-1}(1)$,
Боб~--- $y \in f^{-1}(0)$. Их цель~--- найти хотя бы одну позицию $i$, в которой $x_i = 1 \land y_i = 0$.

\begin{theorem}[Карчмер--Вигдерсон]
    \label{th:KW-theorem}
    Любую формулу для $f$ можно переделать в коммуникационный протокол для $\KW[f]$ с таким же деревом и
    обратно. В частности, минимальная глубина формулы для $f$ равна коммуникационной сложности $\KW[f]$.

    Аналогичное утверждение верно для монотонных формул и $\KWm$.
\end{theorem}

\begin{proof}
    Функцию, которую считает гейт формулы $u$, будем обозначать $f_u$; аналогично, прямоугольник,
    соответствующий вершине протокола $v$, будем обозначать $R_v$.
    
    Начнем с <<простой>> части и построим по формуле коммуникационный протокол. Пусть Алиса получила на
    вход строку $x \in f^{-1}(1)$, а Боб строку $y \in f^{-1}(0)$. Назовём гейт формулы $u$
    \deftext{хорошим}, если $f(x) \neq f(y)$.

    Будем строить протокол, начиная с корня дерева формулы. По условию задачи, корень~--- это хорошая
    вершина, целью Алисы и Боба на каждом раунде будет являться поиск хорошего предка. И тогда,
    перемещаясь на каждом раунде в такого предка, Алиса и Боб найдут хороший лист, в котором и будет тот
    вход формулы, на котором строки Алисы и Боба различаются. Таким образом, для того, чтобы завершить
    доказательство, нам достаточно показать, как найти хорошего предка, передав не более одного
    бита. Рассмотрим текущую вершину $u$ с предками $a, b$ (НУО считаем, что $f_u(x) = 1 \wedge f_u(y) =
    0)$. У нас возможны два случая.
    \begin{enumerate}
        \item В вершине $u$ написан значок $\wedge$. Тогда заметим, что $f_a(x) = f_b(x) = 1$, и при этом
            либо $f_a(y) = 0$, либо $f_b(y) = 0$. Таким образом, Боб может однозначно определить, какой
            из предков является хорошим, и сообщить это Алисе, передав один бит.
        \item В вершине $u$ написан значок $\vee$. Тогда заметим, что $f_a(y) = f_b(y) = 0$, и при этом
            либо $f_a(x) = 1$, либо $f_b(x) = 1$. Таким образом, Алиса может однозначно определить, какой
            из предков является хорошим, и сообщить это Бобу, передав один бит.
    \end{enumerate}


    Теперь по протоколу построим формулу. По индукции, начиная с листьев протокола, для каждой вершины
    протокола $v$ мы предъявим формулу для такой функции $f_v$, что $f_v(X_v) = 1$ и $f_v(Y_v) = 0$,
    где $R_v \coloneqq X_v \times Y_v$.

    Рассмотрим лист протокола $\ell$ и заметим, что прямоугольник $R_{\ell} \coloneqq X_{\ell} \times
    Y_{\ell}$ одноцветный, поэтому все строки $x \in X_{\ell}$ отличаются от всех строк $y \in Y_{\ell}$
    в какой-то фиксированной позиции $i_{\ell}$, в частности это означает, что выполнен один из двух
    следующих случаев:
    \begin{itemize}
        \item для всех $x \in X_{\ell}, y \in Y_{\ell}$ бит $x_{i_{\ell}} = 1$ и $y_{i_{\ell}} = 0$,
            тогда мы можем определить $f_{\ell} \coloneqq x_{i_{\ell}}$;
        \item для всех $x \in X_{\ell}, y \in Y_{\ell}$ бит $x_{i_{\ell}} = 0$ и $y_{i_{\ell}} = 1$,
            тогда мы можем определить $f_{\ell} \coloneqq \neg x_{i_{\ell}}$.
    \end{itemize}
    Других случаев не существует, поскольку если найдутся такие $x, x' \in X$, что $x_{i_{\ell}} \neq
    x'_{i_{\ell}}$, то они одновременно не могут отличаться в позиции $i_l$ ни от какого $y$ (случай
    $y_{i_{\ell}} \neq y'_{i_{\ell}}$ аналогичен).

    Построим теперь формулу для функции $f_v$, если у нас уже есть формулы для функций $f_a$ и $f_b$, где
    $a, b$~--- потомки вершины $v$. Заметим, что прямоугольники $R_a$ и $R_b$ получены рассечением
    прямоугольника $R_v$ на две части либо вертикальным, либо горизонтальным сечением. Рассмотрим эти
    случай отдельно и заметим, что:
    \begin{itemize}
        \item если сечение было горизонтальным, то $Y_a = Y_b = Y_v$ и $X_v = X_a \cup X_b$ и в таком
            случае нам подойдет формула $f_v \coloneqq f_a \vee f_b$;
        \item если сечение было вертикальным, то $X_a = X_b = X_v$ и $Y_v = Y_a \cup Y_b$ и в таком
            случае нам подойдет формула $f_v \coloneqq f_a \wedge f_b$.
    \end{itemize}
\end{proof}


Других перспективных способов доказывать нижние оценки на размеры формул сейчас особо нет.

\begin{remark}
    Известные нижние оценки на размеры формул в базисе де Моргана: Parity: $n^2$, Andreev's function:
    $n^3$.
\end{remark}



\begin{theorem}
    \label{KW is a complete relation}
    $\KWm[f]$~--- <<полное>> отношение: для любого отношения $S \subseteq X\times Y\times Z$ найдётся
    монотонная функция $f_S$, что вычисление $S$ сводится к вычислению $\KWm[f_S]$, в частности,
    $\DCC(\KWm[f_S]) \ge \DCC(S)$.
\end{theorem}

\begin{proof}
    Рассмотрим покрытие матрицы отношения $M_S$ одноцветными прямоугольниками, пусть это $C_1, C_2,
    \ldots, C_m$. $f_S\colon \{0, 1\}^m \to \{0, 1\}$:

    Для каждой строки $x\in X$ выпишем те $C_i$, которые она пересекает. Получим битовую строку $x_S \in
    \{0, 1\}^n$. Положим $f_S(x_S) = 1$.

    Для каждого столбца $y\in Y$ выпишем те $C_i$, которые он \emph{не} пересекает. Получим битовую
    строку $y_S \in \{0, 1\}^n$. Положим $f_S(y_S) = 0$.

    Проверим сначала корректность определения и монотонность $f_S$. От противного, пусть для каких-то
    $x\in X$ и $y\in Y$ выполнено $y_S \geqslant x_S$. Рассмотрим позицию $(x, y)$, она принадлежит хотя
    бы одному прямоугольнику, пусть $(x, y)\in C_i$. Тогда $(x_S)_i = 1$, а $(y_S)_i = 0$. Противоречие.

    Вообще говоря, можем доопределить до полной монотонной функции, но это не требуется определением
    $\KWm$.

    Теперь проверим условие теоремы. Пусть мы хотим посчитать $S(x, y)$. С помощью протокола для
    $\KWm[f_S]$ находим $k$ такое, что $(x_S)_k = 1$, $(y_S)_k = 0$. Тогда $x\in C_k$, $y\in
    C_k$. Тогда ответ для $S(x, y)$~--- это цвет прямоугольника $C_k$.

\end{proof}

Почему это полезно? У $f$ получилось $m = \chi(S)$ входов. Если у $S$ небольшое покрытие, то битов
мало. Поэтому, чтобы доказать нижнюю оценку на монотонные формулы, нужно найти отношение с маленьким
покрытием, но с нижней оценкой на коммуникацию: $L(f_S) = \DCC(\KWm[f_S]) \ge \DCC(S)$. 

Такие отношения есть, например, $\Search_{\varphi}$. Есть покрытие размера, равного количеству клозов в
$\phi$. С другой стороны, для $\DCC(\Search_{\varphi})$ есть нижние оценки.