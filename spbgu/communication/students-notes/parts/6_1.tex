%2020-03-25
\section{Улучшенная верхняя оценка для \texorpdfstring{$\Rpub(\GT)$}{R[pub](GT)}}

По теореме~\ref{R-pub(GT) simple} $\Rpub(\GT) \le \bigO{\log n \cdot \log\log n}$. Докажем лучшую оценку.

\begin{theorem}
    $\Rpub(\GT) \le \bigO{\log n}$.
\end{theorem}

\begin{proof}
    Потребуем ошибку в протоколе для $\EQ$ не более $\frac{1}{100}$. Идём по дереву бинпоиска, оказываясь
    в вершине, перепроверяем, всё ли ок. Если не ок, поднимаемся наверх по ребру.

    Следим за величиной <<расстояние до правильного листа>>. Покажем, что через $100\log n$ итераций с
    большой вероятностью будет нулём: по неравенству Маркова вероятность ошибки $\le \frac{1}{100}$.

    Можно воспользоваться оценками Чернова, тогда вероятность ошибки $\le 2^{-\Omega(\log n)} =
    \bigO{\frac{1}{n}}$. Это можно использовать для правильного понижения
    ошибки.
\end{proof}

\section{Нижняя оценка на \texorpdfstring{$\DPLL_\oplus$}{DPLL(⊕)} через \texorpdfstring{$\R^{pub}(\Search_\phi)$}{R[pub](Search(φ))}}

\begin{theorem}
    \label{DPLL-oplus and Search}
    Пусть $T = \DPLL(\oplus)(\varphi)$. Тогда $\Rpub(\Search_{\varphi}) \le \log T\log\log T$.
\end{theorem}

\begin{proof}
    Как в доказательстве теоремы~\ref{DPLL and Search} для обычного $\DPLL$, только теперь надо проверить
    выполненность системы линейных уравнений. Чтобы проверить, выполняется ли система, нужно проверить
    $k$ равенств. Делаем это с помощью протокола для $\EQ$ с вероятностью ошибки не более $1/\log T$.
\end{proof}

\begin{remark}
    Можно избавиться от $\log\log T$ аналогично тому, как мы сделали это в $\GT$.
    Тогда получим $\Rpub(\Search_{\varphi}) \le \log T$, то есть $T \ge 2^{\Rpub(\Search_{\varphi})}$.
\end{remark}

\begin{proof}
    Оставлено как упражнение.
\end{proof}