\section{Corruption bound}

Назовём $\mu$ сбалансированной мерой, если $\mu(f^{-1}(0)) = \mu(f^{-1}(1)) = \frac{1}{2}$ (на самом деле для наших целей хватит, чтобы мера прообразов нуля и единицы была просто не меньше какой-то константы).

Скажем, что прямоугольник $R$ $1$-biased, если $\mu(R \cap f^{-1}(0)) \leq \frac{\mu(R \cap f^{-1}(1))}{8}$ (здесь, опять же, вместо $8$ может быть любая другая константа).

$$corr(f) = \max_{\mu \text{~-- сбалансированная}} \min_{R \text{~-- 1-biased}} \log{\frac{1}{\mu(R)}}$$

\begin{lemma}
$R_{\varepsilon}(f) \geq \Omega(corr(f))$ для достаточно маленького $\varepsilon$.
\end{lemma}

\begin{proof}
$2^{-corr(f)} = 2^{-corr_{\mu}(f)} \geq \mu(R)$, где $\mu$ --- какая-то конкретная мера, $R$ --- любой $1$-biased прямоугольник.

Тогда чтобы покрыть константную долю прообразов единицы $1$-biased прямоугольниками, нам понадобится порядка $2^{corr(f)}$ таких прямоугольников, что даёт оценку на количество листьев протокола.

Почему доля прообразов единицы, которые мы покрываем $1$-biased прямоугольниками, должна быть константной? Если мы покроем слишком много единичных входов прямоугольниками, в которых доля нулей $\geq \frac{\mu(R)}{8}$, то суммарная ошибка будет слишком большой (скажем, пусть мы покроем $\frac{1}{2}$ долю единиц не $1$-biased прямоугольниками, тогда суммарная ошибка будет хотя бы $\frac{1}{2} \cdot \frac{1}{2} \cdot \frac{1}{8} = \frac{1}{32}$).

\end{proof}
