\subsection{Нижняя оценка}

\begin{theorem}
    $\DCC(\Disj^{\le \log n}_n) = \Omega(\log^2 n)$.
\end{theorem}

\begin{proof}
    Пусть $k = \log n$. Обозначим множество множеств размера $\le k$ за $Z_k$.
    $|Z_k| = \sum\limits_{i = 0}^k \binom{n}{i}$. $M_{f}$~--- матрица $|Z_k| \times |Z_k|$. Ниже мы
    покажем, что эта матрица полного ранга, то есть ранга $|Z_k|$. Тогда $\DCC(f) \ge \log(2|Z_k| - 1) =
    \Omega(\log \binom{n}{k}) = \Omega(\log^2 n)$. 
\end{proof}

\begin{remark}
    На самом деле, нам нужно только $\log k = o(\log n)$. При таком условии получаем $\DCC(\Disj^{\le
        k}_n) = \Omega(k \log n)$.
\end{remark}

\begin{theorem}[Разборов]
    Для любого $k$ матрица $M_{\Disj^{\le k}_n}$ имеет полный ранг над $\field_2$.
\end{theorem}

\begin{proof}
    Обозначим матрицу как $A$. Нужно показать, что нет вектора-строки $\lambda$ такого, что $\lambda A =
    0$.

    Над $\field_2$ это значит, что сумма любого подмножества строк матрицы не равна нулю. Введём
    формальные переменные $x_1, \dots, x_n$. Каждая строка матрицы $A$ соответствует множеству
    $S \subseteq [n]$ ($|S| \le k$), сопоставим этой строке моном $\chi_S = \prod\limits_{i \in S}
    x_i$. Каждый столбец матрицы $A$ соответствует множеству $S \subseteq [n]$ ($|S| \leqslant k$),
    сопоставим этому столбцу полную подстановку $\rho_S$: $\rho_S(x_i) = 0$ для всех $i \in S$, для
    остальных $\rho_S(x_i) = 1$.

    Рассмотрим клетку $(a, b)$. Покажем, что значение в этой клетке~--- это $\rho_b(\chi_a)$. Если $a$ и
    $b$ не пересекаются, то все переменные монома $\chi_a$ будут присвоены $1$, поэтому значение после
    подстановки будет единицей. Если пересекаются по $i$, то $\rho_b$ подставит в $x_i$ ноль, и
    переменная $x_i$ есть в $\chi_a$, поэтому $\rho_b(\chi_a)$ будет равно нулю.

    Для вектора $\lambda$ определим $q_{\lambda} = \sum\limits_{S \in Z_k} \lambda_S \cdot \chi_S$.
    \begin{lemma}
        Пусть $\rho$~--- подстановка с $\le k$ нулями. Множество её нулей обозначим как $T$ (то есть
        $\rho = \rho_T$). Тогда $\rho(q_\lambda) = (\lambda A)_T$.
    \end{lemma}

    \begin{proof}
        $\rho_T(q_\lambda) = \sum\limits_{S\in Z_k} \lambda_S \cdot \rho_T(\chi_S) =
        \sum\limits_{S \in Z_k} \lambda_S \cdot A_{S, T} = (\lambda A)_T$.
    \end{proof}

    Осталось показать, что для любого ненулевого $\lambda$ найдётся $\rho$ с $\le k$ нулями, что
    $\rho(q_\lambda)$ будет не нулём. Пусть $S_0$~--- наибольшее (по включению) множество, что
    $\lambda_{S_0} \neq 0$. Такое $S_0$ существует, так как $\lambda \neq 0$. Определим частичную
    подстановку $\rho'$: $\rho'(x_i) = 1$ для всех $i \not\in S_0$. Заметим, что $\rho'(q_\lambda)$~---
    ненулевой многочлен: член $\lambda_{S_0} \cdot \chi_{S_0}$ ни с чем не сократится ($\lambda_{S_0}
    \neq 0$; все $S \supsetneq S_0$ имеют нулевой коэффициент по максимальности $S_0$). Этот ненулевой
    многочлен мультилинеен, поэтому существует подстановка всех его переменных (которых не более $k$, так
    как $|S_0| \le k$), обращающая его в не ноль: $\rho''(\rho'(q_\lambda)) \neq 0$.

    Тогда определим подстановку $\rho = \rho'' \circ \rho'$, у неё $\le k$ нулей, что и
    требовалось.
\end{proof}


\section{Вероятностная коммуникация}

\subsection{Определения}
Чарли посылает случайные биты Алисе и Бобу. Далее они общаются детерминировано. Сложность протокола~---
это максимальное количество переданных между Алисой и Бобом битов. 

Вычисляем функцию:
\begin{enumerate}[a)]
    \item с двусторонней ошибкой: вероятность прийти в лист с правильным ответом не меньше $\frac{2}{3}$;
    \item с односторонней ошибкой: один из ответов всегда верный, другой с вероятностью хотя бы
        $\frac{1}{2}$.
\end{enumerate}

\subsubsection{Публичные случайные биты}
Чарли посылает одинаковые случайные биты Алисе и Бобу.

Два способа формализовать:
\begin{enumerate}
    \item Также, как с приватными случайными битами (см. ниже), но гарантируется, что $r_1 = r_2$.
    \item Задано распределение на детерминированных протоколах. Вначале Чарли выбирает из этого
        распределения протокол, а затем Алиса и Боб коммуницируют детерминировано (без случайных битов).
\end{enumerate}

Эти два определения эквивалентны, но мы этого не доказывали.

\subsubsection{Приватные случайные биты}
Чарли посылает независимые случайные строки $r_1$ и $r_2$, иначе говоря, у каждого игрока свой источник
случайных битов. Протокол~--- дерево как в детерминированном случае, только функция в узле $v$ теперь
либо $f_v(x, r_1)$, либо $f_v(y, r_2)$. 

\section{Верхняя оценка для \texorpdfstring{$\EQ$}{EQ}}
По теореме~\ref{R-pub(EQ)} $\Rpub(\EQ) = \bigO{1}$.

\begin{theorem}
    \label{th:R-pr(EQ)}
    $\Rpri(\EQ) = \bigO{\log n}$.
\end{theorem}

\begin{proof}
    Пусть $z$~--- формальная переменная. Определим многочлены $p_x = \sum\limits_i x_i z^i$, $p_y =
    \sum\limits_i y_i z^i$. Строки равны $\iff$ $p_x = p_y$.

    Пусть не равны. В большом поле $\field$ у нетривиального многочлена $(p_x - p_y)$ степени $\le n$ в
    случайной точке будет не ноль с большой вероятностью: $\ge 1 - \frac{n}{|\field|}$.

    Пусть $|\field| \approx 3n$. Алиса посылает случайную точку в поле $\field$ и значение $p_x$ на этой
    точке. Боб сравнивает со своим значением. Итого $2 \log |\field| + 1 = \bigO{\log n}$.
\end{proof}

\section{Связь между \texorpdfstring{$\DCC$}{D} и \texorpdfstring{$\Rpub$}{R[pub]}}

\begin{theorem}
    \label{th:pri-to-det}%
    $\DCC(f) \le 2^{\Rpri[\varepsilon](f)} (\log\left(\frac{1}{1/2 - \varepsilon}\right) +
    \Rpri[\varepsilon](f))$.
\end{theorem}

\begin{corollary}
    $\Rpri[\varepsilon](\EQ) = \Theta(\log n)$.
\end{corollary}

\begin{proof}
    Оценку снизу получаем из $\DCC(\EQ) = n + 1$ и данной теоремы. Сверху оценили в теореме~\ref{th:R-pr(EQ)}.
\end{proof}

Заметим, что верхней оценки на $\DCC$ через $\Rpub$ быть не может: $\DCC(\EQ) = n + 1$, $\Rpub(\EQ) = \bigO{1}$.

\begin{proof}[Доказательство теоремы \ref{th:pri-to-det}]
    Пусть дан вероятностный протокол с приватными случайными битами, построим детерминированный.

    Пусть даны $(x, y)$. Посчитаем для каждого $1$-листа вероятность, что мы в него придём. Если сумма
    вероятностей по листам $\ge \frac{1}{2}$, то вход принимается. Проблема в том, что Алиса и Боб не
    знают $(x, y)$~--- каждый знает только свой вход.

    Алиса для каждого $1$-листа $i$ самостоятельно считает вероятность того, что она сделает все шаги в
    верную сторону, обозначим её за $a_i$. Боб делает то же, это вероятности $b_i$. Теперь им нужно
    проверить, что $\sum\limits_i a_ib_i > 1 - \varepsilon$, где сумма берётся по всем $1$-листам,
    которых не более $2^{\Rpri[\varepsilon](f)}$.

    Просто пошлём вектор $\{a_i\}$ с некоторой точностью. Будем посылать первые $\ell = \log(\frac{1}{1/2
        - \varepsilon}) + \Rpri[\varepsilon](f)$ битов каждого числа. Тогда ошибка в одном числе будет не
    больше $2^{-\ell} = \frac{1/2 - \varepsilon}{2^{\Rpri[\varepsilon](f)}}$. Тогда суммарная ошибка не
    более $(1 / 2 - \varepsilon)$~--- все $b_i$ посчитаны без погрешностей и их сумма от $0$ до $1$,
    поэтому просто сложили ошибки всех $a_i$.

    Если $\sum a_ib_i >  1 - \varepsilon$, то $\sum a'_ib_i >  1 - \varepsilon - (\frac{1}{2} -
    \varepsilon) = \frac{1}{2}$. Обратное тоже верно.
\end{proof}

\section{Связь между \texorpdfstring{$\Rpub$}{R[pub]} и \texorpdfstring{$\Rpri$}{R[pr]}}
Какой вообще зазор между сложностями с приватными и с публичными случайными битами?
\begin{proposition}
    $\Rpub[\varepsilon](f) \le \Rpri[\varepsilon](f)$.
\end{proposition}

Оказывается, от публичных битов можно избавиться, а зазор всегда не больше $\bigO{\log n}$.
\begin{theorem}[Newman, \cite{Newman91-pri-pub}]
    $\Rpri[2 \varepsilon](f) \le \Rpub[\varepsilon](f) + \bigO{\log n}$.
\end{theorem}

\begin{proof}
    Алиса хотела бы просто послать свои приватные биты, а потом использовать имеющийся протокол. Однако,
    случайных битов может быть много.

    Пусть мы используем не все возможные случайные публичные строки, а только строки $r_1, r_2, \dots,
    r_t$, тогда переслать достаточно номер такой строки, то есть $\log t$ битов.

    Зафиксируем $t, x, y$. Выберем $r_1, r_2, \dots, r_t$ случайно. Строка $r_i$ плохая, если с ним ней
    неправильный.
    $$
        \Pr\limits_{r_1, \dots, r_t}\left[\ge 2\varepsilon t \text{ штук из $\{r_i\}$ плохие}\right] =
        \Pr\left[\sum_i \One_{r_i\text{ плохая}} \ge 2 \varepsilon t\right].
    $$

    Вероятность неверного ответа для каждой строки $r_i$ исходного протокола не более $\varepsilon$, то
    есть $\Exp[\One_{r_i\text{ плохой}}] \le \varepsilon$, и они выбираются независимо. Тогда по
    неравенству Чернова $\Pr\left[\sum\limits_i \One_{r_i\text{ плохой}} \ge 2 \varepsilon t\right] \le
    2^{-1^2\varepsilon t / c}$. Пусть $t = 2cn / \varepsilon$, тогда вероятность меньше $2^{-2n}$.

    Теперь применим union bound по всем возможным входам, получим $< 2^{2n} \times 2^{-2n} = 1$. Значит,
    есть набор $\{r_i\}$, для которого у всех входов ошибка не более $2\varepsilon$.
\end{proof}

% (???) связь недетерминированной и рандомизированной сложности