\section{$\DISJ$ over product distributions}

Мы докажем почти совпадающие нижнюю и верхнюю оценки.

\begin{theorem}
$\exists \mu : D_{\varepsilon}^{\mu}(\DISJ_n) \geq \Omega(\sqrt{n})$
\end{theorem}

\begin{proof}

На высоком уровне идея доказательства следующая. У $\DISJ$ существует маленькое покрытие $0$-прямоугольниками, но все $1$-прямоугольники в покрытии <<должны быть>> довольно маленькими. Так что мы сосредоточимся на анализе <<почти полностью единичных>> прямоугольников.

При этом для, например, равномерного распределения на входах мера единиц очень маленькая, так как два случайных подмножества $[n]$ пересекаются с большой вероятностью. Давайте рассмотрим меру $\mu$, равномерную на подмножествах размера $\sqrt{n}$, т.е., каждое из таких подмножеств выпадает игроку с вероятностью ${{n}\choose{\sqrt{n}}}^{-1}$. 

Для этой меры $\mu(f^{-1}(1)) = Pr[x \cap y = \varnothing] \approx (\frac{n -  \sqrt{n}}{n})^{\sqrt{n}} \approx \frac{1}{e}$

Для доказательства нам понадобится следующая лемма, которую докажем чуть позже:

\begin{lemma}
$R = S \times T$, $R$ --- прямоугольник, $\varepsilon$-близкий к $1$-прямоугольнику (т.е., доля нулей не более $\varepsilon$). Тогда либо $|S| \leq 2^{-c\sqrt{n}} {{n}\choose{\sqrt{n}}}$, либо $|T| \leq 2^{-c\sqrt{n}} {{n}\choose{\sqrt{n}}}$ 
\end{lemma}

Заметим, что из леммы сразу следует доказательство теоремы. Мы не можем покрыть слишком большую долю единиц прямоугольниками, не $\varepsilon$-близкими к полностью единичным, иначе ошибка будет слишком большая --- скажем, если мы покроем долю $\frac{1}{8}$, то получим ошибку протокола не менее $\frac{\varepsilon}{8e}$, что плохо, если хотим ошибку меньше.

Тогда как минимум константную долю меры единиц нужно покрыть $\varepsilon$-несбалансированными прямоугольниками. Пусть $R = S \times T$ --- самый большой по мере $\mu$ прямоугольник, $\varepsilon$-близкий к $1$-прямоугольнику. Н.у.о. $|T| \leq |S|$

$\mu(R) = \sum \mu(s,t) = (\sum \lambda(s))(\sum \rho(t)) \leq 1 \cdot \frac{|T|}{{{n}\choose{\sqrt{n}}}} \leq 2^{-c\sqrt{n}}$

Тогда нам понадобится порядка $2^{c\sqrt{n}}$ таких прямоугольников.

\end{proof}


Теперь докажем лемму.

\begin{proof}

Предположим,  что $|S| > 2^{-c\sqrt{n}} {{n}\choose{\sqrt{n}}}$, иначе всё уже хорошо.

Покажем, что тогда $T$ достаточно маленькое. Выберем $S' \subseteq S : \forall s \in S' : Pr_{y \in T}[\DISJ(x, y) = 1] \geq 1 - 2\varepsilon$. Это означает, что мы выбираем те строки, в которых доля единиц хотя бы $1 - 2\varepsilon$. Так как $S \times T$ близок к $1$-прямоугольнику, $|S'| \geq \frac{|S|}{2}$.

Выберем последовательность строк $S'' = (x_1, \ldots, x_{\sqrt{n}/3})$ из $S'$ таких, что $|\bigcup x_i| \geq \frac{n}{6}$. Добьёмся этого тем, что на каждом шаге будет выбирать такое $x_{i + 1}$, что $|x_{i + 1} \backslash \bigcup_{j \leq i} x_j| \geq \frac{\sqrt{n}}{2}$.

$x_1$ --- любой элемент. Далее выбираем жадно. Почему каждый раз сможем найти подходящий элемент?

Пусть уже выбрали $(x_1, \ldots, x_k)$. Тогда $|\bigcup x_i| \leq k \sqrt{n} \leq n/3$.

Сколько бывает множеств, которые нам не подходят? Как минимум $\sqrt{n}/2$ элементов плохого множества лежат в $|\bigcup x_i|$.

$\sum_{i = \sqrt{n}/2}^{\sqrt{n}}{{n/3}\choose{i}} {{2n/3}\choose{\sqrt{n} - i}} \leq n {{n/3}\choose{\sqrt{n}/2}} {{2n/3}\choose{\sqrt{n}/2}} \approx 2^{-c\sqrt{n}}{{n}\choose{\sqrt{n}}}$.

Как получить последнее равенство? Источники утверждают, что по формуле Стирлинга, но для тех, кто, как и я, не умеет считать, предлагаю вероятностный метод. Давайте выберем случайное подмножество ${{n}\choose{\sqrt{n}}}$. При этом у нас зафиксировано некоторое разделение $[n]$ на два множества размера $n/3$ и $2n/3$, и нам подходят только те сочетания, которые берут поровну элементов из них. В среднем же мы берём из множества размера $n/3$ около трети элементов выборки. Применим оценки Чернова и получим, что вероятность получить подходящее множество порядка $2^{-c\sqrt{n}}$, что и требовалось.

После того, как выбрали последовательность $S'' = (x_1, \ldots, x_{\sqrt{n}/3})$, выберем в $T' \subseteq T$ такие $y$, что они пересекаются не более чем с $4\varepsilon$-долей $x_i \in S''$.

Из определений следует, что $|T'| \geq \frac{|T|}{2}$. Тогда $\forall y \in T' :  \exists (1 - 4\varepsilon) \frac{\sqrt{n}}{3} \text{ множеств } x_{i_j} \in S'' : y \subseteq [n] \backslash \bigcup_j x_{i_j}$.

$|[n] \backslash \bigcup_j x_{i_j}| \leq n - (1 - 4\varepsilon)\frac{n}{6} \leq \frac{8n}{9}$ при достаточно малом $\varepsilon$.

Тогда разных $y$ в $T'$ может быть не более ${{\sqrt{n}/3}\choose{4\varepsilon\sqrt{n}/3}}{{8n/9}\choose{\sqrt{n}}} < 2^{-c\sqrt{n}}{{n}\choose{\sqrt{n}}}$ (тоже можно получить с использованием формулы Стирлинга/вероятностного метода).

Лемма доказана.

\end{proof}

\begin{theorem}
Для любого product distribution $\mu = \lambda \times \rho$ $D^{\mu}(\DISJ) = \mathcal{O}(\sqrt{n} \log{n})$.
\end{theorem}

\begin{proof}
Несмотря на то, что в определении distributional communication complexity фигурируют детерминированные протоколы, мы построим вероятностный. У него будет некоторая маленькая ошибка $\varepsilon$ и матожидание количества пересланных бит $\mathcal{O}(\sqrt{n}\log{n})$. Из этого можно легко получить детерминированный протокол, ещё чуть-чуть понизив ошибку, разрешив пересылать чуть-чуть побольше бит, чем матожидание и, наконец, зафиксировав <<самые лучшие>> случайные биты.

Итак, построим вероятностный протокол. У Алисы множество $x \sim \lambda$, у Боба множество $y \sim \rho$, нужно понять, пересекаются ли они.

\begin{enumerate}
    \item Если $|x| \leq \sqrt{n}$, Алиса пересылает $x$ Бобу, тратя $\sqrt{n}\log{n}$ бит, Боб присылает ответ на задачу.
    \item Иначе $|x| > \sqrt{n}$. Боб проверяет, что $\varepsilon_y = Pr_{x \sim \lambda} [x \cap y = \varnothing | |x| > \sqrt{n}] \leq \frac{\varepsilon}{\sqrt{n}}$. Если да, отвечает, что множества пересекаются. Иначе с помощью публичных случайных бит генерируется последовательность $X_1, \ldots, X_i, \ldots$, где $X_i \sim \lambda|_{> \sqrt{n}}$. Боб посылает Алисе первый такой номер $j$, что $X_j$ не пересекается с $y$.
    \item $x := x \backslash X_j$, $U := U \backslash X_j$, $\lambda := \lambda|_{x \subseteq U}$. После этого игроки возвращаются к первому шагу.
\end{enumerate}

В этом протоколе не более $\sqrt{n}$ фаз, так как на каждой фазе вселенная уменьшается на $\sqrt{n}$ элементов. Ошибка случается только на втором шаге с $\varepsilon_y$, применяя union bound, получаем, что суммарно за все шаги ошибка не больше $\sqrt{n} \cdot \frac{\varepsilon}{\sqrt{n}} = \varepsilon$.

Матожидание длины номера множества, который пересылает Боб на каждой фазе:

$c_i = E_{R, X \sim \lambda, y \sim \rho} [\mathcal{O}(\log{j})] \leq \mathcal{O}(E_y[\log{E_{X, R}[j]}])$.

$E[j] = \frac{1}{Pr[X \cap y = \varnothing | |X| > \sqrt{n}} = \frac{1}{\varepsilon_y} \leq \frac{\sqrt{n}}{\varepsilon}$.

Итого $c_i = \mathcal{O}(\log{n} + \log{\frac{1}{\varepsilon}})$.
\end{proof}