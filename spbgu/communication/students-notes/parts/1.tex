\section{Введение}

\subsection{Детерминированная коммуникация}

$f\colon X\times Y\to Z$, все конечные. Алиса и Боб не ограничены в силе, в конце оба должны знать
ответ. \deftext{Детерминированная коммуникационная сложность} функции $f$~--- $\measure{CC}(f) =
\DCC(f) = \min\limits_{\text{protocol}} \max\limits_{x \in X,\ y \in Y} (\#\text{ переданных битов})$. 

Тривиальные верхние оценки:
\begin{enumerate}
    \item $\leq \min(\log |x| + \log |z|, \log |y| + \log |z|, \log |x| + \log |y|)$
    \item Для предикатов $f\colon\{0, 1\}^n \times \{0, 1\}^n \to \{0, 1\}$: $\leq n + 1$
\end{enumerate}

В приложениях часто нужно считать promise-функции (например, гарантируется, что $|x| + |y| = n$~---
множество таких $(x, y)$ не является декартовым произведением) или отношения: $S \subseteq X \times Y
\times Z$, по $x \in X$ и $y \in Y$ ищем любой подходящий $z \in Z$. Многие теоремы, к сожалению, будут
работать только для функций.

Примеры функций:
\begin{enumerate}
    \item $\Parity(x, y) = (x + y) \bmod 2$, где $x, y \in \{0, 1\}$. $\DCC(\Parity) = 2$.
    % \item Pointer Chasing: $x$ и $y$ "--- функции $\{0, 1\}^n\to \{0, 1\}^n$. Хотят посчитать
    % $x(y(\ldots(x(1^n))\ldots))$ ($k$ раз), $k$ обычно константа. $|x| = |y| = n2^n$. Тривиальный
    % протокол: $\DCC \leq nk$.
    \item $\CIS$~--- Clique (у Алисы) versus Independent Set (у Боба) на одном графе, ищут
        пересечение~--- это максимум одна вершина.
\end{enumerate}

\begin{theorem}[Yannakakis 1991]
    $\DCC(\CIS)\leqslant \O(\log^2 n)$
\end{theorem}

\begin{proof}
    Рассмотрим три случая:
    \begin{itemize}
        \item у Алисы есть вершина $v$ маленькой степени $\leq n / 2$, то она посылает эту вершину
            Бобу, и они уменьшают граф до соседей вершины $v$, то есть хотя бы вдвое;
        \item аналогично для Боба, если есть вершина степени $\geq n / 2$;
        \item иначе у Алисы все вершины большой степени, а у Боба~--- маленькой, тогда пересечение пусто.
    \end{itemize}

    Раундов $\bigO{\log n}$, за каждый пересылаем $\bigO{\log n}$.
\end{proof}

[G{\"{o}}{\"{o}}s 2014]: нижняя оценка на $\DCC(\CIS)$ (не меньше $\log^{1 + \varepsilon} n$, точно не помним).

Продолжаем примеры функций:
\begin{enumerate}
        \setcounter{enumi}{2}
    \item $\DCC(\EQ) \leq n + 1$.
    \item $\DCC(\GT) \leq n + 1$.
        Сравниваем как числа, то есть лексикографически.
    \item $\DCC(\Disj) \leq n + 1$. $\Disj$: верно ли, что нет общей единички у $x$ и $y$.
\end{enumerate}

Оказывается, для $\EQ$, $\GT$, $\Disj$ это точные границы.


%\subsection{Нижняя оценка на $\DPLL$ через $\DCC(\Search_{\varphi})$}

$\varphi$ "--- формула, хотим проверить её на выполнимость. $\DPLL$-алгоритм: расщепляемся каждый раз по
какой-нибудь переменной (получается дерево).

Докажем, что есть невыполнимая формула, на которой работает экспоненциально долго. То есть любое дерево
решений для этой формулы будет иметь экспоненциальный размер. Пусть $\varphi(\vec x, \vec y)$~---
невыполнимая формула в КНФ, $C$~--- множество её клозов (дизъюнктов). Отношение $\Search_{\varphi}
\subseteq X \times Y \times C$, по $(x, y)$ ищем невыполненный клоз (любой, если несколько).

\begin{theorem}
    \label{th:DPLL-and-Search}
    $\DPLL(\varphi) \geq \DCC(\Search_{\varphi})$.
\end{theorem}

\begin{proof}
    Найдём центроид дерева расщепления (то есть размер его поддерева от $1/3$ до $2/3$ от размера всего
    дерева). Алиса и Боб проверяют, дойдут ли они в эту вершину. Это можно сделать независимо, и потом
    послать всего два бита. Рекурсивно вызовемся от одной из частей.

    Пусть осталась одна вершина. Во-первых, это лист (посмотрим в исходном дереве на путь,
    соответствующий подстановке, его последняя вершина никогда не будет удалена). Во-вторых, в нём
    написан клоз, который опровергается (по тем же причинам).

    Таким образом, дерево $T$ даёт протокол размера $2\log_{3/2} |T| = \bigO{\log |T|}$.
\end{proof}

Есть формулы, у которых $\DCC(\Search_{\varphi}) \geq \Omega(n / \log n)$: [Beame et al. 2007] (статья
<<нечитаемая>>), [Göös, Pitassi 2014] (статья <<понятная и простая>>, разберём). 

В обеих статьях $\Search_{\varphi}$ сводится к $\Disj$.

\section{Вероятностная коммуникация с публичными случайными битами}

В вероятностной коммуникации есть две модели: с публичными случайными битами и с приватными случайными
битами. Пока обсудим только первую.

У Алисы и Боба есть общие случайные биты $r$. $\Rpub[\varepsilon](f) = \min\limits_{\pi} \max\limits_{x, y}(\#\text{ переданных бит})$, где протоколы $\pi$ удовлетворяют
$\Pr_{r} [\pi(x, y)\neq f(x, y)] \leqslant \varepsilon$.

$\#\text{ переданных бит}$ считается в худшем случае по всем случайным битам, но большой разницы нет, если вместо этого считать матожидание.

Зависимость от $\\varepsilon$?
\begin{theorem}
$\varepsilon, \varepsilon' < 1/3$. Тогда
$\RCC[\varepsilon'](f) \le \bigO{\RCC[\varepsilon](f)\times \log\frac{1}{\varepsilon'}}$
\end{theorem}
\begin{proof}
Повторим протокол $100\log(1/\varepsilon')$ раз.
Оценка Чернова.
\end{proof}

\begin{theorem}
\label{R-pub(EQ)}
$\Rpub[1 / 2](\EQ) = 2$, причём ошибка односторонняя.
\end{theorem}
\begin{proof}
Сравнивают $\avg{x, r}$ и $\avg{y, r}$ для случайного $r$.
\end{proof}
\begin{corollary}
$\Rpub[\varepsilon](\EQ) = \bigO(\log \frac{1}{\varepsilon})$.
\end{corollary}

$\RCC(\GT) \le \bigO{\log n}$ "--- нетривиально (покажем позже). Пока что докажем более простую оценку:

\begin{theorem}
\label{R-pub(GT) simple}
$\Rpub(\GT) \le \bigO{\log n\cdot \log\log n}$.
\end{theorem}
\begin{proof}
Найдём первую позицию различия.
Бинарный поиск. Для сравнения используем $\EQ$ для префикса с ошибкой не более $\frac{c}{\log n}$ для $c < 1$ (т. е. $\bigO{\log \log n}$ повторений).

Суммарную ошибку оценим по union bound как $\log n \cdot \frac{c}{\log n} \leq c$.
\end{proof}