\chapter{Методы получения нижних и верхних оценок}
\section{Fooling Set}
Метод, которым мы доказали нижнюю оценку для $\EQ$ и $\GT$ "--- Fooling Set.

\begin{lemma}[Fooling Set]
Пусть $S\subseteq X\times Y$ "--- одноцветная комбинаторная диагональ: все $x$ различны и все $y$ различны, и для каждого $(x_a, y_a)\in S$ $f(x_a, y_a) = c$. Пусть при этом $\forall a=(x_a, y_a), b=(x_b, y_b)\in S$ $a$ и $b$ не могут лежать в одном прямоугольнике покрытия: $f(x_a, y_b)\neq c$ или $f(x_b, y_a)\neq c$.

Тогда для покрытия $S$ необходимо хотя бы $|S|$ прямоугольников.

В частности, $\D(f) \geqslant \log |S|$.
\end{lemma}

\begin{theorem}
$\D(\DISJ) = n + 1$.
\end{theorem}
\begin{proof}
Пусть $S = \{(a, \bar a)\mid a\in \{0, 1\}^n\}$. Тогда у всех $(a, \bar a)$ ответ 1, но для $a\neq b$ хотя бы одна из пар $(a, \bar b)$ или $(b, \bar a)$ имеет пересечение и ответ 0.
\end{proof}

\section{Rectangle Size}
Ещё один метод "--- Rectangle Size.
\begin{lemma}[Rectangle Size]
Пусть размер любого одноцветного прямоугольника имеет долю меньше $\eps$ от размера всего $M_f$. Тогда $\chi>1/\eps$.

В частности, $\D(f)>\log 1/\eps$.
\end{lemma}
\begin{remark}
Бывает полезно рассматривать покрытие не всей матрицы, а только какого-то из цветов в отдельности.
\end{remark}

Скалярное произведение (Inner Product): $\IP(x, y) = \sum x_iy_i \bmod 2$.

\begin{theorem}
$\D(\IP) = n + 1$.
\end{theorem}
\begin{proof}
Покажем, что любой одноцветный прямоугольник имеет площадь не более $2^n$. Тогда прямоугольников хотя бы $2^n$.

Пусть $R = A\times B$ и имеет цвет $c$. Пусть $\dim A = k$, и пусть $A$ лежит в линейной оболочке $\langle a_1, a_2, \ldots, a_k\rangle$, где все $a_i\in A$. Заметим, что $|A|\leqslant 2^k$.
Все элементы $b\in B$ удовлетворяют системе $a_i\cdot b = c$ для всех $i\in [k]$. У такой системы не более $2^{n-k}$ решений, поэтому $|B|\leqslant 2^{n-k}$.
\end{proof}

Мы уже доказывали, что $\D(\DISJ) \geqslant n + 1$. Докажем ещё раз (правда, с худшей константой), используя технику Rectangle Size.
\begin{theorem}
$\D(\DISJ) = \Omega(n + 1)$.
\end{theorem}
\begin{proof}
Посмотрим только на клетки, у которых $\DISJ(x, y) = 1$. Их $3^n$ (каждый бит принадлежит либо $x$, либо $y$, либо не принадлежит обоим).

Теперь посмотрим на $1$-прямоугольник $A\times B$. $\forall a\in A\ \forall b\in B\ a\cap b = \emptyset$. Значит, $A' \asgn \cup_{a\in A} a$ не пересекается с $B'\asgn \cup_{b\in B} b$. Тогда $|A'|+|B'|\leqslant n$.
Заметим, что $|A|\leqslant 2^{|A'|}$ (все подмножества $A'$), $|B|\leqslant 2^{|B'|}$.
Таким образом, $|A||B|\leqslant 2^{|A'|+|B'|}\leqslant 2^n$.

Следовательно, прямоугольников хотя бы $\frac{3^n}{2^n} = 1.5^n$.
Тогда $\D(\DISJ)\geqslant n\log_2 (1.5) = \Omega(n)$.
\end{proof}

Рассмотрим обобщение техники Rectangle Size:
\begin{lemma}[Rectangle Measure]
Пусть \emph{мера} любого одноцветного прямоугольника имеет долю меньше $\eps$ от меры всего $M_f$. Тогда $\chi>1/\eps$.

В частности, $\D(f)>\log 1/\eps$.
\end{lemma}

\begin{remark}
Здесь тоже бывает полезно рассматривать покрытие не всей матрицы, а только какого-то из цветов в отдельности.
\end{remark}

\begin{remark}
Меру всего множества (всех клеток $M_f$ или всех клеток какого-то цвета) обычно выбираем равной единице.
\end{remark}

\begin{remark}
Fooling Set включается в Rectangle Measure.
\end{remark}
\begin{proof}
Если $S$ "--- Fooling Set, то введём $\mu(x, y) = \frac{\One_{(x, y)\in S}}{|S|}$.
\end{proof}

\section{Fooling Set для случайных функций}

Оценим, насколько хорош Fooling Set <<в целом>>.

Будем рассматривать случайные функции $f\colon \{0, 1\}^n\times \{0, 1\}^n\to \{0, 1\}$, где каждая клетка $M_f$ заполняется случайно и независимо значениями.

\begin{theorem}
Для случайной функции с высокой вероятностью $\D(f)\geqslant\Omega(n)$.
\end{theorem}
\begin{proof}
Всего функций такого вида $2^{2^{2n}}$.

Протоколов с $l$ листьями:
\begin{itemize}
    \item $2^l$ "--- ответ в каждом листе;
    \item $(2\cdot 2^{2^n})^l$ "--- чей ход и функция игрока в каждой внутренней вершине (их $l - 1$);
    \item число упорядоченных бинарных деревьев с $l$ листьями не больше числа Каталана $C_{l - 1} \leqslant 2^{2l}$ (но нас устроит и, например, оценка $C_{2l}$).
\end{itemize}
Итого получаем, что протоколов не более $2^{(2^n + 4) l}$.

Значит, для почти всех функций (т. е. с высокой вероятностью) выполняется $l\geqslant c\cdot 2^n$. Тогда $\D(f)\geqslant cn$ с высокой вероятностью. То есть почти все функции "--- <<сложные>>.
\end{proof}

\begin{theorem}
Максимальный Fooling Set случайной функции с высокой вероятностью имеет размер $\O(\log n)$.
\end{theorem}
\begin{proof}
Оценим вероятность, что для случайной функции найдётся Fooling Set размера $t$.
Это должен быть комбинаторный квадрат с одноцветной диагональю (вероятность $2\cdot 2^{-t}$), и для каждой из $t(t-1)/2$ пар клеток диагонали хотя бы в одном из углов образовываемого прямоугольника должен быть другой цвет (вероятность $(3/4)^{t(t-1)/2}$). Способов выбрать $t$ клеток в качестве диагональных не больше $\binom{n^2}{t} \leqslant n^{2t}$.
Тогда с высокой вероятностью $t\leqslant \O(\log n)$.
\end{proof}

Получается, Fooling Set даёт нижнюю оценку на коммуникационную сложность случайной функции не лучше $\Omega(\log\log n)$. А реальная сложность $\Omega(n)$.

\section{Rank Technique}
\subsection{Связь \texorpdfstring{$\rk(M_f)$}{rk(M(f))} и \texorpdfstring{$\D(f)$}{D(f)}}
\begin{remark}
$\rk(A+B)\leqslant \rk(A) + \rk(B)$
\end{remark}

Пусть $\{R_i\}$ "--- 1-прямоугольники разбиения матрицы $M_f$, их $\chi_1(f)$ штук. Обозначим за $\One_{R_i}$ матрицу размерности как $M_f$, в которой единицы только на месте прямоугольника $R_i$.
Тогда $M_f = \sum_{i} \One_{R_i}$.
Тогда $\rk_\mathbb{F}(M_f)\leqslant \sum \rk_\mathbb{F}(\One_{R_i}) = \chi_1$ для любого поля $\mathbb{F}$, так как ранг 1-прямоугольника равен единице.

Выберем поле $\mathbb{R}$, чтобы $\rk$ был побольше.
\begin{theorem}
$\chi_1\geqslant \rk_\mathbb{R}(M_f)$.
Более того, $\D(f)\geqslant \log (2\rk_\mathbb{R}(M_f) - 1)$
\end{theorem}
\begin{proof}
Первую часть уже доказали.

Заметим, что $\rk(\overline{M_f})\leqslant \chi_0$ (аналогично доказанному неравенству).

Кроме того, $\rk(M_f+\overline{M_f}) = \One$, поэтому их ранги отличаются не более, чем на единицу: $\rk(M_f) \leqslant \One + \rk(-\overline{M_f}) = 1 + \rk(\overline{M_f})$, аналогично в другую сторону.

Тогда $2\rk(M_f)-1\leqslant \rk(M_f) + \rk(\overline{M_f})\leqslant \chi_0+\chi_1 = \chi$, а $\D(f)\geqslant \log \chi$.
\end{proof}

\begin{theorem}
$\D(f)\leqslant \rk_{\mathbb{F}_2}(M_f)+1\leqslant \rk_{\mathbb{R}}(M_f)+1$
\end{theorem}
\begin{proof}
$M_f = \sum^{\rk(M_f)}_{i = 1} M_i$, где $M_i$ "--- матрицы ранга 1. Каждая строка такой матрицы $M_i$ либо $v_i$, либо $\vec 0$.
Тогда Алиса для каждой матрицы $M_i$ посылает тип строки $x$: $\vec 0$ или $v_i$ (то есть по биту на каждую $M_i$).

Боб складывает $v_i$ для тех матриц, в которых у Алисы $v_i$ (а не $\vec 0$), и получает строку, совпадающую со строкой $(M_f)_x$. В этой строке он смотрит на бит в позиции $y$ "--- это и будет ответ. Итого переслано $\rk(M_f) + 1$ бит.
\end{proof}

\begin{theorem}[Lovett, 2013]
$\D(f)\leqslant \polylog(\rk_\mathbb{R}(M_f))\cdot \sqrt{\rk_\mathbb{R}(M_f)}$
\end{theorem}
Доказывать не будем.
План: если ранг маленький, тогда есть большой одноцветный прямоугольник, с помощью него уменьшим размер задачи, рекурсивно спустимся.

Давно открытая проблема "--- log-rank conjecture: $\D(f)\leqslant \polylog(\rk_\mathbb{R}(M_f))$?
Известно, что для $\log$ вместо $\polylog$ утверждение неверно: есть пример со степенью логарифма $\approx 1.4$ или что-то вроде.

\subsection{Оценки рангов матриц \texorpdfstring{$\IP$}{IP} и \texorpdfstring{$\DISJ$}{DISJ}}

\begin{theorem}
\label{rk(M-IP) lower bound}
$\rk_{\mathbb{R}}(M_{\IP}) \geqslant 2^n - 1$.
\end{theorem}

\begin{proof}
$\rk(AB)\leqslant \min(\rk(A), \rk(B))$, поэтому если докажем нижнюю оценку на ранг $M^2$, то и для $M$ докажем.

\paragraph{Доказательство с лекции.}
На диагонали $M^2$: в клетке $(x, x)$ при $x\neq 0$ стоит $\sum_y\langle x, y\rangle \langle y, x\rangle = \sum_y \langle x, y\rangle^2$. Сколько единичных слагаемых получится? У нас одно линейное уравнение над $\mathbb{F}_2$, поэтому $2^{n-1}$. Так что на диагонали везде $2^{n-1}$, кроме $(0, 0)$.

Не на диагонали $M^2$: в клетке $(x, z)$ при ненулевых различных $x$ и $z$ стоит $\sum_y\langle x, y\rangle \langle y, z\rangle$. Сколько единичных слагаемых? $x$ и $z$ линейно независимы, поэтому ранг системы равен $2$, поэтому решений $2^{n-2}$.

В строчке $0$ и столбце $0$ все значения равны нулю.

Ранг такой матрицы не меньше $2^n - 2$: это диагональная матрица (но с одним нулём на диагонали) плюс матрица из одинаковых чисел.
\mycomment{Можно ли отсюда получить оценку $\ge 2^n - 1$, мне неочевидно. На консультации было предложено другое доказательство.}

\paragraph{Доказательство с консультации.}
Перейдём к $\pm 1$-матрице: $M'_{x,y} = (-1)^{\langle x, y\rangle}$. Заметим, что $M' = \One - 2M$. Если мы покажем, что ранг $M'^2$ равен $2^n$, тогда ранг $M'$ равен $2^n$, тогда $M$ имеет ранг не меньше $2^n - 1$.

$M'^2_{x, z} = \sum_y (-1)^{\langle x, y\rangle}\cdot (-1)^{\langle y, z\rangle} = \sum_y (-1)^{\langle y, x + z\rangle}$. Если $x = z$, то $x + z = \vec 0$, то есть сумма равна $2^n$ "--- это числа на диагонали $M'^2$. Если $x\neq z$, то ровно половина $y$ будет давать скалярное произведение $1$, а другая половина "--- $0$, поэтому сумма получится равной $0$.

Получили, что матрица $M'^2$ "--- диагональная, причём все элементы на диагонали не нули. Значит, она имеет ранг $2^n$.
\end{proof}

\begin{theorem}
$\rk(M_{\DISJ}) = 2^n$, то есть $M_{\DISJ}$ имеет полный ранг.
\end{theorem}
\begin{proof}
Обозначим $M_n = M_{\DISJ_n}$. Заметим, что $M_n = M_{n-1} \otimes M_1$ (тензорное произведение), тогда $M_n = M^{\otimes n}_1$. Известно, что $\rk(A\otimes B) = \rk(A)\rk(B)$. $\rk(M_1) = 2$, поэтому $\rk(M_n) = 2^n$.
\end{proof}

Матрицы с большим rigidity "--- меняем мало элементов, ранг всё равно остаётся большим. Явное построение таких матриц "--- открытый вопрос. Они дают сложные функции для схем и другие важные последствия. Случайные матрицы удовлетворяют этому свойству.

\subsection{Ранг матрицы случайной функции}
Мы уже знаем, что сложность случайных функций почти всегда большая. Будет ли большим ранг матрицы?

\begin{theorem}
С константной вероятностью $\mathbb{F}_2$-ранг случайной $\{0, 1\}$-матрицы $n\times n$ будет полным.
\end{theorem}
\begin{proof}
Выбираем каждую новую строку так, чтобы не попасть в оболочку предыдущих. Вероятность успеха: $p = (1-2^{-n})\cdot (1-2^{-n+1})\cdot\ldots (1-2^{-n+(n-1)})$.
Воспользуемся тем, что $\log_2(1-x)\geqslant -2x$ для $x\leqslant 1/2$:
$$\log p = \sum_{i=0}^{n-1} \log(1-2^{-n+i})\geqslant -2\sum_{i=0}^{n-1} 2^{-n+i} \geqslant -2.$$
Так что $p\geqslant 1/4$.
\end{proof}

\subsection{Fooling Set\texorpdfstring{$\implies$}{ => }Rank Technique}

\begin{theorem}
Пусть есть Fooling Set размера $k$. Тогда для любого поля $\mathbb{F}$ выполняется $\rk_\mathbb{F}(M_f) \geqslant \sqrt{k} - 1$.
\end{theorem}
\begin{proof}
Пусть $S$ "--- Fooling Set, причём на диагонали у него единички (если нули, перейдём к $\overline{M_f}$ "--- отсюда появится <<$-1$>> в оценке). $S$ образует комбинаторный квадрат, назовём его $M'$. 

$A\odot B$ "--- покоординатное умножение матриц.
$A\odot B$ является подматрицей $A\otimes B$, поэтому $\rk(A\odot B)\leqslant \rk(A)\cdot \rk(B)$.

Заметим, что $M'\odot M'^T = E_{|S|}$. Тогда $|S| = \rk(E_{|S|})\leqslant \rk(M')^2\leqslant \rk(M_f)^2$.
\end{proof}

\begin{corollary}
Для $\IP$ размер Fooling Set не больше $(n+1)^2$. (То есть Fooling Set даёт лишь логарифмическую нижнюю оценку на $\D(\IP)$.)
\end{corollary}
\begin{proof}
В поле $\mathbb{F}_2$ выполняется $M_{\IP} = X\times Y^T$, поэтому $\rk_{\mathbb{F}_2} M_{\IP}\leqslant \min(n, n) = n$. Применяя только что доказанную теорему, получаем требуемое.
\end{proof}

См. \href{https://youtu.be/kmdKu0TYRZ4?t=5654}{рисунок} про то, какие техники для чего работают.