\section{Corruption of $\DISJ$}

Материал этой секции изложен в соответствии со статьёй Разборова <<On the distributional complexity of disjointness>>.

Мы докажем нижнюю оценку на $corr(\DISJ)$. Немного с другими константами, нежели в определении из предыдущей секции, но, как мы уже выяснили, это совершенно не важно.

Пусть $A = \DISJ^{-1}(1)$, $B = \DISJ^{-1}(0)$.

Построим меру $\mu$ со следующими свойствами:

\begin{itemize}
    \item $\mu(A) = \frac{3}{4}$,
    \item для любого $R$ --- $1$-biased прямоугольника $\mu(R \cap B) \geq \alpha \mu(R \cap A) - 2^{-\varepsilon n}$.
\end{itemize}

Если нам удастся построить такую меру, то нижняя оценка на вероятностную коммуникационную сложность сразу последует. Что на самом деле означает второе свойство меры? Если мера $R$ не экспоненциально мала, слагаемое $2^{-\varepsilon n}$ не вносит значительного вклада, и нули составляют константную долю <<веса>> $R$. Если в определении $1$-biased прямоугольника у нас была константа меньше, чем $\alpha$, то мы получаем немедленное противоречие. Значит, мера каждого такого $R$ должна быть экспоненциально маленькой. Это и является оценкой на $corr(\DISJ)$.

Итак, построим меру $\mu$. Пусть $n = 4m - 1$.

\begin{enumerate}
    \item Выберем случайное разбиение $[n] = z_x \cup z_y \cup \{i\}$, $|z_x| = |z_y|$,
    \item $x \subseteq z_x \cup \{i\}$, $y \subseteq z_y \cup \{i\}$, каждый элемент выбирается в подмножество uniformly independently.
\end{enumerate}

Случайную величину $(z_x, z_y, i)$ обозначим $t$.

Рассмотрим также распределения $(x_0, y_0) = (x, y) \mid_{i \notin x, i \notin y}$ и $(x_1, y_1) = (x, y) \mid_{i \in x, i \in y}$.

Входы Алисы и Боба будут распределены следующим образом: с вероятностью $\frac{3}{4}$ мы выбираем их из распределения $(x_0, y_0)$, с вероятностью $\frac{1}{4}$ --- из $(x_1, y_1)$. Это сразу обеспечивает первое свойство меры. Докажем второе. Зафиксируем конкретный $1$-biased прямоугольник $R = S \times T$.

Рассмотрим вероятности $p_x(t) = Pr[x \in S \mid (z_x, z_y, i) = t]$, $p_y(t) = Pr[y \in T \mid (z_x, z_y, i) = t$, а также $p_{x, 1}(t) = Pr[x_1 \in S \mid (z_x, z_y, i) = t]$ и $p_{x, 0}(t)$, $p_{y, 1}(t)$, $p_{y, 0}(t)$, определённые аналогично.

\begin{remark}
$p_x(t) = \frac{1}{2}(p_{x, 0}(t) + p_{x, 1}(t))$, аналогично $p_y(t) = \frac{1}{2}(p_{y, 0}(t) + p_{y, 1}(t))$
\end{remark}

\begin{proof}
Это формула полной вероятности, учитывая, что $i \in S$ с вероятностью $\frac{1}{2}$, а условие на $y$ никак не влияет, т.к. если зафиксировали $t$, то всё независимо.
\end{proof}

\begin{remark}
$p_x(z_x, z_y, i)$ и $p_{y, 0}(z_x, z_y, i)$ зависят только от $z_y$.
\end{remark}
\begin{proof}
В первом случае мы выбираем случайное подмножество $[n] \backslash z_y$, во втором --- случайное подмножество $z_y$.
\end{proof}

Хотим показать, что $Pr[(x_1, y_1) \in R] \geq \Omega(Pr[(x_0, y_0) \in R]) - 2^{-\varepsilon n}$.

$Pr[(x_1, y_1) \in R] = E_t[p_{x, 1}(t) \cdot p_{y, 1}(t)] \geq E_t[p_{x, 1}(t) \cdot p_{y, 1}(t) \cdot (1 - \chi(t))]$, где $\chi(t)$ --- это характеристическая функция, показывающая, что $t$ --- <<плохое>>. Мы назовём $t$ плохим, если $p_{x, 1}(t) \leq \frac{1}{3}p_{x, 0}(t) - 2^{-\varepsilon n}$ ($x$-плохое) или $p_{y, 1} \leq \frac{1}{3} p_{y, 0}(t) - 2^{-\varepsilon n}$.

Тогда $E_t[p_{x, 1}(t) \cdot p_{y, 1}(t) \cdot (1 - \chi(t))] \geq E_t[\frac{1}{3}p_{x, 0}(t)\frac{1}{3}p_{y, 0}(t)(1 - \chi(t))] - 2^{-\varepsilon n} \geq \Omega(E[p_{x, 0}(t)p_{y, 0}(t)(1 - \chi(t))]) - 2^{-\varepsilon n}$. Если у нас получится убрать отсюда $\chi(t)$, то мы докажем теорему.

\begin{lemma}
$Pr[t \text{ is } x\text{-bad } \mid z_y = z] \leq \frac{1}{5}$, аналогично для $y$-bad.
\end{lemma}

\begin{proof}
Заметим, что, раз мы зафиксировали $z_y$, то $p_x = \text{const}$. Разберём случаи.

1). $p_x \leq 2^{-\varepsilon n}$. Тогда условие $p_{x, 1}(t) \leq \frac{1}{3}p_{x, 0}(t) - 2^{- \varepsilon n}$ не можем выполниться --- все вероятности слишком маленькие, и $t$ не может быть $x$-плохим.

2). $p_x > 2^{-\varepsilon n}$. Предположим, что условие леммы не выполняется, т.е. $Pr[(z_x, z_y, i) \text{ is } x\text{-bad } \mid z_y = z] > \frac{1}{5}$.

Пусть множество $M$ --- это случайный элемент $S \cap (co-Z)^{2m}$. Заметим, что $p_{x, 1}(t) = 2 p_x Pr[i \in M]$. Аналогично $p_{x, 0}(t) = 2 p_x Pr[i \notin M]$. Тогда из того, что $t$ --- плохое, следует $Pr[i \in M] \leq \frac{1}{3} Pr[i \notin M] \Rightarrow Pr[i \in M] \leq \frac{1}{4}$, и это верно для $> \frac{1}{5}$ элементов $i \in co-Z$.

Рассмотрим случайные величины $\chi(i_1 \in M), \chi(i_2 \in M), \ldots, \chi(i_{2m} \in M)$, где $\{i_1, \ldots, i_{2m}\} = co-Z$.

Посчитаем энтропию $M$. $m(2 - 4\varepsilon - o(1)) \leq H(M) \leq \sum H(\chi(i \in M)) \leq 2m \frac{4}{5} + 2m \frac{1}{5} H(\frac{1}{4})  \leq 1.93m$, противоречие. Первое неравенство здесь использует то, что $p_x > 2^{-\varepsilon n}$; это стандартное неравенство для энтропии, подробнее мы не объясняли.
\end{proof}

Теперь, когда у нас доказана лемма, можем завершить доказательство теоремы. Нам достаточно показать, что $E[p_{x, 0}(t) p_{y, 0}(t) \chi(t)] \leq \frac{4}{5}E[p_{x, 0}(t)p_{y, 0}(t)]$.

Зафиксируем $z_y = z$ и докажем неравенство для каждого конкретного $z$. Также будем рассматривать отдельно $\chi_x(t)$ и $\chi_y(t)$, проведём доказательство для $\chi_x(t)$ с константой $\frac{2}{5}$, для $\chi_y(t)$ всё будет аналогично.

После того, как мы зафиксировали $z$, $p_{y, 0}$ превратилось в константу, и можно вынести его из-под матожидания.

$E[p_{x, 0}(t)p_{y, 0}(t)\chi_x(t) \mid z_y = z] = p_{y, 0} \cdot E[p_{x, 0}(t) \chi_x(t) \mid z_y = z] \leq p_{y, 0} \cdot E[2p_{x} \chi_x(t) \mid z_y = z] = 2p_{y, 0}p_x \cdot E[\chi_x(t) \mid z_y = z]$.

Воспользовавшись леммой, получим, что это не больше, чем $\frac{2}{5} p_{y, 0}p_x$. При этом $p_{y, 0}p_x = E[p_{x, 0}(t)p_{y, 0}(t) \mid z_y = z]$. Почему? $p_{y, 0}$ --- константа, которую можно спокойно внести обратно под матожидание. Про $p_x$ заметим, что $p_x = E[p_{x, 0}(z_x, z_y, i) \mid z_y = z] = Pr[x_0 \in S \mid z_y = z]$, так как при условии $(z_y = z)$ $x_0$ принимает все значения из $(co-Z)^m$ с одинаковой вероятностью ${{2m}\choose{m}}^{-1}$.
