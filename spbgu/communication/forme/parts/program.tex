\centerline{\textsc{This plan includes some parts that were not presented on the lectures}}

\centerline{\textsc{Lecture 1}}

\begin{enumerate}
    \item Model. Two words about motivation. Functions and relations (promise problems).
    \item Deterministic protocols. Trivial bounds ($\log |X| + \log |Z|, \dots$) $f: X \times Y \to Z$.
    \item Examples:
        \begin{itemize}
            \item $D(\Parity) = 2$;
            \item Pointer Chasing. $D \le k \log n$.
            \item Clique and Ind. Set. $D \le \log^2 n$ (important problem!) [Yan. 91].
            \item \skp Median of multiset. $D \le \log^2 n$ (Can you do better?).
            \item Equality. $D \le n + 1$.
            \item Greater Than. $D \le n + 1$.
            \item Disjointness. $D \le n + 1$.                
        \end{itemize}
    \item Randomized protocols. Two and one-sided errors. Public and private randomness (focus on public
        bits). (Can we consider worst-case on random bits?)
    \item Error reduction.
    \item Examples:
        \begin{itemize}
            \item Equality. $R \le O\left(\log \frac{1}{\varepsilon}\right)$. One-sided!
            \item Greater Than. $R \le O(\log n \log\log n)$. Better? (Nissan?)
            \item Disjointness. $R \le O(n)$.      
        \end{itemize}
    \item Applications ($\Search_{\varphi}$):
        \begin{itemize}
            \item $\DPLL$ algorithms;
            \item \skp $\DPLL(\oplus)$ algorithms.
        \end{itemize}
\end{enumerate}


\centerline{\textsc{Lecture 2}}

\begin{enumerate}
    \item Deterministic protocols. Formal Definition (trees). Size and depth.
    \item Rectangles and tilings. Th. Protocols gives $2^{C_p}$ tiling.
        Discussion about $\Search_{\varphi}$.
    \item Balancing protocols.
    \item $\EQ, \GT$ lower bounds.
    \item $D \le O(\log^2 \chi)$ (covering!).
\end{enumerate}


\centerline{\textsc{Lecture 3}}

\begin{enumerate}
    \item $D \le O(\min( \log^2 \chi_0, \log^2 \chi_1))$ (tiling).
    \item Karchmer--Wigderson relation.
    \item Monotone Karchmer--Wigderson is a complete relation.
\end{enumerate}

\centerline{\textsc{Lecture 4}}

\begin{enumerate}
    \item Fooling sets. $\chi$.
        \begin{itemize}
            \item $\GT$.
            \item $\DISJ$. $\{x, \bar{x}\}$.
        \end{itemize}
    \item Rectangle size. $\mu(R) \le \delta \Rightarrow \chi \ge \frac{1}{\delta}$.
        \begin{itemize}
            \item $\IP$. ($|A| \times |B| \le 2^n$).
            \item $\DISJ$. ($|A| + |B| \le n$).
        \end{itemize}
    \item Fooling set of a random function.
    \item Rank technique (over $\mathbb{R}$). $\log(2 \rk(M_f) - 1)$ ($M$ and $\bar{M}$).
        \begin{itemize}
            \item $\IP$. ($M_f^2$).
            \item $\DISJ$.
                $\begin{pmatrix}
                    1 & 1\\
                    1 & 0
                \end{pmatrix}^{\oplus n}$
        \end{itemize}
    \item Rank of random function.
    \item $D(f) \le \rk(M_f)$.
    \item Rank over any field $\Rightarrow$ Fooling set. ($A \bigodot A^T$). No fooling set for $\IP$.
\end{enumerate}



\centerline{\textsc{Lecture 5}}

\begin{enumerate}
    \item Non-deterministic communication. Three definitions and equivalence.
    \item $\P = \NP \cap \coNP$.
    \item $C^a(f) = O\left(\frac{n}{RS^a(f)}\right)$.
    \item $\DISJ^k$.

\end{enumerate}

\centerline{\textsc{Lecture 6}}

\begin{enumerate}
    \item $\DISJ^k$.
    \item Randomized protocols.
    \item $\EQ$. Public and private random bits.
    \item $D(f) \le 2^{R_{\varepsilon}(f)} \left( \log(\frac{1}{2} - \varepsilon)^{-1} +
        R_{\varepsilon}(f) \right)$ (private).
    \item Newman's Theorem. 
\end{enumerate}



\centerline{\textsc{Lecture 7}}

\begin{enumerate}
    \item Efficient randomized protocol for $\DISJ^k$.
    \item The notion of ``critical block sensensitivity''.
    \item Critical block sensitivity of Tseitin formulas.
\end{enumerate}

\centerline{\textsc{Lecture 8}}

\begin{enumerate}
    \item $R(S \times \mathrm{VER}) \ge \mathrm{cbs}(S)$. (G{\"{o}}{\"{o}}s, Pitassi 14).
\end{enumerate}

\centerline{\textsc{Lecture 9}}

\begin{enumerate}
    \item Discrepancy. $R(\IP) = \Omega(n)$.
    \item Disjoines over product distributions. Lower bound.
\end{enumerate}

\centerline{\textsc{Lecture 10}}

\begin{enumerate}
    \item Disjoines over product distributions. Upper bound.
    \item Corruption bound.
    \item Corruption of $\DISJ$.
\end{enumerate}