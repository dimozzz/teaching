\documentclass[a4paper, 12pt]{article}
% math symbols
\usepackage{amssymb}
\usepackage{amsmath}
\usepackage{mathrsfs}
\usepackage{mathseries}


\usepackage[margin = 2cm]{geometry}

\tolerance = 1000
\emergencystretch = 0.74cm



\pagestyle{empty}
\parindent = 0mm

\renewcommand{\coursetitle}{DM/ML}
\setcounter{curtask}{1}

\setmathstyle{СПбГУ}{Дискретная математика}{1 курс}


\begin{document}

\libproblem{combinatorics}{sum-of-cubes}
\libproblem{combinatorics}{bernulli-ineq}
\libproblem{combinatorics}{unique-lift-weight}
\libproblem{discrete-math}{gray-codes}
\libproblem{other}{linear-system-solutions}
\libproblem{combinatorics}{sum-of-inv-squares}
\libproblem{discrete-math}{inversions-fixed-points}
\libproblem{discrete-math}{chem-alchem-query}

\begin{definition*}
    Последовательность \deftext{гармонических чисел} $H_j$, где $j = 1, 2, 3, \dots$ будем определять
    как:
    $$
        H_j \coloneqq 1 + \frac{1}{2} + \frac{1}{3} + \cdots + \frac{1}{j}.
    $$

    Например,
    $$
        H_4 \coloneqq 1 + \frac{1}{2} + \frac{1}{3} + \frac{1}{4} = \frac{25}{12}.
    $$
\end{definition*}

\libproblem{discrete-math}{harmonic-num-one-and-half}
\libproblem{discrete-math}{dzeta-one-half-ineq}
\libproblem{discrete-math}{erdos-szekeres}




\end{document}



%%% Local Variables:
%%% mode: latex
%%% TeX-master: t
%%% End:
