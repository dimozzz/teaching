\documentclass[a4paper, 12pt]{article}
% math symbols
\usepackage{amssymb}
\usepackage{amsmath}
\usepackage{mathrsfs}
\usepackage{mathseries}


\usepackage[margin = 2cm]{geometry}

\tolerance = 1000
\emergencystretch = 0.74cm



\pagestyle{empty}
\parindent = 0mm

\renewcommand{\coursetitle}{DM/ML}
\setcounter{curtask}{1}

\setmathstyle{СПбГУ}{ДМ. Задание 2}{1 курс}

\begin{document}

\task{
    Являются ли следующие функции линейными:
    \begin{enumcyr}
        \item $(x \to y)(y \to x)$;
        \item $(1001100101100110)$?
    \end{enumcyr}
}



\task{
    Найдите число функций от $n$ переменных в каждом из классов:
    сохраняющие ноль/сохраняющие единицу/самодвойственные/линейные.
    \hinttext{Комментарий:} точное число монотонных функций не известно.
}

\task{
    Докажите, что все пять классов функций: сохраняющие ноль/сохраняющие
    единицу/самодвойственные/линейные/монотонные попарно различны, и ни один из них не лежит ни в каком
    другом.
}


\libproblem{discrete-math}{mon-func-mon-formula}

\thcomment{
    В данной задаче подразумевается, что $x$ и $y$~--- это последовательности переменных, т.е. $x \coloneqq
    (x_1, x_2, \dots, x_n)$ и $y \coloneqq (y_1, y_2, \dots, y_n)$.
}


\begin{definition*}
    Рассмотрим пропозициональные формулы, которые используют константу $1$, конъюнкцию $\land$ и сумму по
    модулю два $\oplus$ (приоритет $\land$ выше, чем $\oplus$). \deftext{Мономом} будем называть
    константу $1$ и конъюнкцию нескольких переменных. \deftext{Полиномом (многочленом) Жегалкина} назовем формулу
    вида $m_1 \oplus m_2 \oplus \dots \oplus m_k$, где $m_i$~--- различные мономы, $k \ge 0$. Пример:
    $x_1 x_2 \oplus x_2 \oplus 1$.
\end{definition*}

\libproblem{discrete-math}{polynomial-representation.mod}

\task{
    Предъявите функцию, у которой длина полинома Жегалкина в $2^n$ раз превосходит длину ее СДНФ.
}

\libproblem{complexity}{cnf-dnf-size-easy}
\libproblem{discrete-math}{and-or-swap-eq}

\task{
    Полны ли следующие системы функций:
    \begin{enumcyr}
        \item $\{xy, x \oplus y, x \to y\}$,
        \item $\{(0100), (01101001), (01111111)\}$?
    \end{enumcyr}
}

\libproblem{discrete-math}{peirce-arrow}


\end{document}
