Рассмотрим какую-нибудь вычислительную модель $\mathfrak{C}$ для подсчета булевых функций на каком-нибудь
множестве: $f\colon X \to \{0, 1\}$. Если модель $\mathfrak{C}$ достаточно мощная, например, булевы
схемы, машины Тьюринга, или почти любая \emph{естественная} модель вычислений, то на текущий момент мы не
знаем как предъявлять функции, являющиеся \emph{сложными} для модели $\mathfrak{C}$. Из этого <<правила>>
есть исключения, в частности, если мы разрешим использовать только монотонные вычисления (например
разрешим только монотонные операции в схемах), то нижние оценки все-таки удается получить.

С практической точки зрения монотонные вычисления важны, поскольку позволяют контролировать относительную
погрешность вычислений. С теоретической точки зрения оказалось, что подобные вычисления тесно связаны с:
\begin{itemize}
    \item со сложностью доказательств, и позволяют переносить оценки из одной области на другую;
    \item с алгоритмами для задачи выполнимости булевых формул;
    \item коммуникационной сложностью для стандартных и не очень моделей вычислений.
\end{itemize}

В курсе мы попробуем понять как же доказывать нижние оценки на монотонные вычисления при помощи:
классических техник Разборова, Алона -- Боппаны; новых техник, таких, как ``lifting''. А также мы
попробуем разобраться с тем как монотонные вычисления связаны с другими областями теории сложности и 
понять чем же монотонные вычисления отличаются от других.

Для курса необязательно знать коммуникационную сложность и сложность доказательств (все необходимые
утверждения будут доказаны), однако знание этих предметов может облегчить понимание материала.