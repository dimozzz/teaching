\centerline{\textsc{Lecture 1}}

\begin{enumerate}
    \item Зачем нужны экспандеры? Задача о суперконцентраторе. Задача о понижении ошибки в
        $\RP$. Случайные блуждания.
    \item Двудольные и общие экспандеры.
%    \item Magical graphs (параметры).
%    \item Хранение множества $S \subseteq [n]$ размера $n$ с запросом в $1$ бит.
    \item Понижение ошибки в $\RP$ без доп. случайных битов. Явные и сильно явные конструции.
    \item $\Search_{\varphi}$ и деревья решений.
\end{enumerate}


\centerline{\textsc{Lecture 2} \mater{L-20. Видео 2}}

\begin{enumerate}
    \item Magical graphs. Конструкция суперконцентратора.
        \mater{
            HLW 1.2, 1.3
            CSC-N
        }
    \item Матрица смежности графа. Лапласиан графа (хотим полином $x^T M x = \sum\limits_{(i, j) \in E}
        (x_i - x_j)^2$).
        \mater{
            LT-N. 3.1
        }
    \item Свойства матрицы смежности:
        \begin{itemize}
            \item симм., следовательно есть собственный базис над $\mathbb{R}$;
            \item $(1, \dots, 1)$~--- максимальный собственный вектор.
            \item $\lambda_k = 1$ тогда и только тогда, когда есть $k$ комп. связности.
            \item $\lambda_n = -1$ тогда и только тогда, когда есть двудольная компонента.
            \item $\sum \lambda_i$~--- число петель.
        \end{itemize}
        \mater{
            Только видео.
        }
    \item Собственные числа, как оптимизационная задача. $\lambda_2 = \max\limits_{...} \frac{|x^T A
        x|}{\norm{x}^2}$.
        \mater{
            LT-N. 3.2
        }
    \item Cheeger's inequality. $\frac{1 - \alpha}{2} \le \varphi(S) \le \sqrt{2 (1 - \alpha)}$
        \begin{itemize}
            \item Док-во. Рассмотрим $x = |\bar{S}| 1_S - |S| 1_{\bar{S}}$.
        \end{itemize}
        \mater{
            CSC-N.
            Доказываем только нижнюю оценку.
        }
\end{enumerate}

\centerline{\textsc{Lecture 3} \mater{L-20. Видео 3}}

\begin{enumerate}
    \item Mixing Lemma. \mater{CSC-N, HLW 2.4}
    \item Нижние оценки на собсвенные числа для графов константной степени. \mater{CSC-N стр. 11}
    \item Существование экспандеров (вероятностная кострукция). \mater{CSC-N стр. 12-14}
\end{enumerate}

\centerline{\textsc{Lecture 4} \mater{L-20. Видео 4 (конец) и 5}}

\begin{enumerate}
    \item Zig-zag и сбалансированное подстановочное произведение. \mater{CSC-N}
    \item Явная и сильно-явная конструкция экспандеров. \mater{CSC-N}
    \item Zig-zag произведение, оценка на собственные числа. $\alpha + 2 \beta + \alpha \beta$. \mater{CSC-N}
\end{enumerate}

\centerline{\textsc{Lecture 5 \mater{L-20. Видео 5}}} (будет уточнение)

\begin{enumerate}
    \item Cбалансированное подстановочное произведение, оценка на собственные числа. $\left(1 -
        \frac{\varepsilon \delta^2}{24}\right)$. \mater{CSC-N}
    \item Любой граф --- экспандер. \mater{CSC-N раздел 5, часть, как упражнения;}        
    \item Случайные блуждания и вероятностный алгоритм для $\UPATH$. Схема. \mater{CSC-N раздел 5}
    \item Алгоритм Рейнгольда.
\end{enumerate}


\centerline{\textsc{Lecture 6} (HLW Раздел 3)}

\begin{enumerate}
    \item Случайные блуждания и расстояние до равномерного распределения.
    \item Случайные блуждания и вероятность выхода из подмножества вершин $\Pr[B_t] \le (\alpha +
        \beta)^t$.
    \item Случайные блуждания и вероятностный алгоритм для $\UPATH$, завершение.
    \item Понижение ошибки в классе $\RP$.
    \item Обобщение задачи выхода из подмножества. Понижение ошибки в классе $\BPP$.
\end{enumerate}


\centerline{\textsc{Lecture 7}}

\begin{enumerate}
    \item Применения случайных блужданий. \mater{продолжение лекции 6}
    \item Графы Кэли. \mater{HLW, CSC-N, LT-N}
    \item Собственные числа графов Кэли. \mater{HLW, CSC-N, LT-N}
    \item Примеры: куб, цикл. \mater{HLW, CSC-N, LT-N}
    \item Cобственные числа пути. \mater{CG}
    \item Случайные генераторы. \mater{CG. 5.8}
    \item Конструкция графов Рамануджана (без доказательства). \mater{CSC-N}
\end{enumerate}

\centerline{\textsc{Lecture 8} \mater{SH. 1-6}}

\begin{enumerate}
    \item Коды, исправляющие ошибки.
    \item Граница Хемминга.
    \item Жадный алгоритм, граница Гилберта.
    \item Код Хемминга.
    \item Задача о шляпах. \mater{См. таблицу}
    \item Вероятностная конструкция кодов.
    \item Линейные коды.
    \item Жадный алгоритм для линейных кодов.
\end{enumerate}


\centerline{\textsc{Lecture 9} \mater{SH. 7-10, 11 (только начало)}}

\begin{enumerate}
    \item Граница Синглтона.
    \item Код Рида--Соломона.
    \item Ошибки и стирания, еще раз о кодах Рида--Соломона.
    \item Композиция кодов. Каскадные коды.
    \item Расстояние каскадного кода. Наивный алгоритм декодирования.
\end{enumerate}

\centerline{\textsc{Lecture 10} \mater{SH. 11-12, 15-17}}

\begin{enumerate}
    \item Каскадный код со стираниями. Алгоритм декодирования.
    \item Теорема Форни.
    \item Код Адамара.
    \item Оценка Плоткина.
    \item Усиление оценки Синглтона.
\end{enumerate}

\centerline{\textsc{Lecture 12}}

\begin{enumerate}
    \item Если хочется побыстрее: коды Земора.
    \item Коды БЧХ.
\end{enumerate}


\centerline{\textsc{Lecture 13}}

\begin{enumerate}
    \item Код Адамара 2.
    \item Вероятностное декодирование кода Адамара.
    \item Списочное декодирование.
\end{enumerate}