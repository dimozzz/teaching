\centerline{\textsc{Lecture 1}}

\begin{enumerate}
    \item Зачем нужны экспандеры? Задача о суперконцентраторе. Задача о понижении ошибки в
        $\RP$. Случайные блуждания.
    \item Двудольные и общие экспандеры.
    \item Magical graphs (параметры).
    \item Хранение множества $S \subseteq [n]$ размера $n$ с запросом в $1$ бит.
    \item Понижение ошибки в $\RP$ без доп. случайных битов. Явные и сильно явные конструции.
\end{enumerate}


\centerline{\textsc{Lecture 2}}

\begin{enumerate}
    \item Конструкция суперконцентратора.
    \item Матрица смежности графа. Лапласиан графа (хотим полином $x^T M x = \sum\limits_{(i, j) \in E} (x_i - x_j)^2$).
    \item Свойства матрицы смежности:
        \begin{itemize}
            \item симм., следовательно есть собственный базис над $\mathbb{R}$;
            \item $(1, \dots, 1)$~--- максимальный собственный вектор.
            \item $\lambda_k = 1$ тогда и только тогда, когда есть $k$ комп. связности.
            \item $\lambda_n = -1$ тогда и только тогда, когда есть двудольная компонента.
            \item $\sum \lambda_i$~--- число петель.
        \end{itemize}
    \item Собственные числа, как оптимизационная задача. $\lambda_2 = \max\limits_{...} \frac{|x^T A x|}{\norm{x}^2}$
    \item Cheeger's inequality. $\frac{1 - \alpha}{2} \le \varphi(S) \le \sqrt{2 (1 - \alpha)}$
        \begin{itemize}
            \item Док-во. Рассмотрим $x = |\bar{S}| 1_S - |S| 1_{\bar{S}}$.
        \end{itemize}
\end{enumerate}

\centerline{\textsc{Lecture 3 (video)}}

\begin{enumerate}
    \item Mixing Lemma.
    \item Нижние оценки на собсвенные числа для графов константной степени.
    \item Существование экспандеров (вероятностная кострукция).
\end{enumerate}

\centerline{\textsc{Lecture 4 (video)}}

\begin{enumerate}
    \item Существование экспандеров (вероятностная кострукция, продолжение).
    \item Zig-zag и сбалансированное подстановочное произведение.
    \item Явная конструкция экспандеров.
\end{enumerate}

\centerline{\textsc{Lecture 5 (video)}}

\begin{enumerate}
    \item Сильно-явная конструкция экспандеров.
    \item Zig-zag произведение, оценка на собственные числа. $\alpha + 2 \beta + \alpha \beta$.
    \item Cбалансированное подстановочное произведение, оценка на собственные числа. $\left(1 -
        \frac{\varepsilon \delta^2}{24}\right)$.
\end{enumerate}


\centerline{\textsc{Lecture 6}}

\begin{enumerate}
    \item Случайные блуждания и задача $\UPATH$.
    \item Случайные блуждания и расстояние до равномерного распределения.
    \item Понижение ошибки в классах $\RP$ и $\BPP$ при помощи случайного блуждания.
    \item Графы Кэли.
\end{enumerate}
