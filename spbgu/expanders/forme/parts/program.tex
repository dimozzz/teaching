\centerline{\textsc{Lecture 1}}

\begin{enumerate}
    \item Зачем нужны экспандеры? Задача о суперконцентраторе. Задача о понижении ошибки в
        $\RP$. Случайные блуждания.
    \item Двудольные и общие экспандеры.
    \item Magical graphs (параметры).
    \item Хранение множества $S \subseteq [n]$ размера $n$ с запросом в $1$ бит.
    \item Понижение ошибки в $\RP$ без доп. случайных битов. Явные и сильно явные конструции.
\end{enumerate}


\centerline{\textsc{Lecture 2}}

\begin{enumerate}
    \item Конструкция суперконцентратора.
    \item Матрица смежности графа. Лапласиан графа (хотим полином $x^T M x = \sum\limits_{(i, j) \in E} (x_i - x_j)^2$).
    \item Свойства матрицы смежности:
        \begin{itemize}
            \item симм., следовательно есть собственный базис над $\mathbb{R}$;
            \item $(1, \dots, 1)$~--- максимальный собственный вектор.
            \item $\lambda_k = 1$ тогда и только тогда, когда есть $k$ комп. связности.
            \item $\lambda_n = -1$ тогда и только тогда, когда есть двудольная компонента.
            \item $\sum \lambda_i$~--- число петель.
        \end{itemize}
    \item Собственные числа, как оптимизационная задача. $\lambda_2 = \max\limits_{...} \frac{|x^T A x|}{\norm{x}^2}$
    \item Cheeger's inequality. $\frac{1 - \alpha}{2} \le \varphi(S) \le \sqrt{2 (1 - \alpha)}$
        \begin{itemize}
            \item Док-во. Рассмотрим $x = |\bar{S}| 1_S - |S| 1_{\bar{S}}$.
        \end{itemize}
\end{enumerate}

\centerline{\textsc{Lecture 3 (video)}}

\begin{enumerate}
    \item Mixing Lemma.
    \item Нижние оценки на собсвенные числа для графов константной степени.
    \item Существование экспандеров (вероятностная кострукция).
\end{enumerate}

\centerline{\textsc{Lecture 4 (video)}}

\begin{enumerate}
    \item Существование экспандеров (вероятностная кострукция, продолжение).
    \item Zig-zag и сбалансированное подстановочное произведение.
    \item Явная конструкция экспандеров.
\end{enumerate}

\centerline{\textsc{Lecture 5 (video)}}

\begin{enumerate}
    \item Сильно-явная конструкция экспандеров.
    \item Zig-zag произведение, оценка на собственные числа. $\alpha + 2 \beta + \alpha \beta$.
    \item Cбалансированное подстановочное произведение, оценка на собственные числа. $\left(1 -
        \frac{\varepsilon \delta^2}{24}\right)$.
\end{enumerate}


\centerline{\textsc{Lecture 6}}

\begin{enumerate}
    \item Случайные блуждания и задача $\UPATH$.
    \item Случайные блуждания и расстояние до равномерного распределения.
\end{enumerate}

\centerline{\textsc{Lecture 7}}

\begin{enumerate}
    \item Лапласиан графа.
    \item Второе собственное число произвольного графа.
    \item Понижение ошибки в классах $\RP$ и $\BPP$ при помощи случайного блуждания.
    \item Графы Кэли.
\end{enumerate}

\centerline{\textsc{Lecture 8}}

\begin{enumerate}
    \item Собственные числа графов Кэли.
    \item Примеры: куб, цикл.
    \item Семплеры:
        \begin{enumerate}
            \item оценка $q = O\left(\frac{1}{ \varepsilon^2}\log \frac{1}{\delta}\right)$, $r = n +
                \log \frac{1}{\delta} - \log q$;
            \item наивный (усредняющий) $q = O\left( \frac{1}{\varepsilon^2}
                \log\frac{1}{\varepsilon^2}\right)$, $r = n q$;
            \item попарно-независимый $q = O\left( \frac{1}{\delta \varepsilon^2}\right)$, $r = 2n$;
            \item медиана (блуждание по экспандеру);
            \item на экспандере (среднее по соседям, булев).
        \end{enumerate}
\end{enumerate}

\centerline{\textsc{Lecture 9}}

\begin{enumerate}
    \item Собственные числа графов Кэли.
    \item Семплеры. Продолжение.
    \item Хиттеры.
    \item Коды, исправляющие ошибки.
    \item Граница Хемминга.
    \item Жадный алгоритм, граница Гилберта.
    \item Код Хемминга.
    \item Задача о шляпах.
\end{enumerate}


\centerline{\textsc{Lecture 10}}

\begin{enumerate}
    \item Вероятностная конструкция кодов.
    \item Линейные коды.
    \item Жадный алгоритм для линейных кодов.
    \item Граница Синглтона.
    \item Код Рида--Соломона.
\end{enumerate}

\centerline{\textsc{Lecture 11}}

\begin{enumerate}
    \item Ошибки и стирания, еще раз о кодах Рида--Соломона.
    \item Композиция кодов. Каскадные коды.
    \item Расстояние каскадного кода. Наивный алгоритм декодирования.
    \item Каскадный код со стираниями. Алгоритм декодирования.
    \item Код Адамара.
\end{enumerate}

\centerline{\textsc{Lecture 12}}

\begin{enumerate}
    \item Если хочется побыстрее: коды Земора.
    \item Оценка Плоткина.
    \item Усиление оценки Синглтона.
    \item Коды БЧХ.
\end{enumerate}


\centerline{\textsc{Lecture 13}}

\begin{enumerate}
    \item Код Адамара 2.
    \item Вероятностное декодирование кода Адамара.
    \item Списочное декодирование.
\end{enumerate}