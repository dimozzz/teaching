\documentclass[a4paper, 12pt]{article}
% math symbols
\usepackage{amssymb}
\usepackage{amsmath}
\usepackage{mathrsfs}
\usepackage{mathseries}


\usepackage[margin = 2cm]{geometry}

\tolerance = 1000
\emergencystretch = 0.74cm



\pagestyle{empty}
\parindent = 0mm

\renewcommand{\coursetitle}{DM/ML}
\setcounter{curtask}{1}


\newlang{\UPATH}{UPATH}

\begin{document}

\setmathstyle{2020}{Экспандеры и коды исправляющие ошибки}{Блиц}

\section*{Часть I}
\begin{enumerate}
    \item Что такое зиг-заг произведение?
    \item Что такое подстановочное прозведение?
    \item У регулярного графа $3$ компоненты связности. Что вы можете сказать о собственных числа матрица
        смежности?
    \item Что означет, что у приведенной матрицы смежности графа есть собственное число $-1$?
    \item Что такое сэмплер?
    \item Как можно задать граф Кэли?
    \item Является ли простой путь графом Кэли?
    \item Приведите пример графа Кэли (кроме куба и цикла).
    \item Что такое $2$-независимое множество?
    \item Что такое лапласиан графа?
    \item Сформулируйте mixing лемму.
    \item Какую задачу решает алгоритм Рейнгольда?
    \item Оцените диаметр экспандера с константной степенью вершин.
    \item Сколько случайных бит вам понадобится для подсчета среднего значения функции $f\colon \{0, 1\}^n
        \to \{0, 1\}$? Ответ обоснуйте.
    \item Опишите результат сбалансированного подстановочного произведения треугольника и ребра.
    \item Оцените минимальную степень графа, у которого второе собственное число приведенной матрицы
        смежности равнялось бы $\frac{1}{100}$.
    \item Дайте определение алгебраического экспандера в терминах лапласиана графа.
    \item Приведите пример суперконцентратора, если разрешается использовать любое число ребер.
\end{enumerate}

\section*{Часть II}
\begin{enumerate}
    \item Существует ли код $\Sigma^{n} \to \Sigma^{n^3}$, исправляющий $25\%$ ошибок?
    \item Существует ли код $\Sigma^{n} \to \Sigma^{n + 2}$, исправляющий $1$ ошибку?
    \item Существует ли код $\Sigma^{n} \to \Sigma^{n^2}$, исправляющий $1\%$ ошибок?
    \item В чем отличие кодов БЧХ и кодов Рида--Соломона?
    \item Как получить границу Хемминга?
    \item Что такое проверочная матрица кода?
    \item Что такое каскадный код?
    \item Что такое списочное декодирование?
    \item Оцените максимальное количество таких булевых векторов в $\mathbb{R}^n$, что никакая пара из
        них не образует острый угол.
    \item Пусть код Рида--Соломона исправляет $e$ ошибок. Сколько <<стираний>> может исправить этот код?
    \item Сколько существует кодовых слов длины $n$ с проверочной матрицей:
        $$
            \left( \begin{array}{ccccc}
                     1 & 0 & 0 & 0 & \dots \\
                     0 & 1 & 0 & 0 & \dots
                   \end{array}
            \right)
        $$
\end{enumerate}


\end{document}

