\documentclass[12pt, fleqn, a4paper]{article}


\usepackage{amsmath}
\usepackage{amssymb}
\usepackage{amsfonts}
\usepackage{textcomp}
\usepackage{amsthm}
\usepackage{mathtools}
\usepackage{xspace}
\usepackage[classfont = bold]{complexity}
\usepackage{fullpage}
\usepackage[russian, english]{babel}
\usepackage[utf8]{inputenc}
\usepackage[
    sorting = ydnt,
    style = alphabetic,
    maxbibnames = 99,
    backend = biber
    ]{biblatex}
\addbibresource{main.bib}



\newtheorem{conjecture}{Conjecture}[section]
\theoremstyle{definition}
\newtheorem{theorem}{Theorem}[section]
\newtheorem*{theorem*}{Theorem}
\newtheorem{lemma}{Lemma}[section]
\newtheorem{corollary}{Corollary}[section]
\newtheorem{proposition}{Proposition}[section]
\newtheorem{fact}{Fact}[section]
\newtheorem{problem}{Problem}[section]
\newtheorem{exercise}{Exercise}[section]
\newtheorem{example}{Example}[section]
\newtheorem{definition}{Definition}[section]
\newtheorem{remark}{Remark}[section]
\newtheorem{algorithm}{Algorithm}[section]


\newcommand{\class}[1]{\mathbf{#1}}
\newcommand{\co}{\mathrm{co}}
\newcommand{\alg}[1]{\mathit{#1}}
\newcommand{\lang}[1]{\mathtt{#1}}


% classes (1)

\newcommand{\DTime}{\class{DTime}}
\newcommand{\RTime}{\class{RTime}}
\newcommand{\UTime}{\class{UTime}}
\newcommand{\NTime}{\class{NTime}}
\newcommand{\BPTime}{\class{BPTime}}


\renewcommand{\P}{\class{P}}
\newcommand{\ZPP}{\class{ZPP}}
\newcommand{\RP}{\class{RP}}
\newcommand{\coRP}{\co\class{RP}}
\newcommand{\UP}{\class{UP}}
\newcommand{\coUP}{\co\class{UP}}
\newcommand{\NP}{\class{NP}}
\newcommand{\coNP}{{\co}\class{NP}}
\newcommand{\BPP}{\class{BPP}}
\newcommand{\SigmaP}[1]{\Sigma^{#1}\class{P}}
\newcommand{\PH}{\class{PH}}
\newcommand{\PP}{\class{PP}}
\newcommand{\IP}{\class{IP}}
\newcommand{\OP}{\class{\oplus P}}


\newcommand{\EXP}{\class{EXP}}
\newcommand{\MIP}{\class{MIP}}
\newcommand{\NEXP}{\class{NEXP}}
\newcommand{\coNEXP}{{\co}\class{NEXP}}
\newcommand{\MAEXP}{\class{MA}_\class{EXP}}


% classes (2)

\newcommand{\Ppoly}{\class{P}/\class{poly}}
\newcommand{\NC}{\class{NC}}


\newcommand{\DSpace}{\class{DSpace}}
\newcommand{\NSpace}{\class{NSpace}}
\newcommand{\PSPACE}{\class{PSPACE}}

\newcommand{\EXPSPACE}{\class{EXPSPACE}}


% algorithms and proof systems


\newcommand{\DPLL}{\alg{DPLL}}
\newcommand{\OBDD}{\alg{OBDD}}
\newcommand{\pOBDD}{\pi\text{-}\alg{OBDD}}
\newcommand{\DPLLL}{\alg{DPLL}_{lin}}
\newcommand{\ResL}{\alg{Res}_{lin}}
\newcommand{\SemL}{\alg{Sem}_{lin}}



% languages


\newcommand{\SAT}{\lang{SAT}}
\newcommand{\GNI}{\lang{GNI}}
\newcommand{\MAJSAT}{\lang{MAJ}\text{-}\lang{SAT}}
\newcommand{\QBF}{\lang{QBF}}



% other

\newcommand{\poly}{\mathrm{poly}}
\newcommand{\Nat}{\mathbb{N}}
\newcommand{\bool}{\{0, 1\}}

\newcommand{\Img}{\mathop{\mathrm{Im}}}

\DeclareMathOperator*{\supp}{supp}
\DeclareMathOperator*{\Exp}{E}
\DeclareMathOperator*{\rk}{rk}




%%% Local Variables:
%%% mode: latex
%%% TeX-master: t
%%% End:


\begin{document}

	\setcounter{curtask}{9}

\mytitle{2 (на 3.10)}

\begin{task}
    Докажите, что множество всех рациональных чисел меньших $\pi$ разрешимо.
\end{task}

\begin{task}
    Существует ли алгоритм, проверяющий, работает ли данная программа
    полиномиальное время?
\end{task}

\begin{task}
    Приведите пример двух непересекающихся неперечислимых множеств.
\end{task}

\begin{task}
    Докажите, что для каждой вычислимой функции $f$ найдется
    псевдообратная вычислимая функция $g$. А именно, $g$ определена на
    множестве значений $f$, и для всех $x$ из области определения $f$
    выполняется $f(g(f(x))) = f(x)$.
\end{task}

\begin{task}
    Приведите пример неразрешимого множества $A \subseteq \Nat \times \Nat$,
    такого, что все его горизонтальные и вертикальные сечения
    разрешимы (т.е. для любого $x$ разрешимы $A \cap \{\{x\} \times \Nat\}$
    и $A \cap \{\Nat \times \{x\}\}$)
\end{task}

\begin{task}
    Докажите, что существует язык, который можно распознать с памятью $2^n$ ($n$~---
    длина слова), но нельзя с памятью $n$. (подсказка: диагонализация)
\end{task}

\end{document}




\newlang{\UPATH}{UPATH}

\begin{document}

\setmathstyle{2020}{Экспандеры и коды исправляющие ошибки}{Блиц}

\section*{Часть I}
\begin{enumerate}
    \item Что такое зиг-заг произведение?
    \item Что такое подстановочное прозведение?
    \item У регулярного графа $3$ компоненты связности. Что вы можете сказать о собственных числа матрица
        смежности?
    \item Что означет, что у приведенной матрицы смежности графа есть собственное число $-1$?
    \item Что такое сэмплер?
    \item Как можно задать граф Кэли?
    \item Является ли простой путь графом Кэли?
    \item Приведите пример графа Кэли (кроме куба и цикла).
    \item Что такое $2$-независимое множество?
    \item Что такое лапласиан графа?
    \item Сформулируйте mixing лемму.
    \item Какую задачу решает алгоритм Рейнгольда?
    \item Оцените диаметр экспандера с константной степенью вершин.
    \item Сколько случайных бит вам понадобится для подсчета среднего значения функции $f\colon \{0, 1\}^n
        \to \{0, 1\}$? Ответ обоснуйте.
    \item Опишите результат сбалансированного подстановочного произведения треугольника и ребра.
    \item Оцените минимальную степень графа, у которого второе собственное число приведенной матрицы
        смежности равнялось бы $\frac{1}{100}$.
    \item Дайте определение алгебраического экспандера в терминах лапласиана графа.
    \item Приведите пример суперконцентратора, если разрешается использовать любое число ребер.
\end{enumerate}

\section*{Часть II}
\begin{enumerate}
    \item Существует ли код $\Sigma^{n} \to \Sigma^{n^3}$, исправляющий $25\%$ ошибок?
    \item Существует ли код $\Sigma^{n} \to \Sigma^{n + 2}$, исправляющий $1$ ошибку?
    \item Существует ли код $\Sigma^{n} \to \Sigma^{n^2}$, исправляющий $1\%$ ошибок?
    \item В чем отличие кодов БЧХ и кодов Рида--Соломона?
    \item Как получить границу Хемминга?
    \item Что такое проверочная матрица кода?
    \item Что такое каскадный код?
    \item Что такое списочное декодирование?
    \item Оцените максимальное количество таких булевых векторов в $\mathbb{R}^n$, что никакая пара из
        них не образует острый угол.
    \item Пусть код Рида--Соломона исправляет $e$ ошибок. Сколько <<стираний>> может исправить этот код?
    \item Сколько существует кодовых слов длины $n$ с проверочной матрицей:
        $$
            \left( \begin{array}{ccccc}
                     1 & 0 & 0 & 0 & \dots \\
                     0 & 1 & 0 & 0 & \dots
                   \end{array}
            \right)
        $$
\end{enumerate}


\end{document}

