\documentclass[a4paper, 12pt]{article}
% math symbols
\usepackage{amssymb}
\usepackage{amsmath}
\usepackage{mathrsfs}
\usepackage{mathseries}


\usepackage[margin = 2cm]{geometry}

\tolerance = 1000
\emergencystretch = 0.74cm



\pagestyle{empty}
\parindent = 0mm

\renewcommand{\coursetitle}{DM/ML}
\setcounter{curtask}{1}

\newcommand{\norm}[1]{\left\lVert#1\right\rVert}
\newlang{\UPATH}{UPATH}

\begin{document}

\setmathstyle{2020}{Экспандеры и коды исправляющие ошибки}{Вопросы к экзамену}

\section*{Часть I}
\begin{enumerate}
    \item Существование двудольных экспандеров. Конструкция супеконцентраторов.
    \item Хранение множества $S \subseteq [n]$ размера $n$ с запросом в $1$ бит. Понижение ошибки в $\RP$
        без доп. случайных битов.
    \item Реберное расширение. Матрица смежности графа и ее спектр (примеры использования). Связность
        графа. Cheeger's inequality.
    \item Expander mixing lemma. Нижняя оценка на второе собственное число.
    \item Существование алгебраических экспандеров.
    \item Подстановочные произведения. Зиг-заг произведение. Оценка собственного числа в зиг-заг произведении. Явная
        конструкция алгебраического экспандера.
    \item Оценка собственного числа в сбалансированном подстановочном произведении произведении.
        Алгоритм Рейнгольда.
    \item Случайные блуждания и задача $\UPATH$ (полностью).
    \item Спектр графов Кэли. Примеры (цикл, гиперкуб). Второе собственное число произвольного
        графа. Понижение ошибки в классе $\BPP$ с доп. случайными битами. 
    \item Сэмплеры. Наивный сэмплер. Попарно-независимый сэмплер. $2$-независимое множество. Конструкция
        $2$-независимого множества.
    \item Сэмплер, основанный на медиане усреднений. Случайное блуждание по экспандеру. Булев сэмплер из
        экспандера.
\end{enumerate}

\section*{Часть II}
\begin{enumerate}
    \item Коды исправляющие ошибки. Граница Хемминга. Граница Гилберта. Объем шаров. Случайный
        код. Код Хемминга. Задача о волшебниках.
    \item Линейные коды. Жадный алгоритм. Граница Варшамова--Гилберта. Граница Синглтона. Код
        Рида--Соломона.
    \item Код Рида--Соломона. Декодирование кода Рида--Соломона.
    \item Каскадный код. Каскадный код с кодом Рида--Соломона.
    \item Коды с расстоянием более $\frac{n}{2}$. Оценка Плоткина. Усиление границы Синглтона.
    \item Коды Земора.
    \item Коды БЧХ.
    \item Декодирование списком. Существование списочных кодов для классической границы Хемминга.
	\item Декодирование списком. Связь кодового расстояния и числа исправляемых ошибок для декодирования
        списком длины $(n + 1)$. 
\end{enumerate}


\end{document}

