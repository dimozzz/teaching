\centerline{\textsc{Раздел: экспандеры и их применения}}

\begin{enumerate}
    \item Матрица графа, комбинаторный смысл ее собственных чисел. Алгебраический экспандер. Лемма о
        перемешивании.
    \item Свойство реберного расширения алгебраического экспандера.
    \item Нижняя оценка на второе собственное число.
    \item Существование алгебраических экспандеров: большинство $d$-регулярных графов являются
        экспандерами.
    \item Произведения графов: матричное произведение, тензорное произведение, простой и сбалансированный
        варианты подстановочного произведения, зигзаг-произведение.
    \item Оценка второго собственного числа в графе зигзаг-произведения. Явное построение экспандеров
        (рекурсивная конструкция с зигзаг-произведением).
    \item Оценка второго собственного числа для сбалансированного подстановочного произведения. Вторая
        явная конструкция экспандера (рекурсивная конструкция с подстановочным произведением).
    \item Вычисление спектра для графа аффинной плоскости. Использование графа аффинной плоскости в явных
        конструкциях экспандеров.
    \item Графы Кэли. Спектр графа Кэли для конечных абелевых групп. Примеры. Построение экспандера из
        кодов, исправляющих ошибки. Графы Рамануджана (без доказательства).
    \item Понижение ошибки в $\RP$ и $\BPP$ в полиномиальное число раз без использования
        дополнительных случайных битов. В экспоненциальное число раз с использованием случайных битов.
    \item Алгоритм Рейнголда: решение задачи $\UPATH$ детерминированным алгоритмом с логарифмической
        памятью.
\end{enumerate}


\centerline{\textsc{Раздел: коды исправляющие ошибки}}
\begin{enumerate}
    \item Коды, исправляющие ошибки. Границы Хемминга, Гилберта. Случайные коды. Линейные коды: граница
        Варшамова-Гилберта. Проверочная матрица, код Хемминга.
    \item Оценка Синглетона. Код Рида-Соломона. Алгоритм Берлекэмпа-Велча.
    \item Каскадные коды. Декодирующий алгоритм, позволяющий исправлять $d_1 e_2$ ошибок.
    \item Теорема Форни. Код Форни-Возенкрафта-Юстенсена.
    \item Коды с большими расстояниями. Оценка Плоткина.
    \item Код Адамара и его локальное декодирование. Каскадный код из кода Адамара и Рида-Соломона.
    \item Код Рида-Маллера и его локальное декодирование.
    \item Декодирование списком при выполнении оценки Хемминга. Кодовое расстояние и декодирование
        списком.
    \item Декодирование списком кодов Адамара. Оценка Джонсона. Оценка Элайеса-Бассалыго.
    \item Декодирование списком кода Рида-Соломона. Декодирование списком каскадного года из
        Рида-Соломона и Адамара.
    \item Локальное декодирование списком для кодов Адамара.
    \item Локальное декодирование списком для кодов Рида-Маллера.
    \item Коды, основанные на экспандерах: коды Земора.
\end{enumerate}

\centerline{\textsc{Раздел: немного о дизайнах}}
\begin{enumerate}
    \item Энтропия и ее свойства. Полуаддитивность. Оценка для $\binom{n}{1} + \binom{n}{2} + \cdots +
        \binom{n}{k}$.
    \item Условная энтропия и ее свойства. Обобщенная полуаддитивность. Пересекающиеся множества и графы.
    \item Энтропия и однозначно декодируемые коды.
    \item Дизайны и их свойства. Семейство регулярных подмножеств. Разностные множества и
        дизайны. Разностное множество из квадратичных вычетов.
    \item Конечные геометрии. Конечные проективные и аффинные плоскости, их свойства. Построение
        проективных плоскостей порядка $p^k$.
\end{enumerate}