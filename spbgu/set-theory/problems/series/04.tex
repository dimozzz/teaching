\documentclass[a4paper, 12pt]{article}
% math symbols
\usepackage{amssymb}
\usepackage{amsmath}
\usepackage{mathrsfs}
\usepackage{mathseries}


\usepackage[margin = 2cm]{geometry}

\tolerance = 1000
\emergencystretch = 0.74cm



\pagestyle{empty}
\parindent = 0mm

\renewcommand{\coursetitle}{DM/ML}
\setcounter{curtask}{1}

\setmathstyle{}{Теория множеств}{1 курс}


\begin{document}

\task{
    Выведите принцип возвратной индукции из принципа минимального элемента.
}

\begin{definition*}
    Пусть $X \subseteq \mathbb{N}$ и $X \ne \emptyset$. Будем называть $X$ \deftext{ограниченным}, если
    $\exists n \in \mathbb{N}\, \forall u \in X\, (u \leq n)$. Будем говорить, что $x \in X$ является
    \deftext{наибольшим} в $X$, если $\forall u \in X\, (u \leq x)$. 
\end{definition*}

\task{
    покажите, что если непустое множество натуральных чисел ограничено, то оно содержит наибольший
    элемент, причём такой элемент единственнен.
}

\task{
    Покажите, что для всех $k, m, n \in \mathbb{N}$ верно следующее:
    \begin{enumcyr}
        \item $(k + m) + n = k + (m + n)$;
        \item $m + n = n + m$.
    \end{enumcyr}
}

%\task{
%    Докажите теорему о возвратной рекурсии.
%}

\task{
    Покажите, что для любых $X$, $Y$ и $Z$ верно следующее:
    \begin{enumcyr}
        \item $\abs{X \times Y} = \abs{Y \times X}$;
        \item $\abs{(X \times Y) \times Z} = \abs{X \times (Y \times Z)}$;
        \item если $\abs{X} = \abs{Y}$, то $\abs{X \times Z} = \abs{Y \times Z}$.
    \end{enumcyr}
}


\task{
    Пусть $\abs{X} \leq \abs{Y}$ и $X \ne \emptyset$. Покажите, что существует сюрьекция из $Y$ на $X$.
}

\task{
    Пусть существует сюрьекция из $Y$ на $X$. Покажите, что $\abs{X} \leq \abs{Y}$.
}

\breakline

\libproblem[6][02]{set-theory}{relation-func-crit}
\task[7][02]{
    Если $f\colon X \rightarrow Y$ и $g\colon Y \rightarrow Z$, то $f \circ g\colon X \rightarrow Z$,
    причём $(f \circ g)(x) = g(f(x))$ для всех $x \in X$.
}
\libproblem[8][02]{set-theory}{reduce-left-right}
\libproblem[9][02]{set-theory}{inverse-left-right}
\libproblem[10][02]{set-theory}{choice-equiv-c-prime}

\task[4][03]{
    Покажите, что:
    \begin{enumcyr}
        \item для любого $n \in \mathbb{N}$ не существует инъекции из $n + 1$ в $n$;
        \item для любых $n, m \in \mathbb{N}$, если $m < n$, то не существует инъекции из $n$ в $m$.
        \item для любых $n, m \in \mathbb{N}$, если $n \ne m$, то не существует биекции между $n$ и $m$.
    \end{enumcyr}
}


\task[7][03]{
    Покажите, что множество конечно тогда и только тогда, когда оно $\mathrm{T}$-конечно.
}

\task[8][03]{
    Используя лишь обычную индукцию, докажите, что для любого $n \in \mathbb{N}$:
    $$
        \forall X\, ((X \subseteq n \wedge X \ne \emptyset) \rightarrow \funccplx{Min}(X) \ne \emptyset).
    $$
    Выведите отсюда принцип минимального элемента.
}



\end{document}
