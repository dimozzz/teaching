\documentclass[12pt, fleqn, a4paper]{article}


\usepackage{amsmath}
\usepackage{amssymb}
\usepackage{amsfonts}
\usepackage{textcomp}
\usepackage{amsthm}
\usepackage{mathtools}
\usepackage{xspace}
\usepackage[classfont = bold]{complexity}
\usepackage{fullpage}
\usepackage[russian, english]{babel}
\usepackage[utf8]{inputenc}
\usepackage[
    sorting = ydnt,
    style = alphabetic,
    maxbibnames = 99,
    backend = biber
    ]{biblatex}
\addbibresource{main.bib}



\newtheorem{conjecture}{Conjecture}[section]
\theoremstyle{definition}
\newtheorem{theorem}{Theorem}[section]
\newtheorem*{theorem*}{Theorem}
\newtheorem{lemma}{Lemma}[section]
\newtheorem{corollary}{Corollary}[section]
\newtheorem{proposition}{Proposition}[section]
\newtheorem{fact}{Fact}[section]
\newtheorem{problem}{Problem}[section]
\newtheorem{exercise}{Exercise}[section]
\newtheorem{example}{Example}[section]
\newtheorem{definition}{Definition}[section]
\newtheorem{remark}{Remark}[section]
\newtheorem{algorithm}{Algorithm}[section]


\newcommand{\class}[1]{\mathbf{#1}}
\newcommand{\co}{\mathrm{co}}
\newcommand{\alg}[1]{\mathit{#1}}
\newcommand{\lang}[1]{\mathtt{#1}}


% classes (1)

\newcommand{\DTime}{\class{DTime}}
\newcommand{\RTime}{\class{RTime}}
\newcommand{\UTime}{\class{UTime}}
\newcommand{\NTime}{\class{NTime}}
\newcommand{\BPTime}{\class{BPTime}}


\renewcommand{\P}{\class{P}}
\newcommand{\ZPP}{\class{ZPP}}
\newcommand{\RP}{\class{RP}}
\newcommand{\coRP}{\co\class{RP}}
\newcommand{\UP}{\class{UP}}
\newcommand{\coUP}{\co\class{UP}}
\newcommand{\NP}{\class{NP}}
\newcommand{\coNP}{{\co}\class{NP}}
\newcommand{\BPP}{\class{BPP}}
\newcommand{\SigmaP}[1]{\Sigma^{#1}\class{P}}
\newcommand{\PH}{\class{PH}}
\newcommand{\PP}{\class{PP}}
\newcommand{\IP}{\class{IP}}
\newcommand{\OP}{\class{\oplus P}}


\newcommand{\EXP}{\class{EXP}}
\newcommand{\MIP}{\class{MIP}}
\newcommand{\NEXP}{\class{NEXP}}
\newcommand{\coNEXP}{{\co}\class{NEXP}}
\newcommand{\MAEXP}{\class{MA}_\class{EXP}}


% classes (2)

\newcommand{\Ppoly}{\class{P}/\class{poly}}
\newcommand{\NC}{\class{NC}}


\newcommand{\DSpace}{\class{DSpace}}
\newcommand{\NSpace}{\class{NSpace}}
\newcommand{\PSPACE}{\class{PSPACE}}

\newcommand{\EXPSPACE}{\class{EXPSPACE}}


% algorithms and proof systems


\newcommand{\DPLL}{\alg{DPLL}}
\newcommand{\OBDD}{\alg{OBDD}}
\newcommand{\pOBDD}{\pi\text{-}\alg{OBDD}}
\newcommand{\DPLLL}{\alg{DPLL}_{lin}}
\newcommand{\ResL}{\alg{Res}_{lin}}
\newcommand{\SemL}{\alg{Sem}_{lin}}



% languages


\newcommand{\SAT}{\lang{SAT}}
\newcommand{\GNI}{\lang{GNI}}
\newcommand{\MAJSAT}{\lang{MAJ}\text{-}\lang{SAT}}
\newcommand{\QBF}{\lang{QBF}}



% other

\newcommand{\poly}{\mathrm{poly}}
\newcommand{\Nat}{\mathbb{N}}
\newcommand{\bool}{\{0, 1\}}

\newcommand{\Img}{\mathop{\mathrm{Im}}}

\DeclareMathOperator*{\supp}{supp}
\DeclareMathOperator*{\Exp}{E}
\DeclareMathOperator*{\rk}{rk}




%%% Local Variables:
%%% mode: latex
%%% TeX-master: t
%%% End:


\begin{document}

	\setcounter{curtask}{9}

\mytitle{2 (на 3.10)}

\begin{task}
    Докажите, что множество всех рациональных чисел меньших $\pi$ разрешимо.
\end{task}

\begin{task}
    Существует ли алгоритм, проверяющий, работает ли данная программа
    полиномиальное время?
\end{task}

\begin{task}
    Приведите пример двух непересекающихся неперечислимых множеств.
\end{task}

\begin{task}
    Докажите, что для каждой вычислимой функции $f$ найдется
    псевдообратная вычислимая функция $g$. А именно, $g$ определена на
    множестве значений $f$, и для всех $x$ из области определения $f$
    выполняется $f(g(f(x))) = f(x)$.
\end{task}

\begin{task}
    Приведите пример неразрешимого множества $A \subseteq \Nat \times \Nat$,
    такого, что все его горизонтальные и вертикальные сечения
    разрешимы (т.е. для любого $x$ разрешимы $A \cap \{\{x\} \times \Nat\}$
    и $A \cap \{\Nat \times \{x\}\}$)
\end{task}

\begin{task}
    Докажите, что существует язык, который можно распознать с памятью $2^n$ ($n$~---
    длина слова), но нельзя с памятью $n$. (подсказка: диагонализация)
\end{task}

\end{document}



\setmathstyle{}{Теория множеств}{1 курс}


\begin{document}

\task{
    Выведите принцип возвратной индукции из принципа минимального элемента.
}

\begin{definition*}
    Пусть $X \subseteq \mathbb{N}$ и $X \ne \emptyset$. Будем называть $X$ \deftext{ограниченным}, если
    $\exists n \in \mathbb{N}\, \forall u \in X\, (u \leq n)$. Будем говорить, что $x \in X$ является
    \deftext{наибольшим} в $X$, если $\forall u \in X\, (u \leq x)$. 
\end{definition*}

\task{
    покажите, что если непустое множество натуральных чисел ограничено, то оно содержит наибольший
    элемент, причём такой элемент единственнен.
}

\task{
    Покажите, что для всех $k, m, n \in \mathbb{N}$ верно следующее:
    \begin{enumcyr}
        \item $(k + m) + n = k + (m + n)$;
        \item $m + n = n + m$.
    \end{enumcyr}
}

%\task{
%    Докажите теорему о возвратной рекурсии.
%}

\task{
    Покажите, что для любых $X$, $Y$ и $Z$ верно следующее:
    \begin{enumcyr}
        \item $\abs{X \times Y} = \abs{Y \times X}$;
        \item $\abs{(X \times Y) \times Z} = \abs{X \times (Y \times Z)}$;
        \item если $\abs{X} = \abs{Y}$, то $\abs{X \times Z} = \abs{Y \times Z}$.
    \end{enumcyr}
}


\task{
    Пусть $\abs{X} \leq \abs{Y}$ и $X \ne \emptyset$. Покажите, что существует сюрьекция из $Y$ на $X$.
}

\task{
    Пусть существует сюрьекция из $Y$ на $X$. Покажите, что $\abs{X} \leq \abs{Y}$.
}

\breakline

\libproblem[6][02]{set-theory}{relation-func-crit}
\task[7][02]{
    Если $f\colon X \rightarrow Y$ и $g\colon Y \rightarrow Z$, то $f \circ g\colon X \rightarrow Z$,
    причём $(f \circ g)(x) = g(f(x))$ для всех $x \in X$.
}
\libproblem[8][02]{set-theory}{reduce-left-right}
\libproblem[9][02]{set-theory}{inverse-left-right}
\libproblem[10][02]{set-theory}{choice-equiv-c-prime}

\task[4][03]{
    Покажите, что:
    \begin{enumcyr}
        \item для любого $n \in \mathbb{N}$ не существует инъекции из $n + 1$ в $n$;
        \item для любых $n, m \in \mathbb{N}$, если $m < n$, то не существует инъекции из $n$ в $m$.
        \item для любых $n, m \in \mathbb{N}$, если $n \ne m$, то не существует биекции между $n$ и $m$.
    \end{enumcyr}
}


\task[7][03]{
    Покажите, что множество конечно тогда и только тогда, когда оно $\mathrm{T}$-конечно.
}

\task[8][03]{
    Используя лишь обычную индукцию, докажите, что для любого $n \in \mathbb{N}$:
    $$
        \forall X\, ((X \subseteq n \wedge X \ne \emptyset) \rightarrow \funccplx{Min}(X) \ne \emptyset).
    $$
    Выведите отсюда принцип минимального элемента.
}



\end{document}
