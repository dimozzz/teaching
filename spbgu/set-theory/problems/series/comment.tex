\documentclass[12pt, fleqn, a4paper]{article}


\usepackage{amsmath}
\usepackage{amssymb}
\usepackage{amsfonts}
\usepackage{textcomp}
\usepackage{amsthm}
\usepackage{mathtools}
\usepackage{xspace}
\usepackage[classfont = bold]{complexity}
\usepackage{fullpage}
\usepackage[russian, english]{babel}
\usepackage[utf8]{inputenc}
\usepackage[
    sorting = ydnt,
    style = alphabetic,
    maxbibnames = 99,
    backend = biber
    ]{biblatex}
\addbibresource{main.bib}



\newtheorem{conjecture}{Conjecture}[section]
\theoremstyle{definition}
\newtheorem{theorem}{Theorem}[section]
\newtheorem*{theorem*}{Theorem}
\newtheorem{lemma}{Lemma}[section]
\newtheorem{corollary}{Corollary}[section]
\newtheorem{proposition}{Proposition}[section]
\newtheorem{fact}{Fact}[section]
\newtheorem{problem}{Problem}[section]
\newtheorem{exercise}{Exercise}[section]
\newtheorem{example}{Example}[section]
\newtheorem{definition}{Definition}[section]
\newtheorem{remark}{Remark}[section]
\newtheorem{algorithm}{Algorithm}[section]


\newcommand{\class}[1]{\mathbf{#1}}
\newcommand{\co}{\mathrm{co}}
\newcommand{\alg}[1]{\mathit{#1}}
\newcommand{\lang}[1]{\mathtt{#1}}


% classes (1)

\newcommand{\DTime}{\class{DTime}}
\newcommand{\RTime}{\class{RTime}}
\newcommand{\UTime}{\class{UTime}}
\newcommand{\NTime}{\class{NTime}}
\newcommand{\BPTime}{\class{BPTime}}


\renewcommand{\P}{\class{P}}
\newcommand{\ZPP}{\class{ZPP}}
\newcommand{\RP}{\class{RP}}
\newcommand{\coRP}{\co\class{RP}}
\newcommand{\UP}{\class{UP}}
\newcommand{\coUP}{\co\class{UP}}
\newcommand{\NP}{\class{NP}}
\newcommand{\coNP}{{\co}\class{NP}}
\newcommand{\BPP}{\class{BPP}}
\newcommand{\SigmaP}[1]{\Sigma^{#1}\class{P}}
\newcommand{\PH}{\class{PH}}
\newcommand{\PP}{\class{PP}}
\newcommand{\IP}{\class{IP}}
\newcommand{\OP}{\class{\oplus P}}


\newcommand{\EXP}{\class{EXP}}
\newcommand{\MIP}{\class{MIP}}
\newcommand{\NEXP}{\class{NEXP}}
\newcommand{\coNEXP}{{\co}\class{NEXP}}
\newcommand{\MAEXP}{\class{MA}_\class{EXP}}


% classes (2)

\newcommand{\Ppoly}{\class{P}/\class{poly}}
\newcommand{\NC}{\class{NC}}


\newcommand{\DSpace}{\class{DSpace}}
\newcommand{\NSpace}{\class{NSpace}}
\newcommand{\PSPACE}{\class{PSPACE}}

\newcommand{\EXPSPACE}{\class{EXPSPACE}}


% algorithms and proof systems


\newcommand{\DPLL}{\alg{DPLL}}
\newcommand{\OBDD}{\alg{OBDD}}
\newcommand{\pOBDD}{\pi\text{-}\alg{OBDD}}
\newcommand{\DPLLL}{\alg{DPLL}_{lin}}
\newcommand{\ResL}{\alg{Res}_{lin}}
\newcommand{\SemL}{\alg{Sem}_{lin}}



% languages


\newcommand{\SAT}{\lang{SAT}}
\newcommand{\GNI}{\lang{GNI}}
\newcommand{\MAJSAT}{\lang{MAJ}\text{-}\lang{SAT}}
\newcommand{\QBF}{\lang{QBF}}



% other

\newcommand{\poly}{\mathrm{poly}}
\newcommand{\Nat}{\mathbb{N}}
\newcommand{\bool}{\{0, 1\}}

\newcommand{\Img}{\mathop{\mathrm{Im}}}

\DeclareMathOperator*{\supp}{supp}
\DeclareMathOperator*{\Exp}{E}
\DeclareMathOperator*{\rk}{rk}




%%% Local Variables:
%%% mode: latex
%%% TeX-master: t
%%% End:


\begin{document}

	\setcounter{curtask}{9}

\mytitle{2 (на 3.10)}

\begin{task}
    Докажите, что множество всех рациональных чисел меньших $\pi$ разрешимо.
\end{task}

\begin{task}
    Существует ли алгоритм, проверяющий, работает ли данная программа
    полиномиальное время?
\end{task}

\begin{task}
    Приведите пример двух непересекающихся неперечислимых множеств.
\end{task}

\begin{task}
    Докажите, что для каждой вычислимой функции $f$ найдется
    псевдообратная вычислимая функция $g$. А именно, $g$ определена на
    множестве значений $f$, и для всех $x$ из области определения $f$
    выполняется $f(g(f(x))) = f(x)$.
\end{task}

\begin{task}
    Приведите пример неразрешимого множества $A \subseteq \Nat \times \Nat$,
    такого, что все его горизонтальные и вертикальные сечения
    разрешимы (т.е. для любого $x$ разрешимы $A \cap \{\{x\} \times \Nat\}$
    и $A \cap \{\Nat \times \{x\}\}$)
\end{task}

\begin{task}
    Докажите, что существует язык, который можно распознать с памятью $2^n$ ($n$~---
    длина слова), но нельзя с памятью $n$. (подсказка: диагонализация)
\end{task}

\end{document}



\setmathstyle{}{Теория множеств}{1 курс}


\begin{document}


\task[2][03]{
    Покажите, что для любых $m, n \in \mathbb{N}$,
    $$
        n \subsetneq m \to n < m.
    $$
}

Докажем следующее утверждение, из которого будет следовать наше, по индукции:
$$
    \forall x \in \mathbb{N}, m < x \to m + 1 < x \lor m + 1 = x.
$$

\textbf{База}: $m < 0 \to m + 1 < 0 \lor m + 1 = 0$. Никакое $m$ не меньше нуля (см. слайды),
следовательно верно. 

\textbf{Переход}. Пусть $m < n \to m + 1 < n \lor m + 1 = n$. Мы хотим показать:
$$
    m < n + 1 \to m + 1 < n + 1 \lor m + 1 = n + 1.
$$
    
Пусть $m < n + 1$, тогда выполнено $m < n \lor m = n$ (см. слайды). В первом случае по предположению
индукции $m + 1 \in n + 1 \lor m + 1 = n + 1$, а во стором $m + 1 = n \cup \{n\} = n + 1$ и, как
следствие $m + 1 \in n + 1 \lor m + 1 = n + 1$.

\vspace{0.3cm}
Пользуясь этим утверждением мы можем доказать искомое утверждение:
$\forall x, x \subsetneq m \to x < m$.

\textbf{База:} $0 < m$ верно для любого $m$ (см. слайды).
\textbf{Переход:} пусть $n + 1 \subsetneq m$, тогда $n \subsetneq m$ и, по предположению: $n \in m$, но,
по доказанному выше это означает, что $n + 1 < m \lor n + 1 = m$, при этом вторая часть неверна, так как
$n + 1 \subsetneq m$, и следовательно $n + 1 < m$.


\begin{proposition*}
    Существует наименьшее индуктивное множество.
\end{proposition*}

Данный факт нам поможет реализовать наивная идея: давайте возьмем <<пересечение всех индуктивных
множеств>>. К сожалению, мы не можем взять совокупность объектов пересечь их и заявить, что у на
получилось множество. Однако, в данном случае нам помогут аксиомы выделения и бесконечности.

Пусть $X$~--- какое-то индуктивное множество, которое существует по аксиоме бесконечности. Определим
множество $Y$ таким образом:
$$
    Y \coloneqq \{x \in X \mid \forall z,\ \mathrm{Ind}(z) \to x \in z\}.
$$

Нетрудно понять, что если $Y$ индуктивное, что оно минимальное, поскольку любой его элемент обязан
принадлежать любому другому индуктивному. Осталось понять, что $Y$ индуктивное. Заметим, что $\emptyset
\in Y$, поскольку $\emptyset$ входит в любое индуктивное. Также заметим, что если какое-то множество $s$
входит в любое индуктивное, что $s \cup \{s\}$ также входит в любое индуктивное, следовательно $Y$
является индуктивным.


\begin{proposition*}
    Если $\alpha$~--- ординал, и $X \in \alpha$, то $X$~--- ординал.
\end{proposition*}

\begin{proof}
    Пусть $v \in X$ и $u \in v$. Тогда $v \in \alpha$, а следовательно и $u \in \alpha$, т.к $\alpha$~---
    транзитивное множество.

    Но, заметим, что $\in_{\alpha}$~--- это полный порядок. При этом $v \le X$ и $u \le v$, а
    следовательно $u \le X$ (т.к. отношение порядка транзитивно). А следовательно $u \in X$ по
    определению порядка.
\end{proof}


\end{document}
