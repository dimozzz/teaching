\documentclass[a4paper, 12pt]{article}
% math symbols
\usepackage{amssymb}
\usepackage{amsmath}
\usepackage{mathrsfs}
\usepackage{mathseries}


\usepackage[margin = 2cm]{geometry}

\tolerance = 1000
\emergencystretch = 0.74cm



\pagestyle{empty}
\parindent = 0mm

\renewcommand{\coursetitle}{DM/ML}
\setcounter{curtask}{1}

\setmathstyle{}{Теория множеств}{1 курс}


\begin{document}


\task[2][03]{
    Покажите, что для любых $m, n \in \mathbb{N}$,
    $$
        n \subsetneq m \to n < m.
    $$
}

Докажем следующее утверждение, из которого будет следовать наше, по индукции:
$$
    \forall x \in \mathbb{N}, m < x \to m + 1 < x \lor m + 1 = x.
$$

\textbf{База}: $m < 0 \to m + 1 < 0 \lor m + 1 = 0$. Никакое $m$ не меньше нуля (см. слайды),
следовательно верно. 

\textbf{Переход}. Пусть $m < n \to m + 1 < n \lor m + 1 = n$. Мы хотим показать:
$$
    m < n + 1 \to m + 1 < n + 1 \lor m + 1 = n + 1.
$$
    
Пусть $m < n + 1$, тогда выполнено $m < n \lor m = n$ (см. слайды). В первом случае по предположению
индукции $m + 1 \in n + 1 \lor m + 1 = n + 1$, а во стором $m + 1 = n \cup \{n\} = n + 1$ и, как
следствие $m + 1 \in n + 1 \lor m + 1 = n + 1$.

\vspace{0.3cm}
Пользуясь этим утверждением мы можем доказать искомое утверждение:
$\forall x, x \subsetneq m \to x < m$.

\textbf{База:} $0 < m$ верно для любого $m$ (см. слайды).
\textbf{Переход:} пусть $n + 1 \subsetneq m$, тогда $n \subsetneq m$ и, по предположению: $n \in m$, но,
по доказанному выше это означает, что $n + 1 < m \lor n + 1 = m$, при этом вторая часть неверна, так как
$n + 1 \subsetneq m$, и следовательно $n + 1 < m$.


\begin{proposition*}
    Существует наименьшее индуктивное множество.
\end{proposition*}

Данный факт нам поможет реализовать наивная идея: давайте возьмем <<пересечение всех индуктивных
множеств>>. К сожалению, мы не можем взять совокупность объектов пересечь их и заявить, что у на
получилось множество. Однако, в данном случае нам помогут аксиомы выделения и бесконечности.

Пусть $X$~--- какое-то индуктивное множество, которое существует по аксиоме бесконечности. Определим
множество $Y$ таким образом:
$$
    Y \coloneqq \{x \in X \mid \forall z,\ \mathrm{Ind}(z) \to x \in z\}.
$$

Нетрудно понять, что если $Y$ индуктивное, что оно минимальное, поскольку любой его элемент обязан
принадлежать любому другому индуктивному. Осталось понять, что $Y$ индуктивное. Заметим, что $\emptyset
\in Y$, поскольку $\emptyset$ входит в любое индуктивное. Также заметим, что если какое-то множество $s$
входит в любое индуктивное, что $s \cup \{s\}$ также входит в любое индуктивное, следовательно $Y$
является индуктивным.

\end{document}
