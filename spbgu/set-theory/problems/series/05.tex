\documentclass[a4paper, 12pt]{article}
% math symbols
\usepackage{amssymb}
\usepackage{amsmath}
\usepackage{mathrsfs}
\usepackage{mathseries}


\usepackage[margin = 2cm]{geometry}

\tolerance = 1000
\emergencystretch = 0.74cm



\pagestyle{empty}
\parindent = 0mm

\renewcommand{\coursetitle}{DM/ML}
\setcounter{curtask}{1}

\setmathstyle{}{Теория множеств}{1 курс}


\begin{document}

\task{
    Дайте убедительное (хотя и неформальное) обоснование счётности следующих множеств:
    \begin{enumcyr}
        \item множество всех целых чисел;
        \item множество всех рациональных чисел;
        \item множество всех периодических дробей;
        \item множество всех полиномов с целыми коэффициентами;
        \item множество всех полиномов с рациональными коэффициентами;
        \item множество всех вещественных алгебраических чисел (\hinttext{комментарий:} свободно
            пользоваться базовыми результатами из области алгебры и анализа, включая обычные свойства
            вещественных чисел).
    \end{enumcyr}
}

\libdefinition{sets-d-infinity}


\task{
    Покажите, что если множество конечно, то оно конечно по Дедекинду.
}


\task{
    Пусть $f\colon X \rightarrow \mathbb{R}$ такая функция, что множество:
    $$
        \left\{\sum_{s \in S} f(s) \mid S \subseteq X \text{ и } S \text{ конечно} \right\}
    $$
    ограничено, т.е. существует такое $N \in \mathbb{N}$, что для любого конечного $S \subseteq X$,
    $$
        \abs{\sum_{s \in S} f(s)} \leq N.
    $$
    Покажите, что $\{x \in X \mid f(x) \neq 0\}$ не более чем счётно.
}

\begin{definition*}
    Будем называть множество $X$ \deftext{континуальным}, если оно равномощно $2^{\mathbb{N}}$.
\end{definition*}


В следующих задачах можно пользоваться теоремой.

\begin{theorem*}
    $\mathbb{R}$ равномощно $2^{\mathbb{N}}$.
\end{theorem*}

\task{
    $\mathbb{R} \times \mathbb{R}$, $\mathbb{R} \times \mathbb{R} \times \mathbb{R}$ и т.д. континуальны.
}


\task{
    Множество всех трансцендентных чисел континуально.
}


\task{
    Пусть $X \cup Y = \mathbb{R}$. Тогда хотя бы одно из $X$ и $Y$ континуально.
}





\breakline


\libproblem[10][02]{set-theory}{choice-equiv-c-prime}

\task[4][03]{
    Покажите, что:
    \begin{enumcyr}
        \item для любого $n \in \mathbb{N}$ не существует инъекции из $n + 1$ в $n$;
        \item для любых $n, m \in \mathbb{N}$, если $m < n$, то не существует инъекции из $n$ в $m$.
        \item для любых $n, m \in \mathbb{N}$, если $n \ne m$, то не существует биекции между $n$ и $m$.
    \end{enumcyr}
}


\task[7][03]{
    Покажите, что множество конечно тогда и только тогда, когда оно $\mathrm{T}$-конечно.
}

\task[8][03]{
    Используя лишь обычную индукцию, докажите, что для любого $n \in \mathbb{N}$:
    $$
        \forall X\, ((X \subseteq n \wedge X \ne \emptyset) \rightarrow \funccplx{Min}(X) \ne \emptyset).
    $$
    Выведите отсюда принцип минимального элемента.
}

\task[2][04]{
    Покажите, что если непустое множество натуральных чисел ограничено, то оно содержит наибольший
    элемент, причём такой элемент единственнен.
}

\task[3][04]{
    Покажите, что для всех $k, m, n \in \mathbb{N}$ верно следующее:
    \begin{enumcyr}
        \item $(k + m) + n = k + (m + n)$;
        \item $m + n = n + m$.
    \end{enumcyr}
}

\task[4][04]{
    Покажите, что для любых $X$, $Y$ и $Z$ верно следующее:
    \begin{enumcyr}
        \item $\abs{X \times Y} = \abs{Y \times X}$;
        \item $\abs{(X \times Y) \times Z} = \abs{X \times (Y \times Z)}$;
        \item если $\abs{X} = \abs{Y}$, то $\abs{X \times Z} = \abs{Y \times Z}$.
    \end{enumcyr}
}


\task[5][04]{
    Пусть $\abs{X} \leq \abs{Y}$ и $X \ne \emptyset$. Покажите, что существует сюрьекция из $Y$ на $X$.
}

\task[6][04]{
    Пусть существует сюрьекция из $Y$ на $X$. Покажите, что $\abs{X} \leq \abs{Y}$.
}



\end{document}
