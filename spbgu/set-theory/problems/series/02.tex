\documentclass[12pt, fleqn, a4paper]{article}


\usepackage{amsmath}
\usepackage{amssymb}
\usepackage{amsfonts}
\usepackage{textcomp}
\usepackage{amsthm}
\usepackage{mathtools}
\usepackage{xspace}
\usepackage[classfont = bold]{complexity}
\usepackage{fullpage}
\usepackage[russian, english]{babel}
\usepackage[utf8]{inputenc}
\usepackage[
    sorting = ydnt,
    style = alphabetic,
    maxbibnames = 99,
    backend = biber
    ]{biblatex}
\addbibresource{main.bib}



\newtheorem{conjecture}{Conjecture}[section]
\theoremstyle{definition}
\newtheorem{theorem}{Theorem}[section]
\newtheorem*{theorem*}{Theorem}
\newtheorem{lemma}{Lemma}[section]
\newtheorem{corollary}{Corollary}[section]
\newtheorem{proposition}{Proposition}[section]
\newtheorem{fact}{Fact}[section]
\newtheorem{problem}{Problem}[section]
\newtheorem{exercise}{Exercise}[section]
\newtheorem{example}{Example}[section]
\newtheorem{definition}{Definition}[section]
\newtheorem{remark}{Remark}[section]
\newtheorem{algorithm}{Algorithm}[section]


\newcommand{\class}[1]{\mathbf{#1}}
\newcommand{\co}{\mathrm{co}}
\newcommand{\alg}[1]{\mathit{#1}}
\newcommand{\lang}[1]{\mathtt{#1}}


% classes (1)

\newcommand{\DTime}{\class{DTime}}
\newcommand{\RTime}{\class{RTime}}
\newcommand{\UTime}{\class{UTime}}
\newcommand{\NTime}{\class{NTime}}
\newcommand{\BPTime}{\class{BPTime}}


\renewcommand{\P}{\class{P}}
\newcommand{\ZPP}{\class{ZPP}}
\newcommand{\RP}{\class{RP}}
\newcommand{\coRP}{\co\class{RP}}
\newcommand{\UP}{\class{UP}}
\newcommand{\coUP}{\co\class{UP}}
\newcommand{\NP}{\class{NP}}
\newcommand{\coNP}{{\co}\class{NP}}
\newcommand{\BPP}{\class{BPP}}
\newcommand{\SigmaP}[1]{\Sigma^{#1}\class{P}}
\newcommand{\PH}{\class{PH}}
\newcommand{\PP}{\class{PP}}
\newcommand{\IP}{\class{IP}}
\newcommand{\OP}{\class{\oplus P}}


\newcommand{\EXP}{\class{EXP}}
\newcommand{\MIP}{\class{MIP}}
\newcommand{\NEXP}{\class{NEXP}}
\newcommand{\coNEXP}{{\co}\class{NEXP}}
\newcommand{\MAEXP}{\class{MA}_\class{EXP}}


% classes (2)

\newcommand{\Ppoly}{\class{P}/\class{poly}}
\newcommand{\NC}{\class{NC}}


\newcommand{\DSpace}{\class{DSpace}}
\newcommand{\NSpace}{\class{NSpace}}
\newcommand{\PSPACE}{\class{PSPACE}}

\newcommand{\EXPSPACE}{\class{EXPSPACE}}


% algorithms and proof systems


\newcommand{\DPLL}{\alg{DPLL}}
\newcommand{\OBDD}{\alg{OBDD}}
\newcommand{\pOBDD}{\pi\text{-}\alg{OBDD}}
\newcommand{\DPLLL}{\alg{DPLL}_{lin}}
\newcommand{\ResL}{\alg{Res}_{lin}}
\newcommand{\SemL}{\alg{Sem}_{lin}}



% languages


\newcommand{\SAT}{\lang{SAT}}
\newcommand{\GNI}{\lang{GNI}}
\newcommand{\MAJSAT}{\lang{MAJ}\text{-}\lang{SAT}}
\newcommand{\QBF}{\lang{QBF}}



% other

\newcommand{\poly}{\mathrm{poly}}
\newcommand{\Nat}{\mathbb{N}}
\newcommand{\bool}{\{0, 1\}}

\newcommand{\Img}{\mathop{\mathrm{Im}}}

\DeclareMathOperator*{\supp}{supp}
\DeclareMathOperator*{\Exp}{E}
\DeclareMathOperator*{\rk}{rk}




%%% Local Variables:
%%% mode: latex
%%% TeX-master: t
%%% End:


\begin{document}

	\setcounter{curtask}{9}

\mytitle{2 (на 3.10)}

\begin{task}
    Докажите, что множество всех рациональных чисел меньших $\pi$ разрешимо.
\end{task}

\begin{task}
    Существует ли алгоритм, проверяющий, работает ли данная программа
    полиномиальное время?
\end{task}

\begin{task}
    Приведите пример двух непересекающихся неперечислимых множеств.
\end{task}

\begin{task}
    Докажите, что для каждой вычислимой функции $f$ найдется
    псевдообратная вычислимая функция $g$. А именно, $g$ определена на
    множестве значений $f$, и для всех $x$ из области определения $f$
    выполняется $f(g(f(x))) = f(x)$.
\end{task}

\begin{task}
    Приведите пример неразрешимого множества $A \subseteq \Nat \times \Nat$,
    такого, что все его горизонтальные и вертикальные сечения
    разрешимы (т.е. для любого $x$ разрешимы $A \cap \{\{x\} \times \Nat\}$
    и $A \cap \{\Nat \times \{x\}\}$)
\end{task}

\begin{task}
    Докажите, что существует язык, который можно распознать с памятью $2^n$ ($n$~---
    длина слова), но нельзя с памятью $n$. (подсказка: диагонализация)
\end{task}

\end{document}



\setmathstyle{}{Теория множеств}{1 курс}


\begin{document}

Определим \deftext{упорядоченную пару} $X_1$ и $X_2$ как
$$
    \left( X_1, X_2 \right) \coloneqq \{\{ X_1 \}, \{ X_1, X_2 \}\}.
$$
        
Заметим, что такие пары обладают следующим свойством:
\begin{equation*}
    (X_1, X_2) = (Y_1, Y_2) \quad \text{тогда и только тогда, когда} \quad
    X_1 = Y_1~\, \text{и}~\, X_2 = Y_2.
\end{equation*}

Далее, для любых $X$ и $Y$ выражение
\begin{align*}
  {X \times Y} \coloneqq&
                          \{ (x, y) \mid x \in X \wedge y \in Y \}\\
  =& \{ u \mid \exists x\, \exists y\, (u = (x, y) \wedge x \in X \wedge y \in Y)\}\\
  =& \{ u \in \mathcal{P} (\mathcal{P} (X \cup Y)) \mid \exists x\, \exists y\, (u = (x, y)) \wedge x \in
     X \wedge y \in Y)\}
\end{align*}
задаёт множество, которое мы будем называть \deftext{прямым} (или \deftext{декартовым})
\deftext{произведением} $X$ и $Y$. Под \deftext{бинарными отношениями на} $X$ будем понимать произвольные
подможества $X \times X$. Бинарное отношение $R \subseteq X \times X$ на $X$ будем называть:
\begin{itemize}
    \item \deftext{рефлексивным}, если $\forall x\, (x \in X \rightarrow xRx)$;
    \item \deftext{иррефлексивным}, если $\neg \exists x\, (x \in X \wedge xRx)$;
    \item \deftext{транзитивным}, если $\forall x\, \forall y\, \forall z\, ((xRy \wedge yRz) \rightarrow
        xRz)$;
    \item \deftext{симметричным}, если $\forall x\, \forall y\, (xRy \rightarrow yRx)$;
    \item \deftext{антисимметричным}, если $\forall x\, \forall y\, ((xRy \wedge yRx) \rightarrow x =
        y)$.
\end{itemize}
Здесь для удобства мы пишем $xRy$ вместо $(x, y) \in R$.

Будем говорить, что отношение $R$ является:
\begin{itemize}
    \item \deftext{предпорядком} на $X$, если $R$ рефлексивно и транзитивно;
    \item \deftext{строгим частичным порядком} на $X$, если $R$ иррефлексивно и транзитивно;
    \item \deftext{частичным порядком} на $X$, если $R$ рефлексивно, антисимметрично и транзитивно;
    \item \deftext{эквивалентностью} на $X$, если $R$ рефлексивно, симметрично и транзитивно.
\end{itemize}
Данные базовые понятия играют весьма важную роль в математике.

\begin{definition*}
    Пусть $\approx$~--- эквивалентность на $X$. Для каждого $x \in X$ под \deftext{классом
        эквивалентности $x$ по $\approx$} понимается множество:
    $$
        [x]_{\approx} \coloneqq \{u \in X \mid x \approx u\}.
    $$

    \deftext{Фактор-множество} $X$ по $\approx$ будем определять как
    $$
        X_{/\approx} \coloneqq \{ [x]_{\approx} \mid x \in X\},
    $$
    т.е. как множество всех классов эквивалентности элементов $X$ по $\approx$.
\end{definition*}


Для отношений эквивалентности имеется простая характеризация. Чтобы её сформулировать, давайте введём
пару вспомогательных понятий.
\begin{definition*}
    Будем называть $Y$ (\deftext{взаимно}, или \deftext{попарно}) \deftext{дизъюнктным}, если оно
    удовлетворяет условию
    $$
        \forall u\, \forall v\, ((u \in Y \wedge v \in Y \wedge u \ne v) \rightarrow u \cap v = \emptyset).
    $$
    Будем говорить, что $Y$ является \deftext{разбиением} $X$, если $Y$ дизъюнктно, $\emptyset \not \in
    Y$ и $\bigcup Y = X$. Элементы всякого разбиения $X$ обязаны быть подмножествами $X$.
\end{definition*}


\libproblem{set-theory}{partition-equivalence}
\libproblem{set-theory}{preorder-equivalence}
\libproblem{set-theory}{partial-order-and-id}
\libproblem{set-theory}{ordered-pair-prod-subset}
\libproblem{set-theory}{relation-basic}
\libproblem{set-theory}{relation-func-crit}


\task{
    Если $f\colon X \rightarrow Y$ и $g\colon Y \rightarrow Z$, то $f \circ g\colon X \rightarrow Z$,
    причём $(f \circ g)(x) = g(f(x))$ для всех $x \in X$.
}

\begin{definition}
    Пусть $f\colon X \rightarrow Y$. Говорят, что на $f$ можно \deftext{сокращать справа}, если для любых
    $g_1\colon Z_1 \rightarrow X$ и $g_2\colon Z_2 \rightarrow X$,
    $$
        g_1 \circ f = g_2 \circ f \quad \text{влечёт} \quad g_1 = g_2.
    $$

    По аналогии, на $f$ можно \deftext{сокращать слева}, если для любых $g_1\colon Y \rightarrow Z_1$ и
    $g_2\colon Y \rightarrow Z_2$,
    $$
        f \circ g_1 = f \circ g_2 \quad \text{влечёт} \quad g_1\ =\ g_2.
    $$
\end{definition}

\libproblem{set-theory}{reduce-left-right}

\begin{definition*}
    Пусть $f\colon X \rightarrow Y$. Под \deftext{правой обратной} к $f$ понимают любую такую $g\colon Y
    \rightarrow X$, что:
    $$
        f \circ g = \mathrm{id}_X.
    $$

    По аналогии, под \deftext{левой обратной} к $f$ понимают любую такую $g\colon Y \rightarrow X$, что:
    $$
        g \circ f = \mathrm{id}_Y.
    $$
\end{definition*}

\libproblem{set-theory}{inverse-left-right}
\libproblem{set-theory}{choice-equiv-c-prime}

\end{document}



%%% Local Variables:
%%% mode: latex
%%% TeX-master: t
%%% End:
