\documentclass[a4paper, 12pt]{article}
% math symbols
\usepackage{amssymb}
\usepackage{amsmath}
\usepackage{mathrsfs}
\usepackage{mathseries}


\usepackage[margin = 2cm]{geometry}

\tolerance = 1000
\emergencystretch = 0.74cm



\pagestyle{empty}
\parindent = 0mm

\renewcommand{\coursetitle}{DM/ML}
\setcounter{curtask}{1}

\setmathstyle{}{Теория множеств}{1 курс}


\begin{document}

\task{
    Покажите, что гомоморфизм $f$ из ч.у.м. $\mathfrak{A}$ в ч.у.м. $\mathfrak{B}$ является изоморфизмом
    тогда и только тогда, когда существует такой гомоморфизм $g$ из $\mathfrak{B}$ в $\mathfrak{A}$, что:
    $$
        f \circ g = \mathrm{id}_A \quad \text{и} \quad g \circ f = \mathrm{id}_B.
    $$
}
    
%\task{
%    Покажите строго, что всякое ч.у.м. с конечным носителем фундировано.
%}

%\task{
%    Пусть $\mathfrak{A} = \avg{A, \leq}$~--- ч.у.м., причём $A$ конечно. Покажите, что для каждого $a \in
%    A$ существует такой максимальный в $\mathfrak{A}$ элемент $a'$, что $a \leq a'$.
%}


\task{
    Пусть $\mathfrak{A} = \avg{A, \leq}$~--- ч.у.м., причём $A$ конечно. Тогда существует линейный
    порядок $\preccurlyeq$ на $A$ такой, что $\leq \subseteq \preccurlyeq$.
}


\task{
    Пусть $\mathfrak{A} = \avg{A, \leq_A}$~--- в.у.м., удовлетворяющий следующим условиям:
    \begin{enumcyr}
        \item в $\mathfrak{A}$ есть наименьший элемент, но нет наибольшего;
        \item в каждом непустом подмножестве $A$, имеющем вернюю грань, есть наибольший элемент.
    \end{enumcyr}
    Покажите, что $\avg{A, \leq_A}$ изоморфно $\avg{\mathbb{N}, \leq}$, где $\leq$~--- естественный
    порядок на $\mathbb{N}$.
}

\begin{definition*}
    Ч.у.м.\ $\mathfrak{A} = \avg{A, \leq_A}$ будем называть \deftext{плотным}, если:
    $$
        \forall a_1 \in A\, \forall a_2 \in A\, (a_1 <_A a_2 \rightarrow
        \exists a_{1.5} \in A\, (a_1 <_A a_{1.5} <_A a_2)).
    $$
    Под ч.у.м. \deftext{без концов} будем понимать произвольное ч.у.м., в котором нет ни наибольшего, ни
    наименьшего элементов.
\end{definition*}


\task{
    Пусть $\mathfrak{A} = \avg{A, \leq_A}$ и $\mathfrak{B} = \avg{B, \leq_B}$~--- два плотных л.у.м.\ без
    концов, причём $A$ и $B$ счётны. Тогда $\mathfrak{A}$ и $\mathfrak{B}$ изоморфны.
}

%\task{
%    С точностью до изоморфизма существует ровно четыре плотных л.у.м. со счётными носителями.
%}


%\task{
%    Постройте два плотных л.у.м.\ без концов с континуальными носителями, которые не изоморфны.
%}


%\task{
%    Докажите, что сложение и умножение на $\mathrm{Ord}$ ассоциативны.
%}

%\task{
%    Докажите, что ни сложение, ни умножение на $\mathrm{Ord}$ не коммутативно.
%}

\task{
    Аккуратно докажите теорему о классовой трансфинитной рекурсии.
}


\task{
    Докажите, что сложение и умножение на $\mathrm{Card}$ ассоциативны и коммутативны.
}



\task{
    Докажите, что $\mathrm{Card}$ не является множеством.
}


\task{
    У всякого векторного пространства есть базис.
}


\task{
    Пусть $\mathfrak{A} = \avg{A, \leq}$~--- ч.у.м. Тогда существует линейный порядок $\preccurlyeq$ на
    $A$ такой, что $\leq \subseteq \preccurlyeq$.
}

\begin{definition*}
    Цепь $S$ в ч.у.м.\ $\mathfrak{A}$ будем называть \deftext{максимальной}, если $S$ максимальна по
    включению среди всех цепей в $\mathfrak{A}$, т.е. не существует цепи $S'$ в $\mathfrak{A}$ такой, что
    $S \subsetneq S'$.
\end{definition*}


\task{[Принцип максимума Хаусдорфа]
    Пусть $\mathfrak{A} = \avg{A, \leq}$~--- ч.у.м. Тогда для каждой цепи $S$ в $\mathfrak{A}$ найдётся
    такая максимальная цепь $S'$ в $\mathfrak{A}$, что $S \subseteq S'$.
}


\breakline




\breakline

\task[3][04]{
    Покажите, что для всех $k, m, n \in \mathbb{N}$ верно следующее:
    \begin{enumcyr}
        \item $(k + m) + n = k + (m + n)$;
        \item $m + n = n + m$.
    \end{enumcyr}
}



\task[2][05]{
    Покажите, что если множество конечно, то оно конечно по Дедекинду.
}


\task[3][05]{
    Пусть $f\colon X \rightarrow \mathbb{R}$ такая функция, что множество:
    $$
        \left\{\sum_{s \in S} f(s) \mid S \subseteq X \text{ и } S \text{ конечно} \right\}
    $$
    ограничено, т.е. существует такое $N \in \mathbb{N}$, что для любого конечного $S \subseteq X$,
    $$
        \abs{\sum_{s \in S} f(s)} \leq N.
    $$
    Покажите, что $\{x \in X \mid f(x) \neq 0\}$ не более чем счётно.
}

\task[4][05]{
    $\mathbb{R} \times \mathbb{R}$, $\mathbb{R} \times \mathbb{R} \times \mathbb{R}$ и т.д. континуальны.
}


\task[5][05]{
    Множество всех трансцендентных чисел континуально.
}


\task[6][05]{
    Пусть $X \cup Y = \mathbb{R}$. Тогда хотя бы одно из $X$ и $Y$ континуально.
}

\task[4][06]{
    Покажите, что ч.у.м.\ $\avg{A, \leq}$ фундировано тогда и только тогда, когда не существует
    инъективной функции $f$ из $\mathbb{N}$ в $A$ такой, что $f(n + 1) < f(n)$ для всех $n \in
    \mathbb{N}$.
}

\end{document}
