\documentclass[12pt, fleqn, a4paper]{article}


\usepackage{amsmath}
\usepackage{amssymb}
\usepackage{amsfonts}
\usepackage{textcomp}
\usepackage{amsthm}
\usepackage{mathtools}
\usepackage{xspace}
\usepackage[classfont = bold]{complexity}
\usepackage{fullpage}
\usepackage[russian, english]{babel}
\usepackage[utf8]{inputenc}
\usepackage[
    sorting = ydnt,
    style = alphabetic,
    maxbibnames = 99,
    backend = biber
    ]{biblatex}
\addbibresource{main.bib}



\newtheorem{conjecture}{Conjecture}[section]
\theoremstyle{definition}
\newtheorem{theorem}{Theorem}[section]
\newtheorem*{theorem*}{Theorem}
\newtheorem{lemma}{Lemma}[section]
\newtheorem{corollary}{Corollary}[section]
\newtheorem{proposition}{Proposition}[section]
\newtheorem{fact}{Fact}[section]
\newtheorem{problem}{Problem}[section]
\newtheorem{exercise}{Exercise}[section]
\newtheorem{example}{Example}[section]
\newtheorem{definition}{Definition}[section]
\newtheorem{remark}{Remark}[section]
\newtheorem{algorithm}{Algorithm}[section]


\newcommand{\class}[1]{\mathbf{#1}}
\newcommand{\co}{\mathrm{co}}
\newcommand{\alg}[1]{\mathit{#1}}
\newcommand{\lang}[1]{\mathtt{#1}}


% classes (1)

\newcommand{\DTime}{\class{DTime}}
\newcommand{\RTime}{\class{RTime}}
\newcommand{\UTime}{\class{UTime}}
\newcommand{\NTime}{\class{NTime}}
\newcommand{\BPTime}{\class{BPTime}}


\renewcommand{\P}{\class{P}}
\newcommand{\ZPP}{\class{ZPP}}
\newcommand{\RP}{\class{RP}}
\newcommand{\coRP}{\co\class{RP}}
\newcommand{\UP}{\class{UP}}
\newcommand{\coUP}{\co\class{UP}}
\newcommand{\NP}{\class{NP}}
\newcommand{\coNP}{{\co}\class{NP}}
\newcommand{\BPP}{\class{BPP}}
\newcommand{\SigmaP}[1]{\Sigma^{#1}\class{P}}
\newcommand{\PH}{\class{PH}}
\newcommand{\PP}{\class{PP}}
\newcommand{\IP}{\class{IP}}
\newcommand{\OP}{\class{\oplus P}}


\newcommand{\EXP}{\class{EXP}}
\newcommand{\MIP}{\class{MIP}}
\newcommand{\NEXP}{\class{NEXP}}
\newcommand{\coNEXP}{{\co}\class{NEXP}}
\newcommand{\MAEXP}{\class{MA}_\class{EXP}}


% classes (2)

\newcommand{\Ppoly}{\class{P}/\class{poly}}
\newcommand{\NC}{\class{NC}}


\newcommand{\DSpace}{\class{DSpace}}
\newcommand{\NSpace}{\class{NSpace}}
\newcommand{\PSPACE}{\class{PSPACE}}

\newcommand{\EXPSPACE}{\class{EXPSPACE}}


% algorithms and proof systems


\newcommand{\DPLL}{\alg{DPLL}}
\newcommand{\OBDD}{\alg{OBDD}}
\newcommand{\pOBDD}{\pi\text{-}\alg{OBDD}}
\newcommand{\DPLLL}{\alg{DPLL}_{lin}}
\newcommand{\ResL}{\alg{Res}_{lin}}
\newcommand{\SemL}{\alg{Sem}_{lin}}



% languages


\newcommand{\SAT}{\lang{SAT}}
\newcommand{\GNI}{\lang{GNI}}
\newcommand{\MAJSAT}{\lang{MAJ}\text{-}\lang{SAT}}
\newcommand{\QBF}{\lang{QBF}}



% other

\newcommand{\poly}{\mathrm{poly}}
\newcommand{\Nat}{\mathbb{N}}
\newcommand{\bool}{\{0, 1\}}

\newcommand{\Img}{\mathop{\mathrm{Im}}}

\DeclareMathOperator*{\supp}{supp}
\DeclareMathOperator*{\Exp}{E}
\DeclareMathOperator*{\rk}{rk}




%%% Local Variables:
%%% mode: latex
%%% TeX-master: t
%%% End:


\begin{document}

	\setcounter{curtask}{9}

\mytitle{2 (на 3.10)}

\begin{task}
    Докажите, что множество всех рациональных чисел меньших $\pi$ разрешимо.
\end{task}

\begin{task}
    Существует ли алгоритм, проверяющий, работает ли данная программа
    полиномиальное время?
\end{task}

\begin{task}
    Приведите пример двух непересекающихся неперечислимых множеств.
\end{task}

\begin{task}
    Докажите, что для каждой вычислимой функции $f$ найдется
    псевдообратная вычислимая функция $g$. А именно, $g$ определена на
    множестве значений $f$, и для всех $x$ из области определения $f$
    выполняется $f(g(f(x))) = f(x)$.
\end{task}

\begin{task}
    Приведите пример неразрешимого множества $A \subseteq \Nat \times \Nat$,
    такого, что все его горизонтальные и вертикальные сечения
    разрешимы (т.е. для любого $x$ разрешимы $A \cap \{\{x\} \times \Nat\}$
    и $A \cap \{\Nat \times \{x\}\}$)
\end{task}

\begin{task}
    Докажите, что существует язык, который можно распознать с памятью $2^n$ ($n$~---
    длина слова), но нельзя с памятью $n$. (подсказка: диагонализация)
\end{task}

\end{document}



\setmathstyle{}{Теория множеств}{1 курс}


\begin{document}

\task{
    Покажите, что гомоморфизм $f$ из ч.у.м. $\mathfrak{A}$ в ч.у.м. $\mathfrak{B}$ является изоморфизмом
    тогда и только тогда, когда существует такой гомоморфизм $g$ из $\mathfrak{B}$ в $\mathfrak{A}$, что:
    $$
        f \circ g = \mathrm{id}_A \quad \text{и} \quad g \circ f = \mathrm{id}_B.
    $$
}
    
%\task{
%    Покажите строго, что всякое ч.у.м. с конечным носителем фундировано.
%}

%\task{
%    Пусть $\mathfrak{A} = \avg{A, \leq}$~--- ч.у.м., причём $A$ конечно. Покажите, что для каждого $a \in
%    A$ существует такой максимальный в $\mathfrak{A}$ элемент $a'$, что $a \leq a'$.
%}


\task{
    Пусть $\mathfrak{A} = \avg{A, \leq}$~--- ч.у.м., причём $A$ конечно. Тогда существует линейный
    порядок $\preccurlyeq$ на $A$ такой, что $\leq \subseteq \preccurlyeq$.
}


\task{
    Пусть $\mathfrak{A} = \avg{A, \leq_A}$~--- в.у.м., удовлетворяющий следующим условиям:
    \begin{enumcyr}
        \item в $\mathfrak{A}$ есть наименьший элемент, но нет наибольшего;
        \item в каждом непустом подмножестве $A$, имеющем вернюю грань, есть наибольший элемент.
    \end{enumcyr}
    Покажите, что $\avg{A, \leq_A}$ изоморфно $\avg{\mathbb{N}, \leq}$, где $\leq$~--- естественный
    порядок на $\mathbb{N}$.
}

\begin{definition*}
    Ч.у.м.\ $\mathfrak{A} = \avg{A, \leq_A}$ будем называть \deftext{плотным}, если:
    $$
        \forall a_1 \in A\, \forall a_2 \in A\, (a_1 <_A a_2 \rightarrow
        \exists a_{1.5} \in A\, (a_1 <_A a_{1.5} <_A a_2)).
    $$
    Под ч.у.м. \deftext{без концов} будем понимать произвольное ч.у.м., в котором нет ни наибольшего, ни
    наименьшего элементов.
\end{definition*}


\task{
    Пусть $\mathfrak{A} = \avg{A, \leq_A}$ и $\mathfrak{B} = \avg{B, \leq_B}$~--- два плотных л.у.м.\ без
    концов, причём $A$ и $B$ счётны. Тогда $\mathfrak{A}$ и $\mathfrak{B}$ изоморфны.
}

%\task{
%    С точностью до изоморфизма существует ровно четыре плотных л.у.м. со счётными носителями.
%}


%\task{
%    Постройте два плотных л.у.м.\ без концов с континуальными носителями, которые не изоморфны.
%}


%\task{
%    Докажите, что сложение и умножение на $\mathrm{Ord}$ ассоциативны.
%}

%\task{
%    Докажите, что ни сложение, ни умножение на $\mathrm{Ord}$ не коммутативно.
%}

\task{
    Аккуратно докажите теорему о классовой трансфинитной рекурсии.
}


\task{
    Докажите, что сложение и умножение на $\mathrm{Card}$ ассоциативны и коммутативны.
}



\task{
    Докажите, что $\mathrm{Card}$ не является множеством.
}


\task{
    У всякого векторного пространства есть базис.
}


\task{
    Пусть $\mathfrak{A} = \avg{A, \leq}$~--- ч.у.м. Тогда существует линейный порядок $\preccurlyeq$ на
    $A$ такой, что $\leq \subseteq \preccurlyeq$.
}

\begin{definition*}
    Цепь $S$ в ч.у.м.\ $\mathfrak{A}$ будем называть \deftext{максимальной}, если $S$ максимальна по
    включению среди всех цепей в $\mathfrak{A}$, т.е. не существует цепи $S'$ в $\mathfrak{A}$ такой, что
    $S \subsetneq S'$.
\end{definition*}


\task{[Принцип максимума Хаусдорфа]
    Пусть $\mathfrak{A} = \avg{A, \leq}$~--- ч.у.м. Тогда для каждой цепи $S$ в $\mathfrak{A}$ найдётся
    такая максимальная цепь $S'$ в $\mathfrak{A}$, что $S \subseteq S'$.
}


\breakline




\breakline

\task[3][04]{
    Покажите, что для всех $k, m, n \in \mathbb{N}$ верно следующее:
    \begin{enumcyr}
        \item $(k + m) + n = k + (m + n)$;
        \item $m + n = n + m$.
    \end{enumcyr}
}



\task[2][05]{
    Покажите, что если множество конечно, то оно конечно по Дедекинду.
}


\task[3][05]{
    Пусть $f\colon X \rightarrow \mathbb{R}$ такая функция, что множество:
    $$
        \left\{\sum_{s \in S} f(s) \mid S \subseteq X \text{ и } S \text{ конечно} \right\}
    $$
    ограничено, т.е. существует такое $N \in \mathbb{N}$, что для любого конечного $S \subseteq X$,
    $$
        \abs{\sum_{s \in S} f(s)} \leq N.
    $$
    Покажите, что $\{x \in X \mid f(x) \neq 0\}$ не более чем счётно.
}

\task[4][05]{
    $\mathbb{R} \times \mathbb{R}$, $\mathbb{R} \times \mathbb{R} \times \mathbb{R}$ и т.д. континуальны.
}


\task[5][05]{
    Множество всех трансцендентных чисел континуально.
}


\task[6][05]{
    Пусть $X \cup Y = \mathbb{R}$. Тогда хотя бы одно из $X$ и $Y$ континуально.
}

\task[4][06]{
    Покажите, что ч.у.м.\ $\avg{A, \leq}$ фундировано тогда и только тогда, когда не существует
    инъективной функции $f$ из $\mathbb{N}$ в $A$ такой, что $f(n + 1) < f(n)$ для всех $n \in
    \mathbb{N}$.
}

\end{document}
