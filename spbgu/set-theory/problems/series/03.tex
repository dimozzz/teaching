\documentclass[a4paper, 12pt]{article}
% math symbols
\usepackage{amssymb}
\usepackage{amsmath}
\usepackage{mathrsfs}
\usepackage{mathseries}


\usepackage[margin = 2cm]{geometry}

\tolerance = 1000
\emergencystretch = 0.74cm



\pagestyle{empty}
\parindent = 0mm

\renewcommand{\coursetitle}{DM/ML}
\setcounter{curtask}{1}

\setmathstyle{}{Теория множеств}{1 курс}


\begin{document}

\task{
    Покажите, что для любого $n \in \mathbb{N}$ верно $n \subseteq n + 1 \wedge \neg \exists x\, n
    \subsetneq x \subsetneq n + 1$.
}


\task{
    Покажите, что для любых $m, n \in \mathbb{N}$,
    $$
        m < n \quad \text{тогда и только тогда, когда} \quad m \subsetneq n.
    $$
}

\task{
    Покажите, что для всех $n, m \in \mathbb{N}$ верно следующее:
    \begin{enumcyr}
        \item $n \ne 0 \leftrightarrow \exists k \in \mathbb{N}\, n = k + 1$;
        \item $n + 1 = m + 1 \leftrightarrow n = m$.
    \end{enumcyr}
}

\task{
    Покажите, что:
    \begin{enumcyr}
        \item для любого $n \in \mathbb{N}$ не существует инъекции из $n + 1$ в $n$;
        \item для любых $n, m \in \mathbb{N}$, если $m < n$, то не существует инъекции из $n$ в $m$.
        \item для любых $n, m \in \mathbb{N}$, если $n \ne m$, то не существует биекции между $n$ и $m$.
    \end{enumcyr}
}


\begin{definition*}
    Будем говорить, что \deftext{$X$ содержит $n$ элементов}, где $n \in \mathbb{N}$, и писать $\abs{X} =
    n$, если существует биекция между $n$ и $X$. Далее, $X$ будем называть \deftext{конечным}, если
    $\abs{X} = n$ для некоторого $n \in \mathbb{N}$, и \deftext{бесконечным}, если оно не конечно.

    Свойство <<быть конечным>> можно определять по-разному. Для произвольного $Y$ обозначим
    $$
        \funccplx{Max}(Y) \coloneqq
        \{u \in Y \mid \neg \exists v\, (v \in Y \wedge u \subsetneq v)\}.
    $$

    Элементы $\funccplx{Max}(Y)$ будем называть \deftext{$\subsetneq$-максимальными} в $Y$. Также будем
    говорить, что $X$ \deftext{конечно по Тарскому}, или \deftext{$\mathrm{T}$-конечно}, если оно
    удовлетворяет условию:
    $$
        \forall Y\, ((Y \subseteq \mathcal{P}(X) \wedge Y \ne \emptyset) \rightarrow \funccplx{Max}(Y)
        \ne \emptyset),
    $$
    и \deftext{бесконечно по Тарскому}, или \deftext{$\mathrm{T}$-бесконечно}, если оно не
    $\mathrm{T}$-конечно.
\end{definition*}

\task{
    Покажите, что каждое натуральное число $\mathrm{T}$-конечно.
}

\task{
    Покажите, что множество всех натуральных чисел $\mathrm{T}$-бесконечно.
}

\task{
    Покажите, что множество конечно тогда и только тогда, когда оно $\mathrm{T}$-конечно.
}

\task{
    Используя лишь обычную индукцию, докажите, что для любого $n \in \mathbb{N}$:
    $$
        \forall X\, ((X \subseteq n \wedge X \ne \emptyset) \rightarrow \funccplx{Min}(X) \ne \emptyset).
    $$
    Выведите отсюда принцип минимального элемента.
}

\breakline

\libproblem[6][02]{set-theory}{relation-func-crit}
\task[7]{
    Если $f\colon X \rightarrow Y$ и $g\colon Y \rightarrow Z$, то $f \circ g\colon X \rightarrow Z$,
    причём $(f \circ g)(x) = g(f(x))$ для всех $x \in X$.
}
\libproblem[8][02]{set-theory}{reduce-left-right}
\libproblem[9][02]{set-theory}{inverse-left-right}
\libproblem[10][02]{set-theory}{choice-equiv-c-prime}



\end{document}
