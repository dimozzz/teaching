\documentclass[a4paper, 12pt]{article}
% math symbols
\usepackage{amssymb}
\usepackage{amsmath}
\usepackage{mathrsfs}
\usepackage{mathseries}


\usepackage[margin = 2cm]{geometry}

\tolerance = 1000
\emergencystretch = 0.74cm



\pagestyle{empty}
\parindent = 0mm

\renewcommand{\coursetitle}{DM/ML}
\setcounter{curtask}{1}
\usepackage{dashrule}

\renewcommand{\coursetitle}{\textsc{StT}}


\begin{document}

\setmathstyle{}{Правила проведения экзамена}{}

Экзамен проводится в форме опроса без времени на подготовку ответов, включающего $9$ вопросов по всему курсу.

Здесь участвуют: 
\begin{itemize}
    \item все определения (в том числе из упражнений);
    \item все формулировки утверждений;
    \item все простые доказательства;
    \item идеи всех доказательств;        
    \item решения сравнительно простых упражений.
\end{itemize}

За каждый ответ ставится
$$
    + [5 \text{ баллов}] \qquad \pm [4 \text{ балла}] \qquad
    \mp [3 \text{ балла}] \qquad - [0 \text{ баллов}]
$$

Обозначим за $\Sigma$ сумму балов, набранных данным студентом.

\begin{enumerate}[label = {\realasbuk*)}, ref = \realasbuk*]
    \item Если $\Sigma < 22$, то ставится <<неудовлетворительно>>.
    \item Если $22 \le \Sigma < 30$, то ставится <<удовлетворительно>>.
    \item Если $30 \le \Sigma < 35$, то ставится <<хорошо>>.
    \item Если $\Sigma \ge 35$, то ставится <<отлично>>.
\end{enumerate}


По усмотрению принимающего в пограничных ситуациях может быть задан ровно один
дополнительный вопрос.

\vspace{0.1cm}

\hdashrule{1.01\textwidth}{1pt}{3pt}

\vspace{0.05cm}

\begin{flushright}
    17.10.2020
\end{flushright}

\vspace{2cm}

\begin{center}
    \begin{tikzpicture}
        \node[graduate, shield, minimum size = 3cm] at (0, 0) {};
        \node[graduate, sword, minimum size = 3cm, mirrored] at (8, 0) {};
    \end{tikzpicture}    
\end{center}


\end{document}