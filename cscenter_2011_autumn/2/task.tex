\setcounter{curtask}{7}

\mytitle{2 (на 21.09)}

\begin{task}
    Докажите, что множество всех рациональных чисел меньших $e$ разрешимо.
\end{task}

\begin{task}
    Существует ли алгоритм, проверяющий, работает ли данная программа
    полиномиальное время?
\end{task}

\begin{task}
    Приведите пример двух непересекающихся неперечислимых множеств.
\end{task}

\begin{task}
    Докажите, что множество вещественных чисел не является счетным.
    (Подсказка: применить диагонализацию).
\end{task}

\begin{task}
    Докажите, что для каждой вычислимой функции $f$ найдется
    псевдообратная вычислимая функция $g$. А именно, $g$ определена на
    множестве значений $f$, и для всех $x$ из области определения $f$
    выполняется $f(g(f(x))) = f(x)$.
\end{task}

\begin{task}
    Приведите пример неразрешимого множества $A \subseteq \Nat \times \Nat$,
    такого, что все его горизонтальные и вертикальные сечения
    разрешимы (т.е. для любого $x$ разрешимы $A \cap \{\{x\} \times \Nat\}$
    и $A \cap \{\Nat \times \{x\}\}$)
\end{task}

\begin{task}
    Докажите, что существует конкретная машина Тьюринга, для которой
    задача о ее остановке на входе $x$ неразрешима.
\end{task}
\breakline

\begin{ptask}{4}
    Покажите, что любое бесконечное перечислимое множество можно
    записать в виде: ${a(1), a(2), a(3), \dots}$, где $a$~---
    вычислимая функция, все значения которой различны.
\end{ptask}

\begin{ptask}{5}
    Докажите, что любое бесконечное перечислимое множество содержит
    бесконечное разрешимое подмножество.
\end{ptask}