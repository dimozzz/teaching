\setcounter{curtask}{1}

\mytitle{Контрольная работа 1}

\begin{task}
    Предъявите такое число $x \in \mathbb{R}$, что $x$ не является
    вычислимым, а множество рациональных числел меньших $x$~---
    перечислимо.
\end{task}

\begin{task}
    Докажите, что существует пара программ $A, B$ такая, что $A$
    печатает текст $B$, а $B$ печатает текст $A$
\end{task}

\begin{task}
    Покажите, что число $\pi$ является вычислимым.
\end{task}

\begin{task}
    Пусть отношение $R(x, y)$ задает примитивно рекурсивное множество
    (т.е. множество $\{(x, y) \mid R(x, y) = 1\}$), докажите, что
    отношения $S(x, z) = \exists (y \le z) R(x, y)$ и
    $T(x, z) = \forall (y \le z) R(x, y)$ также задают примитивно
    рекурсивные множества.
\end{task}


\breakline

\begin{task}(*)
    Пусть $S$~--- разрешимое множество натуральных чисел. Разложим все
    числа из $S$ на простые множители, из данных простых составим
    множество $D$. Верно ли что $D$ разрешимо?
\end{task}

\begin{task}(*)
    Является ли перечислимым множество всех программ, вычисляющим
    сюръективные функции? А коперечислимым?
\end{task}

\begin{task}(*)
    Докажите, что существует счетное число непересекающихся
    перечислимых множеств, никакие два из которых нильзя отделит
    разрешимым.
\end{task}