\setcounter{curtask}{14}

\mytitle{3 (на 28.09)}

\begin{task}
    Существует ли алгоритм, проверяющий, считает ли данный алгоритм
    полиномиально вычислимую функцию.
\end{task}

\begin{task}
    Докажите, что существует программа, которая печатает квадрат
    своего номера.
\end{task}

\begin{task}
    Докажите, что существует тройка попарно различных программ $A, B,
    C$ таких, что $A$ печатает текст $B$, $B$ печатает текст $C$, а
    $C$ печатает текст $A$.
\end{task}

\begin{task}
    Докажите, что существуют перечислимые множества $A, B$, которые не
    могут быть отделены разрешимым множеством, т.е не существует
    такого разрешимого множества $C$, что $A \subseteq C$ и $B \cap C
    = \emptyset$
\end{task}

\begin{task}
    Напишите на языке $C++$ (или аналогичном) программу, которая
    печатает свой текст.
\end{task}

\begin{task}
    Докажите, что: {\it a)} существует универсальное перечислимое множество ---
    $U \subseteq \mathbb{N} \times \mathbb{N}$, что для любого
    перечислимого $A$ найдется такое $a \in \mathbb{N}$, что $A = \{x
    \mid (a, x) \in U\}$
    {\it b)} не существует универсального вычислимого множества
    {\it с)} найдется $x \in \mathbb{N}$ такой, что $\{x\} = \{y
    \mid (x, y) \in U\}$
\end{task}

\breakline

\begin{ptask}{8}
    Существует ли алгоритм, проверяющий, работает ли данная программа
    полиномиальное время?
\end{ptask}

\begin{ptask}{9}
    Приведите пример двух не пересекающихся неперечислимых множеств.
\end{ptask}

\begin{ptask}{11}
    Докажите, что для каждой вычислимой функции $f$ найдется
    псевдообратная вычислимая функция $g$. А именно, $g$ определена на
    множестве значений $f$, и для всех $x$ из области определения $f$
    выполняется $f(g(f(x))) = f(x)$.
\end{ptask}

\begin{ptask}{13}
    Докажите, что существует конкретная машина Тьюринга (конкретная
    программа), для которой задача о ее остановке на входе $x$ неразрешима.
\end{ptask}