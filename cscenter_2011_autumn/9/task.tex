\setcounter{curtask}{44}

\mytitle{9 (на 17.11)}

Предикат, заданный на множестве натуральных чисел называется
арифметичным, если он выражается при помощи формулы исчисления
предикатов в сигнатуре $(=, +, \times)$ в естественной интерпретации
на множестве натуральных чисел.

\begin{task}
    Докажите, что следующие предикаты являются арифметичными:
    {\it a)} $x < y$
    {\it б)} $x = 0$
    {\it в)} $x = 1$
    {\it г)} $x = c$, где $c$~--- константа
    {\it д)} $x~mod~b = r$
    {\it е)} $a$~--- степень двойки
    {\it ж)} $a$~--- степень четверки
\end{task}

\begin{task}
	$\mathbb{Z} + \mathbb{Z}$~--- это две копии целых чисел, причем
    все числа из второй копии больше чисел из первой. Докажите, что
    $(\mathbb{Z}, <, =)$ элементарно эквивалентна $(\mathbb{Z} +
    \mathbb{Z}, <, =)$.
\end{task}

\begin{task}
    Будет ли интерпретация $(\mathbb{N}, =, <)$ элементарно
    эквивалентна:
    а) $(\mathbb{N} + \mathbb{N}, =, <)$
    б) $(\mathbb{N} + \mathbb{Z}, =, <)$
\end{task}

\breakline

\begin{ptask}{33}
    Докажите, что если у формулы существует резолюционный вывод, то
    существует и вывод размера $O(2^n)$, где $n$~--- количество переменных.
\end{ptask}

\begin{ptask}{34}
    Приведите пример формулы длины $n$ такой, что ее минимальный
    размер в КНФ $\Omega(2^n)$.
\end{ptask}

\begin{ptask}{37} (критерий Поста)
	Пусть $F = {f_1, \dots, f_k}$~--- набор булевых функций от $n$
    переменных. Будем говорить, что $F$ принадлежит классу функций,
    если все функции из множества $F$ принадлежат данному классу.
    
    Пусть теперь $F$ не принадлежит ни одному из перечисленных
    классов.

    в) постройте конъюнкцию из композиций функций из $F$ и докажите,
    что набор $F$ является базисом булевых функций от $n$ аргументов
    (указание: использовать полином Жегалкина).

\end{ptask}

\begin{ptask}{38}
    Приведите к предваренной нормальной форме формулу:
    $\forall x A(x) \rightarrow \forall x B(x)$
\end{ptask}

\begin{ptask}{40}
    Докажите, что предикат $y = x + 2011$ невыразим в интерпретации
    $(\mathbb{Z}, =, x \mapsto x^2)$.
\end{ptask}

\begin{ptask}{42}
    $(\mathbb{Q}, =, +)$
\end{ptask}

\begin{ptask}{43}
    $(Z, =, <, +, 0, 1)$ (подсказка: рассмотреть предикат четности)
\end{ptask}