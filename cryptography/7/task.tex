\setcounter{curtask}{29}

\mytitle{7 (на 10.04)}

\begin{task}
    Пусть $G: \{0, 1\}^{n} \rightarrow \{0, 1\}^{m}$ такая функция, что $G(U_n)$
    вычислительно неотличимо от $U_m$. Докажите, что $G(U_n)G(U_n)G(U_n) \dots
    G(U_n)$ ($p(n)$ раз) вычислительно неотличимо от $U_{m p(n)}$.
\end{task}

\begin{task}
    Покажите, что существуют такие вычислительно неразличимые случайные величины
    $\alpha_n$ и $\beta_n$, которые различаются схемами полиномиального размера.
\end{task}

\begin{task}
    Пусть $G$~--- это $2n$-генератор. $G_0$ и $G_1$~--- это первая и вторая половины
    выхода. Пусть $s \in \{0, 1\}^n$, определим $f_s(x)$. Является ли $f_s$
    семейством псевдослучайных функций?
\end{task}

\begin{task}
    Докажите, что если существует схема кодирования с открытым ключом для
    однобитового сообщения, то существует и схема кодирования с открытым ключом для
    произвольных сообщений полиномиальной длины.
\end{task}

\breakline

\begin{ptask}{12}
    Докажите, что функция $f(xy) = prime(x) + prime(y)$ не является односторонней,
    где $x$, $y$~--- бинарные строки длины $n$, а $prime(x)$~--- минимальное простое
    число, большее $x$.
\end{ptask}

\begin{ptask}{21}
    Докажите, что если $(NP, PISamp) \subseteq Heur_{\frac{1}{n}}P$, то $(NP, PISamp)
    \subseteq HeurP$.
\end{ptask}

\begin{ptask}{22}
    Пусть $EXP \neq NEXP$, докажите что существует задача поиска $(L, D) \in
    \widetilde{(NP, PSamp)}$,
    которая детерминированно не сводится к задаче из класса $\widetilde{(NP, U)}$.
\end{ptask}

\begin{ptask}{27}
    Пусть существует надежная криптосистема (PKCS), докажите, что тогда существует
    trapdoor permutation family.
\end{ptask}

\begin{ptask}{28}
    Докажите, что если существует односторонняя функция, то существует и
    неуниверсальная односторонняя функция.
\end{ptask}
