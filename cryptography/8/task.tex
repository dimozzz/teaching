\setcounter{curtask}{33}

\mytitle{8 (на 17.04)}

\begin{task}
    Пусть существует протокол привязки в биту. Пусть Мерлин и Артур (Артур~---
    полиномиальный алгоритм) знают граф $G$, и Мерлин хочет убедить Артура в том, что
    данный граф раскрашивается в $3$ цвета. Покажите, что существует такой протокол,
    что с вероятностью $1 - \frac{1}{n}$ Боб может узнать является ли граф $G$ $3$
    раскрашиваемым, но после выполнения данного протокола Боб не сможет узнать
    раскраску графа с вероятностью $1 - \frac{1}{n}$.
\end{task}


\breakline

\begin{ptask}{12}
    Докажите, что функция $f(xy) = prime(x) + prime(y)$ не является односторонней,
    где $x$, $y$~--- бинарные строки длины $n$, а $prime(x)$~--- минимальное простое
    число, большее $x$.
\end{ptask}

\begin{ptask}{21}
    Докажите, что если $(NP, PISamp) \subseteq Heur_{\frac{1}{n}}P$, то $(NP, PISamp)
    \subseteq HeurP$.
\end{ptask}

\begin{ptask}{27}
    Пусть существует надежная криптосистема (PKCS), докажите, что тогда существует
    trapdoor permutation family.
\end{ptask}

\begin{ptask}{28}
    Докажите, что если существует односторонняя функция, то существует и
    неуниверсальная односторонняя функция.
\end{ptask}

\begin{ptask}{30}
    Покажите, что существуют такие вычислительно неразличимые случайные величины
    $\alpha_n$ и $\beta_n$, которые различаются схемами полиномиального размера.
\end{ptask}

\begin{ptask}{31}
    Пусть $G$~--- это $2n$-генератор. $G_0$ и $G_1$~--- это первая и вторая половины
    выхода. Пусть $s \in \{0, 1\}^n$, определим $f_s(x) =
    G_{s_1}(G_{s_2}(\dots))$. Является ли $f_s$ семейством псевдослучайных функций?
\end{ptask}

\begin{ptask}{32}
    Докажите, что если существует схема кодирования с открытым ключом для
    однобитового сообщения, то существует и схема кодирования с открытым ключом для
    произвольных сообщений полиномиальной длины.
\end{ptask}