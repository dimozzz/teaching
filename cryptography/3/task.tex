\setcounter{curtask}{14}

\mytitle{2 (на 05.03)}

\begin{task}
    Зафиксируем язык $L$. Для каждого алгоритма найдется такое семейство
    распределений, что алгоритм работает либо не верно на этом языке, либо алгоритм
    работает не полиномиальное время. Докажите, что существует такое семейство
    распределений $D$, что $(L, D) \notin AvgP$.
\end{task}

\begin{task}
    Докажите, что $Avg_{\frac{1}{n^{100}}}P \notin AvgP$.
\end{task}

\begin{task}
    Пусть односторонние функции существуют. Постройте такую одностороннюю функцию,
    сохраняющую длину, что для любого $x$ $cyc_{f}(x) > 2^{\frac{|x|}{2}}$.
\end{task}

\begin{task}
    Пусть односторонние функции существуют.
    Докажите, что существует такая односторонняя а) функция б) перестановка, что
    $f(0^n) = 0^n$.
\end{task}


\breakline

\begin{ptask}{3}
    Докажите, что а) сильная б) слабая односторонняя функция не может иметь
    полиномиально ограниченный образ.
\end{ptask}

\begin{ptask}{12}
    Докажите, что функция $f(xy) = prime(x) + prime(y)$ не является односторонней,
    где $x$, $y$~--- бинарные строки длины $n$, а $prime(x)$~--- минимальное простое
    число, большее $x$.
\end{ptask}

\begin{ptask}{13}
    Пусть $cyc_{f}(x)$~--- это минимальное $n$, что $f^{(n)}(x) = x$. Докажите, что
    среднее значение $cyc_{f}$ на строчках длины $n$ не может быть ограничена
    полиномом для слабой односторонней функции.
\end{ptask}

