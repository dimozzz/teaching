\setcounter{curtask}{18}

\mytitle{4 (на 12.03)}

\begin{task}
    Докажите, что в определении $AvgBPP$ обе константы  можно понизить до $2^{-n}$.
\end{task}

\begin{task}
    Покажите, что $AvgBPP \subseteq HeurBPP$.
\end{task}

\begin{task}
    Докажите, что если $(BH, U^{BH}) \in Avg_{\frac{1}{n}}P$, то $(NP, U) \in AvgP$.
\end{task}

\begin{task}
    Докажите, что если $(NP, PISamp) \subseteq Heur_{\frac{1}{n}}P$, то $(NP, PISamp)
    \subseteq HeurP$.
\end{task}

\begin{task}
    Пусть $EXP \neq NEXP$, докажите что существует задача поиска $(L, D) \in
    \widetilde{(NP, PSamp)}$,
    которая детерминированно не сводится к задаче из класса $\widetilde{(NP, U)}$.
\end{task}

\breakline

\begin{ptask}{12}
    Докажите, что функция $f(xy) = prime(x) + prime(y)$ не является односторонней,
    где $x$, $y$~--- бинарные строки длины $n$, а $prime(x)$~--- минимальное простое
    число, большее $x$.
\end{ptask}

\begin{ptask}{14}
    Зафиксируем язык $L$. Для каждого алгоритма найдется такое семейство
    распределений, что алгоритм работает либо не верно на этом языке, либо алгоритм
    работает не полиномиальное время. Докажите, что существует такое семейство
    распределений $D$, что $(L, D) \notin AvgP$.
\end{ptask}

\begin{ptask}{17}
    Пусть односторонние функции существуют.
    Докажите, что существует такая односторонняя а) функция б) перестановка, что
    $f(0^n) = 0^n$.
\end{ptask}



