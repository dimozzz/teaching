\setcounter{curtask}{21}

\mytitle{5 (на 10.11)}

\begin{task}(арифметика Пресбургера)
    Рассмотрим интерпретацию $(Z, =, <, +, 0, 1)$.
    Докажите, что:
    а) элиминация кванторов невозможна
    б) $(Z, =, <, +, 0, 1, \equiv_c)$ допускает элиминацию кванторов
    ($\equiv_c$~--- сравнение по модулю $c$, по предикату для каждого $c$)
\end{task}

\begin{task}(резолюция для исчисления предикатов)
    Система доказывает противоречивость замкнутой формулы следующим
    образом: формула приводится в предваренную нормальную форму, затем
    проводится скулемизация (избавляемся от функциональных
    символов). Получается формула $\forall x_1 x_2 \dots
    \phi(x_1, \dots, x_n)$. Для формулы $\phi$ делается резолюционный
    вывод. Докажите корректность и полноту данного метода.
\end{task}

\begin{task}
    Приведите пример совместной теории:
    а) без конечных моделей
    б) наименьшая модель, которой имеет мощность не менее $2^{\aleph_0}$
\end{task}

\breakline


\begin{ptask}{13}
    Будет ли интерпретация $(\mathbb{N}, =, <)$ элементарно
    эквивалентна:
    б) $(\mathbb{N} + \mathbb{Z}, =, <)$
\end{ptask}

\begin{ptask}{14}
    в) Пусть единичный квадрат разрезан на несколько меньших
    квадратов. Докажите, что все они имеют рациональные стороны.
\end{ptask}

\begin{ptask}{19}
    $(\mathbb{N}, =, S)$
\end{ptask}

\begin{ptask}{20}
    $(\mathbb{N}, =, S, P)$, где $P$~--- предикат ``быть степенью двойки''.
\end{ptask}