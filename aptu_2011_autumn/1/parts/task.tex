\setcounter{curtask}{1}

\begin{task}
    Докажите, что следующая формула является тавтологией для любой
    формулы $A$: $\neg\neg A \rightarrow A$
\end{task}

\begin{task}
    Докажите, что для любых формул $A$, $B$, $C$ верна формула:
    \begin{center}
        $(A \rightarrow B) \rightarrow ((B \rightarrow C) \rightarrow (A
	    \rightarrow C))$
    \end{center}
\end{task}

\begin{task} (Правило сечения)
    Докажите, что если: $\Gamma_1 \mapsto A$ и $\Gamma_2, A \mapsto
    B$, то $\Gamma_1, \Gamma_2 \mapsto B$
\end{task}

\begin{task} (Теорема о компактности)
    Пусть $\Gamma$ множество формул. $\Gamma$ непротиворечиво тогда и
    только тогда, когда любое конечное подмножество $\Gamma$ непротиворечиво.
\end{task}

\vspace{0.5cm}
Методом резолюций называется метод доказательства противоречивости
множества дизъюнктов. Данный метод содержит единственное правило
вывода: $\frac{x \vee \alpha \ \ \ \neg x \vee \beta}{\alpha \vee
  \beta}$. Множество дизъюнктов считается противоречивым (имеет вывод
в резолюции), если из него можно вывести пустой дизъюнкт.

\begin{task} (Корректность)
    Формула $F$ является тавтологией, если $\neg F$ (записанная в КНФ)
    имеет вывод в резолюциях.
\end{task}

\begin{task}
    Пусть формула $\phi \rightarrow \psi$~--- тавтология. Докажите,
    что найдется такая формула $\tau$, содержащая только общие для
    $\phi$ и $\psi$ переменные, что $\phi \rightarrow \tau$ и
    $\tau \rightarrow \psi$ будут тавтологиями.
\end{task}