\setcounter{curtask}{6}

\mytitle{2 (на 28.09)}

\begin{task}
    Докажите: {\it a)} если у формулы есть резолюционный вывод, то
    есть вывод длиной не более $O(2^n)$ {\it б)} полноту системы
    резолюций.
\end{task}

Предикат, заданный на множестве натуральных чисел называется
арифметичным, если он выражается при помощи формулы исчисления
предикатов в сигнатуре $(=, +, \times)$ в естественной интерпретации
на множестве натуральных чисел.

\begin{task}
    Докажите, что следующие предикаты являются арифметичными:
    {\it a)} $x < y$
    {\it б)} $x = 0$
    {\it в)} $x = 1$
    {\it г)} $x = c$, где $c$~--- константа
    {\it д)} $x~mod~b = r$
    {\it е)} $a$~--- степень двойки
    {\it ж)} $a$~--- степень четверки
\end{task}

\begin{task}
    Приведите к предваренной нормальной форме формулу:
    $\forall x A(x) \rightarrow \forall x B(x)$
\end{task}

\begin{task}
    Докажите, что:
    {\it a)} для любого $k$ найдется сколь угодно большое $b$, что
    $b + 1, 2b + 1, 3b + 1, \dots, kb + 1$~--- взаимно простые числа.
    {\it б)} для любой последовательности натуральный чисел $x_0, x_1,
    \dots$ найдутся таки $a$ и $b$, что $x_i = a_i~ mod ~ b(i + 1) +
    1$.
    {\it в)} предикат: $a$~--- степень шестерки арифметичен.
\end{task}

\breakline

\begin{ptask}{1}
    Докажите, что следующая формула является тавтологией для любой
    формулы $A$: $\neg\neg A \rightarrow A$
\end{ptask}

\begin{ptask}{2}
    Докажите, что для любых формул $A$, $B$, $C$ верна формула:
    \begin{center}
        $(A \rightarrow B) \rightarrow ((B \rightarrow C) \rightarrow (A
	    \rightarrow C))$
    \end{center}
\end{ptask}

\begin{ptask}{5}
    Пусть формула $\phi \rightarrow \psi$~--- тавтология. Докажите,
    что найдется такая формула $\tau$, содержащая только общие для
    $\phi$ и $\psi$ переменные, что $\phi \rightarrow \tau$ и
    $\tau \rightarrow \psi$ будут тавтологиями.
\end{ptask}