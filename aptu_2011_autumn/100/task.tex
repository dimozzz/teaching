\begin{task}
    Докажите, что любой частичный порядок продолжается до линейного.
\end{task}

\begin{task}
    Докажите, что теория $Th(\mathbb{Z}, S, =, 0)$ не является
    конечно-аксиоматизируемой.
\end{task}

\begin{task}
    Будет ли теория $Th(\mathbb{N}, =, <)$ конечно аскиоматизируемой?
\end{task}

\begin{task}
    Построите две неизоморфные модели теории $Th(\mathbb{Q}, <, =)$
    (плотный линейный порядок без первого и последнего элемента)
    мощности континуум.
\end{task}

\begin{task}
    Рассмотрим интерпритацию $(\mathbb{N}, =, S, P)$, где $P$~---
    предикат ``быть степенью двойки''.
    Выразимы ли в ней следующие предикаты:
    а) $x$ четный.
\end{task}

\begin{task}
    Какое минимальное количество булевых функций от $n~ (n > 1)$
    переменных составляют полный базис?
\end{task}

\begin{task}
    Докажите, что $\mathbb{N}^{\mathbb{R}} \le 2^{\mathbb{R}}$
\end{task}

\begin{task}
    Докажите, что $\mathbb{R}^{\mathbb{N}} \le \mathbb{R}$
\end{task}

\begin{task}
    Приведите к предваренной нормальной форме формулу:
    $\exists x A(x) \rightarrow  \forall y \exists x B(x, y)$
\end{task}

\begin{task}
    Выразимы ли следующие предикаты в интерпретации
    $(\mathbb{Z}, =, <)$:
    а) $y = 2x$
    б) $y = x + 1$
\end{task}
