\documentclass[a4paper, 12pt]{article}
% math symbols
\usepackage{amssymb}
\usepackage{amsmath}
\usepackage{mathrsfs}
\usepackage{mathseries}


\usepackage[margin = 2cm]{geometry}

\tolerance = 1000
\emergencystretch = 0.74cm



\pagestyle{empty}
\parindent = 0mm

\renewcommand{\coursetitle}{DM/ML}
\setcounter{curtask}{1}

\setmathstyle{26.09}{Задание 3}{CS Center}
\setcounter{curtask}{13}

\begin{document}

\libproblem{discrete-math}{turing-example-xx}

\task{
    Пусть $\langcplx{LINEQ}$~--- язык выполнимых систем рациональных линейный
    уравнений. $\langcplx{LINEQ}$ состоит из пар $(A, b)$, где $A$~--- матрица $m \times n$, а $b$~---
    такой рациональный вектор размерности $m$, что система $Ax = b$ имеет решения. Докажите, что язык
    $\langcplx{LINEQ}$ лежит в классе $\NP$.
}

\libproblem{struct-complexity}{exactly-one-sat-npc}

\task{
    Пусть $L_1, L_2 \in \NP$. Принадлежит ли объединение этих языков $\NP$, а пересечение?
}

\dzcomment{
    Плохая задача.
}

\task{
    Пусть $\P = \NP$, докажите, что $\NP = \coNP$. ($\coNP$~--- класс языков, дополнение которых лежит в
    классе $\NP$).
}

%\libproblem{struct-complexity}{2-sat-p}

\breakline

\libproblem[9]{computability}{hor-ver-cuts}
\libproblem[10]{computability}{subset-split}
\task[12]{
    Рассмотрим пары $(S, k)$, где $S$~--- множество точек на плоскости, $k \in \mathbb{N}$, что выпуклая
    оболочка множества $S$ состоит из не более $k$ точек. Предъявите доказательство того, что выпуклая
    оболочка множества $S$ состоит из не более $k$ точек, которое можно проверить за $\bigO{|S|}$.
}

\end{document}



%%% Local Variables:
%%% mode: latex
%%% TeX-master: t
%%% End:
