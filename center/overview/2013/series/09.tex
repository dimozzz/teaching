\documentclass[a4paper, 12pt]{article}
% math symbols
\usepackage{amssymb}
\usepackage{amsmath}
\usepackage{mathrsfs}
\usepackage{mathseries}


\usepackage[margin = 2cm]{geometry}

\tolerance = 1000
\emergencystretch = 0.74cm



\pagestyle{empty}
\parindent = 0mm

\renewcommand{\coursetitle}{DM/ML}
\setcounter{curtask}{1}

\setmathstyle{06.11}{Задание 9}{CS Center}
\setcounter{curtask}{47}

\begin{document}

\libproblem{complexity}{decision-depth-bound}

\task{
    Постройте такую булеву схему, которая перемножает две квадратных булевых матрицы
    (сложение --- $xor$, а умножением $and$), что ее глубина:
    \begin{enumcyr}
        \item $\bigO{n^2}$;
        \item $\bigO{\log n}$.
    \end{enumcyr}
}

\libproblem{complexity}{ckt-lower-bound-count-hint}
\libproblem{complexity}{uncomputable-easy-ckt}

\breakline

\libproblem[39]{struct-complexity}{pcp-o-log}

\task[41]{
    Докажите, что язык $\SAT[2]$ лежит в $\DSPACE[\log^2 n]$.\\
    \hinttext{Подсказка:} какие есть способы проверки того, что в графе есть путь из одной вершины до
    другой?
}

\task[42]{
    Определим кванторную пропозициональную формулу: она имеет вид:
    $$
        Q_1 x_1 Q_2 x_x \dots Q_n x_n \varphi(x_1, x_2, \dots, x_n),
    $$
    где $\varphi$~--- пропозициональная формула от переменных $x_1, \dots, x_n$, а $Q_i \in \{\exists,
    \forall\}$~--- кванторы. Переменные $x_i$ принимают значения $\{0, 1\}$, истинность формулы
    определяется естественным образом. Обозначим $\langcplx{QBF}$~--- это множество истинных кванторных
    пропозициональных формул. Докажите, что $\langcplx{QBF}$ лежит в $\PSPACE$.
}

\libproblem[43]{struct-complexity}{ucycle-logspace}

\task[45]{
    Рассмотрим функцию $\Maj(x_1, x_2, x_3)$, которая возвращает бит, который чаще
    встречается на входе. Докажите, что $R(\Maj) \le \frac{8}{3}$.
}

\task[46]{
    Рассмотрим функцию $f = \bigvee\limits_{i = 1}^{n} x_i$. Докажите, что $R(f) = n$.
}


\end{document}



%%% Local Variables:
%%% mode: latex
%%% TeX-master: t
%%% End:
