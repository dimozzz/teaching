\documentclass[a4paper, 12pt]{article}
% math symbols
\usepackage{amssymb}
\usepackage{amsmath}
\usepackage{mathrsfs}
\usepackage{mathseries}


\usepackage[margin = 2cm]{geometry}

\tolerance = 1000
\emergencystretch = 0.74cm



\pagestyle{empty}
\parindent = 0mm

\renewcommand{\coursetitle}{DM/ML}
\setcounter{curtask}{1}

\setmathstyle{31.10}{Задание 8}{CS Center}
\setcounter{curtask}{40}

\begin{document}

\libproblem{struct-complexity}{logspace-composition}

\task{
    Докажите, что язык $\SAT[2]$ лежит в $\DSPACE[\log^2 n]$.\\
    \hinttext{Подсказка:} какие есть способы проверки того, что в графе есть путь из одной вершины до
    другой?
}

\task{
    Определим кванторную пропозициональную формулу: она имеет вид:
    $$
        Q_1 x_1 Q_2 x_x \dots Q_n x_n \varphi(x_1, x_2, \dots, x_n),
    $$
    где $\varphi$~--- пропозициональная формула от переменных $x_1, \dots, x_n$, а $Q_i \in \{\exists,
    \forall\}$~--- кванторы. Переменные $x_i$ принимают значения $\{0, 1\}$, истинность формулы
    определяется естественным образом. Обозначим $\langcplx{QBF}$~--- это множество истинных кванторных
    пропозициональных формул. Докажите, что $\langcplx{QBF}$ лежит в $\PSPACE$.
}

\libproblem{struct-complexity}{ucycle-logspace}
\libproblem{struct-complexity}{dspace-1-linear-time}

\begin{definition*}
    Напомним, деревом решений для функции $f$ назовем бинарное корневое дерево, в котором
каждая внутренняя вершина помечена некоторой переменной $x_i$, а лист значением $0$
или $1$. Ребра помечены значениями $0$ или $1$, причем у каждой внутренней вершины
один сын помечен ребром $0$, а другой $1$. Вычисление значения $f(x_1, \dots, x_n)$
начинается от корня. Проходя внутреннюю вершину, мы спрашиваем значение входной
переменной, которая соответствует метке вершины, после чего переходим в
соответствующее поддерево. Дойдя до листа мы выдаем значение, которое написано в нем.

Вероятностным деревом решения называется распределение на деревьях решений. Теперь
определим вероятностную запросовую сложность.

Стоимость дерева $t$ на входе $x$ (${cost}(t, x)$)~--- число переменных, значение
который было опрошено на пути к листу на входе $x$.

Для функции $f$ обозначим за $\mathcal{P}_f$ множество распределений на множестве
деревьев, которые вычисляют функцию $f$. Вероятностной запросовой сложностью функции
$f$ называется величина:
$R(f) = \min\limits_{P \in \mathcal{P}_f} \max\limits_{x \in \{0, 1\}^n}
E_{t \gets P}[{cost}(t, x)]$.

\end{definition*}

\task{
    Рассмотрим функцию $\Maj(x_1, x_2, x_3)$, которая возвращает бит, который чаще
    встречается на входе. Докажите, что $R(\Maj) \le \frac{8}{3}$.
}

\task{
    Рассмотрим функцию $f = \bigvee\limits_{i = 1}^{n} x_i$. Докажите, что $R(f) = n$.
}

\breakline

\libproblem[36]{struct-complexity}{rho-gap-max-sat}
\libproblem[39]{struct-complexity}{pcp-o-log}


\end{document}



%%% Local Variables:
%%% mode: latex
%%% TeX-master: t
%%% End:
