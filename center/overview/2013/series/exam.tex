\documentclass[a4paper, 12pt]{article}
% math symbols
\usepackage{amssymb}
\usepackage{amsmath}
\usepackage{mathrsfs}
\usepackage{mathseries}


\usepackage[margin = 2cm]{geometry}

\tolerance = 1000
\emergencystretch = 0.74cm



\pagestyle{empty}
\parindent = 0mm

\renewcommand{\coursetitle}{DM/ML}
\setcounter{curtask}{1}

\setmathstyle{}{Контрольная работа}{CS Center}
\setcounter{curtask}{55}

\begin{document}

Каждая задача контрольной может быть либо решена, либо не решена. За каждую решенную
задачу вы получаете указанное число баллов.  Если Вы заработали $A$ баллов за работу
в течение семестра и $B$ баллов за контрольную работу, то ваш итоговый балл за
практику $\min \{A + B, 80\}$.

Вы можете самостоятельно выбирать, какие задачи решать, исходя из того, сколько вам
нужно набрать баллов. Задачи следует записать и послать Д.О. Соколову по электронной
почте sokolov.dmt@gmail.com не позже среды 4-го декабря. В четверг 5-го декабря на
занятии можно будет получить результаты и исправить ошибки.


\libproblem[\textbf{15}]{computability}{enum-uncomp-numbers}
\libproblem[\textbf{10}]{computability}{prime-div}
\libproblem[\textbf{10}]{computability}{poly-time}

\task[\textbf{5}]{
    Докажите, что если $\NP \neq \coNP$, то $\P \neq \NP$.
}

\libproblem[\textbf{5}]{struct-complexity}{subgraph-isomorphism}
\libproblem[\textbf{5}]{struct-complexity}{pcp-log-p}
\libproblem[\textbf{5}]{struct-complexity}{pspace-in-exp}
\libproblem[\textbf{15}]{combinatorics}{infeasible-dual-lp}

\task{
    Пусть дан граф $G \coloneqq (V, E)$, $|V| = n$, $|E| = m$, $x \in \mathbb{R}^m$. Рассмотрим следующую
    задачу линейного программирования. $\sum\limits_{e \in E} x_e \rightarrow \max$, $\forall e \in E ~~
    x_e \ge 0$, $\forall v \in V ~~ \sum\limits_{e, v \in \delta(e)} x_e \le 1$, где $\delta(e)$~---
    множество концов ребра $e$.
    \begin{enumcyr}
        \item ($\mathbf{10}$) Докажите, что если граф $G$ двудольный, то оптимум достигается в вершине с
            целочисленными координатами.
        \item ($\mathbf{5}$) Предъявите пример, когда $G$ не является двудольным и максимум не
            целочисленный.
    \end{enumcyr}
}

\task[\textbf{15}]{
    Докажите, что существует такая функция $f\colon \{0, 1\}^{n} \rightarrow \{0, 1\}^n$, которая задает
    аффинное преобразование, что ее схемная сложность не менее $\frac{n^2}{100 \log n}$.
}

\libproblem[\textbf{25}]{cc}{clique-ind}
\libproblem[\textbf{30}]{cc}{median-hard}

\task[\textbf{10}]{
    Рассмотрим функцию $f(x_1, \dots, x_n) = x_1 \oplus x_2 \oplus \dots \oplus x_n$. Докажите, что $D(f)
    = n$ (детерминированная запросовая сложность).
}

\task[\textbf{20}]{
    Докажите, что для достаточно больших $n$ $R(Maj(x_1, x_2, \dots, x_n)) \le n - \frac{1}{4}$
    (вероятностная запросовая сложность).
}

\task[\textbf{10}]{
	Пусть пара случайных величин $X, Y$ совместно распределена на некотором конечном
    множестве. Докажите, что $\entropy(X, Y) \le \entropy(X) + \entropy(Y)$.
}


\libproblem[\textbf{15}]{error-correcting}{BCH-distance}
\libproblem[\textbf{5}]{error-correcting}{reed-solomon-erasure}

\end{document}



%%% Local Variables:
%%% mode: latex
%%% TeX-master: t
%%% End:
