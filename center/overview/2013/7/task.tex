\setcounter{curtask}{35}

\mytitle{7 (на 24.10)}

Напомним: язык $L \in PCP(r(n), q(n))$, если существует система доказательств для
языка $L$, которая использует $O(r(n))$ случайных битов и делает $O(q(n))$ запросов
к битам доказательства. Если $x \in L$, то найдется доказательство, которое
принимается с вероятностью $1$, если $x \notin L$, то любое доказательство
принимается с вероятностью меньше $\frac{1}{2}$.


\begin{task}
    Докажите, что $\mathrm{PCP}(0, \log(n)) = \mathrm{P}$.
\end{task}

\begin{task}
    Докажите, что если $\rho${-gap}{MAX-q-SAT}, где $\rho < 1$ является
    $\mathrm{NP}$-трудной (т.е. для любого языка из класса $\mathrm{NP}$ существует
    полиномиальное сведение к данной задаче, (элементы языка сводятся к выполнимым
    формулам, а элементы не из языка сводятся к формулам, для которых можно выполнить
    не более $(1 - \rho)$ клозов)), то существует $\rho' < 1$, что
    $\rho'${-gap}{MAX-3-SAT}  также является $\mathrm{NP}$-трудной.
\end{task}

\begin{task}

    Рассмотрим язык $INDSET$, состоящий из таких пар $(G, k)$, где $G$~--- граф,
    $k$~--- натуральное число, что в графе $G$ есть независимое множество вершин
    размера $k$ (независимым называется такое множество вершин, что между любыми
    двумя вершинами из данного множества нет ребра). Задача $MAXINDSET$~--- задача
    поиска по графу максимального независимого множества вершин.
    
    Докажите, что:

    а) $3SAT$ сводится к задаче $MAXINDSET$ (подсказка: для каждого клоза создать $7$
    вершин, соответствующих выполняющим наборам).

    б) Существует такая константа $\rho$, что если $\mathrm{P} \neq \mathrm{NP}$, то
    не существует $\rho$-приближенного алгоритма для задачи $MAXINDSET$.
\end{task}

\begin{task}
    Докажите, что язык:
    а) палиндромов над конечным алфавитом;
    б) $L = \{a^nb^n \min n \in \mathbb{N}\}$ не является регулярным (автоматным).
\end{task}

\begin{task}
    Докажите, что если $SAT \in \mathrm{PCP}(o(\log(n)), 1)$, то $P = NP$.
\end{task}

\breakline

\begin{ptask}{29}
    Рассмотрим полиэдр $P = \{x \mid Ax = b, x \ge 0\}$, где $b$~--- вектор с
    целочисленными координатами, а матрица $A$~--- тотально унимодулярна. Докажите,
    что все вершины данного полиэдра имеют целочисленные координаты.
\end{ptask}