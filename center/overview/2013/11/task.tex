\setcounter{curtask}{55}

\mytitle{11 (на 21.11)}

\begin{task}
    Докажите, что коммуникационная сложность функции $GT: \{0, 1\}^n \times \{0,
    1\}^n \rightarrow \{0, 1\}$, которая равна $1$ тогда и только тогда,
    когда $x > y$ (как натуральные числа в двоичной записи), не менее $n$.
\end{task}

\begin{task}
    Пусть у Алисы и Боба есть множества $X, Y \subseteq \{1, \dots, n\}$. Они хотят
    посчитать функцию $AVG(X, Y)$, которая возвращает среднее значение (не
    обязательно целое) мультимножества $X \cup Y$. Докажите, что для этого им
    достаточно $O(\log(n))$ битов коммуникации.
\end{task}

\begin{task}
    Пусть у Алисы и Боба есть множества $X, Y \subseteq \{1, \dots, n\}$. Они хотят
    посчитать функцию $MED(X, Y)$, которая возвращает медиану мультимножества $X \cup
    Y$. Докажите, что для этого им достаточно:
    а) $O(n)$ б) $O(\log^2(n))$ битов коммуникации.
\end{task}

\begin{task}
    Докажите, что если матрицу $A \in \{0, 1\}^{n \times m}$ можно разбить на $k$
    одноцветных прямоугольников, то ранг данной матрицы (над полем $\mathbb{Z}_2$) не
    превосходит $k$. (подсказка: ранг суммы не превосходит суммы рангов).
\end{task}

%\begin{task}
%    Докажите, что коммуникационная сложность функции $MED$ не превосходит $O(\log(n))$.
%\end{task}


\breakline

\begin{ptask}{41}
	Докажите, что задача $2SAT$ лежит в $DSpace[\log^2(n)]$.    
\end{ptask}

\begin{ptask}{42}
	Определим кванторную пропозициональную формулу: она имеет вид \\
    $Q_1 x_1 Q_2 x_x \dots Q_n x_n \phi(x_1, x_2, \dots, x_n)$, где $\phi$~---
    пропозициональная формула от переменных $x_1, \dots, x_n$, а $Q_i \in \{\exists,
    \forall\}$~--- кванторы. Переменные $x_i$ принимают значения $\{0, 1\}$,
    истинность формулы определяется естественным образом. Обозначим $TQBF$~--- это
    множество истинных кванторных пропозициональных формул. Докажите, что $TQBF$
    лежит в $\mathrm{PSpace}$.
\end{ptask}

\begin{ptask}{43}
    Докажите, что язык графов с циклом лежит в классе $\mathrm{DSpace}[\log(n)]$.
\end{ptask}

\begin{ptask}{51}
	Алиса задумывает целое число от $1$ до $n$. Боб должен отгадать это число,
    задавая Алисе вопросы, требующие ответы да или нет. Алиса может солгать в одном
    из ответов. Стратегия Боба называется адаптивной, если очередной задаваемый
    вопрос может зависеть от ответов, данных Алисой на предыдущих шагах. Стратегия
    называется неадаптивной, если Боб сразу предъявляет список всех своих вопросов,
    не дожидаясь первых ответов Алисы.

	a) Какое минимальное число вопросов Должен задать Боб, чтобы гарантированно
    узнать задуманное Алисой число для $n = 200$. (для адаптивной стратегии)?
    
	b) Какое минимальное число неадаптивных вопросов должен задать Боб для $n = 150$.
    
	c) Какое минимальное число адаптивных вопросов должен задать Боб для $n = 150$.
\end{ptask}

\begin{ptask}{52} (The Hat Problem)
    В комнате находятся $n$ человек. На каждом человеке находится шляпа черного или
    белого цвета. Шляпа выдаются случайным образом независимо друг от друга. Каждый
    человек может видеть шляпы всех остальных людей, но не может видеть свою. Каждого
    человека спрашивают, не хочет ли он попробовать угадать цвет своей шляпы. Человек
    может попробовать или отказаться. Каждый человек делает выбор, не зная ответы
    остальных людей. Выигрывают или проигрывают люди вместе. Они выигрывают, если
    все, кто решил отвечать отвечают верно, и хотя бы один человек отвечает. Во всех
    других случаях люди проигрывают.

	a) Назовем граф $G$ ориентированным подграфом $n$-мерного гиперкуба, если его
    вершины соответствуют бинарным строкам длины $n$ и если существует ребро $u
    \rightarrow v$, то строки $u, v$ различаются не более, чем в одном бите. Пусть
    $K(G)$~--- количество вершин в графе $G$ со входящей степенью не менее $1$ и
    исходящей $0$. Покажите, что вероятность победы в игре равна максимуму по выборам
    подграфа $n$-мерного гиперкуба величины $K(G) / 2^n$.

    б) Используя факт, что исходящая степень вершин не превосходит $n$, покажите, что
    $K(G) / 2^n \le \frac{n}{n + 1}$ для любого графа $G$ подграфа $n$-мерного
    гиперкуба.

    в) Покажите, что если $n = 2^l - 1$, то существует граф $G$, для которого
	$K(G) / 2^n = \frac{n}{n + 1}$. (Подсказка: используйте коды Хэмминга).
\end{ptask}

\begin{ptask}{53}
	Добавим к кодовым словам кода Хэмминга бит проверки четности: значение
	добавленного бита выбирается так, чтобы число единиц в каждом кодовом слове было
    бы чётно. Понятно, что число кодовых слов при этом не меняется, а их длина
    увеличивается на $1$. Как при этом изменяется кодовое расстояние?
\end{ptask}

\begin{ptask}{54}
    Код $C: \{0, 1\}^k \rightarrow \{0, 1\}^n$ называется систематическим, если
    существуют такие числа $j_1, \dots, j_k$, что для любого $(x_1, \dots, x_k) \in
    \{0, 1\}^k$ в кодовом слове $(y_1, \dots, y_n) = C(x_1, \dots, x_n)$ биты
    $y_{j_1}, \dots, y_{j_k}$ равны соответствующим битам исходного слова. Другими
    словами, все ``информационные биты'' непосредственно входят в кодовое слово.

    Докажите, что всякий линейный код $C: \{0, 1\}^k \rightarrow \{0, 1\}^n$ можно
    переделать в систематический линейный код $C': \{0, 1\}^k \rightarrow \{0,
    1\}^n$, сохранив прежнее множество кодовых слов (и проверочную матрицу).
\end{ptask}