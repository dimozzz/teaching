\setcounter{curtask}{40}

\mytitle{8 (на 31.10)}


\begin{task}
    Пусть функции $f, g: \{0, 1\}^* \rightarrow \{0, 1\}^*$ можно посчитать с
    использованием $O(\log(n))$ памяти (напомним, что память считается только на
    рабочих лентах, входная лента доступна только для чтения, а по выходной ленте
    головка машины Тьюринга движется только слева направо). Докажите, что функцию
    $f(g(x))$ можно также посчитать с использованием $O(\log(n))$ памяти.
\end{task}

\begin{task}
	Докажите, что задача $2SAT$ лежит в $DSpace[\log^2(n)]$.    
\end{task}

\begin{task}
	Определим кванторную пропозициональную формулу: она имеет вид \\
    $Q_1 x_1 Q_2 x_x
    \dots Q_n x_n \phi(x_1, x_2, \dots, x_n)$, где $\phi$~--- пропозициональная
    формула от переменных $x_1, \dots, x_n$, а $Q_i \in \{\exists, \forall\}$~---
    кванторы. Переменные $x_i$ принимают значения $\{0, 1\}$, истинность формулы
    определяется естественным образом. Обозначим $TQBF$~--- это множество истинных
    кванторных пропозициональных формул.
    Докажите, что $TQBF$ лежит в $\mathrm{PSpace}$.
\end{task}

\begin{task}
    Докажите, что язык графов с циклом лежит в классе $DSpace[\log(n)]$.
\end{task}

\begin{task}
    Докажите, что языки из класса $DSpace[1]$ можно распознать за линейное время.
\end{task}


Напомним, деревом решений для функции $f$ назовем бинарное корневое дерево, в котором
каждая внутренняя вершина помечена некоторой переменной $x_i$, а лист значением $0$
или $1$. Ребра помечены значениями $0$ или $1$, причем у каждой внутренней вершины
один сын помечен ребром $0$, а другой $1$. Вычисление значения $f(x_1, \dots, x_n)$
начинается от корня. Проходя внутреннюю вершину, мы спрашиваем значение входной
переменной, которая соответствует метке вершины, после чего переходим в
соответствующее поддерево. Дойдя до листа мы выдаем значение, которое написано в нем.

Вероятностным деревом решения называется распределение на деревьях решений. Теперь
определим вероятностную запросовую сложность.

Стоимость дерева $t$ на входе $x$ (${cost}(t, x)$)~--- число переменных, значение
который было опрошено на пути к листу на входе $x$.

Для функции $f$ обозначим за $\mathcal{P}_f$ множество распределений на множестве
деревьев, которые вычисляют функцию $f$. Вероятностной запросовой сложностью функции
$f$ называется величина:
$R(f) = \min\limits_{P \in \mathcal{P}_f} \max\limits_{x \in \{0, 1\}^n}
E_{t \gets P}[{cost}(t, x)]$.


\begin{task}
    Рассмотрим функцию $f = Maj(x_1, x_2, x_3)$, которая возвращает бит, который чаще
    встречается на входе. Докажите, что $R(f) \le \frac{8}{3}$
\end{task}

\begin{task}(!Исправлено!)
    Рассмотрим функцию $f = \bigvee\limits_{i = 1}^{n} x_i$. Докажите, что $R(f) = n$.
\end{task}




\breakline

\begin{ptask}{36}
    Докажите, что если $\rho${-gap}{MAX-q-SAT}, где $\rho < 1$ является
    $\mathrm{NP}$-трудной (т.е. для любого языка из класса $\mathrm{NP}$ существует
    полиномиальное сведение к данной задаче, (элементы языка сводятся к выполнимым
    формулам, а элементы не из языка сводятся к формулам, для которых можно выполнить
    не более $(1 - \rho)$ клозов)), то существует $\rho' < 1$, что
    $\rho'${-gap}{MAX-3-SAT}  также является $\mathrm{NP}$-трудной.
\end{ptask}


\begin{ptask}{39}
    Докажите, что если $SAT \in \mathrm{PCP}(o(\log(n)), 1)$, то $P = NP$.
\end{ptask}
