\setcounter{curtask}{31}

\mytitle{6 (на 17.10)}

\begin{task}
	Рассмотрим множество матриц $\mathcal{A} \subseteq \{-1, 0, 1\}^{n \times m}$,
	которые обладают следующими свойствами:
    \begin{itemize}
		\item в любом столбце не более двух ненулевых элементов;
    	\item существует такое разбиение множества строк на классы $X, Y$, что
		    для любого столбца $j$, в котором ровно два ненулевых элемента верно:
            $\sum\limits_{i \in X}{a_{i,j}} = \sum\limits_{i \in Y}{a_{i,j}}$.
    \end{itemize}

    Докажите, что:
    
    а) данное множество замкнуто относительно перехода к подматрице;
    
    б) если в любом столбце матрицы $A \in \mathcal{A}$ два ненулевых элемента, то
	$det(A) = 0$;
   
    в) если $A \in \mathcal{A}$, то $A$~--- тотально унимодулярна.
\end{task}


\begin{task}
    Пусть дан граф $G = (V, E)$, $|V| = n$, $|E| = m$, $x \in
    \mathbb{R}^m$. Рассмотрим следующую задачу линейного программирования.
    $\sum_{e \in E} x_e \rightarrow max$, $\forall e \in E ~~ x_e \ge 0$,
    $\forall v \in V ~~ \sum_{e, v \in \delta(e)} x_e \le 1$, где $\delta(e)$~---
    множество концов ребра $e$.

    Докажите, что если граф $G$ двудольный, то матрица данной задачи является
    тотально унимодулярной.
\end{task}

%\begin{task}
%    Предъявите такую несовместную задачу линейного программирования, что двойственная
%    задача будет также несовместна.
%\end{task}

%\begin{task}
 %   Пусть $G = (V, E)$~--- двудольный граф. Реберное покрытие $G$~--- это такое
  %  множество $X \subseteq E$, что любая вершина $v \in V$ инцидентна хотя бы одному
%    ребру из множества $X$.

 % 	а) Опишите задачу линейного программирования для нахождения минимального
  %  реберного покрытия.

%    б) Докажите, что если граф регулярный, то в минимальном покрытие не более
%    $\frac{|V|}{2}$ ребер.

%    в) Постройте двойственную задачу линейного программирования и изучите ее
%    комбинаторный смысл.
%\end{task}


\vspace{2cm}
Пусть $f: \{0, 1\}^n \rightarrow \{0, 1\}$ некоторая функция. Деревом решений для
функции $f$ назовем бинарное корневое дерево, в котором каждая внутренняя вершина
помечена некоторой переменной $x_i$, а лист значением $0$ или $1$. Ребра помечены
значениями $0$ или $1$, причем у каждой внутренней вершины один сын помечен ребром
$0$, а другой $1$. Вычисление значения $f(x_1, \dots, x_n)$ начинается от
корня. Проходя внутреннюю вершину, мы спрашиваем значение входной переменной, которая
соответствует метке вершины, после чего переходит в соответствующее поддерево. Дойдя
до листа мы выдаем значение, которое написано в нем.

%Стоимость дерева $t$ на входе $x$ (${cost}(t, x)$)~--- число переменных, значение
%который было опрошено на пути к листу на входе $x$.

Запросовой сложностью функции $f$ называется величина $D(f)$ равная минимальной
глубине дерева, которое считает данную функцию.

%$= \min\limits_{t \in \mathcal{T}_f} \max\limits_{x \in \{0, 1\}^n} {cost}(t, x)$, где
%$\mathcal{T}_f$~--- множество деревьев, которые вычисляют функцию $f$.

%Для функции $f$ обозначим за $\mathcal{P}_f$ множество распределений на множестве
%деревьев, которые вычисляют функцию $f$. Вероятностной запросовой сложностью функции
%$f$ называется величина:
%$R(f) = \min\limits_{P \in \mathcal{P}_f} \max\limits_{x \in \{0, 1\}^n}
%E[{cost}(t, x)]$, где $E$~--- мат. ожидание по соответствующему распределению.


\begin{task}
    Рассмотрим функцию $f = Maj(x_1, x_2, x_3)$, которая возвращает бит, который чаще
    встречается на входе. Докажите, что $D(f) = 3$.

    %б) $R(f) \le \frac{8}{3}$
\end{task}

\begin{task}
    Рассмотрим функцию $f = \bigvee\limits_{i = 1}^{n} x_i$. Докажите, что $D(f) = n$.
\end{task}

\breakline

\begin{ptask}{24}
	Докажите, что если существует унарный $NP$-полный язык, то $P = NP$.    
\end{ptask}

\begin{ptask}{25}
	Рассмотрим полиэдр $P = \{x \mid Ax \le b \}$. Матрица $A$ имеет размер
    $m$ на $n$ и ее строки линейно независимы. Доказать, что число вершин этого
    полиэдра не превосходит числа сочетаний из $n$ по $m$.    
\end{ptask}

\begin{ptask}{26}
   Докажите, что если полиэдр $P = \{x \mid Ax \le b\} \subseteq \mathbb{R}^n$ не пуст, то
   он имеет вершину тогда и только тогда, когда ранг матрицы $A$ равен $n$.
\end{ptask}

\begin{ptask}{27}
    Докажите, что любая точка выпуклого многогранника в $\mathbb{R}^n$ есть выпуклая
    комбинация не более, чем $n + 1$ вершины.
\end{ptask}

\begin{ptask}{29}
    Рассмотрим полиэдр $P = \{x \mid Ax = b, x \ge 0\}$, где $b$~--- вектор с
    целочисленными координатами, а матрица $A$~--- тотально унимодулярна. Докажите,
    что все вершины данного полиэдра имеют целочисленные координаты.
\end{ptask}


\begin{ptask}{30}
    Пусть дан граф $G = (V, E)$, $|V| = n$, $|E| = m$, $x \in
    \mathbb{R}^m$. Рассмотрим следующую задачу линейного программирования.
    $\sum_{e \in E} x_e \rightarrow max$, $\forall e \in E ~~ x_e \ge 0$,
    $\forall v \in V ~~ \sum_{e, v \in \delta(e)} x_e \le 1$, где $\delta(e)$~---
    множество концов ребра $e$.

    а) Какой ``физический'' смысл у данной задачи? А если вектор $x$ имеет
    целочисленные координаты?

    б) Докажите, что если граф $G$ двудольный, то оптимум достигается в вершине с
    целочисленными координатами.
\end{ptask}