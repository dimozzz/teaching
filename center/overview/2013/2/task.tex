\setcounter{curtask}{6}

\mytitle{2 (на 19.09)}

\begin{task}
    Доказать, что полуразрешимое множество является перечислимым.
\end{task}

\begin{task}
    Докажите, что множество всех рациональных чисел меньших $e$ разрешимо.
\end{task}

\begin{task}
    Пусть $S \subseteq \mathbb{N}$ состоит из таких чисел $n$, что в десятичной
    записи числа $\pi$ содержится $n$ девяток подряд. Разрешимо ли множество $S$?
\end{task}

\begin{task}
    Приведите пример неразрешимого множества $A \subseteq \Nat \times \Nat$,
    такого, что все его горизонтальные и вертикальные сечения
    разрешимы (т.е. для любого $x$ разрешимы $A \cap \{\{x\} \times \Nat\}$
    и $A \cap \{\Nat \times \{x\}\}$)
\end{task}

\begin{task}
    Пусть $X$, $Y$~--- перечислимые множества. Докажите, что всегда
    найдутся такие перечислимые $X' \subseteq X$, $Y' \subseteq Y$,
    что $X' \cup Y' = X \cup Y$ и $X' \cap Y' = \emptyset$.
\end{task}

\begin{task}
    Приведите пример двух непересекающиеся неперечислимых множеств.
\end{task}

\begin{task}
    Рассмотрим пары $(S, k)$, где $S$~--- множество точек на плоскости, $k
    \in \mathbb{N}$, что выпуклая оболочка множества $S$ состоит из не более $k$
    точек. Предъявите доказательство того, что выпуклая оболочка множества $S$
    состоит из не более $k$ точек, которое можно проверить за $O(|S|)$.
\end{task}

\breakline

\begin{ptask}{5}
    Приведите пример:
    
    а) неперечислимого множества;

    б) такого неперечислимого множества, что дополнение его также не является
    перечислимым.
\end{ptask}