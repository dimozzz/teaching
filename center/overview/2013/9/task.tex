\setcounter{curtask}{47}

\mytitle{9 (на 6.11)}

\begin{task}
    Предъявите функцию, которая существенно зависит от всех своих аргументов,
    детерминированная запросовая сложность которой не превосходит $\log(n)$.
\end{task}

\begin{task}
    Постройте такую булеву схему, которая перемножает две квадратных булевых матрицы
    (сложение --- $xor$, а умножением $and$), что ее глубина:
    а) $O(n^2)$
    б) $O(\log(n))$.
\end{task}

\begin{task}
    а) Сколько существует булевых функций от $n$ переменных?
    б) Сколько существует булевых схем от $n$ переменных размера $s$?
    в) Докажите, что существует булева функция от $n$ переменных, для подсчета
    которой необходима схема размером не менее $\frac{2^n}{100n}$
\end{task}

\begin{task}
    Семейство булевых функций $f_n: \{0, 1\}^n \rightarrow \{0, 1\}$ имеет схемную
    сложность не более $s(n)$, если для любого $n$ найдется такая булева схема
    размера не более $s(n)$, что она вычисляет функцию $f_n$.
    Докажите, что существует неразрешимый язык, схемная сложность которого не
    превосходит $n$.
\end{task}

\breakline

\begin{ptask}{39}
    Докажите, что если $SAT \in \mathrm{PCP}(o(\log(n)), 1)$, то $P = NP$.
\end{ptask}

\begin{ptask}{41}
	Докажите, что задача $2SAT$ лежит в $DSpace[\log^2(n)]$.    
\end{ptask}

\begin{ptask}{42}
	Определим кванторную пропозициональную формулу: она имеет вид \\
    $Q_1 x_1 Q_2 x_x \dots Q_n x_n \phi(x_1, x_2, \dots, x_n)$, где $\phi$~---
    пропозициональная формула от переменных $x_1, \dots, x_n$, а $Q_i \in \{\exists,
    \forall\}$~--- кванторы. Переменные $x_i$ принимают значения $\{0, 1\}$,
    истинность формулы определяется естественным образом. Обозначим $TQBF$~--- это
    множество истинных кванторных пропозициональных формул. Докажите, что $TQBF$
    лежит в $\mathrm{PSpace}$.
\end{ptask}

\begin{ptask}{43}
    Докажите, что язык графов с циклом лежит в классе $\mathrm{DSpace}[\log(n)]$.
\end{ptask}

\begin{ptask}{45}
    Рассмотрим функцию $f = Maj(x_1, x_2, x_3)$, которая возвращает бит, который чаще
    встречается на входе. Докажите, что $R(f) \le \frac{8}{3}$
\end{ptask}

\begin{ptask}{46}
    Рассмотрим функцию $f = \bigvee\limits_{i = 1}^{n} x_i$. Докажите, что $R(f) = n$.
\end{ptask}