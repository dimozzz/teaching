\documentclass[12pt, fleqn, a4paper]{article}


\usepackage{amsmath}
\usepackage{amssymb}
\usepackage{amsfonts}
\usepackage{textcomp}
\usepackage{amsthm}
\usepackage{mathtools}
\usepackage{xspace}
\usepackage[classfont = bold]{complexity}
\usepackage{fullpage}
\usepackage[russian, english]{babel}
\usepackage[utf8]{inputenc}
\usepackage[
    sorting = ydnt,
    style = alphabetic,
    maxbibnames = 99,
    backend = biber
    ]{biblatex}
\addbibresource{main.bib}



\newtheorem{conjecture}{Conjecture}[section]
\theoremstyle{definition}
\newtheorem{theorem}{Theorem}[section]
\newtheorem*{theorem*}{Theorem}
\newtheorem{lemma}{Lemma}[section]
\newtheorem{corollary}{Corollary}[section]
\newtheorem{proposition}{Proposition}[section]
\newtheorem{fact}{Fact}[section]
\newtheorem{problem}{Problem}[section]
\newtheorem{exercise}{Exercise}[section]
\newtheorem{example}{Example}[section]
\newtheorem{definition}{Definition}[section]
\newtheorem{remark}{Remark}[section]
\newtheorem{algorithm}{Algorithm}[section]


\newcommand{\class}[1]{\mathbf{#1}}
\newcommand{\co}{\mathrm{co}}
\newcommand{\alg}[1]{\mathit{#1}}
\newcommand{\lang}[1]{\mathtt{#1}}


% classes (1)

\newcommand{\DTime}{\class{DTime}}
\newcommand{\RTime}{\class{RTime}}
\newcommand{\UTime}{\class{UTime}}
\newcommand{\NTime}{\class{NTime}}
\newcommand{\BPTime}{\class{BPTime}}


\renewcommand{\P}{\class{P}}
\newcommand{\ZPP}{\class{ZPP}}
\newcommand{\RP}{\class{RP}}
\newcommand{\coRP}{\co\class{RP}}
\newcommand{\UP}{\class{UP}}
\newcommand{\coUP}{\co\class{UP}}
\newcommand{\NP}{\class{NP}}
\newcommand{\coNP}{{\co}\class{NP}}
\newcommand{\BPP}{\class{BPP}}
\newcommand{\SigmaP}[1]{\Sigma^{#1}\class{P}}
\newcommand{\PH}{\class{PH}}
\newcommand{\PP}{\class{PP}}
\newcommand{\IP}{\class{IP}}
\newcommand{\OP}{\class{\oplus P}}


\newcommand{\EXP}{\class{EXP}}
\newcommand{\MIP}{\class{MIP}}
\newcommand{\NEXP}{\class{NEXP}}
\newcommand{\coNEXP}{{\co}\class{NEXP}}
\newcommand{\MAEXP}{\class{MA}_\class{EXP}}


% classes (2)

\newcommand{\Ppoly}{\class{P}/\class{poly}}
\newcommand{\NC}{\class{NC}}


\newcommand{\DSpace}{\class{DSpace}}
\newcommand{\NSpace}{\class{NSpace}}
\newcommand{\PSPACE}{\class{PSPACE}}

\newcommand{\EXPSPACE}{\class{EXPSPACE}}


% algorithms and proof systems


\newcommand{\DPLL}{\alg{DPLL}}
\newcommand{\OBDD}{\alg{OBDD}}
\newcommand{\pOBDD}{\pi\text{-}\alg{OBDD}}
\newcommand{\DPLLL}{\alg{DPLL}_{lin}}
\newcommand{\ResL}{\alg{Res}_{lin}}
\newcommand{\SemL}{\alg{Sem}_{lin}}



% languages


\newcommand{\SAT}{\lang{SAT}}
\newcommand{\GNI}{\lang{GNI}}
\newcommand{\MAJSAT}{\lang{MAJ}\text{-}\lang{SAT}}
\newcommand{\QBF}{\lang{QBF}}



% other

\newcommand{\poly}{\mathrm{poly}}
\newcommand{\Nat}{\mathbb{N}}
\newcommand{\bool}{\{0, 1\}}

\newcommand{\Img}{\mathop{\mathrm{Im}}}

\DeclareMathOperator*{\supp}{supp}
\DeclareMathOperator*{\Exp}{E}
\DeclareMathOperator*{\rk}{rk}




%%% Local Variables:
%%% mode: latex
%%% TeX-master: t
%%% End:


\begin{document}

	\setcounter{curtask}{9}

\mytitle{2 (на 3.10)}

\begin{task}
    Докажите, что множество всех рациональных чисел меньших $\pi$ разрешимо.
\end{task}

\begin{task}
    Существует ли алгоритм, проверяющий, работает ли данная программа
    полиномиальное время?
\end{task}

\begin{task}
    Приведите пример двух непересекающихся неперечислимых множеств.
\end{task}

\begin{task}
    Докажите, что для каждой вычислимой функции $f$ найдется
    псевдообратная вычислимая функция $g$. А именно, $g$ определена на
    множестве значений $f$, и для всех $x$ из области определения $f$
    выполняется $f(g(f(x))) = f(x)$.
\end{task}

\begin{task}
    Приведите пример неразрешимого множества $A \subseteq \Nat \times \Nat$,
    такого, что все его горизонтальные и вертикальные сечения
    разрешимы (т.е. для любого $x$ разрешимы $A \cap \{\{x\} \times \Nat\}$
    и $A \cap \{\Nat \times \{x\}\}$)
\end{task}

\begin{task}
    Докажите, что существует язык, который можно распознать с памятью $2^n$ ($n$~---
    длина слова), но нельзя с памятью $n$. (подсказка: диагонализация)
\end{task}

\end{document}



\renewcommand{\coursetitle}{}
\setcounter{curtask}{1}

\begin{document}

\libproblem[\textbf{10}]{computability}{enum-uncomp-numbers}
\libproblem[\textbf{10}]{computability}{prime-div}
\libproblem[\textbf{10}]{computability}{poly-time}

\task[\textbf{5}]{
    Докажите, что если $\NP \neq \coNP$, то $\P \neq \NP$.
}

\libproblem[\textbf{10}]{struct-complexity}{subgraph-isomorphism}
\libproblem[\textbf{5}]{struct-complexity}{pcp-log-p}
\libproblem[\textbf{5}]{struct-complexity}{pspace-in-exp}

\task{
    Пусть дан граф $G \coloneqq (V, E)$, $|V| = n$, $|E| = m$, $x \in \mathbb{R}^m$. Рассмотрим следующую
    задачу линейного программирования. $\sum\limits_{e \in E} x_e \rightarrow \max$, $\forall e \in E ~~
    x_e \ge 0$, $\forall v \in V ~~ \sum\limits_{e, v \in \delta(e)} x_e \le 1$, где $\delta(e)$~---
    множество концов ребра $e$.
    \begin{enumcyr}
        \item ($\mathbf{10}$) Докажите, что если граф $G$ двудольный, то оптимум достигается в вершине с
            целочисленными координатами.
        \item ($\mathbf{5}$) Предъявите пример, когда $G$ не является двудольным и максимум не
            целочисленный.
    \end{enumcyr}
}

\task[15]{
    Докажите, что существует такая функция $f\colon \{0, 1\}^{n} \rightarrow \{0, 1\}^n$, которая задает
    аффинное преобразование, что ее схемная сложность не менее $\frac{n^2}{100 \log n}$.
}

\libproblem[\textbf{20}]{cc}{clique-ind}
\libproblem[\textbf{25}]{cc}{median-hard}

\task[\textbf{5}]{
    Рассмотрим функцию $f(x_1, \dots, x_n) = x_1 \oplus x_2 \oplus \dots \oplus x_n$. Докажите, что $D(f)
    = n$ (детерминированная запросовая сложность).
}

\libproblem[\textbf{15}]{error-correcting}{BCH-distance}
\libproblem[\textbf{20}]{computability}{non-separable}
\libproblem[\textbf{10}]{cc}{rand-eq}


\end{document}