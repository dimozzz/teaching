\documentclass[a4paper, 12pt]{article}
% math symbols
\usepackage{amssymb}
\usepackage{amsmath}
\usepackage{mathrsfs}
\usepackage{mathseries}


\usepackage[margin = 2cm]{geometry}

\tolerance = 1000
\emergencystretch = 0.74cm



\pagestyle{empty}
\parindent = 0mm

\renewcommand{\coursetitle}{DM/ML}
\setcounter{curtask}{1}

\renewcommand{\coursetitle}{}
\setcounter{curtask}{1}

\begin{document}

\libproblem[\textbf{10}]{computability}{enum-uncomp-numbers}
\libproblem[\textbf{10}]{computability}{prime-div}
\libproblem[\textbf{10}]{computability}{poly-time}

\task[\textbf{5}]{
    Докажите, что если $\NP \neq \coNP$, то $\P \neq \NP$.
}

\libproblem[\textbf{10}]{struct-complexity}{subgraph-isomorphism}
\libproblem[\textbf{5}]{struct-complexity}{pcp-log-p}
\libproblem[\textbf{5}]{struct-complexity}{pspace-in-exp}

\task{
    Пусть дан граф $G \coloneqq (V, E)$, $|V| = n$, $|E| = m$, $x \in \mathbb{R}^m$. Рассмотрим следующую
    задачу линейного программирования. $\sum\limits_{e \in E} x_e \rightarrow \max$, $\forall e \in E ~~
    x_e \ge 0$, $\forall v \in V ~~ \sum\limits_{e, v \in \delta(e)} x_e \le 1$, где $\delta(e)$~---
    множество концов ребра $e$.
    \begin{enumcyr}
        \item ($\mathbf{10}$) Докажите, что если граф $G$ двудольный, то оптимум достигается в вершине с
            целочисленными координатами.
        \item ($\mathbf{5}$) Предъявите пример, когда $G$ не является двудольным и максимум не
            целочисленный.
    \end{enumcyr}
}

\task[15]{
    Докажите, что существует такая функция $f\colon \{0, 1\}^{n} \rightarrow \{0, 1\}^n$, которая задает
    аффинное преобразование, что ее схемная сложность не менее $\frac{n^2}{100 \log n}$.
}

\libproblem[\textbf{20}]{cc}{clique-ind}
\libproblem[\textbf{25}]{cc}{median-hard}

\task[\textbf{5}]{
    Рассмотрим функцию $f(x_1, \dots, x_n) = x_1 \oplus x_2 \oplus \dots \oplus x_n$. Докажите, что $D(f)
    = n$ (детерминированная запросовая сложность).
}

\libproblem[\textbf{15}]{error-correcting}{BCH-distance}
\libproblem[\textbf{20}]{computability}{non-separable}
\libproblem[\textbf{10}]{cc}{rand-eq}


\end{document}