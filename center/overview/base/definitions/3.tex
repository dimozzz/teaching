Пусть $f: \{0, 1\}^n \rightarrow \{0, 1\}$ некоторая функция. Деревом решений для функции $f$ назовем бинарное корневое
дерево, в котором каждая внутренняя вершина помечена некоторой переменной $x_i$, а лист значением $0$ или $1$. Ребра помечены
значениями $0$ или $1$, причем у каждой внутренней вершины один сын помечен ребром $0$, а другой $1$. Вычисление значения
$f(x_1, \dots, x_n)$ начинается от корня. Проходя внутреннюю вершину, мы спрашиваем значение входной переменной, которая
соответствует метке вершины, после чего переходит в соответствующее поддерево. Дойдя до листа мы выдаем значение, которое
написано в нем.
