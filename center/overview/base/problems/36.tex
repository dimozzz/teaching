Пусть дан граф $G = (V, E)$, $|V| = n$, $|E| = m$, $x \in \mathbb{R}^m$. Рассмотрим следующую задачу
линейного программирования. $\sum\limits_{e \in E} x_e \rightarrow max$, $\forall e \in E ~~ x_e \ge 0$,
$\forall v \in V ~~ \sum\limits_{E_v} x_e \le 1$, где $E_v$~--- множество ребер, инцидентных вершине $v$.
\begin{enumcyr}
	\item Какой ``физический'' смысл у данной задачи? А если вектор $x$ имеет целочисленные координаты?
    \item Докажите, что если граф $G$ двудольный, то оптимум достигается в вершине с целочисленными
	    координатами.
\end{enumcyr}