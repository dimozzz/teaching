\documentclass[a4paper, 12pt]{article}
% math symbols
\usepackage{amssymb}
\usepackage{amsmath}
\usepackage{mathrsfs}
\usepackage{mathseries}


\usepackage[margin = 2cm]{geometry}

\tolerance = 1000
\emergencystretch = 0.74cm



\pagestyle{empty}
\parindent = 0mm

\renewcommand{\coursetitle}{DM/ML}
\setcounter{curtask}{1}

\setmathstyle{}{Задание 10}{CS Center}
\setcounter{curtask}{51}

\begin{document}

\libproblem{struct-complexity}{logspace-composition}

\begin{definition*}
    Класс $\NSPACE[f(n)]$~--- это класс языков, для которых существует система доказательств, требующая
    $\bigO{f(n)}$ памяти.

    $\PSPACE \coloneqq \bigcup\limits_i \DSPACE[n^i]$.
\end{definition*}

\task{
    Определим кванторную пропозициональную формулу: она имеет вид:
    $$
        Q_1 x_1 Q_2 x_x \dots Q_n x_n \varphi(x_1, x_2, \dots, x_n),
    $$
    где $\varphi$~--- пропозициональная формула от переменных $x_1, \dots, x_n$, а $Q_i \in \{\exists,
    \forall\}$~--- кванторы. Переменные $x_i$ принимают значения $\{0, 1\}$, истинность формулы
    определяется естественным образом. Обозначим $\langcplx{QBF}$~--- это множество истинных кванторных
    пропозициональных формул. Докажите, что $\langcplx{QBF}$ лежит в $\PSPACE$.
}

\libproblem{struct-complexity}{dspace-1-linear-time}
\libproblem{struct-complexity}{poly-savitch-th}

\begin{definition*}
    $\Lclass$~--- класс языков для которых существует ДМТ, которая использует $\bigO{\log n}$ памяти.
\end{definition*}

\libproblem{struct-complexity}{sat-logspace-np-logspace}

\breakline

\libproblem[41]{struct-complexity}{rho-gap-max-sat}
\libproblem[42]{complexity}{maxindset-inapprox}
\libproblem[49]{struct-complexity}{ucycle-logspace}


\end{document}



%%% Local Variables:
%%% mode: latex
%%% TeX-master: t
%%% End:
