\setcounter{curtask}{37}

\mytitle{7 (на 5.11)}

\begin{task}
    Докажите $NP$ полноту следующей задачи:
    на вход подается пара графов $(G_1, G_2)$, необходимо определить, изоморфен ли
    граф $G_2$ подграфу графа $G_1$.
\end{task}

\begin{task}
    Докажите, что существует язык, для которого любой алгоритм, работающий время
    $O(n^2)$ решает его правильно ровно на половине входов какой-то длины, но
    этот язык распознается алгоритмом, работающим время $O(n^3)$.
\end{task}

\begin{task}
    Докажите, что $PH \subseteq PSpace$.
\end{task}

\begin{task}
    Докажите, что язык $L = \{(\phi, 1^k) \mid$ функция, заданная формулой $\phi$, не может быть посчитана формулой размера $k$ $\}$
    лежит в $PH$.
\end{task}


\begin{task}
    Пусть функции $f, g: \{0, 1\}^* \rightarrow \{0, 1\}^*$ можно посчитать с
    использованием $O(\log(n))$ памяти (напомним, что память считается только на
    рабочих лентах, входная лента доступна только для чтения, а по выходной ленте
    головка машины Тьюринга движется только слева направо). Докажите, что функцию
    $f(g(x))$ можно также посчитать с использованием $O(\log(n))$ памяти.
\end{task}

Определим классы $EXP = \bigcup\limits_{i}DTime[2^{n^i}]$,  $NEXP = \bigcup\limits_{i}NTime[2^{n^i}]$.

\begin{task}
	Пусть $P = NP$, докажите, что  $EXP = NEXP$.
\end{task}

\begin{task}
	Докажите, что если все унарные языки из $NP$ лежат в $P$, то $EXP = NEXP$.
\end{task}


\breakline

\begin{ptask}{23}
    Докажите, что если множество $A$ задается предикатом из класса $\Sigma_n$, то
    множество $A \times A$ также задается предикатом из класса $\Sigma_n$.
\end{ptask}

\begin{ptask}{24}
    Докажите, что дизъюнктное объединение двух множеств, задаваемых предикатом из
    класса $\Sigma_n$, задается предикатом из класса $\Sigma_{n + 1}$, а также
    предикатом из класса $\Pi_{n + 1}$.
\end{ptask}


\begin{ptask}{35}
    Докажите, что язык графов с циклом разрешим алгоритмом, который использует
    $O(\log(n))$ памяти, где $n$~--- число вершин графа.
\end{ptask}


