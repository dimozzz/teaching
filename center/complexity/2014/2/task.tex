\setcounter{curtask}{9}

\mytitle{2 (на 24.09)}

\begin{task}
    Докажите, что множество всех рациональных чисел меньших $\pi$ разрешимо.
\end{task}

\begin{task}
    Существует ли алгоритм, проверяющий, работает ли данная программа
    полиномиальное время? (т.е. на каждом входе алгоритм делает не более $p(|x|)$
    шагов, где $p$~--- полином, а $x$~--- вход алгоритма).
\end{task}

\begin{task}
    Существует ли алгоритм, проверяющий, что данная программа считает полиномиально
    вычислимую функцию. (т.е. такую функцию, для которой есть алгоритм, который
    работает полиномиальное время).
\end{task}

\begin{task}
    Докажите, что для каждой вычислимой функции $f$ найдется
    псевдообратная вычислимая функция $g$. А именно, $g$ определена на
    множестве значений $f$, и для всех $x$ из области определения $f$
    выполняется $f(g(f(x))) = f(x)$.
\end{task}

\begin{task}
    Приведите пример неразрешимого множества $A \subseteq \Nat \times \Nat$,
    такого, что все его горизонтальные и вертикальные сечения
    разрешимы (т.е. для любого $x$ разрешимы $A \cap \{\{x\} \times \Nat\}$
    и $A \cap \{\Nat \times \{x\}\}$)
\end{task}


\begin{task}
    Докажите, что существует программа, которая печатает квадрат
    своего номера (подсказака: используйте теорему Клини).
\end{task}

\begin{task}
    Пусть $S$~--- разрешимое множество натуральных чисел. Разложим все
    числа из $S$ на простые множители, из данных простых составим
    множество $D$. Верно ли что $D$ разрешимо?
\end{task}



\breakline


\begin{ptask}{6}
    Является ли множество всюду останавливающихся алгоритмов перечислимым? А его
    дополнение?
\end{ptask}

