\setcounter{curtask}{55}

\mytitle{9 (на 26.11)}


\begin{task}
    Покажите, что если $P^A = NP^A$, то $P^A = PH^A$
\end{task}

\begin{task}
    Докажите, что если $P = NP$, то существует язык из $EXP$, схемная сложность которого не менее $\frac{2^n}{10n}$.
\end{task}

\breakline

\begin{ptask}{41}
    Пусть функции $f, g: \{0, 1\}^* \rightarrow \{0, 1\}^*$ можно посчитать с
    использованием $O(\log(n))$ памяти (напомним, что память считается только на
    рабочих лентах, входная лента доступна только для чтения, а по выходной ленте
    головка машины Тьюринга движется только слева направо). Докажите, что функцию
    $f(g(x))$ можно также посчитать с использованием $O(\log(n))$ памяти.
\end{ptask}


\begin{ptask}{49}
    Докажите, что $NEXP^{NEXP^{NEXP^{NEXP}}} \notin SIZE[\frac{2^n}{100n}]$ (подсказка: см. доказательство теоремы Каннана).
\end{ptask}

\begin{ptask}{50}
    Приведите пример разрешимого языка из $P/poly$, но не лежащего в классе $BPP$.
\end{ptask}

\begin{ptask}{52}
    Докажите, что $BPP = P^{BPP}$, где $P^{BPP}$~--- множество языков, которые можно распознать при помощи полиномиальной по
    времени детерминировнной машины Тьюринга, которая может делать оракульные запросы к языку из класса $BPP$.
\end{ptask}

\begin{ptask}{53}
	Докажите, что $L \in ZPP$ тогда и только тогда, когда существует полиномиальная по времени вероятностная машина Тьюринга,
    которая выдает $\{0, 1, ?\}$, и для всех $x \in \{0, 1\}^*$ с вероятностью $1$ $M(x) \in \{L(x), ?\}$ и $\Pr[M(x) = ?] \le
    \frac{1}{2}$.
\end{ptask}

\begin{ptask}{54}
	Докажите, что $ZPP = RP \cap coRP$.
\end{ptask}
