\setcounter{curtask}{31}

\mytitle{6 (на 29.10)}

Определим класс $coNP$: $L \in coNP \Leftrightarrow \bar{L} \in NP$, где
$\bar{L}$~--- дополнение языка $L$.


\begin{task}
	Докажите:
    а) $P \subseteq NP \cap coNP$
    б) $P = NP \Rightarrow NP = coNP$
\end{task}

\begin{task}
	Докажите, что язык формул, где каждый клоз либо хорновский, либо состоит из двух
    литералов, $NP$-полный.
\end{task}

\begin{task}
    Рассмотрим язык $Exactly One 3SAT$, который состоит из таких булевых формул в
    $3$-КНФ, что существует такой выполняющий набор, что в каждом клозе выполнен
    равно один литерал. Докажите, что $Exactly One 3SAT$ является $NP$~полным.
\end{task}

\begin{task}
    Докажите, что язык графов с циклом разрешим алгоритмом, который использует $O(n
    \log(n))$ памяти, где $n$~--- число вершин графа.
\end{task}

\begin{task}
    Докажите, что язык графов с циклом разрешим алгоритмом, который использует
    $O(\log(n))$ памяти, где $n$~--- число вершин графа.
\end{task}


\begin{task}
    Покажите, что существует язык, который разрешим алгоритмом, использующим
    $O(n^{10})$ памяти, но при этом не существует алгоритма, который бы разрешал
    данный язык и использовал при этом $O(n)$ памяти (подсказка: используйте метод
    диагонализации).
\end{task}

\breakline


\begin{ptask}{23}
    Докажите, что если множество $A$ задается предикатом из класса $\Sigma_n$, то
    множество $A \times A$ также задается предикатом из класса $\Sigma_n$.
\end{ptask}

\begin{ptask}{24}
    Докажите, что дизъюнктное объединение двух множеств, задаваемых предикатом из
    класса $\Sigma_n$, задается предикатом из класса $\Sigma_{n + 1}$, а также
    предикатом из класса $\Pi_{n + 1}$.
\end{ptask}