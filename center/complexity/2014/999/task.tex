\setcounter{curtask}{1}

\begin{center}
	{\bf Контрольная работа}.
\end{center}


\noindent{\bf Правила}.
Каждая задача контрольной может быть либо решена, либо не решена. За каждую решенную задачу вы получаете указанное число
баллов. Контрольная делится на две части (по темам: вычислимость и сложность), если Вы заработали $A$ баллов за часть работы, а
также в течение семестра вы получили $B$ баллов со соответствующей теме, то за данную тему вы получаете $\min \{A + B,
40\}$. Баллы за разные темы суммируются. Вы можете самостоятельно выбирать, какие задачи решать, исходя из того, сколько вам нужно
набрать баллов. Задачи следует записать и послать Д.О. Соколову по электронной почте sokolov.dmt@gmail.com не позднее 00:00
понедельник 1-ого декабря. В среду 3-го декабря на занятии будет апелляция.

\breakline


\setcounter{curtask}{1}

\begin{center}
    \textbf{Часть I}
\end{center}

\begin{task}($\mathbf{10}$)
    Докажите, что существует пара различных программ $A, B$ такая, что $A$ печатает текст $B$, а $B$ печатает текст $A$.
\end{task}

\begin{task}($\mathbf{10}$)
    Назовем множество {\it иммунным}, если оно бесконечно, но не содержит бесконечных перечислимых подмножеств. Перечислимое
	множество называется {\it простым}, если его дополнение иммунно.  Докажите, что простые множества существуют.
\end{task}

\begin{task}($\mathbf{5}$)
    Предъявите такое число $x \in \mathbb{R}$, что множество рациональных чисел меньших $x$~--- не является перечислимым.
\end{task}

\begin{task}($\mathbf{10}$)
    Докажите, что существуют перечислимые множества $A, B$, которые не могут быть отделены разрешимым множеством, т.е не
	существует такого разрешимого множества $C$, что $A \subseteq C$ и $B \cap C = \emptyset$.
\end{task}

\begin{task}($\mathbf{10}$)
    Докажите, что не существует универсального разрешимого множества.
\end{task}

\begin{task}($\mathbf{7}$)
    Является ли разрешимым множество программ, которые считают линейную по каждому аргументу функцию от своих переменных.
\end{task}




\begin{center}
    \textbf{Часть II}
\end{center}


\begin{task}($\mathbf{10}$)
    Докажите включение $NP^{BPP^{BPP}} \subseteq \Sigma_3$
\end{task}

\begin{task}($\mathbf{5}$)
    Докажите, что $P/poly \nsubseteq EXP$
\end{task}

\begin{task}($\mathbf{5}$)
    Докажите следующие включения: $P \subseteq BPP \subseteq PH \subseteq EXP$.
\end{task}

\begin{task}($\mathbf{8}$)
    Докажите, что существует язык, для которого любой алгоритм, работающий время
    $O(n^{\log(n)})$ решает его неправильно на всех входах какой-то длины, но
    этот язык распознается алгоритмом, работающим время $O(n^{2 \log(n)})$.
\end{task}


Язык $L \in BPL$ тогда и только тогда, когда существует такая вероятностная машина Тьюринга $M$, что $M$ использует $O(\log(n))$
памяти и $\Pr[M(x) = L(x)] \ge \frac{2}{3}$ (машина $M$ может не останавливаться при некоторых значениях случайных битов).

\begin{task}($\mathbf{20}$)
    Докажите, что $BPL \subseteq P$.
\end{task}

\begin{task}($\mathbf{10}$)
    Пусть $NP \subseteq DTime[n^{\log(n)}]$, докажите, что $PH \subseteq \bigcup\limits_{k}DTime[n^{\log^k(n)}]$.
\end{task}

\begin{task}($\mathbf{8}$)
    Докажите, что язык $L = \{(\phi, 1^k) \mid$ функция, заданная формулой $\phi$, не может быть посчитана формулой размера $k$
    $\}$ лежит в $PH$.
\end{task}