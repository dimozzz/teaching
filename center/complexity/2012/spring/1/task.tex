\setcounter{curtask}{1}

\mytitle{1 (на 14.02)}

\begin{task}
    Покажите, что язык выполнимых формул в 2-КНФ принадлежит классу $P$.
\end{task}

\begin{task}
    Определим класс $EXP = \bigcup\limits_{i}DTime[2^{n^i}]$. Докажите, что
    $NP \subseteq EXP$.
\end{task}

\begin{task}
    Хорновской формулой называется формула в КНФ, в которой в каждый дизъюнкт
    максимум одна переменная входит без отрицания. Покажите, что множество
    хорновских выполнимых формул содержится в классе $P$.
\end{task}

\begin{task}
    а) Докажите, что число $n$ простое тогда и только тогда, когда для каждого
    простого делителя $q$ числа $n - 1$ существует $a \in {2, 3, \dots, n - 1}$ при котором
    $a^{n - 1} = 1~mod~n$, а $a^{\frac{n - 1}{q}} \ne 1~mod~n$
    б) докажите, что язык простых чисел лежит в $NP$.
\end{task}

\begin{task}
    Пусть $L_1, L_2 \in NP$. Принадлежит ли объединение этих языков $NP$, а пересечение?
\end{task}

\begin{task}
    Пусть $L_1, L_2 \in NP \cap coNP$. Докажите, что $L_1 \oplus L_2 \in NP \cap coNP$.
\end{task}
