\setcounter{curtask}{02}

\mytitle{4 (на 17.10)}

\begin{task}
	Пусть задан конечный алфавит $W$. И конечное множество правил перехода для строк
    $s_1 \to s_2$. Строки называются равными, если при помощи данных (применяя их к
    подстрокам исходной строки) правил одну можно перевести в другую. Докажите
    неразрешимость задачи проверки двух строк на равенство.
\end{task}

\begin{task} (Равенство слов в полугруппе)
    Пусть задан конечный алфавит $W$. И конечное множество правил перехода для строк
    $s_1 \Leftrightarrow s_2$. Строки называются равными, если при помощи данных (применяя их к
    подстрокам исходной строки) правил одну можно перевести в другую. Докажите
    неразрешимость задачи проверки двух строк на равенство.
\end{task}

\begin{task}
    Докажите, что существуют вычислимые не главные нумерации.
\end{task}

\begin{task} (простые множества Поста)
    Назовем множество {\it иммунным}, если оно бесконечно, но не
    содержит бесконечных перечислимых подмножеств. Перечислимое
    множество называется {\it простым}, если его дополнение иммунно.
    Докажите, что простые множества существуют.
\end{task}


\breakline

\begin{ptask}{15}
	Докажите, что найдется такой язык разрешимый за $2^{n^2}$ памяти, что любой
    алгоритм, использующий $n$ памяти ошибается не менее, чем на двух словах из языка.
\end{ptask}

\begin{ptask}{16}
    Докажите, что существует тройка попарно различных программ $A, B,
    C$ таких, что $A$ печатает текст $B$, $B$ печатает текст $C$, а
    $C$ печатает текст $A$.
\end{ptask}

\begin{ptask}{17}
    Докажите, что существуют перечислимые множества $A, B$, которые не
    могут быть отделены разрешимым множеством, т.е не существует
    такого разрешимого множества $C$, что $A \subseteq C$ и $B \cap C
    = \emptyset$
\end{ptask}

\begin{ptask}{19}
    Докажите, что:
    {\it b)} не существует универсального вычислимого множества
    {\it с)} найдется $x \in \mathbb{N}$ такой, что $\{x\} = \{y
    \mid (x, y) \in U\}$
\end{ptask}