\setcounter{curtask}{1}

\mytitle{на зачет 1}

\begin{task}
    Предъявите такое число $x \in \mathbb{R}$, что $x$ не является
    вычислимым, а множество рациональных чисел меньших $x$~---
    перечислимо.
\end{task}

\begin{task}
    Докажите, что существует пара программ $A, B$ такая, что $A$
    печатает текст $B$, а $B$ печатает текст $A$
\end{task}


\begin{task}
    Пусть $S$~--- разрешимое множество натуральных чисел. Разложим все
    числа из $S$ на простые множители, из данных простых составим
    множество $D$. Верно ли что $D$ разрешимо?
\end{task}

\begin{task}
    Является ли перечислимым множество всех программ, вычисляющим
    инъективные функции? А коперечислимым?
\end{task}


\breakline

\begin{task}(*)
    Является ли перечислимым множество всех программ, вычисляющим
    сюръективные функции? А коперечислимым?
\end{task}

\begin{task}(*)
    Пусть $f, g$~--- вычислимые всюду определенные функции. Докажите,
    что найдутся такие номера машин Тьюринга $n$ и $m$, что алгоритм с
    номером $f(n)$ ведется себя так же, как и алгоритм с номером $m$,
    а алгоритм с номером $g(m)$ так же, как и алгоритм с номером $n$.
\end{task}

\begin{task}(*)
    Докажите, что существует счетное число не пересекающихся
    перечислимых множеств, никакие два из которых нельзя отделить
    разрешимым.
\end{task}

\begin{task}(*) (простые множества Поста)
    Назовем множество {\it иммунным}, если оно бесконечно, но не
    содержит бесконечных перечислимых подмножеств. Перечислимое
    множество называется {\it простым}, если его дополнение иммунно.
    Докажите, что простые множества существуют.
\end{task}
