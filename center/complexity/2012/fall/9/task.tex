\setcounter{curtask}{40}

\mytitle{9 (на 11.12)}

\begin{task}
    Докажите, что если $\Sigma_i = \Pi_i$, то $PH = \Sigma_i$
\end{task}

\begin{task}
    Пусть $NP \subseteq DTime[n^{\log(n)}]$, докажите, что $PH \subseteq \bigcup\limits_{k}DTime[n^{\log^k(n)}]$
\end{task}

\begin{task}
    Докажите, что если язык $A$ сводится по Тьюрингу к языку $B \in \Sigma_{i}$, то
    $A \in \Sigma_{i + 1}$.
\end{task}

\breakline

\begin{ptask}{30}
    Изменится ли класс $\widetilde{P}$, если из определения убрать условие
    ``$\widetilde{P} \subseteq \widetilde{NP}$''.
\end{ptask}

\begin{ptask}{31}
    Докажите $NP$ полноту следующей задачи:
    на вход подается пара графов $(G_1, G_2)$, необходимо определить, изоморфен ли
    граф $G_2$ подграфу графа $G_1$.
\end{ptask}

\begin{ptask}{36}
    Докажите, что существует язык, для которого любой алгоритм, работающий время
    $O(n^2)$ решает его правильно ровно на половине входов какой-то длины, но
    этот язык распознается алгоритмом, работающим время $O(n^3)$.
\end{ptask}

\begin{ptask}{37}
	Докажите, что язык формул, где каждый клоз либо хорновский, либо состоит из двух
    литералов, $NP$-полный.
\end{ptask}

\begin{ptask}{38}
    Докажите, что $DSpace[n] \ne NP$.
    (Подсказка: рассмотреть язык из класса $DSpace[n^2]$).
\end{ptask}