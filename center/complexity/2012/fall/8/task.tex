\setcounter{curtask}{37}

\mytitle{8 (на 21.11)}

\begin{task}
	Докажите, что язык формул, где каждый клоз либо хорновский, либо состоит из двух
    литералов, $NP$-полный.
\end{task}

\begin{task}
    Докажите, что $DSpace[n] \ne NP$.
    (Подсказка: рассмотреть язык из класса $DSpace[n^2]$).
\end{task}

\begin{task}
	Докажите, что $SAT$ не является $NP$-полной задачей относительно сведений,
    сохраняющих размер входа.
\end{task}

\breakline

\begin{ptask}{29}
    а) Докажите, что число $n$ простое тогда и только тогда, когда для каждого
    простого делителя $q$ числа $n - 1$ существует $a \in {2, 3, \dots, n - 1}$ при котором
    $a^{n - 1} = 1~mod~n$, а $a^{\frac{n - 1}{q}} \ne 1~mod~n$
\end{ptask}

\begin{ptask}{30}
    Изменится ли класс $\widetilde{P}$, если из определения убрать условие
    ``$\widetilde{P} \subseteq \widetilde{NP}$''.
\end{ptask}

\begin{ptask}{31}
    Докажите $NP$ полноту следующей задачи:
    на вход подается пара графов $(G_1, G_2)$, необходимо определить, изоморфен ли
    граф $G_2$ подграфу графа $G_1$.
\end{ptask}

\begin{ptask}{36}
    Докажите, что существует язык, для которого любой алгоритм, работающий время
    $O(n^2)$ решает его правильно ровно на половине входов какой-то длины, но
    этот язык распознается алгоритмом, работающим время $O(n^3)$.
\end{ptask}
