\setcounter{curtask}{1}

\mytitle{1 (на 18.09)}

\begin{task} 
    Докажите, что есть взаимнооднозначное соответсвие между $\mathbb{N}$ и
    $\mathbb{N} \times \mathbb{N}$.
    
    а) предъявите в любом виде. (Разобрано!)
    
    б) в явном виде (формулой).
\end{task}

\begin{task}
    Докажите, что есть взаимнооднозначного соответсвия между $\mathbb{N}$ и
    $2^{\mathbb{N}}$.
\end{task}

\begin{task} (Разобрано!)
    Докажите, что любую булеву функцию можно выразить в базисе $1, \wedge, \oplus$.
\end{task}

\begin{task}
    Пусть формула $\phi \rightarrow \psi$~--- тавтология. Докажите,
    что найдется такая формула $\tau$, содержащая только общие для
    $\phi$ и $\psi$ переменные, что $\phi \rightarrow \tau$ и
    $\tau \rightarrow \psi$ будут тавтологиями.
\end{task}

\begin{task} (Разобрано!)
    Докажите, что среди 6-ти людей есть либо 3 попарно знакомых, либо 3 попарно незнакомых.
\end{task}

\begin{task}
    Докажите эквивалентность следующих формул: $\neg A \wedge \neg B, \neg (A \vee B)$
\end{task}

\begin{task}
    Являеются ли базисом следующие наборы функций:

    а) $\wedge, \vee$

    б) $1, f(x, y) = x \oplus y$
\end{task}

\begin{task}
    Запишите в КНФ функцию $x_1 \oplus x_2 \oplus \dots \oplus x_n$. Можно ли короче?
\end{task}
