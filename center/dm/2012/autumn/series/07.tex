\documentclass[a4paper, 12pt]{article}
% math symbols
\usepackage{amssymb}
\usepackage{amsmath}
\usepackage{mathrsfs}
\usepackage{mathseries}


\usepackage[margin = 2cm]{geometry}

\tolerance = 1000
\emergencystretch = 0.74cm



\pagestyle{empty}
\parindent = 0mm

\renewcommand{\coursetitle}{DM/ML}
\setcounter{curtask}{1}
\setcounter{curtask}{35}

\setmathstyle{20.11}{Дискретная математика}{CS Центр}


\begin{document}

\libproblem{discrete-math}{euler-cycle-examples}
\libproblem{discrete-math}{lipshire-roads}
\libproblem{discrete-math}{street-of-talks}
\libproblem{discrete-math}{marriages-and-strangers}

\breakline

\libproblem[32]{discrete-math}{no-triangles-edge-counting}
\libproblem[33]{discrete-math}{boy-girl-even-group}

\task[34]{
    В связном графе на каждом ребре написали положительное число. Весом остовного
    дерева назовем сумму весов, входящих в него ребер. Докажите, что осточное дерево,
    на котором достигается минимум суммы чисел совпадает с одним из остовных
    деревьев, на котором достигается минимум квадратов чисел.
}

\end{document}



%%% Local Variables:
%%% mode: latex
%%% TeX-master: t
%%% End:
