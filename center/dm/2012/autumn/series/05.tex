\documentclass[a4paper, 12pt]{article}
% math symbols
\usepackage{amssymb}
\usepackage{amsmath}
\usepackage{mathrsfs}
\usepackage{mathseries}


\usepackage[margin = 2cm]{geometry}

\tolerance = 1000
\emergencystretch = 0.74cm



\pagestyle{empty}
\parindent = 0mm

\renewcommand{\coursetitle}{DM/ML}
\setcounter{curtask}{1}
\setcounter{curtask}{24}

\setmathstyle{30.10}{Дискретная математика}{CS Центр}


\begin{document}

\libproblem{discrete-math}{vertex-del-connectivity}
\libproblem{discrete-math}{graph-2-points-deg}

\task{
    Докажите, что если в неориентированном графе $n - k$ ребер, то в нем не менее $k$
    компонент связности.
}

\libproblem{discrete-math}{grid-connectivity}

\task{
    В связном графе на каждом ребре написали положительное число. Весом остовного
    дерева назовем сумму весов, входящих в него ребер. Докажите, что одно из
    ребер с минимальным весом входит в минимальное остовное дерево.
}

\breakline

\task[17]{
    Пусть $K$~--- линейное пространство над $\mathbb{R}$ размерности $n$.
    $f\coloneqq K^m \to \mathbb{R}$~--- полилинейная форма, т.е.:
    \begin{itemize}
	    \item $\forall i \in \mathbb{N}, \alpha, \beta \in K ~~ f(x_1, \dots, \alpha x_i + \beta y_i,
            \dots, x_m) = \alpha f(x_1, \dots, x_i, \dots, x_m) + \beta f(x_1, \dots, y_i, \dots, x_m)$.
    \end{itemize}
    Какова размерность линейного пространства:
    \begin{enumcyr}
        \item всех полилинейных форм;
        \item антисимметричных полилинейных форм (от перестановки соседних аргументов значение меняется
            на противоположное);
        \item симметричных полилинейных форм (от перестановки аргументов значение не меняется).
    \end{enumcyr}
}

\libproblem[18]{combinatorics}{binom-sum-three}

\end{document}



%%% Local Variables:
%%% mode: latex
%%% TeX-master: t
%%% End:
