\documentclass[a4paper, 12pt]{article}
% math symbols
\usepackage{amssymb}
\usepackage{amsmath}
\usepackage{mathrsfs}
\usepackage{mathseries}


\usepackage[margin = 2cm]{geometry}

\tolerance = 1000
\emergencystretch = 0.74cm



\pagestyle{empty}
\parindent = 0mm

\renewcommand{\coursetitle}{DM/ML}
\setcounter{curtask}{1}

\setmathstyle{Тест-3}{Дискретная математика}{CS Центр}


\begin{document}

\task{
    Докажите, что каждое множество, состоящее из $n$ отличных от нуля вещественых
    чисел, содержит подмножество $A$ мощности строго большей, чем $\frac{n}{3}$, в
    котором нет троек $a_1, a_2, a_3$, удовлетворяющих условию $a_1 + a_2 = a_3$.
}

\task{
    Чему равно число корневых бинарных деревьев с $n$ листьями.
}


\libproblem{combinatorics}{chaos-permutation-lim}

\task{
    Докажите, что из любого двусвязного графа, степени всех вершин которого больше
    двух, можно удалить вершину так, чтобы граф остался двусвязным.
}

\libproblem{discrete-math}{ramsey-3-4}

\task{
    Докажите, что в $2k$-регулярном графе есть $2$-регулярный подграф на всех вершинах.
}

\task{
    В шеренгу стоит $nm + 1$ человек, докажите, что найдется либо $m + 1$ человек,
    стоящие по росту слева направо, либо $n + 1$ справа налево.
}

\libproblem{discrete-math}{odd-cycle-chrom-number}


\end{document}



%%% Local Variables:
%%% mode: latex
%%% TeX-master: t
%%% End:
