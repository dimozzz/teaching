\setcounter{curtask}{1}

\mytitle{(Контрольное!)}

\begin{task}
    В некоторых клетках прямоугольной таблицы стоят звездочки, причем в каждой строке
    есть хотя бы одна звездочка. Известно, что строк больше, чем столбцов. Докажите,
    что найдется звездочка, в строке которой меньше звезд, чем в столбце.
\end{task}

\begin{task}
    (Полигамный вариант!).
    Каждому юноше нравится несколько девушек, причем любому набору из $k$ юношей
    нравится не менее $km$ девушек. Докажите, что каждому юноше можно так выделить
    гарем из $m$ нравящихся ему девушек, чтобы гарем не пересекались.
\end{task}

\begin{task}
    Докажите, что на всех ребрах и диагоналях произвольного выпуклого
    $2n + 1$-угольника можно расставить стрелочки так, чтобы суммы получившихся
    векторов равнялась нулю.
\end{task}

\begin{task}
    Докажите, что из любого графа можно выкинуть не более половины ребер так, чтобы
    он стал двудольным.
\end{task}

\begin{task}
    Докажите, что если любые два нечетных цикла в графе имеют общую вершину, то
    хроматическое число графа не более $5$.
\end{task}

\begin{task}
    Докажите, что в $2k$-регулярном графе есть $2$-регулярный подграф на всех вершинах.
\end{task}

\begin{task}
    Докажите, что если простой связный граф $G$ имеет ровно две вершины, при удалении
    которых он не теряет связность, то $G$~--- это путь.
\end{task}
