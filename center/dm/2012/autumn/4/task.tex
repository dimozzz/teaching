\setcounter{curtask}{20}

\mytitle{4 (на 16.10)}

\begin{task}
    Автобусные билеты имеют 6-ти значные номера (могут начинаться с нуля). Билет
    называется счастливым, если сумма первых трех цифр равна сумме оставшихся. Каких
    билетов больше счастливых или с суммой 27.
\end{task}

\begin{task}
    Докажите, что любое число можно представить в виде суммы различных натуральных
    чисел столькими способами, сколькими его можно представить в виде суммы не
    обязательно различных нечетных слагаемых.
\end{task}

\begin{task}
    Посчитайте число способов разбить $n$ на $k$ слагаемых. Разбиения отличающиеся
    порядкой считаются различными.
\end{task}

\begin{task}(Перестановочное неравенство)
    Пусть $x_1 \le x_2 \le \dots \le x_n$, $y_1 \le y_2 \le \dots \le y_n$. Докажите,
    что для любой перестановки $\sigma$ выполнено $x_1 y_n + x_2 y_{n - 1} + \dots +
    x_n y_1 \le x_1 y_{\sigma(1)} + x_2 y_{\sigma(2)} + \dots + x_n y_{\sigma(n))} \le
    x_1 y_1 + x_2 y_2 + \dots + x_n y_n$.
\end{task}

\breakline

\begin{ptask}{17}
    Какова размерность линейного пространства:
	в) симметричных полилинейных форм (от перестановки аргументов значение не
    	меняется)
\end{ptask}

\begin{ptask}{18}
    Посчитайте сумму:
    $C_{n}^{0} + C_{n}^{3} + C_{n}^{6} + C_{n}^{9} + C_{n}^{12} + \dots$
\end{ptask}

\begin{ptask}{19}
    Пусть $(x_1, \dots, x_{2n})$~--- перестановка. Назовем перестановку удобной, если
    какие-то два соседних элемента различаются в ней ровно на $n$. Докажите, что
    удобных перестановок больше половины.
\end{ptask}