\setcounter{curtask}{24}

\mytitle{5 (на 30.10)}


\begin{task}
    Докажите, что из произвольного связного графа можно выкинуть вершину так, чтобы
    граф остался связным.
\end{task}

\begin{task}
    Докажите, что любом графе есть две вершины одинаковой степени.
\end{task}

\begin{task}
    Докажите, что если в неориентированном графе $n - k$ ребер, то в нем не менее $k$
    компонент связности.
\end{task}

\begin{task}
    Имеется сетка $n \times n$. Разрешается разрезать любое ребро сетки. Какое
    максимальное количество ребер можно разрезать без потери связности?
\end{task}

\begin{task}
    В связном графе на каждом ребре написали положительное число. Весом остовного
    дерева назовем сумму весов, входящих в него ребер. Докажите, что одно из
    ребер с минимальным весом входит в минимальное остовное дерево.
\end{task}

\breakline

\begin{ptask}{17}
    Какова размерность линейного пространства:
	в) симметричных полилинейных форм (от перестановки аргументов значение не
    	меняется)
\end{ptask}

\begin{ptask}{18}
    Посчитайте сумму:
    $C_{n}^{0} + C_{n}^{3} + C_{n}^{6} + C_{n}^{9} + C_{n}^{12} + \dots$
\end{ptask}