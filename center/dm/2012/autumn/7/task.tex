\setcounter{curtask}{35}

\mytitle{7 (на 20.11)}

\begin{task}
    На плоскости нарисовано несколько окружностей. Докажите, что области, на которые
    эти окружности разбивают плосткость, можно покрасить в черный и белый цвета в
    шахматном порядке.
\end{task}

\begin{task}
    В графстве Липшир из усадбы каждого джентельмена исходит ровно 10 дорог к другим
    усадьбам. При этом каждый джентельмен может доехать до каждого. Однажды одну из
    дорог перекопали, докажите, что каждый джентельмен по прежнему может добраться до
    каждого. 
\end{task}

\begin{task}
    На улице Болтунов живет $n$ юношей и $n$ девушек, причем каждый юноша ровно к $k$
    девушками, а каждая девушка с ровно $k$ юношами. а) докажите, что все юноши и
    девушки могут одновременно гоорить со своими знакомыми по телефону б) докажите,
    что за $k$ часов каждый юноша может поговорить со всеми знакомыми девушками по
    часу.
\end{task}

\begin{task}
    Есть $n$ юношей и $n$ девушек. Каждый юноша знает хотя бы одну девушку. Тогда
    можно некоторый юношей поженить на знакомых девушках так, чтобы женатые юноши не
    знали незамужних девушек.
\end{task}

\breakline

\begin{ptask}{32}
    Сколько может быть ребер в неориентированном графе на $2n$ вершинах без циклов
    длины $3$.
\end{ptask}

\begin{ptask}{33}
    В классе поровну мальчиков и девочек. Каждый мальчик дружит с четным числом
    девочек. Докажите, что можно выбрать группу из мальчиков так, чтобы каждая
    девочка дружила с четным числом мальчиков из этой группы.
\end{ptask}

\begin{ptask}{34}
    В связном графе на каждом ребре написали положительное число. Весом остовного
    дерева назовем сумму весов, входящих в него ребер. Докажите, что осточное дерево,
    на котором достигается минимум суммы чисел совпадает с одним из остовных
    деревьев, на котором достигается минимум квадратов чисел.
\end{ptask}