\setcounter{curtask}{15}

\mytitle{3 (на 9.10)}

\begin{task}
	Определим расстояние между перестановками, как минимальное количество
	транспозиций, которое необходимо сделать, чтобы перевести одну перестановку в
	другую. Какое максимальное расстояние между перестановками может быть?
\end{task}

\begin{task}
	Пусть $G$~--- конечная группа, докажите, что найдется такое $n$, что $G$
    вкладывается в группу $S_n$ (т.е. существует инъекция из $G$ в $S_n$, сохраняющая
    групповую операцию).
\end{task}

\begin{task}
    Пусть $K$~--- линейное пространство над $\mathbb{R}$ размерности $n$.
    $f: K^m \to \mathbb{R}$~--- полилинейная форма, т.е.:
    \begin{itemize}
	    \item $\forall i, a, b ~~ f(x_1, \dots, a x_i + b y_i, \dots, x_m) =
    		a f(x_1, \dots, x_i, \dots, x_m) + b f(x_1, \dots, y_i, \dots, x_m)$
    \end{itemize}
    Какова размерность линейного пространства:
    а) всех полилинейных форм
    б) антисимметричных полилинейных форм (от перестановки соседних аргументов
    значение меняется на противоположное)
    в) симметричных полилинейных форм (от перестановки аргументов значение не меняется)
\end{task}

\begin{task}
    Посчитайте сумму:
    $C_{n}^{0} + C_{n}^{3} + C_{n}^{6} + C_{n}^{9} + C_{n}^{12} + \dots$
\end{task}

\begin{task}
    Пусть $(x_1, \dots, x_{2n})$~--- перестановка. Назовем перестановку удобной, если
    какие-то два соседних элемента различаются в ней ровно на $n$. Докажите, что
    удобных перестановок больше половины.
\end{task}

