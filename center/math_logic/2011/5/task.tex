\setcounter{curtask}{24}

\mytitle{5 (на 12.10)}

\begin{task} ({\it машина Минского})
    Рассмотрим следующую модель вычислений:
    \begin{itemize}
	    \item есть конечное число счетчиков: $r_1, \dots, r_k$;
    	\item программа представляет собой пронумерованный набор
    		команд;
    	\item команды бывают следующий типов:
		    \begin{itemize}
        	    \item $inc(r_j)$~--- увеличить содержимое счетчика
		            $r_j$ на $1$ и перейти к следующей по номеру команде;
                \item $dec(r_j)$~--- уменьшить содержимое счетчика
		            $r_j$ на $1$ и перейти к следующей по номеру команде;
               	\item $jz(r_j, i)$~--- если $r_j$~--- равно нулю то
            		перейти к команде с номером $i$, иначе перейти к
                    следующей по номеру команде;
                \item $halt$~--- остановить машину.
		    \end{itemize}
       	\item ответ хранится в заранее заданном счетчике.
    \end{itemize}

    Докажите, что:
    а) задача останова машины Минского неразрешима для некоторого $k$
    б) задача останова машины Минского неразрешима для $k = 3$.
\end{task}

\begin{task}
    Пусть $g(x_1, \dots, x_k) = y_0$, где $y_0 = \min \{y \mid f(x_1,
      \dots, x_k, y) = 0\}$. Покажите, что при вычислимой не всюду
      определенной $f$, $g$ может быть невычислимой.
\end{task}

\begin{task}
    Покажите, что функция обратная к примитивно рекурсивной биекции
    $f: \mathbb{N} \rightarrow \mathbb{N}$
    может не быть примитивно рекурсивной.
\end{task}

\begin{task}
	Пусть $g(x_1, \dots, x_k, y)$~--- примитивно рекурсивная функция.
    Докажите, что функция $f(x_1, \dots, x_k, y, z) =
	\begin{cases}
		\sum\limits_{i = 0}^{z} g(x_1, \dots, x_k, y + i),~~ y \le z  \\
		0,~~ y > z
	\end{cases}$
\end{task}

\begin{task}
    Доказать, что следующие функции являются примитивно рекурсивными:
    а)$x^y$
    б)$x!$
    в)$min(x, y)$
    г)$max(x, y)$
    д)предикат равенства.
\end{task}

Общерекурсивная функция~--- частично рекурсивная функция, определенная
для всех значений.

\begin{task}
    Пусть $f$~--- общерекурсивная. Докажите (не пользуясь
    вычислительной эквивалентностью с машинами Тьюринга), что если
    изменить значение в конечном числе точек, то получится
    общерекурсивная функция.
\end{task}

\breakline

\begin{ptask}{16}
    Докажите, что существует тройка попарно различных программ $A, B,
    C$ таких, что $A$ печатает текст $B$, $B$ печатает текст $C$, а
    $C$ печатает текст $A$.
\end{ptask}

\begin{ptask}{19}
    Докажите, что:
    {\it в)} найдется $x \in \mathbb{N}$ такой, что $\{x\} = \{y
    \mid (x, y) \in U\}$
\end{ptask}

\begin{ptask}{20}
    Докажите, что существуют вычислимые не главные нумерации.
\end{ptask}

\begin{ptask}{21} ({\it проблема соответствия Поста})
    Придумать алгоритм, или доказать, что его не существует для
    следующей задачи: по двум бесконечным словам $a, b \in \Sigma^*$
    определить существует ли последовательность $i_1, i_2, \dots, i_n$
	такая, что $\forall k ~~ a_{i_k} = b_{i_k}$.
\end{ptask}

\begin{ptask}{23} ({\it простые множества Поста})
    Назовем множество {\it иммунным}, если оно бесконечно, но не
    содержит бесконечных перечислимых подмножеств. Перечислимое
    множество называется {\it простым}, если его дополнение иммунно.
    Докажите, что простые множества существуют.
\end{ptask}