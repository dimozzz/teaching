\setcounter{curtask}{1}

\mytitle{Контрольная работа 2}

\begin{task}
    Приведите к предваренной нормальной форме формулу:
    $\exists x A(x) \rightarrow \exists x \forall y B(x, y)$
\end{task}

\begin{task}
    Выразимы ли следующие предикаты в интерпретации
    $(\mathbb{Z}, =, <)$:
    а) $y = 2x$
    б) $y = x + 1$
\end{task}

\begin{task}
    Допускает ли интерпретация $(\mathbb{N}, =, S)$ элиминацию
    кванторов?
\end{task}

\begin{task}
    Какое минимальное количество булевых функций от $n~ (n > 1)$
    переменных составляют полный базис?
\end{task}


\breakline

\begin{task}(*)
    Будет ли теория $Th(\mathbb{Q}, <, =)$ (плотный линейный порядок
    без первого и последнего элемента) категоричной в мощности континуум?
\end{task}

Спектром теории (формулы) называется множество мощностей ее конечных нормальных
моделей.

\begin{task}(*)
    Приведите пример конечной теории, спектр которой~--- все степени
    простых чисел.
\end{task}

\begin{task}(*)
    Приведите пример конечной теории, спектр которой~--- все составные
    числа.
\end{task}

\begin{task}(*)
    Приведите пример теории, спектр которой состоит из всех
    положительных чисел кратных $3$.
\end{task}

\begin{task}(*)
    Верно ли, что для любого бесконечного множества $A$ найдется
    линейный порядок такой же мощности?
\end{task}