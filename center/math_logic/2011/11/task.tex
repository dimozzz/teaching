\setcounter{curtask}{50}

\mytitle{11 (на 30.11)}

\begin{task} (upward Löwenheim–Skolem Theorem)
    Пусть теория $T$ имеет бесконечную модель мощности $k$, докажите,
    что данная теория имеет модель любой мощности большей $k$
    (подсказка: добавить в сигнатуру необходимое число констант).
\end{task}

Пусть $I$~--- интерпретация. $Th(I)$~--- множество формул верной в
данной интерпретации.
\begin{task}
    Будет ли теория $Th((Z, <, =))$ конечно аксиоматизируемой.
\end{task}

\begin{task}
    Будет ли теория $Th((N, <, =))$ конечно аксиоматизируемой.
\end{task}


\breakline

\begin{ptask}{37} (критерий Поста)
	Пусть $F = {f_1, \dots, f_k}$~--- набор булевых функций от $n$
    переменных. Будем говорить, что $F$ принадлежит классу функций,
    если все функции из множества $F$ принадлежат данному классу.
    
    Пусть теперь $F$ не принадлежит ни одному из перечисленных
    классов.

    в) постройте конъюнкцию из композиций функций из $F$ и докажите,
    что набор $F$ является базисом булевых функций от $n$ аргументов
    (указание: использовать полином Жегалкина).

\end{ptask}

\begin{ptask}{45}
	$\mathbb{Z} + \mathbb{Z}$~--- это две копии целых чисел, причем
    все числа из второй копии больше чисел из первой. Докажите, что
    $(\mathbb{Z}, <, =)$ элементарно эквивалентна $(\mathbb{Z} +
    \mathbb{Z}, <, =)$.
\end{ptask}

\begin{ptask}{46}
    Будет ли интерпретация $(\mathbb{N}, =, <)$ элементарно
    эквивалентна:
    а) $(\mathbb{N} + \mathbb{N}, =, <)$
    б) $(\mathbb{N} + \mathbb{Z}, =, <)$
\end{ptask}

\begin{task}{48}
    Приведите пример:
    а) совместной теории без конечных моделей
    б) совместной теории наименьшая модель, которой имеет мощность не менее
    $2^{\mathbb{N}}$. (подсказка: рассмотреть несчетную сигнатуру,
    для подобных сигнатур все доказанные результаты остаются верными)
    в) совместной конечной теории (задается конечным набором аксиом) без конечных моделей
\end{task}