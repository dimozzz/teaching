Пусть $F \coloneqq {f_1, \dots, f_k}$~--- набор булевых функций от $n$ переменных. Будем говорить, что
$F$ принадлежит классу функций, если все функции из множества $F$ принадлежат данному классу.

Докажите, что:
\begin{enumcyr}
    \item если все функции из $F$ являются сохраняющими ноль/сохраняющими
        единицу/самодвойственными/монотонными/линейными, то любые композиции функций из $F$ принадлежат
        тому же классу;
    \item если для каждого из перечисленных классов в $F$ найдется функция, то при помощи композиции
        функций из $F$ можно построить константы и отрицание;
    \item если для каждого из перечисленных классов в $F$ найдется функция, то при помощи композиции
        функций из $F$ можно построить конъюнкцию.
    \item если для каждого из перечисленных классов в $F$ найдется функция, то $F$ является базисом, для
        всех булевых функций.
\end{enumcyr}