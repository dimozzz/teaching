На химической конференции присутствовало $k$ учёных: химиков и алхимиков, причём химиков было больше, чем
алхимиков. Известно, что на любой вопрос химики всегда отвечают правду, а алхимики иногда говорят правду,
а иногда лгут. Оказавшийся на конференции математик про каждого учёного хочет установить, химик тот или
алхимик. Для этого он любому учёному может задать вопрос: <<Кем является такой-то: химиком или
алхимиком?>>, и, в частности, может спросить, кем является сам этот учёный. Докажите, что математик может
установить это за:
\begin{enumcyr}
    \item $4k$ вопросов;
    \item $2k - 2$ вопросов.
\end{enumcyr}
