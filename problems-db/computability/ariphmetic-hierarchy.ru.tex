Покажите, что:
\begin{enumcyr}
    \item $\Sigma_1$~--- это множество перичислимых предикатов, а $\Pi_1$~--- коперечислимых;
    \item $Q \in \Sigma_k$ тогда и только тогда, когда $Q$ можно представить в виде: $Q(x) = \exists y_1
        \forall y_2 \exists y_3 \dots P(x, y_1, y_2, \dots, y_n)$, где $P$~--- разрешимый предикат
        (соответственно $Q \in \Pi_k \Leftrightarrow Q(x) = \forall y_1 \exists y_2 \forall y_3 \dots
        P(x, y_1, y_2, \dots, y_n)$);
    \item $\Sigma_k \cup \Pi_k \subseteq \Sigma_{k + 1} \cap \Pi_{k + 1}$;
    \item каждый арифметичный предикат содержится в $\Sigma_k$ для некоторого $k$;
    \item все предикаты из $\Sigma_k$ являются арифметичными.
\end{enumcyr}