Используя теорему Клини докажите, что:
\begin{enumcyr}
    \item существует алгоритм, который на всех входах выводит свой номер;
    \item существует алгоритм, который всюду применим и выдает $1$ на числе, которое является квадратом
        его номера, а на всех остальных входах выдает ноль;
    \item существуют два различных алгоритма $\alg{A}$ и $\alg{B}$, что алгоритм $\alg{A}$ печатает
        $\sharp \alg{B}$, а алгоритм $\alg{B}$ печатает $\sharp \alg{A}$.
\end{enumcyr}