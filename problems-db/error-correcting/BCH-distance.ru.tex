Коды БЧХ (Боуза--Чоудхури--Хоквингема) позволяют исправлять некоторое фиксированное число ошибок в
двоичном кодовом слове. Чтобы описать кодовые слова БЧХ, мы рассмотрим поле $\mathbb{F}_n$ из $n = 2^m$
элементов и многочлены $P(z) = a_0 + a_1 z + a_2 z^2 + \dots + a_{n - d} z^{n - d}$ над этим
полем. Кодовыми словами будут таблицы значений таких многочленов во всех элементах поля (таким образом,
длина кодового слова равна $n$). Мы ограничимся рассмотрением только таких многочленов, которые в каждой
точке поля принимают значение $0$ или $1$. Все такие последовательности битов длины $n$ и образуют код
БЧХ. Докажите, что расстояние данного кода не менее $d$.