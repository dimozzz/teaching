Предикат, заданный на множестве натуральных чисел ($\mathbb{N} = \{0, 1, 2, \dots\}$) назовем
\deftext{арифметичным}, если он выражается с помощью формулы исчисления предикатов в сигнатуре $(+,
\times, =)$ в естественной интерпретации на множестве натуральных чисел. Докажите, что следующие
предикаты являются арифметичными:
\begin{enumcyr}
    \item $x < y$;
    \item $x = 0$;
    \item $x = 1$;
    \item $x = c$, где $c$~--- это некоторая натуральная константа;
    \item $a \bmod b = r$;
    \item $a$~--- это степень двойки;
    \item $a$~--- это степень четверки.
\end{enumcyr}