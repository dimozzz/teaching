\deftext{Коммуникационный протокол} для функции $f\colon X \times Y \to Z$~--- это корневое двоичное
дерево, которое описывает совместное вычисление Алисой и Бобом функции $f$. В этом дереве каждая
внутренняя вершина $v$ помечена меткой $a$ или $b$, означающей очередь хода Алисы или Боба
соответственно. Для каждой вершины, помеченной $a$, определена функция $g_v\colon X \to \{0, 1\}$,
которая говорит Алисе, какой бит нужно послать, если вычисление находится в этой вершине. Аналогично, для
каждой вершины $v$ с пометкой $b$ определена функция $h_v\colon Y \to \{0, 1\}$, которая определяет бит,
который Боб должен отослать в этой вершине. Каждая внутренняя вершина имеет двух потомков, ребро к
первому потомку помечено $0$, а ребро ко второму потомку помечено $1$. Каждый лист помечен значением из
множества $Z$.

Каждая пара входов $(x, y)$ определяет путь от корня до листа в описанном двоичном дереве естественным
обрзом. Будем говорить, что коммуникационный протокол вычисляет функцию $f$, если для всех пар $(x, y)
\in X \times Y$, этот путь заканчивается в листе с пометкой $f(x, y)$.

\deftext{Коммуникационной сложностью} функции $f$ назовем наименьшую глубину протокола, вычисляющего
функцию $f$, и будем ее обозначать $\DCC(f)$.