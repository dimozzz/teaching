Пусть $Z$~--- некоторое множество. \deftext{Деревом решений} назовем такое корневое бинарное дерево, что:
\begin{itemtask}
    \item каждая внутренняя вершина помечена некоторой переменной $x_i$, где $i \in [n]$, а листы
        помечены элементами $Z$;
    \item ребра помечены значениями $0$ или $1$, причем у каждой внутренней вершины один сын помечен
        ребром $0$, а другой $1$.
\end{itemtask}

Любая подстановка $\rho \in \{0, 1\}^n$ индуцирует путь от корня до листа естественным образом. Будем
говорить, что дерево решений вычисляет функцию $f\colon \{0, 1\}^n \rightarrow Z$, если $f(\rho) =
z_{\rho}$, где $z_{\rho}$~--- пометка в листе, который соответствует $\rho$.