Будем называть \deftext{кодом} функцию $C\colon \{a_1, \dots, a_n\} \to \{0, 1\}^*$, сопоставляющую
буквам некоторого алфавита \deftext{кодовые слова}. Если любое сообщение, которое получено применением
кода $C$, декодируется однозначно (т.е. только единственным образом разрезается на образы $C$), то такой
код называется \deftext{однозначно декодируемым}.

Код называется \deftext{префиксным (беспрефиксным, prefix-free)}, если никакое кодовое слово не является
префиксом другого кодового слова.