Рассмотрим следующую эвристику для решения задачи коммивояжера в предположении, что расстояния на рёбрах
удовлетворяют неравенству треугольника. Алгоритм начинает с цикла длины $1$~--- произвольной вершины. На
каждом шаге алгоритм находит вершину $u$, не принадлежащую текущему циклу, расстояние от которой до
ближайшей вершины $v$ цикла минимально, и вставляет $u$ в цикл сразу после $v$.

Покажите, что рассмотренный алгоритм~--- $2$-приближённый, то есть полученный гамильтонов цикл имеет
суммарную длину не более чем в два раза большую оптимальной.

\answer{
    Добавляемая вершина выбирается по тому же принципу, что и в алгоритме Прима нахождения наименьшего
    остовного дерева. Вес наименьшего остовного дерева не превосходит длины наименьшего гамильтонова
    цикла. Из неравенства треугольника получается, что длина гамильтонова цикла не превосходит удвоенного
    веса наименьшего остовного дерева.
}