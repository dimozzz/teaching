Рассмотрим следующую эвристику для решения задачи коммивояжера в предположении, что расстояния на рёбрах
удовлетворяют неравенству треугольника. Алгоритм начинает с цикла длины $1$~--- произвольной вершины. На
каждом шаге алгоритм находит вершину $u$, не принадлежащую текущему циклу, расстояние от которой до
ближайшей вершины $v$ цикла минимально, и вставляет $u$ в цикл сразу после $v$.

Покажите, что рассмотренный алгоритм не является $\alpha$-приближённым ни для какого $\alpha < 2$.

\answer{
    Граф с вершинами $\{1, 2, \ldots, n\}$, расположенными по кругу, расстояние от $i$ до $j$~--- это $|i
    - j|$ (цикл с метрикой кратчайших путей). Тогда алгоритм начнёт с вершины $1$, потом получит цикл из
    $1$ и $2$, потом --- треугольник $1$, $2$, $3$. Далее, переходя от пути $i$ к пути длины $i + 1$,
    алгоритм заменит ребро $(i - 1, i)$ на рёбра $(i, i + 1)$ и $(i - 1, i + 1)$. В итоге получится обход
    длины $2n - 2$, в то время как длина оптимального обхода~--- $n$.
    
    Если вершины расположены на плоскости по окружности, а расстояния евклидовы, это доказательство тоже
    проходит. Таким образом, получаем нижнюю оценку на точность приближения сразу в двух конкретных
    метриках~--- кратчайших путей и евклидовой.
}    