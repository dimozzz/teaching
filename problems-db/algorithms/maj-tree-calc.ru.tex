Рассмотрим сбалансированное троичное дерево высоты $h$, в листьях которого записана данная
последовательность нулей и единиц длины $3^h$. В каждой вершине дерева будем вычислять функцию $\Maj$ от
трёх её потомков. Нужно вычислить значение в корне.
\begin{enumcyr}
    \item Показите, что любой детерминированный алгоритм в худшем случае должен обойти все $3^h$
        листьев.
    \item Постройте вероятностный алгоритм для решения этой задачи с ожидаемым временем работы $c^h$, для
        некоторого $c < 3$.
\end{enumcyr}

\answer{
    \begin{enumcyr}
        \item Индукция по $h$. Если $h = 1$, то пусть какой-то из трёх листьев не просмотрен. Тогда, если
            два других значения различны, алгоритм ошибётся при каком-то из значений третьего
            листа. Стало быть, все три листа должны быть просмотрены.

            Переход: пусть значение в одном из трёх поддеревьев не вычислено, и пусть значения в двух
            других поддеревьях различны. Тогда алгоритм ошибётся при каком-то из значений третьего
            поддерева. Стало быть, значения всех трёх поддеревьев должны быть вычислены, а для этого, по
            предположению индукции, надо просмотреть $3 \cdot 3^{h - 1}$ листьев.
        \item Рекурсивно вычислим значения поддеревьев, случайно выбирая порядок рекурсивных вызовов, и
            не делая третьего вызова, если первые два дали одинаковый результат. Первые два значения
            совпадут с вероятностью $\frac{1}{3}$, и потому ожидаемое время работы на дереве высоты $h$
            составит
            $$
                \frac{1}{3} \cdot 2 T(h - 1) + \frac{2}{3} \cdot 3 T(h - 1) = \frac{8}{3} T(h - 1).
            $$
            Отсюда $T(h) = \left(\frac{8}{3}\right)^h$.
    \end{enumcyr}
}