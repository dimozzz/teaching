Рассмотрим следующий жадный алгоритм для приближенного решения задачи о поиске наименьшего вершинного
покрытия в графе. На каждом шаге алгоритм выбирает вершину наибольшей степени, добавляет её к вершинному
покрытию, и удаляет её вместе со всеми примыкающими к ней рёбрами. Будем повторять этот процесс вплоть до
удаления всех ребер. 

Покажите, что такой алгоритм не будет $2$-приближённым. \hinttext{Комментарий:} на самом деле он не будет
приближенным ни для какого $\alpha \ge 1$.

\answer{
    Пусть $n = 8$. Построим двудольный граф, в одной доле которого~--- $n$ вершин степени $n - 1$ каждая,
    а другая доля строится так:
    \begin{itemtask}
        \item наибольшая степень~--- $n$;
        \item после удаления этой вершины в первой доле остаются $n$ вершин степени $n-2$ каждая;
        \item следующая степень во второй доле --- $n - 1$;
        \item После её удаления в первой доле остаётся 1 вершина степени $n - 2$ и $n - 1$ вершина
                степени $n - 3$;
        \item и так далее, до одних единиц.         
    \end{itemtask}
    
    Для $n = 8$ во второй доле получается последовательность 8 7 7 6 5 4 4 3 3, и ещё восемь вершин малой
    степени. Алгоритм выберет сперва девять указанных вершин, а потом ему ещё останутся какие-то
    8. Получилось 17, это более чем вдвое больше размера минимального вершинного покрытия (8).
}