Рассмотрим следующий алгоритм для приближенного решения задачи о поиске наименьшего вершинного покрытия в
графе. На каждом шаге алгоритм выбирает произвольное ребро $(u, v)$, добавляет вершины $u$ и $v$ к
вершинному покрытию, и удаляет из графа эти вершины вместе со всеми примыкающими к ним рёбрами. Повторяем
этот процесс вплоть до удаления всех рёбер.
\begin{enumcyr}
    \item Покажите, что этот алгоритм гарантированно находит $2$-приближенное решение задачи о вершинном
        покрытии~--- то есть вершинное покрытие, в котором не более чем в вдвое больше вершин, чем в
        наименьшем покрытии.
    \item Постройте графы сколь угодно большого размера, для которых такой алгоритм всегда находит
        вершинное покрытие, в 2 раза большее оптимального.
\end{enumcyr}

\answer{
    \begin{enumcyr}
        \item Рассматривается наименьшее вершинное покрытие $C$. Всякий раз, когда предложенный алгоритм
            выбирает ребро, хотя бы один из концов этого ребра принадлежит $C$. Поэтому для каждого
            элемента $C$ алгоритм добавляет не более двух вершин.
        \item Граф --- набор рёбер, не связанных между собой.
    \end{enumcyr}
}