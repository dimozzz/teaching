Предположим, что алгоритм двойного дерева имеет доступ к оракулу, который по полученному алгоритмом
эйлерову графу выбирает его оптимальный обход и оптимальную последовательность сокращений этого
обхода. Докажите, что даже в этом случае алгоритм двойного дерева для задачи коммивояжера с неравенством
треугольника~--- не $\alpha$-приближённый ни для какого $\alpha < 5 / 6$.

\answer{
    Пример в метрике кратчайших путей (число вершин во внешних лучах $n\to\infty$). На иллюстрации граф,
    удвоенное MST веса $12n + \bigO{1}$, оптимальное сокращение MST веса $10n + \bigO{1}$, оптимальный
    обход графа веса $6n + \bigO{1}$.

    \begin{adjustbox}{valign = t, minipage = {0.32\textwidth}}
        \centering
        \begin{tikzpicture}[black]
            \node[graph-vert] at (0, 0) (m) {};
            \foreach \i in {1, 2, 3}{
                \node[graph-vert] at ({120 * (\i - 1)}:0.5) (a\i) {};
                \node[graph-vert] at ({120 * (\i - 1)}:1) (b\i) {};
                \draw (a\i) -- (m);
                \draw (b\i) -- (a\i);
                \foreach \j in {1, 2, 3}{
                    \path (b\i) ++({120 * (\i - 1) + 60}:{0.5 * \j}) node[graph-vert] (l\i\j) {};
                    \path (b\i) ++({120 * (\i - 1) - 60}:{0.5 * \j}) node[graph-vert] (r\i\j) {};
                }
                \draw (b\i) -- (l\i1);
                \draw (b\i) -- (r\i1);
                \foreach \j [evaluate = \j as \k using int(\j - 1)] in {2, 3}{
                    \draw (l\i\j) -- (l\i\k);
                    \draw (r\i\j) -- (r\i\k);
                }
            }

            \foreach \i [evaluate = \i as \j using {int(Mod(\i, 3) + 1)}] in {1, 2, 3}{
                \draw (l\i3) -- (r\j3);
            }
        \end{tikzpicture}
    \end{adjustbox}%
    \begin{adjustbox}{valign = t, minipage = {0.32\textwidth}}
        \centering
        \begin{tikzpicture}[black]
            \node[graph-vert] at (0, 0) (m) {};
            \foreach \i in {1, 2, 3}{
                \node[graph-vert] at ({120 * (\i - 1)}:0.5) (a\i) {};
                \node[graph-vert] at ({120 * (\i - 1)}:1) (b\i) {};
                \draw (a\i) to[bend left] (m);
                \draw (m) to[bend left] (a\i);
                \draw (b\i) to[bend left] (a\i);
                \draw (a\i) to[bend left] (b\i);
                \foreach \j in {1, 2, 3}{
                    \path (b\i) ++({120 * (\i - 1) + 60}:{0.5 * \j}) node[graph-vert] (l\i\j) {};
                    \path (b\i) ++({120 * (\i - 1) - 60}:{0.5 * \j}) node[graph-vert] (r\i\j) {};
                }
                \draw (b\i) to[bend left] (l\i1);
                \draw (l\i1) to[bend left] (b\i);
                \draw (b\i) to[bend left] (r\i1);
                \draw (r\i1) to[bend left] (b\i);
                \foreach \j [evaluate = \j as \k using int(\j - 1)] in {2, 3}{
                    \draw (l\i\j) to[bend left] (l\i\k);
                    \draw (l\i\k) to[bend left] (l\i\j);
                    \draw (r\i\j) to[bend left] (r\i\k);
                    \draw (r\i\k) to[bend left] (r\i\j);
                }
            }
        \end{tikzpicture}
    \end{adjustbox}%
    \begin{adjustbox}{valign = t, minipage = {0.32\textwidth}}
        \centering
        \begin{tikzpicture}[black]
            \node[graph-vert] at (0, 0) (m) {};
            \foreach \i in {1, 2, 3}{
                \node[graph-vert] at ({120 * (\i - 1)}:0.5) (a\i) {};
                \node[graph-vert] at ({120 * (\i - 1)}:1) (b\i) {};
                \foreach \j in {1, 2, 3}{
                    \path (b\i) ++({120 * (\i - 1) + 60}:{0.5 * \j}) node[graph-vert] (l\i\j) {};
                    \path (b\i) ++({120 * (\i - 1) - 60}:{0.5 * \j}) node[graph-vert] (r\i\j) {};
                }
            }
            \draw (m) to[bend left] (a1);
            \draw (a1) to[bend left] (m);
            \draw (b1) to[bend left] (a1);
            \draw (a1) to[bend left] (b1);
            \draw (b1) to[bend left] (l11);
            \draw (l11) to[bend left] (b1);
            \draw (b1) to[bend left] (r11);
            \draw (r11) to[bend left] (b1);
            \draw (b2) to[bend left] (r21);
            \draw (r21) to[bend left] (b2);
            \draw (b3) to[bend left] (l31);
            \draw (l31) to[bend left] (b3);

            \draw (b2) -- (l21);
            \draw (b3) -- (r31);

            \draw (l23) -- (r33);

            \foreach \j [evaluate = \j as \k using int(\j - 1)] in {2, 3}{
                \draw (l1\j) to[bend left] (l1\k);
                \draw (l1\k) to[bend left] (l1\j);

                \draw (r2\j) to[bend left] (r2\k);
                \draw (r2\k) to[bend left] (r2\j);

                \draw (r1\j) to[bend left] (r1\k);
                \draw (r1\k) to[bend left] (r1\j);

                \draw (l3\j) to[bend left] (l3\k);
                \draw (l3\k) to[bend left] (l3\j);

                \draw (m) -- (a\j);
                \draw (a\j) -- (b\j);

                \draw (l2\j) -- (l2\k);
                \draw (r3\j) -- (r3\k);
            }

        \end{tikzpicture}
    \end{adjustbox}%

    \begin{center}
        \begin{tikzpicture}[black]
            \node[graph-vert] at (0, 0) (m) {};
            \foreach \i in {1, 2, 3}{
                \node[graph-vert] at ({120 * (\i - 1)}:0.5) (a\i) {};
                \node[graph-vert] at ({120 * (\i - 1)}:1) (b\i) {};
                \draw (a\i) to[bend left] (m);
                \draw (m) to[bend left] (a\i);
                \draw (b\i) to[bend left] (a\i);
                \draw (a\i) to[bend left] (b\i);
                \foreach \j in {1, 2, 3}{
                    \path (b\i) ++({120 * (\i - 1) + 60}:{0.5 * \j}) node[graph-vert] (l\i\j) {};
                    \path (b\i) ++({120 * (\i - 1) - 60}:{0.5 * \j}) node[graph-vert] (r\i\j) {};
                }
                \draw (b\i) -- (l\i1);
                \draw (b\i) -- (r\i1);
                \foreach \j [evaluate = \j as \k using int(\j - 1)] in {2, 3}{
                    \draw (l\i\j) -- (l\i\k);
                    \draw (r\i\j) -- (r\i\k);
                }
            }

            \foreach \i [evaluate = \i as \j using {int(Mod(\i, 3) + 1)}] in {1, 2, 3}{
                \draw (l\i3) -- (r\j3);
            }
        \end{tikzpicture}
    \end{center}

    В евклидовой метрике аналогичный пример дает нижнюю оценку $(8 + \sqrt{3}) / 6 \approx 1.622 < 5 /
    3$.
}