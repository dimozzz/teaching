Рассмотрим следующую структуру данных для поиска в упорядоченном множестве чисел $M$. Зафиксируем
параметр $0 < p < 1$~--- вероятность. Структура состоит из уровней, на каждом из которых содержится
связный список с упорядоченным подмножеством $M$. На нижнем, $1$-м уровне~--- отсортированный связный
список, содержащий все элементы $M$. На каждом следующем уровне~--- список, помогающий искать в
предыдущем уровне; а именно, каждый уровень $i$, если в нём больше одного элемента, продлевается вверх на
уровень $i + 1$ так: наименьший элемент продлевается всегда, а каждый из последующих элементов
продлевается с вероятностью $p$.

Поиск элемента $q$ (или его предшественника, если $q \notin M$) проводится так. Сперва находится
предшественник $q$ на самом верхнем уровне~--- последовательным прохождением по списку верхнего уровня с
начала. Далее на каждом шаге ищется предшественник $q$ на один уровень ниже. Спускаясь с уровня $i + 1$
на предыдущий уровень $i$, алгоритм ищет на этом уровне наибольший элемент, не превосходящий $q$,
последовательно проходя по списку. В итоге алгоритм спускается до нижнего уровня и найденный там элемент
возвращается.

\begin{enumcyr}
    \item Найдите математическое ожидание объёма памяти, требуемого для хранения массива, как функцию от
        $n$ и $p$.
    \item Докажите, что математическое ожидание времени поиска~--- $\bigO{\log n}$, где $n$~---
        количество элементов в $M$. Найдите зависимость времени работы от $p$.
    \item Опишите алгоритмы вставки и удаления элементов для этой структуры данных. Определите
        математическое ожидание их времени работы.
\end{enumcyr}

\answer{
    \hinttext{Подсказка:} рассмотрим путь, который проходится во время поиска, в обратном направлении.
    
    Вставка: поиск предшественника, подбрасывание монет для определения максимального уровня, куда идет
    элемент, вставка во все уровни до максимального. Удаление: просто удалить со всех уровней.

    Вероятность, что конкретный элемент окажется на уровне $r$~--- $1 / {p^r}$. Мат.\ ожидание количества
    уровней~--- $\log_{1 / p}n$.

    Докажем, что мат.\ ожидание количества элементов, посещённых на каждом уровне, не превосходит $1 /
    p$. Рассмотрим путь, который мы прошли для одного поиска, пойдем по нему задом наперед. На каждом
    шаге мы идем вверх если можем, либо идем влево. Вероятность, что мы можем вверх~--- это вероятность,
    что элемент попал на уровень выше, то есть $p$. Мат.\ ожидание количества шагов влево на одном уровне
    поэтому $1 / p$ (сумма геометрической прогрессии).

    Ожидаемое время поиска $\bigO{1 / p \log_{1 / p}{n}}$.
}    