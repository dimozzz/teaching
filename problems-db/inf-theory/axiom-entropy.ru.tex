Назовем функцию $f$, определенную на распределениях, \deftext{формальной мерой информации}, если $f$
удовлетворяет следующим свойствам.
\begin{enumtask}
    \item Если $X, Y$ равномерные распределения на множествах $M$ и $M'$ соответственно, причем, $|M| >
        |M'|$, то $f(X) > f(Y)$.
    \item Если $X, Y$ независимы, то $f(X, Y) = f(X) + f(Y)$.
    \item Пусть $B_q$ распределение с вероятностями: $\Pr[B_q = 1] \coloneqq q$ и $\Pr[B_q = 1] \coloneqq
        1 - q$. $f(B_q)$ непрерывна, как функция от $q$.
    \item Если $B$~--- случайная величина, принимающая значения $\{0, 1\}$, то для любой случайной
        величины $X$ выполнено: $f(BX) = f(B) + \Pr[B = 1] f(X \mid B = 1) + \Pr[B = 0] f(X \mid B = 0)$.
\end{enumtask}

Докажите, что $f(X) = c \entropy(X)$, где $c$~--- абсолютная константа.
\tags{теория информации, энтропия}