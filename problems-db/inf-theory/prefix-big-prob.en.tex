Let $C \subseteq \{0, 1\}^∗$ be a finite prefix-free code, and let $W$ be a random string taking values
from $C$. Then, there exists a string $w \in C$ such that $\Pr[W = w] \ge 2^{-|w|}$.