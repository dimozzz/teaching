Покажите, что:
\begin{enumcyr}
    \item существует полиномиальный от $n$ алгоритм $A$, который получает вход, распределенный согласно
        распределению $X$ с $H_{\infty}(X) \ge n^{100}$ и имеет оракульный доступ к функции
        $f\colon \{0, 1\}^n \to  \{0, 1\}$, который  удовлетворяет следующим свойствам:
        \begin{itemize}
            \item если $\Exp[f(u_n)] \ge \frac{2}{3}$, то $A$ отвечает $1$ с вероятностью хотя бы $0.99$;
            \item если $\Exp[f(u_n)] \le \frac{1}{3}$, то $A$ отвечает $0$ с вероятностью хотя бы $0.99$;
        \end{itemize}
        (такой алгоритм будем называть \deftext{аппроксиматором} функции);
    \item существует аппроксиматора без доступа к случайным битам;
    \item если распределение $X$ находится на расстоянии более $\frac{1}{5}$ от каждого распределения $Y$
        с $H(Y) \ge \frac{n}{2}$, то не существует аппроксиматора, вход которого распределен согласно $X$.
\end{enumcyr}