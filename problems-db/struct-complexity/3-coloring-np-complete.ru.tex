\begin{enumcyr}
    \item Постройте граф со следующими выделенными вершинами: $T, t_1, t_2, r$, со следующим
        свойством: в любой правильной раскраске графа в три цвета вершина $r$ покрашена в тот же цвет,
        что и $T$, тогда и только тогда, когда хотя бы одна из вершин $t_1, t_2$ покрашена в тот же цвет,
        что и $T$.
    \item Постройте граф со следующими выделенными вершинами: $T, t_1, t_2, t_3, r$, со следующим
        свойством: в любой правильной раскраске графа в три цвета вершина $r$ покрашена в тот же цвет,
        что и $T$, тогда и только тогда, когда хотя бы одна из вершин $t_1, t_2, t_3$ покрашена в тот же
        цвет, что и $T$.
    \item Докажите, что язык графов, которые можно раскрасить в три цвета, $\NP$-полон
        (\hinttext{подсказка:} создайте в графе треугольник с вершинами: $\mathrm{True}, \mathrm{False},
        \mathrm{Base}$).
\end{enumcyr}