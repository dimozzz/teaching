Пусть дан граф $G \coloneqq (V, E)$, $|V| = n$, $|E| = m$, $x \in \mathbb{R}^m$. Рассмотрим следующую
задачу линейного программирования:
\begin{itemtask}
    \item $\sum\limits_{e \in E} x_e \rightarrow \max$;
    \item $\forall e \in E ~~ x_e \ge 0$;
    \item $\forall v \in V ~~ \sum\limits_{e \in E_v} x_e \le 1$, где $E_v$~--- множество ребер,
        инцидентных вершине $v$.        
\end{itemtask}

\begin{enumcyr}
	\item Какой <<физический>> смысл у данной задачи? А если вектор $x$ имеет целочисленные координаты?
    \item Докажите, что если граф $G$ двудольный, то оптимум достигается в вершине с целочисленными
	    координатами.
\end{enumcyr}

\answer{
    \begin{enumcyr}
        \item Максимальное паросочетание.
        \item Покажем, что матрица тотально унимодулярна. Заметим, что строки~--- это вершины, а столбцы
            это ребра. В каждом столбце ровно две единицы.

            Пусть нам дали квадратную подматрицу. Если в ней есть строка (или столбец) с одной едницей,
            то можно вырезать ее и столбец (строку), где единица. Определитель от этого не поменяется
            (например, можно посмотреть на разложение по строке). Теперь осталась подматрица, где в
            каждом столбце ровно две единицы. Заметим, что в случае двудольного графа ее определитель равен
            нулю, т.е. можно из строк первой доли вычесть строки второй доли и получить ноль.
    \end{enumcyr}
}    