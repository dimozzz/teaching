Пусть $G \coloneqq (V, E)$~--- двудольный граф. Реберное покрытие $G$~--- это такое множество $X
\subseteq E$, что любая вершина $v \in V$ инцидентна хотя бы одному ребру из множества $X$.
\begin{enumcyr}
    \item Опишите задачу линейного программирования для нахождения минимального реберного покрытия.
    \item Докажите, что если граф регулярный, то в минимальном покрытие не более $\frac{|V|}{2}$ ребер.
    \item Постройте двойственную задачу линейного программирования и изучите ее комбинаторный смысл.
\end{enumcyr}
\tags{линейное программирование, комбинаторика}

\answer{
    \begin{enumcyr}
        \item $\sum\limits_{e \in E} x_e \to \min$,
            \begin{align*}
              \sum\limits_{e \ni v} x_e & \ge 1 \\
              x_e & \ge 0
            \end{align*}
        \item Регулярный, следовательно есть паросочетание, его и выдадим.
        \item $\sum\limits_{v \in V} -y_v \to \min$,
            \begin{align*}
              (u, v) \in E: -y_u - y_v & \ge -1 \\
              y_v & \ge 0
            \end{align*}

            Перепишем:
            $\sum\limits_{v \in V} y_v \to \max$,
            \begin{align*}
              (u, v) \in E: y_u + y_v & \le 1 \\
              y_v & \ge 0
            \end{align*}

            Получилось максимальное независимое множество.
    \end{enumcyr}    
}    