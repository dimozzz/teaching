Говорят, что полиэдр $\mathcal{P} \coloneqq \{x \mid x \ge 0, Ax \le b\}$ обладает свойством
целочисленной декомпозиции, если для любого натурального $k$ всякий целый вектор из $k \cdot \mathcal{P}$
представим в виде суммы $k$ целых векторов из $\mathcal{P}$.
\begin{enumcyr}
    \item Покажите, что если матрица $A$ тотально унимодулярна, а вектор $b$ целый, то $P$ обладает
        свойством целочисленной декомпозиции.
    \item Используя предыдущий пункт покажите, что всякий $k$-регулярный двудольный граф допускает
        раскраску множества ребер в $k$ цветов, при которой одноцветные ребра не имеют общих концов.
\end{enumcyr}

\answer{
    Индукция по $k$. Для $k = 1$ утверждение тривиально.

    Пусть мы доказали необходимое для $k$, покажем для $k + 1$. Пусть $y = k x_0$~--- целочисленный
    вектор, где $x_0 \in \mathcal{P}$. Рассмотрим полиэдр $\mathcal{P}' \coloneqq \{x \mid 0 \le x \le y,
    Ay - kb + b \le Ax \le b\}$. Поскольку $A$ тотально унимодулярна, то у политопа $P'$ вершины
    целочисленны. Действительно, можно переписать условия, задающие политоп $\mathcal{P}'$ более
    наглядным образом:
    $$
        \mathcal{P}' = \{x \mid 0 \le x; -x \le y, -Ax \le (Ay - kb + b), Ax \le b\}).
    $$
    Таким образом матрица будет следующая:
    $$
        \begin{pmatrix}
          A \\
          -A \\
          -I
        \end{pmatrix},
    $$
    и она тотально унимодулярна. Поскольку $x_0 \in \mathcal{P'}$, то $\mathcal{P'}$ не пуст, а поскольку
    его вершины целые, то мы можем взять произвольный целый вектор $x_1 \in \mathcal{P'}$ (как следствие
    $x_1 \in \mathcal{P}$). Заметим, что $y_1 = y - x_1$ целочисленный вектор в $k \cdot \mathcal{P}$ и
    мы можем применить предположение индукции.

    Пункт б). Рассмотрим политоп $\mathcal{P}$ дробных паросочетаний в полном двудольном графе (вспомним,
    что этот политоп тотально унимодулярен). Заметим, что $k$-регулярный граф~--- это целочисленная точка
    в $k \cdot \mathcal{P}$, следовательно данную точку можно представить, как сумму $k$ целочисленных
    точек в политопе $\mathcal{P}$ (каждая из которых является паросочетанием).
}
    