\begin{enumerate}
    
	\item Определение детерминированного коммуникационного протокола. Нижняя оценка $D(f) > n$ сложности функции $f(x, y) = x + y$
		сложения $n$-битовых чисел. Оценка $D(f) = c \log(n)$ для коммуникационной сложности медианы мультимножества. Верхняя
        оценка $D(f) = O(\log^2 n)$ для коммуникационной сложности функции $CIS$.
    \item Связь глубины и размера коммуникационного протокола. Метод прямоугольников для доказательства нижних
		оценок. Доказательство нижних оценок для функций $EQ, DISJ, IP$.
	\item Метод рангов. Гипотеза о связи коммуникационной сложности и ранга матрицы. Теорема Ниссана-Вигдерсона о зазоре между
		коммуникационной сложностью и логарифмом ранга матрицы.


    \item Недетерминированная коммуникационная сложность. Покрытие матрицы. $D(f) \le \min(C^0 + 1, C^1 + 1)$. $D(f) =
		O(N^0(f) N^1(f))$. Верхние и нижние оценки на $N^0(f), N^1(f)$ для функций $EQ, GT$.
    \item Универсальность метода размера прямоугольников для оценки $N^1(f)$ с точностью до $\log(n)$: $N^1(f) \le \log(B^1_*(f))
		+ \log(n) + O(1)$.
	\item Функция $DISJ_{\log(n)}$. Верхняя оценка $O(\log(n))$ на недетерминированную коммуникационную сложность и нижняя оценка
		$O(\log^2(n))$ на детерминированную.

    \item Вероятностные коммуникационные протоколы. Частный и общие источники случайных битов. Протоколы с односторонней и
		двусторонней ошибкой, безошибочные протоколы. Уменьшение ошибки. Примеры протоколов для функции $EQ$.
    \item $R^1_{\epsilon}(f) \ge \log(C^1(f))$. $R_{\epsilon}(f) = \Omega(\log(C^1(f))$.
	\item Теорема о переделывании общего источника в частный.
    \item Распределения на входах. Распределенная коммуникационная сложность. Связь распределенной и вероятностной коммуникационной
		сложность. Нижняя оценка на функцию $DISJ$.
    \item Discrepancy. Оценка распределенной сложности.
 
    \item Коммуникационная сложность подсчета прямого произведения. $R(EQ^{\log(n)}) = \Theta(\log^{2}(n))$.
	\item Коммуникационная сложность подсчета прямого произведения. $B^1_*(f \land g) \ge B^1_*(f) B^1_*(g)$. $D(f^l) = \Omega(l
		(\sqrt{D(f)} - \log(n) - O(1)))$.


    \item Коммуникационная сложность отношений. Метод прямоугольников. $DISJ$.


	\item Multi-party коммуникационная сложность. $NOF$ модель. Метод цилиндров. Discrepancy.
	\item Нижние оценки для $IP_k$ и $DISJ_{n, k}$.
	\item Multi-party коммуникационная сложность. $NIH$ модель. Задача об упаковке множеств.

	\item Применение коммуникационной сложности для доказательства нижних оценок на глубину формул. Теорема
		Карчмара-Вигдерсона. Доказательство суперполиномиальной нижней оценки на монотонные формулы.


\end{enumerate}




%%% Local Variables:
%%% mode: latex
%%% TeX-master: t
%%% End:
