\setcounter{curtask}{1}

\mytitle{1 (на 4.04)}

\begin{task}
    Найдите все собственные числа полного графа с $n$ вершинами 
	а) без петель; б) с петлями.
\end{task}

\begin{task}
    Пусть собственные числа матрицы смежности графа с $n$ вершинами и 
	$m$ ребрами (в графе нет петель и кратных ребер) равны 
	$\lambda_1, \lambda_2, \dots, \lambda_n$. Вычислите 
	а) $\sum_{i=1}^n \lambda_i$; б) $\sum_{i=1}^n \lambda_i^2$;
	в) Какой смысл у $\sum_{i=1}^n \lambda_i^3$.
\end{task}

\begin{task}
    Если $A, B$~--- это симметрические стохастические матрицы (сумма элементов
	по строкам равна 1), а $\lambda(M)$~--- это модуль второго по величине
	собственного числа, то $\lambda(A + B) \le \lambda(A) + \lambda(B)$.
\end{task}

\begin{task}
	Докажите, что в любом $d$-регулярном графе с $n$ вершинами найдется такое
    множество $S$ из $\frac{n}{2}$ вершин, что $E(S,\overline{S})\le \frac{dn}{4}$.
\end{task}

\begin{task}
    Рассмотрим случайный двудольный граф, в котором вершины разбиты на две доли: $L$
	и $R$ по $n$ вершин в каждой доле. Для каждой вершины левой доли выбирается
	независимо случайным образом $d$ соседей из правой доли (кратные ребра
	разрешены). Докажите, что для всех $\epsilon > \frac{1}{d}$ найдется такое число
	$\alpha>0$, что с большой вероятностью выполняется следующее свойство: для
    каждого множества $S \subseteq L$ размера не больше $\frac{\alpha n}{d}$
    выполняется $|\Gamma(S)| \ge |S|(1 - \epsilon)d$.
\end{task}

\begin{task}
    Докажите, что хроматическое число алгебраического $(n,d,\alpha)$-экспандера
    больше, чем $\frac{1}{\alpha}$.
\end{task}

\begin{task}
    Пусть $G$~--- это алгебраический $(n,d,\alpha)$-экспандер. Пусть 
	$k\le \frac{1}{\alpha}$ и $n$ делится на $k$. Докажите, что если покрасить
    вершины  в $k$ цветов так, чтобы каждый цвет использовался ровно $\frac{n}{k}$
	раз, то найдется хотя бы одна вершина, среди соседей которой встречаются все $k$
    цветов.
\end{task}