\begin{enumerate}
    \item Реберное расширение. Матрица смежности графа и ее спектр. Связность графа. Cheeger's inequality.
    \item Expander mixing lemma. Существование алгебраических экспандеров. Нижняя оценка на второе собственное число.
    \item Подстановочные произведения. Зиг-заг произведение. Оценка собственного числа в зиг-заг произведении. Явная
        конструкция алгебраического экспандера. Лемма о разложении. 
    \item Оценка собственного числа в сбалансированном подстановочном произведении произведении. Алгоритм Рейнгольда. Спектр
        аффинной плоскости. Графы Кэли.
    \item Спектр графов Кэли. Примеры (цикл, гиперкуб). Пример графа Рамануджана (без доказательства). Экспандеры из кодов
        исправляющих ошибки. Экспандерные коды и алгоритм декодирования.
    \item Понижение ошибки в классе $\RP$. Сэмплеры. Наивный сэмплер. Попарно-независимый сэмплер. $2$-независимое
        множество. Конструкция $2$-независимого множества.
    \item Матричная конструкция $2$-независимого множества. Сэмплер, основанный на медиане усреднений. Случайное блуждаение
        по экспандеру. Булев сэмплер из экспандера.
\end{enumerate}

\breakline

Темы разобранные на практике.
\begin{enumerate}
    \item Вероятностный алгоритм для задачи $\lang{UPATH}$.
    \item Вероятностный двудольный экспандер.
    \item Случайное блуждаение по экспандеру II.
    \item Хиттеры. Хиттер из сэмплера.
\end{enumerate}