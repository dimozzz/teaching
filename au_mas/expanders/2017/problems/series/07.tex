\documentclass[a4paper, 12pt]{article}
% math symbols
\usepackage{amssymb}
\usepackage{amsmath}
\usepackage{mathrsfs}
\usepackage{mathseries}


\usepackage[margin = 2cm]{geometry}

\tolerance = 1000
\emergencystretch = 0.74cm



\pagestyle{empty}
\parindent = 0mm

\renewcommand{\coursetitle}{DM/ML}
\setcounter{curtask}{1}

%\setmathstyle{Апрель 9}{Теория информации}{2 курс}


\begin{document}

\setcounter{curtask}{25}


\libproblem{expanders}{sampler-bool}

\begin{definition*}
    Хиттером называется вероятностный алгоритм $H^f(n, \varepsilon, \delta)$, который получает оракульный
    доступ к функции $f \in F_{n, \varepsilon}$ и удовлетворяет следующему свойству: $\Pr[f(H^f(n, \varepsilon,
    \delta)) = 1] \ge 1 - \delta$, где $F_{n, \varepsilon}$~--- множество всех функций $g\colon \{0, 1\}^n \to
    \{0, 1\}$, для которых $|g^{-1}(1)| \ge \varepsilon 2^n$.
\end{definition*}

\libproblem{expanders}{sampler-to-hitter}
\libproblem{expanders}{random-walk-entropy}
\libproblem{expanders}{vertex-expansion}


\breakline

\libproblem[11]{expanders}{all-graphs-eigenvalue}
\libproblem[14]{expanders}{random-ind-set}
\libproblem[21]{expanders}{big-con-component}
\libproblem[22]{expanders}{error-bpp}
\libproblem[24]{expanders}{path-eigenvalue}


\end{document}



%%% Local Variables:
%%% mode: latex
%%% TeX-master: t
%%% End:
