Код $C: \{0, 1\}^k \rightarrow \{0, 1\}^n$ называется систематическим, если существуют такие числа $j_1, \dots, j_k$, что для
любого $(x_1, \dots, x_k) \in \{0, 1\}^k$ в кодовом слове $(y_1, \dots, y_n) = C(x_1, \dots, x_n)$ биты $y_{j_1}, \dots,
y_{j_k}$ равны соответствующим битам исходного слова. Другими словами, все <<информационные биты>> непосредственно входят в
кодовое слово. Докажите, что всякий линейный код $C: \{0, 1\}^k \rightarrow \{0, 1\}^n$ можно переделать в систематический
линейный код $C': \{0, 1\}^k \rightarrow \{0, 1\}^n$, сохранив прежнее множество кодовых слов (и проверочную матрицу).