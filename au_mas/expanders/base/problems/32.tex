Алиса задумывает целое число от $1$ до $n$. Боб должен отгадать это число, задавая Алисе вопросы, требующие ответы да или
нет. Алиса может солгать в одном из ответов. Стратегия Боба называется адаптивной, если очередной задаваемый вопрос может
зависеть от ответов, данных Алисой на предыдущих шагах. Стратегия называется неадаптивной, если Боб сразу предъявляет список
всех своих вопросов, не дожидаясь первых ответов Алисы.
\begin{enumcyr}
    \item Какое минимальное число вопросов Должен задать Боб, чтобы гарантированно узнать задуманное Алисой число для $n =
	    200$ (для адаптивной стратегии)?
    \item Какое минимальное число неадаптивных вопросов должен задать Боб для $n = 150$.
    \item Какое минимальное число адаптивных вопросов должен задать Боб для $n = 150$.
\end{enumcyr}