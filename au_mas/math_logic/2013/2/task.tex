\setcounter{curtask}{7}

\mytitle{2 (на 31.09)}

Будем говорить, что множества $A$ и $B$ равны, если $A \le B$ и $B \le A$.

\begin{task}
    Докажите, что $[0, 1] = (0, 1) = [0, 1) = \mathbb{R}$
\end{task}

\begin{task}
    Докажите, что $\mathbb{R}^{\mathbb{N}} \le \mathbb{R}$
\end{task}

\begin{task}
    Докажите, что $\mathbb{N}^{\mathbb{R}} \le 2^{\mathbb{R}}$
\end{task}


\begin{task} (полнота резолюций)
    Пусть $\neg F$~--- противоречивая формула в КНФ, докажите что у нее есть вывод в 
    резолюциях. (подсказка: индукция по числу переменных).
\end{task}


Исчисление предикатов:

\begin{task}
    Придумайте теорию:
    а) без бесконечных моделей
    б) без конечных моделей
\end{task}

Спектром теории (формулы) называется множество всех конечных размеров моделей данной
теории.

\begin{task}
    Предъявите теорию спектр которой:
    а) все числа
    б) все четные числа
\end{task}

\begin{task}
    Предъявите теорию спектр которой:
    а) все степени простых чисел
    б) все числа не являющиеся степенями простого
    в) состоит из квадратов чисел
\end{task}


\breakline

\begin{ptask}{1}
    Докажите, что не существует биекции между $\mathbb{N}$ и $\mathbb{R}$.
\end{ptask}

\begin{ptask}{2}
    Докажите, что $\mathbb{N}^c \le \mathbb{N}$.
\end{ptask}

\begin{ptask}{3}
    Докажите, что:
    а) $\mathbb{R} \le [0, 1]$
    б) $\mathbb{R} \le 2^{\mathbb{N}}$
    в) $\mathbb{R} \times \mathbb{R} \le \mathbb{R}$ (подсказка:
	    придумать явную биекцию квадрата на одну из сторон)
\end{ptask}

\begin{ptask}{3}
    Приведите пример формулы длины $n$ такой, что ее минимальный
    размер в КНФ $\Omega(2^n)$.
\end{ptask}

\begin{ptask}{4} (Корректность)
    Если $\neg F$ (записанная в КНФ) имеет вывод в резолюциях, то
    формула $F$ является тавтологией.
\end{ptask}

\begin{ptask}{5} (Теорема о компактности)
    Пусть $\Gamma$ множество формул. $\Gamma$ непротиворечиво тогда и
    только тогда, когда любое конечное подмножество $\Gamma$ непротиворечиво.
\end{ptask}