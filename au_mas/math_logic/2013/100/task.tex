\setcounter{curtask}{1}

\begin{task}
    Докажите, что любой частичный порядок продолжается до линейного.
\end{task}

\begin{task}
    Будет ли теория $Th(\mathbb{N}, =, <)$ конечно аксиоматизируемой?
\end{task}

\begin{task}
    Докажите, что теория $Th(\mathbb{Z}, S, =, 0)$ не является
    конечно-аксиоматизируемой.
\end{task}

\begin{task}
    Пусть $T$ теория следующего языка: $\{<, R, B\}$, где $R$ (red) и $B$ (blue)
    унарные предикаты.
    
	$T$ содержит все аксиомы DLO, а также: 
	\[ \forall xy \exists zw (x < z < w < y \; \wedge \; R(z) \; \wedge \; B(w)) \]
	\[ \forall x \; (R(x)\; \vee \; B(x)) \]
	\[ \forall x \; (R(x) \leftrightarrow \neg B(x). \]
    
	Докажите, что $T$ имеет одну (с точностью до изоморфизма) модель счетной
    мощности.
\end{task}


\begin{task}
    Постройте теорию, спектр которой:
    а) $p^k$, где $p$ --- проcтое число, а $k > 2$;
    а) $p$, где $p$ --- проcтое число.
\end{task}

\begin{task}
    Докажите, что если для любого $p > p_0$ формула $\phi$ верна в алгебраически
    замкнутом поле в любом поле с характеристикой $p$, данная формула верна и в
    алгебраически замкнутом поле с характеристикой 0.
\end{task}

\begin{task}
  	Существует ли модель $RCF$ (Real Closed Field), которая содержит $\mathbb{R}$ и
    такую точку $r$, что $r$ больше любого натурального числа $n$.
\end{task}

\begin{task}
    Пусть у теории есть счетная модель, докажите, что у нее есть модель любой
    бесконечной мощности.
\end{task}

\begin{task}
    Покажите, что в интерпретации $(\mathbb{Z}, =, <)$ предикат $y = x
    + 1$ невыразим при помощи бескванторной формулы.
\end{task}