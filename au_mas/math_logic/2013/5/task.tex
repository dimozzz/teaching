\setcounter{curtask}{27}

\mytitle{5 (на 5.11)}

\begin{task}
    Предъявите теорию $T$ у которой нет счетной модели, но есть бесконечная модель.
\end{task}

\begin{task}
    Докажите, что для всякого поля $K$ существует его расширение $K'$
    такое, что каждый многочлен с коэффициентами из $K$ имеет корень
    в $K'$ (можно пользоваться тем, что для конкретного многочлена
    такое расширение существует).
\end{task}

\begin{task}
    Докажите, что если формула $\phi$ верна в алгебраически замкнутом
    поле в характеристикой 0, то найдется $p_o$, что для любого $p >
    p_o$ $\phi$ будет верна в алгебраически замкнутом поле с
    характеристикой $p$.
\end{task}

\begin{task}
    Докажите, что если теория конечно аксиоматизируема
    (т.е. существует конечное множество теорем этой теории из которых выводятся все
    остальные), то конечное множество аксиом можно выбрать из самой
    теории, а не из теорем.
\end{task}


\breakline

\begin{ptask}{22}
    Будет ли интерпретация $(\mathbb{Q}, =, <)$ элементарно
    эквивалентна:
    а) $(\mathbb{Q} + \mathbb{Q}, =, <)$
    б) $(\mathbb{Q} + \mathbb{R}, =, <)$
\end{ptask}

\begin{ptask}{23}
    Будет ли интерпретация $(\mathbb{N}, =, <)$ элементарно
    эквивалентна: $(\mathbb{N} + \mathbb{Z}, =, <)$
\end{ptask}

\begin{ptask}{24}
    а) Докажите, что в интерпретации $(Q, =, <, +,$ рациональные
    константы) допустима элиминация кванторов.
    б) Докажите, что интерпретации $(Q, =, <, +,$ рациональные
    константы) и $(\mathbb{R}, =, <, +,$ рациональные константы)
    элементарно эквивалентны.
    в) Пусть единичный квадрат разрезан на несколько меньших
    квадратов. Докажите, что все они имеют рациональные стороны.
\end{ptask}

\begin{ptask}{26}
    Предъявите не менее $2^{|\mathbb{R}|}$ неизоморфных моделей плотного линейного
    порядка континуальной мощности.

    (пункт 0: $\mathbb{Q} + \cdot + \mathbb{R}$ не изоморфно $\mathbb{Q} +
    \mathbb{R}$).
\end{ptask}
