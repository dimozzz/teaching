\setcounter{curtask}{1}

\mytitle{1 (на 5.09)}

\begin{task}
    Докажите, что не существует биекции между $\mathbb{N}$ и $\mathbb{R}$.
\end{task}

Будет говорить, что множество $A$ не больше множества $B$, если
существует биективное отображение из $A$ в подмножество $B$, и строго
меньше, если не существует биекции из $B$ в $A$. Под степенью будем
подразумевать декартову степень.

\begin{task}
    Докажите, что $\mathbb{N}^c \le \mathbb{N}$.
\end{task}

\begin{task}
    Докажите, что $\mathbb{N} < 2^{\mathbb{N}} \le \mathbb{R}$
\end{task}

\begin{task}
    Докажите, что:
    а) $\mathbb{R} \le [0, 1]$
    б) $\mathbb{R} \le 2^{\mathbb{N}}$
    в) $\mathbb{R} \times \mathbb{R} \le \mathbb{R}$ (подсказка:
	    придумать явную биекцию квадрата на одну из сторон)
\end{task}

\begin{task}
    Докажите, что любая функция над конечным полем представляется в
	виде полинома.
\end{task}

\begin{task}
    Приведите пример формулы длины $n$ такой, что ее минимальный
    размер в КНФ $\Omega(2^n)$.
\end{task}