\setcounter{curtask}{38}

\mytitle{7 (на 21.11)}

\begin{task}
    Докажите, что если формула $\phi$ верна в алгебраически замкнутом
    поле в характеристикой 0, то найдется $p_o$, что для любого $p >
    p_o$ $\phi$ будет верна в алгебраически замкнутом поле с
    характеристикой $p$.
\end{task}

\breakline

\begin{ptask}{26}
    Будет ли интерпретация $(\mathbb{N}, =, <)$ элементарно
    эквивалентна: $(\mathbb{N} + \mathbb{Z}, =, <)$
\end{ptask}

\begin{ptask}{27}
    в) Пусть единичный квадрат разрезан на несколько меньших
    квадратов. Докажите, что все они имеют рациональные стороны.
\end{ptask}

\begin{ptask}{33}(резолюция для исчисления предикатов)
    Система доказывает противоречивость замкнутой формулы следующим
    образом: формула приводится в предваренную нормальную форму, затем
    проводится скулемизация (избавляемся от функциональных
    символов). Получается формула $\forall x_1 x_2 \dots
    \phi(x_1, \dots, x_n)$. Для формулы $\phi$ делается резолюционный
    вывод. Докажите корректность и полноту данного метода.
\end{ptask}

\begin{ptask}{36}
    Будет ли теория $Th((Z, <, =))$ конечно аксиоматизируемой.
\end{ptask}

\begin{ptask}{37}
    Будет ли теория $Th((N, <, =))$ конечно аксиоматизируемой.
\end{ptask}