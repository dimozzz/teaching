\setcounter{curtask}{33}

\mytitle{6 (на 7.11)}

\begin{task}(резолюция для исчисления предикатов)
    Система доказывает противоречивость замкнутой формулы следующим
    образом: формула приводится в предваренную нормальную форму, затем
    проводится скулемизация (избавляемся от функциональных
    символов). Получается формула $\forall x_1 x_2 \dots
    \phi(x_1, \dots, x_n)$. Для формулы $\phi$ делается резолюционный
    вывод. Докажите корректность и полноту данного метода.
\end{task}

\begin{task}
    Докажите, что для всякого поля $K$ существует его расширение $K'$
    такое, что каждый многочлен с коэффициентами из $K$ имеет корень
    в $K'$ (можно пользоваться тем, что для конкретного многочлена
    такое расширение существует).
\end{task}

\begin{task}
    Докажите, что если теория конечно аксиоматизируема
    (т.е. существует конечное множество теорем этой теории из которых выводятся все
    остальные), то конечное множество аксиом можно выбрать из самой
    теории, а не из теорем.
\end{task}

Пусть $I$~--- интерпретация. $Th(I)$~--- множество формул верной в
данной интерпретации.

\begin{task}
    Будет ли теория $Th((Z, <, =))$ конечно аксиоматизируемой.
\end{task}

\begin{task}
    Будет ли теория $Th((N, <, =))$ конечно аксиоматизируемой.
\end{task}


\breakline

\begin{ptask}{26}
    Будет ли интерпретация $(\mathbb{N}, =, <)$ элементарно
    эквивалентна: $(\mathbb{N} + \mathbb{Z}, =, <)$
\end{ptask}

\begin{ptask}{27}
    а) Докажите, что в интерпретации $(Q, =, <, +,$ рациональные
    константы) допустима элиминация кванторов.
    б) Докажите, что интерпретации $(Q, =, <, +,$ рациональные
    константы) и $(\mathbb{R}, =, <, +,$ рациональные константы)
    элементарно эквивалентны.
    в) Пусть единичный квадрат разрезан на несколько меньших
    квадратов. Докажите, что все они имеют рациональные стороны.
\end{ptask}
