\setcounter{curtask}{15}

\mytitle{4 (на 09.04)}

\begin{task}
    Рассмотрим теорию $Th(\mathbb{Z}, s)$, $s(x) = x + 1$. Ракие типы входят в
    $S_n(T)$? Какие из них изолированы?
\end{task}

\begin{task}
    Рассмотрим теорию $Th(\mathbb{Z}, s, <)$, $s(x) = x + 1$. Ракие типы входят в
    $S_n(T)$? Какие из них изолированы?
\end{task}

\begin{task}
    Пусть $A \subseteq B \subseteq M$, $\phi$~--- $L_A$ формула. Докажите, что если
    $\phi$ изолирует $t_p^M(a / B)$, то $\phi$ изолирует $t_p^M(a / A)$.
\end{task}

\begin{task}
    Предъявите контрпример к ommiting type theorem, если разрешить иметь несчетный
    язык.
\end{task}


\breakline



\begin{ptask}{6}
    Докажите, что теория алгебраически замкнутых полей фиксированной характеристики
    $\lambda$-категорична для любого $\lambda > \aleph_0$
\end{ptask}

\begin{ptask}{9}
    Определим игру $P_{\omega}$, как $EF_{\omega}$, но первый игрок обязан выбирать
    точки из первой модели, а второй из второй.
	а) докажите, что второй игрок имеет выигрышную стратегию тогда и только тогда,
    когда первая модель вкладывается во вторую.
    б) а что если, первый игрок выбирает на четных шагах из первой структуры, а на
    нечетных из второй?
\end{ptask}

\begin{ptask}{10}
    Пусть $L = \{E\}$, где $E$~--- бинарное отношение. $T$~--- $L$-теория,
    утверждающая, что $E$~--- отношение эквивалентности и бесконечным числом классов.
    а) запишите аксиомы теории $T$.
    б) Сколько неизоморфных моделей у теории $T$ мощности: $\aleph_0$? $\aleph_1$?
    $\aleph_2$? $\aleph_{\omega_1}$?
\end{ptask}