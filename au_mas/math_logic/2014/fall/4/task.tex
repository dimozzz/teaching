\setcounter{curtask}{21}

\mytitle{4 (на 22.10)}


\begin{task}
    Придумайте теорию:
    а) без бесконечных моделей
    б) без конечных моделей
\end{task}

Спектром теории (формулы) называется множество всех конечных размеров моделей данной
теории.

\begin{task}
    Предъявите теорию спектр которой:
    а) все числа
    б) все четные числа
\end{task}

\begin{task}
    Предъявите теорию спектр которой:
    а) все степени простых чисел
    б) все числа не являющиеся степенями простого
    в) состоит из квадратов чисел
\end{task}

\begin{task}
    Докажите, что предикат $x = 2$ невыразим в множестве целых чисел с
    предикатами равенства и $x$ делит $y$.
\end{task}

\begin{task}
    Докажите, что предикат $y = x + 2011$ невыразим в интерпретации
    $(\mathbb{Z}, =, x \mapsto x^2)$.
\end{task}

Две интерпретации одной сигнатуры называются элементарно
эквивалентными, если каждая замкнутая формула в первой интерпретации
верна тогда и только тогда, когда она верна во второй.

\begin{task}
    Будет ли интерпретация $(\mathbb{N}, =, <)$ элементарно
    эквивалентна: $(\mathbb{N} + \mathbb{N}, =, <)$. (Две копии нат. чисел, все
    элементы из второй копии больше элементов из первой).
\end{task}

\begin{task}
    Выразим ли предикат $x = 0$ в интерпретации $(\mathbb{N}, =, <)$
    а) бескванторной формулой
    б) любой формулой
\end{task}


\breakline

\begin{ptask}{20}
    Добавим к исчислению высказываний правило подстановки. Оно разрешает заменить в
    исходной формуле все переменные на произвольные формулы (вхождение одной
    переменной заменяются на одинаковые формулы). Докажите, что класс выводимых
    формул не изменится, но лемма о дедукции перестанет быть верной. 
\end{ptask}
