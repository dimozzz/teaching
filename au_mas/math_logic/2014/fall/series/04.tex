\documentclass[a4paper, 12pt]{article}
% math symbols
\usepackage{amssymb}
\usepackage{amsmath}
\usepackage{mathrsfs}
\usepackage{mathseries}


\usepackage[margin = 2cm]{geometry}

\tolerance = 1000
\emergencystretch = 0.74cm



\pagestyle{empty}
\parindent = 0mm

\renewcommand{\coursetitle}{DM/ML}
\setcounter{curtask}{1}

\setmathstyle{22.10}{Задание 4}{АУ. 5 курс}
\setcounter{curtask}{21}

\begin{document}

\libproblem{math-logic}{th-example-fin-infin-model}

\begin{definition*}
    \deftext{Спектром} теории (формулы) будем называть множество всех конечных размеров моделей данной
    теории.
\end{definition*}

\libproblem{math-logic}{th-specter-even}
\libproblem{math-logic}{th-specter-prime-pow}
\libproblem{math-logic}{divisibility-predicate}
\libproblem{math-logic}{x-plus-2011-x-to-x2}

\begin{definition*}
    Будем говорить, что две интерпретации одной сигнатуры являются \deftext{элементарно эквивалентными},
    если каждая замкнутая формула в первой интерпретации верна тогда и только тогда, когда она верна во
    второй.
\end{definition*}

\libproblem{math-logic}{n-less-equiv-n-plus-n}
\libproblem{math-logic}{x-eq-0-n-less}

\breakline

\libproblem[20]{math-logic}{prop-substitution-and-ded}

\end{document}



%%% Local Variables:
%%% mode: latex
%%% TeX-master: t
%%% End:
