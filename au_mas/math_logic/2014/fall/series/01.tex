\documentclass[a4paper, 12pt]{article}
% math symbols
\usepackage{amssymb}
\usepackage{amsmath}
\usepackage{mathrsfs}
\usepackage{mathseries}


\usepackage[margin = 2cm]{geometry}

\tolerance = 1000
\emergencystretch = 0.74cm



\pagestyle{empty}
\parindent = 0mm

\renewcommand{\coursetitle}{DM/ML}
\setcounter{curtask}{1}

\setmathstyle{10.09}{Задание 1}{АУ. 5 курс}

\begin{document}

\libproblem{set-theory}{real-not-countable}

\begin{definition*}
    Будет говорить, что множество $A$ \deftext{не больше} множества $B$, если существует биективное
    отображение из $A$ в подмножество $B$, и \deftext{строго меньше}, если не существует биекции из $B$ в
    $A$. Под степенью будем подразумевать декартову степень.
\end{definition*}

\libproblem{set-theory}{nn-equiv-n}
\libproblem{set-theory}{basic-isomorphic-sets}
\libproblem{complexity}{cnf-dnf-size-easy}

\begin{definition*}
    \deftext{Методом резолюций} будем называть метод доказательства противоречивости множества
    дизъюнктов. Данный метод содержит единственное правило вывода:
    $$
        \frac{x \vee \alpha \ \ \ \neg x \vee \beta}{\alpha \vee \beta}.
    $$

    Множество дизъюнктов считается противоречивым (имеет вывод в резолюции), если из него можно вывести
    пустой дизъюнкт.
\end{definition*}

\libproblem{proof-complexity}{res-correctness}
\libproblem{math-logic}{propositional-common-vars}
\libproblem{math-logic}{propositional-compactness}


\end{document}



%%% Local Variables:
%%% mode: latex
%%% TeX-master: t
%%% End:
