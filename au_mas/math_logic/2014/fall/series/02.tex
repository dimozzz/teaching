\documentclass[a4paper, 12pt]{article}
% math symbols
\usepackage{amssymb}
\usepackage{amsmath}
\usepackage{mathrsfs}
\usepackage{mathseries}


\usepackage[margin = 2cm]{geometry}

\tolerance = 1000
\emergencystretch = 0.74cm



\pagestyle{empty}
\parindent = 0mm

\renewcommand{\coursetitle}{DM/ML}
\setcounter{curtask}{1}

\setmathstyle{17.09}{Задание 2}{АУ. 5 курс}
\setcounter{curtask}{8}

\begin{document}

\begin{definition*}
    Будем говорить, что множества $A$ и $B$ \deftext{равны} и обозначать $A \sim B$, если $A \le B$ и $B
    \le A$. 
\end{definition*}

\libproblem{set-theory}{basic-isomorphic-real-point}

\task{
    Докажите, что $\mathbb{R}^{\mathbb{N}} \le \mathbb{R}$.
}

\task{
    Докажите, что $\mathbb{N}^{\mathbb{R}} \le 2^{\mathbb{R}}$.
}

\libproblem{proof-complexity}{res-soundness}

\task{
    Какое минимальное количество булевых функций от $n~ (n > 1)$
    переменных составляют полный базис?
}

\libproblem{discrete-math}{polynomial-representation}

\begin{definition*}
    Пусть $f(x_1, \dots, x_n)$~--- булева функция от булевых аргументов. Будем называть функцию $f$:
    \begin{itemtask}
        \item \deftext{сохраняющей ноль}, если $f(0, 0, \dots, 0) = 0$;
        \item \deftext{сохраняющей единицу}, если $f(1, 1, \dots, 1) = 1$;
        \item \deftext{самодвойственной}, если $f(x_1, x_2, \dots, x_n) = \neg f(\neg x_1, \neg x_2,
            \dots, \neg x_n)$;
        \item \deftext{монотонной}, если
            $$
                f(x_1, x_2, \dots, x_{i - 1}, 1, x_{i + 1}, \dots,  x_n) \ge
                f(x_1, x_2, \dots, x_{i - 1}, 0, x_{i + 1}, \dots,  x_n)
            $$
            для всех $i \in [n]$;
        \item \deftext{линейной}, если $f(x_1, x_2, \dots, x_n) = a_0 \oplus a_1x_1 \oplus \dots \oplus
            a_nx_n$, где $a_i$~--- булевы константы.
    \end{itemtask}
\end{definition*}

\libproblem{discrete-math}{basis-post-criteria}

\breakline


\libproblem[1]{set-theory}{real-not-countable}
\libproblem[2]{set-theory}{nn-equiv-n}
\libproblem[3]{set-theory}{basic-isomorphic-sets}
\libproblem[4]{complexity}{cnf-dnf-size-easy}
\libproblem[5]{proof-complexity}{res-correctness}
\libproblem[6]{math-logic}{propositional-common-vars}
\libproblem[7]{math-logic}{propositional-compactness}

\end{document}



%%% Local Variables:
%%% mode: latex
%%% TeX-master: t
%%% End:
