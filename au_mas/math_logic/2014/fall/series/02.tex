\documentclass[12pt, fleqn, a4paper]{article}


\usepackage{amsmath}
\usepackage{amssymb}
\usepackage{amsfonts}
\usepackage{textcomp}
\usepackage{amsthm}
\usepackage{mathtools}
\usepackage{xspace}
\usepackage[classfont = bold]{complexity}
\usepackage{fullpage}
\usepackage[russian, english]{babel}
\usepackage[utf8]{inputenc}
\usepackage[
    sorting = ydnt,
    style = alphabetic,
    maxbibnames = 99,
    backend = biber
    ]{biblatex}
\addbibresource{main.bib}



\newtheorem{conjecture}{Conjecture}[section]
\theoremstyle{definition}
\newtheorem{theorem}{Theorem}[section]
\newtheorem*{theorem*}{Theorem}
\newtheorem{lemma}{Lemma}[section]
\newtheorem{corollary}{Corollary}[section]
\newtheorem{proposition}{Proposition}[section]
\newtheorem{fact}{Fact}[section]
\newtheorem{problem}{Problem}[section]
\newtheorem{exercise}{Exercise}[section]
\newtheorem{example}{Example}[section]
\newtheorem{definition}{Definition}[section]
\newtheorem{remark}{Remark}[section]
\newtheorem{algorithm}{Algorithm}[section]


\newcommand{\class}[1]{\mathbf{#1}}
\newcommand{\co}{\mathrm{co}}
\newcommand{\alg}[1]{\mathit{#1}}
\newcommand{\lang}[1]{\mathtt{#1}}


% classes (1)

\newcommand{\DTime}{\class{DTime}}
\newcommand{\RTime}{\class{RTime}}
\newcommand{\UTime}{\class{UTime}}
\newcommand{\NTime}{\class{NTime}}
\newcommand{\BPTime}{\class{BPTime}}


\renewcommand{\P}{\class{P}}
\newcommand{\ZPP}{\class{ZPP}}
\newcommand{\RP}{\class{RP}}
\newcommand{\coRP}{\co\class{RP}}
\newcommand{\UP}{\class{UP}}
\newcommand{\coUP}{\co\class{UP}}
\newcommand{\NP}{\class{NP}}
\newcommand{\coNP}{{\co}\class{NP}}
\newcommand{\BPP}{\class{BPP}}
\newcommand{\SigmaP}[1]{\Sigma^{#1}\class{P}}
\newcommand{\PH}{\class{PH}}
\newcommand{\PP}{\class{PP}}
\newcommand{\IP}{\class{IP}}
\newcommand{\OP}{\class{\oplus P}}


\newcommand{\EXP}{\class{EXP}}
\newcommand{\MIP}{\class{MIP}}
\newcommand{\NEXP}{\class{NEXP}}
\newcommand{\coNEXP}{{\co}\class{NEXP}}
\newcommand{\MAEXP}{\class{MA}_\class{EXP}}


% classes (2)

\newcommand{\Ppoly}{\class{P}/\class{poly}}
\newcommand{\NC}{\class{NC}}


\newcommand{\DSpace}{\class{DSpace}}
\newcommand{\NSpace}{\class{NSpace}}
\newcommand{\PSPACE}{\class{PSPACE}}

\newcommand{\EXPSPACE}{\class{EXPSPACE}}


% algorithms and proof systems


\newcommand{\DPLL}{\alg{DPLL}}
\newcommand{\OBDD}{\alg{OBDD}}
\newcommand{\pOBDD}{\pi\text{-}\alg{OBDD}}
\newcommand{\DPLLL}{\alg{DPLL}_{lin}}
\newcommand{\ResL}{\alg{Res}_{lin}}
\newcommand{\SemL}{\alg{Sem}_{lin}}



% languages


\newcommand{\SAT}{\lang{SAT}}
\newcommand{\GNI}{\lang{GNI}}
\newcommand{\MAJSAT}{\lang{MAJ}\text{-}\lang{SAT}}
\newcommand{\QBF}{\lang{QBF}}



% other

\newcommand{\poly}{\mathrm{poly}}
\newcommand{\Nat}{\mathbb{N}}
\newcommand{\bool}{\{0, 1\}}

\newcommand{\Img}{\mathop{\mathrm{Im}}}

\DeclareMathOperator*{\supp}{supp}
\DeclareMathOperator*{\Exp}{E}
\DeclareMathOperator*{\rk}{rk}




%%% Local Variables:
%%% mode: latex
%%% TeX-master: t
%%% End:


\begin{document}

	\setcounter{curtask}{9}

\mytitle{2 (на 3.10)}

\begin{task}
    Докажите, что множество всех рациональных чисел меньших $\pi$ разрешимо.
\end{task}

\begin{task}
    Существует ли алгоритм, проверяющий, работает ли данная программа
    полиномиальное время?
\end{task}

\begin{task}
    Приведите пример двух непересекающихся неперечислимых множеств.
\end{task}

\begin{task}
    Докажите, что для каждой вычислимой функции $f$ найдется
    псевдообратная вычислимая функция $g$. А именно, $g$ определена на
    множестве значений $f$, и для всех $x$ из области определения $f$
    выполняется $f(g(f(x))) = f(x)$.
\end{task}

\begin{task}
    Приведите пример неразрешимого множества $A \subseteq \Nat \times \Nat$,
    такого, что все его горизонтальные и вертикальные сечения
    разрешимы (т.е. для любого $x$ разрешимы $A \cap \{\{x\} \times \Nat\}$
    и $A \cap \{\Nat \times \{x\}\}$)
\end{task}

\begin{task}
    Докажите, что существует язык, который можно распознать с памятью $2^n$ ($n$~---
    длина слова), но нельзя с памятью $n$. (подсказка: диагонализация)
\end{task}

\end{document}



\setmathstyle{17.09}{Задание 2}{АУ. 5 курс}
\setcounter{curtask}{8}

\begin{document}

\begin{definition*}
    Будем говорить, что множества $A$ и $B$ \deftext{равны} и обозначать $A \sim B$, если $A \le B$ и $B
    \le A$. 
\end{definition*}

\libproblem{set-theory}{basic-isomorphic-real-point}

\task{
    Докажите, что $\mathbb{R}^{\mathbb{N}} \le \mathbb{R}$.
}

\task{
    Докажите, что $\mathbb{N}^{\mathbb{R}} \le 2^{\mathbb{R}}$.
}

\libproblem{proof-complexity}{res-soundness}

\task{
    Какое минимальное количество булевых функций от $n~ (n > 1)$
    переменных составляют полный базис?
}

\libproblem{discrete-math}{polynomial-representation}

\begin{definition*}
    Пусть $f(x_1, \dots, x_n)$~--- булева функция от булевых аргументов. Будем называть функцию $f$:
    \begin{itemtask}
        \item \deftext{сохраняющей ноль}, если $f(0, 0, \dots, 0) = 0$;
        \item \deftext{сохраняющей единицу}, если $f(1, 1, \dots, 1) = 1$;
        \item \deftext{самодвойственной}, если $f(x_1, x_2, \dots, x_n) = \neg f(\neg x_1, \neg x_2,
            \dots, \neg x_n)$;
        \item \deftext{монотонной}, если
            $$
                f(x_1, x_2, \dots, x_{i - 1}, 1, x_{i + 1}, \dots,  x_n) \ge
                f(x_1, x_2, \dots, x_{i - 1}, 0, x_{i + 1}, \dots,  x_n)
            $$
            для всех $i \in [n]$;
        \item \deftext{линейной}, если $f(x_1, x_2, \dots, x_n) = a_0 \oplus a_1x_1 \oplus \dots \oplus
            a_nx_n$, где $a_i$~--- булевы константы.
    \end{itemtask}
\end{definition*}

\libproblem{discrete-math}{basis-post-criteria}

\breakline


\libproblem[1]{set-theory}{real-not-countable}
\libproblem[2]{set-theory}{nn-equiv-n}
\libproblem[3]{set-theory}{basic-isomorphic-sets}
\libproblem[4]{complexity}{cnf-dnf-size-easy}
\libproblem[5]{proof-complexity}{res-correctness}
\libproblem[6]{math-logic}{propositional-common-vars}
\libproblem[7]{math-logic}{propositional-compactness}

\end{document}



%%% Local Variables:
%%% mode: latex
%%% TeX-master: t
%%% End:
