\documentclass[a4paper, 12pt]{article}
% math symbols
\usepackage{amssymb}
\usepackage{amsmath}
\usepackage{mathrsfs}
\usepackage{mathseries}


\usepackage[margin = 2cm]{geometry}

\tolerance = 1000
\emergencystretch = 0.74cm



\pagestyle{empty}
\parindent = 0mm

\renewcommand{\coursetitle}{DM/ML}
\setcounter{curtask}{1}

\setmathstyle{01.10}{Задание 3}{АУ. 5 курс}
\setcounter{curtask}{15}

\begin{document}

\libproblem{set-theory}{continous-r-r-card}
\libproblem{set-theory}{monotone-r-r-card}

\task{
    Пусть $S = {S_{\alpha} \subseteq 2^{\mathbb{N}}}$, для любых $\alpha, \beta$ либо
    $S_{\alpha} \subset S_{\beta}$, либо $S_{\beta} \subset S_{\alpha}$. Докажите,
    что может быть верно неравенство $\mathbb{N} < S$. (подсказка: придумайте
    адаптивную кодировку (т.е. кодировка символа зависит от предыдущих символов)
    вещественных чисел). 
}

\dzcomment{
    Не понял что написано было, не добавлял в базу.
}

\task{
    Приведите пример трех неизоморфных линейных порядка на счетном множестве.
}

\begin{definition*}
    Доказательство противоречивости множества дизъюнктов в системе Cutting Planes. Представляет собой
    последовательность линейных неравенств:
    $C_1 \ge c_1, C_2 \ge c_2, \dots, C_{m - 1} \ge c_{m - 1}, 0 \ge 1$.

    $j$-ое неравенство получено по одному из следующих правил:
    \begin{itemize}
        \item аксиомы: $x \ge 0$, $1 - x \ge 0$,
	    \item $(x_1 \lor \neg x_2 \lor \dots)$~--- исходный дизъюнкт, тогда
    		$C_j = x_1 + (1 - x_2) + \dots, c_j = 1$;
        \item $С_j = C_i + C_k, c_j = c_i + c_k$, где $i, k < j$;
		\item $C_j = k C_i, c_j = k c_i$, где $i < j$, $k \in \mathbb{N}$;
        \item Пусть $C_i = k a_1 x_1 + k a_2 x_2 + \dots$, тогда $C_j = a_1 x_1 +
    		a_2 x_2 + \dots, c_j =  \lceil c_i \rceil$.
    \end{itemize}
\end{definition*}

\libproblem{proof-complexity}{cp-corr-sound-formula}
\libproblem{math-logic}{prop-substitution-and-ded}

\end{document}



%%% Local Variables:
%%% mode: latex
%%% TeX-master: t
%%% End:
