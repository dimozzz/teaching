\setcounter{curtask}{15}

\mytitle{4 (на 26.10)}

\begin{task}
    Пусть у теории $T$ есть бесконечная модель, докажите, что у теории
    $T$ есть счетная модель.
\end{task}

В следующих задачах требуется описать множество выразимых предикатом в
данной интерпретации. Обычно требуется доказать, что это множество
совпадает с множеством бескванторных формул. Иногда такое доказать не
получится, тогда необходимо добавить выразимый предикат (выразимый с
квантором) и доказать, что выразимые~--- это бескванторные с новым
предикатом.

\begin{task}
    $(M, =)$, где $M$~--- призвольное бесконечное множество.
\end{task}

\begin{task}
    $(\mathbb{Q}, =, +)$
\end{task}

\begin{task}
    $(\mathbb{Q}, =, S)$, где $S$~--- прибавление единицы.
\end{task}

\begin{task}
    $(\mathbb{N}, =, S)$
\end{task}

\begin{task}
    $(\mathbb{N}, =, S, P)$, где $P$~--- предикат ``быть степенью двойки''.
\end{task}

\breakline

\begin{ptask}{12}
	$\mathbb{Z} + \mathbb{Z}$~--- это две копии целых чисел, причем
    все числа из второй копии больше чисел из первой. Докажите, что
    $(\mathbb{Z}, <, =)$ элементарно эквивалентна $(\mathbb{Z} +
    \mathbb{Z}, <, =)$.
\end{ptask}

\begin{ptask}{13}
    Будет ли интерпретация $(\mathbb{N}, =, <)$ элементарно
    эквивалентна:
    б) $(\mathbb{N} + \mathbb{Z}, =, <)$
\end{ptask}

\begin{ptask}{14}
    в) Пусть единичный квадрат разрезан на несколько меньших
    квадратов. Докажите, что все они имеют рациональные стороны.
\end{ptask}