\setcounter{curtask}{24}

\mytitle{5 (на 17.11)}

\begin{task}
    Докажите, что не существует биекции между $\mathbb{N}$ и $\mathbb{R}$.
\end{task}

Будет говорить, что множество $A$ не больше множества $B$, если
существует биективное отображение из $A$ в подмножество $B$, и строго
меньше, если не существует биекции из $B$ в $A$. Под степенью будем
подразумевать декартову степень.

\begin{task}
    Докажите, что $\mathbb{N}^c \le \mathbb{N}$.
\end{task}

\begin{task}
    Докажите, что $\mathbb{N} < 2^{\mathbb{N}} \le \mathbb{R}$
\end{task}

\begin{task}
    Докажите, что:
    а) $\mathbb{R} \le [0, 1]$
    б) $\mathbb{R} \le 2^{\mathbb{N}}$
    в) $\mathbb{R} \times \mathbb{R} \le \mathbb{R}$ (подсказка:
	    придумать явную биекцию квадрата на одну из сторон)
\end{task}

\begin{task}
    Докажите, что для всякого поля $K$ существует его расширение $K'$
    такое, что каждый многочлен с коэффициентами из $K$ имеет корень
    в $K'$ (можно пользоваться тем, что для конкретного многочлена
    такое расширение существует) (подсказка: использовать теорему о компактности).
\end{task}

\begin{task}
    Докажите, что если теория конечно аксиоматизируема
    (т.е. существует конечное множество теорем этой теории из которых выводятся все
    остальные), то конечное множество аксиом можно выбрать из самой
    теории, а не из теорем.
\end{task}

\breakline

\begin{ptask}{13}
    Будет ли интерпретация $(\mathbb{N}, =, <)$ элементарно
    эквивалентна:
    б) $(\mathbb{N} + \mathbb{Z}, =, <)$
\end{ptask}

\begin{ptask}{14}
    в) Пусть единичный квадрат разрезан на несколько меньших
    квадратов. Докажите, что все они имеют рациональные стороны.
\end{ptask}

\begin{ptask}{19}
    $(\mathbb{N}, =, S)$
\end{ptask}

\begin{ptask}{20}
    $(\mathbb{N}, =, S, P)$, где $P$~--- предикат ``быть степенью двойки''.
\end{ptask}

\begin{ptask}{21}(арифметика Пресбургера)
    Рассмотрим интерпретацию $(Z, =, <, +, 0, 1)$.
    Докажите, что:
    а) элиминация кванторов невозможна
    б) $(Z, =, <, +, 0, 1, \equiv_c)$ допускает элиминацию кванторов
    ($\equiv_c$~--- сравнение по модулю $c$, по предикату для каждого $c$)
\end{ptask}

\begin{ptask}{22}(резолюция для исчисления предикатов)
    Система доказывает противоречивость замкнутой формулы следующим
    образом: формула приводится в предваренную нормальную форму, затем
    проводится скулемизация (избавляемся от функциональных
    символов). Получается формула $\forall x_1 x_2 \dots
    \phi(x_1, \dots, x_n)$. Для формулы $\phi$ делается резолюционный
    вывод. Докажите корректность и полноту данного метода.
\end{ptask}