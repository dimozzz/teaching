\documentclass[a4paper, 12pt]{article}
% math symbols
\usepackage{amssymb}
\usepackage{amsmath}
\usepackage{mathrsfs}
\usepackage{mathseries}


\usepackage[margin = 2cm]{geometry}

\tolerance = 1000
\emergencystretch = 0.74cm



\pagestyle{empty}
\parindent = 0mm

\renewcommand{\coursetitle}{DM/ML}
\setcounter{curtask}{1}

\setmathstyle{10.11}{Мат. логика}{}


\begin{document}

\task{
    (Арифметика Пресбургера)
    Рассмотрим интерпретацию $(\mathbb{Z}, =, <, +, 0, 1)$. Докажите, что:
    \begin{enumcyr}
        \item элиминация кванторов невозможна;
        \item $(\mathbb{Z}, =, <, +, 0, 1, \equiv_c)$ допускает элиминацию кванторов ($\equiv_c$~---
            сравнение по модулю $c$, по предикату для каждого $c$).
    \end{enumcyr}
}

\task{
    (Резолюция для исчисления предикатов)
    Система доказывает противоречивость замкнутой формулы следующим образом: формула приводится в
    предваренную нормальную форму, затем проводится скулемизация (избавляемся от функциональных
    символов). Получается формула $\forall x_1 x_2 \dots \varphi(x_1, \dots, x_n)$. Для формулы $\varphi$
    делается резолюционный вывод. Докажите корректность и полноту данного метода.
}

\libproblem{math-logic}{th-example-fin-infin-model}

\breakline

\libproblem[13]{math-logic}{n-equiv-n-plus-z}
\libproblem[14]{math-logic}{unit-square-cut}


\task[19]{
    $(\mathbb{N}, =, S)$
}

\task[20]{
    $(\mathbb{N}, =, S, P)$, где $P$~--- предикат ``быть степенью двойки''.
}

\end{document}
