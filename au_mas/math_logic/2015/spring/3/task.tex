\setcounter{curtask}{14}

\mytitle{3 (на 7.04)}

\begin{task}
    Пусть $M = (\mathbb{Q}, <)$,
    $p = \{\phi(v) \mid M \models \phi(\frac{1}{2})\}$.
    Реализуется ли тип $p$ в модели $M$, если да, то конечно ли число точек
    реализации? 
\end{task}

\begin{task}
    Предъявите счетную модель над счетным языком такую, что найдется несчетное число
    полных $1$-типов.
\end{task}

\begin{task}
    Рассмотрим теорию $Th(\mathbb{Z}, s)$, $s(x) = x + 1$. Ракие типы входят в
    $S_n(T)$? Какие из них изолированы?
\end{task}

\begin{task}
    Рассмотрим теорию $Th(\mathbb{Z}, s, <)$, $s(x) = x + 1$. Ракие типы входят в
    $S_n(T)$? Какие из них изолированы?
\end{task}


\breakline

\begin{task}{8}
    Докажите, что класс ординалов не является множеством.
\end{task}

\begin{task}{9}
    Арифметика кардиналов.
    1. $\aleph_{\alpha} + \aleph_{\beta} = \aleph_{\alpha}  \aleph_{\beta} =
    	\max(\aleph_{\alpha}, \aleph_{\beta})$
    2. Если $\alpha \le \beta$, то $\aleph_{\alpha}^{\aleph_{\beta}} \le 2^{\aleph_{\beta}}$
\end{task}

\begin{task}{11}
    Алиса и Боб играют в игру. Они загадали некоторой предикат $P$ после чего, они
    составляют бесконечное слово $\omega$ по очереди, называя биты. Алиса ходит
    первой, и побеждает, если $P(w) = 1$.Верно ли, что для любого предиката $P$
    существует выигрышная стратегия для одного из игроков?
\end{task}

\begin{task}{12}
    Будет ли теория групп, где порядок каждого элемента равен 2:
    а) $\lambda$-категоричной для любого $\lambda \ge \aleph_0$?
    б) полной?
\end{task}

\begin{task}{13}
    Пусть $L = \{s\}$. $T$ --- $L$-теория, которая говорит, что $s$~--- биекция без
	циклов (т.е $s^{(n)}(x) != x$). Для каких $\lambda$ данная теория будет
    $\lambda$-категоричной.
\end{task}