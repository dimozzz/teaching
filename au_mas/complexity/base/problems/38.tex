Докажите, что:
\begin{enumcyr}
    \item язык простых чисел содержится в классе $\coNP$;
    \item число $n$ простое тогда и только тогда, когда для каждого простого делителя $q$ числа $n - 1$ существует $a \in
	    \{2, 3, \dots, n - 1\}$ при котором $a^{n - 1} = 1 \bmod n$ и $a^{\frac{n - 1}{q}} \neq 1 \bmod n$;
    \item язык простых чисел лежит в $\NP$.
\end{enumcyr}