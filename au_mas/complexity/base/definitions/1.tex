Мы называем алгоритмы $\mathcal{A}$ и $\mathcal{B}$ эквивалентными если:
\begin{itemtask}
    \item $\forall x\  \mathcal{A}(x)$ останавливается $\Leftrightarrow \mathcal{B}(x)$ останавливается;
	\item $\forall x$ если $\mathcal{A}(x)$ останавливается, то и $\mathcal{A}(x) = \mathcal{B}(x)$.
\end{itemtask}

Такую же эквивалентность можно ввесли на множестве натуральных чисел $a \equiv b \Leftrightarrow {<}a{>} \sim
{<}b{>}$. Множество $S \subseteq \mathcal{N}$ называется инвариантным, если $\forall a \in S, b \in \mathbb{N} \setminus S, a
\not\equiv b$.
