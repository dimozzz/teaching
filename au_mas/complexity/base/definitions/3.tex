Ассоциативным исчислением называется конечный набор правил вида $\{s_i~\to~t_i\}_{i \in I}$, где $s_i, t_i$~--- строчки. Говорят,
что строка $y$ выводится из строки $x$, если из строки $x$ можно получить строку $y$, заменяя несколько раз подстроку $s_i$ на
$t_i$. Ассоциативное исчисление называется двусторонним, если наряду с правилом $s \to t$ есть и правило $t \to s$.