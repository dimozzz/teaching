\setcounter{curtask}{21}

\mytitle{4 (на 14.04)}


\begin{task}
    Пусть $f_1(x_{1 1}, \dots, x_{1 n_1}), f_2(x_{2 1}, \dots, x_{2 n_2}), \dots, f_m(x_{m 1}, \dots, x_{m n_m})$~--- произвольные
    булевы функции, зависящие от непересекающегося множества переменных. Докажите, что выполняется неравенство:\\
    $L(f_1(x_{1 1}, \dots, x_{1 n_1}) \oplus f_2(x_{2 1}, \dots, x_{2 n_2}) \oplus \dots \oplus f_m(x_{m 1}, \dots, x_{m n_m}))
    \ge \frac{1}{2} \sum\limits_{i} L(f_i)$, где $L(f)$~--- минимальное количество гейтов в формуле $\{\land, \lor, \neg\}$,
    вычисляющей $f$.
\end{task}

\begin{task}
    Покажите, что у случайной булевой функции $f: \{0, 1\}^n \rightarrow \{0, 1\}$ с большой вероятностью средняя сложность
    $H_{avg}$ не менее $2^{\frac{n}{10}}$ при больших $n$.
\end{task}

\begin{task}
    Докажите, что если существует $S(l)$ псевдослучайный генератор, то существует такая функция $f \in E$, что $H_{wrs}(f)(n) \ge
    S(n)$.
\end{task}

\begin{task}
    Докажите, что если перманент является полной задачей в классе $\sharp P$ относительно сведений, сохраняющих число
    решений, то $NP = RP$.
\end{task}


\breakline


\begin{ptask}{9}
    Докажите, что у любой формулы размера $s$ существует эквивалентная формула глубины $O(\log(s))$.
\end{ptask}

\begin{ptask}{10}
    Какие значения может принимать глубина дерева решений (decision tree) для функции $f: \{0, 1\}^n \rightarrow \{0, 1\}$, где
    все аргументы не являются фиктивными (т.е. для каждого номера $i$ найдется вход $x$, что $f(x) \neq f(x^{i})$).
\end{ptask}

\begin{ptask}{14}
    Докажите, что если $SAT \in \mathrm{PCP}(o(\log(n)), 1)$, то $P = NP$.
\end{ptask}

\begin{ptask}{19}
    Пусть $M[X, X]$~--- $0 / 1$-матрица, которая содержит перестановочную матрицу размера $|X|$ (т.е. ее перманент над
    $\mathbb{R}$ не ноль). а) Докажите, что $R(M) \cdot T(M) \ge |X|^2$, где $T(M)$~--- число единиц в матрице. б) Докажите при
    помощи этой техники, что $L(MOD_2) = \Omega(n^2)$.
\end{ptask}


\begin{ptask}{20}
    Пусть $S_t$~--- биномиальное распределение с $t$ сбалансированными монетами. Докажите, что для любого $\delta < 1$,
    $\sum\limits_{i = 0}^{t + \delta \sqrt{t}} |\Pr[S_t = i] - \Pr[S_{t + \delta \sqrt{t}} = i]| \le 20 \delta$.
\end{ptask}
