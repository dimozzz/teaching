\setcounter{curtask}{6}

\mytitle{2 (на 24.02)}


В данном задании будет считать, что все формулы имеют гейты $\lang, \lor, \neg$. Размером формулы будем называть число листьев в
дереве, которое представляет эту функцию. Для булевой функции $f$ обозначим через $L(f)$ формульную сложность $f$. $B_n$~---
множество всех булевых функций $\{0, 1\}^n \rightarrow \{0, 1\}$

\begin{task}
    Формальной мерой сложности называется отображение $\mathrm{FC}: B_n \to \mathbb{N}$, обладающее следующими свойствами:
    \begin{itemize}
        \item $\mathrm{FC}(x_i) = 1$;
        \item $\mathrm{FC}(f) = \mathrm{FC}(\neg f)$;
    	\item $\mathrm{FC}(f \lor g) \le \mathrm{FC}(f) + \mathrm{FC}(g)$.
	\end{itemize}

	а) Докажите, что $\mathrm{FC}(f \land g) \le \mathrm{FC}(f) + \mathrm{FC}(g)$; б) Покажите, что $L(f)$~--- это формальная мера
    сложности; в) (лемма Патерсона) Докажите, что для любой формальной меры сложности $\mathrm{FC}$ выполняется неравенство:
    $\mathrm{FC}(f) \le L(f)$.
\end{task}

\begin{task}
    Для множеств $A, B \subseteq \{0, 1\}^n$ обозначим через $H(A, B)$~--- множество пар соседей $\{(a, b) \in A \times B \mid
    \rho(a, b) = 1\}$, где $\rho$~--- расстояние Хемминга. Определим $K_{AB} = \frac{|H(A, B)|^2}{|A| |B|}$ и $K(f) = \max
    \{K_{AB} \mid A \subseteq f^{-1}(1), B \subseteq f^{-1}(0)\}$. Докажите, что а) $K(f)$~--- формальная мера сложности; б)
    (теорема Храпченко) $L(f) \ge K(f)$; в) $K(f) \le n^2$; г) $L(Maj) = \Omega(n^2)$.
\end{task}

\begin{task}
    Покажите, что представление $\bigwedge\limits_{i = 1}^{n} x_i$ в виде полинома $\mathbb{F}_q[x_1, \dots, x_n]$ ($q$~--- простое
    число) требуют степень ровно $n$.
\end{task}


\breakline

\begin{ptask}{1}
    Рассмотрим функцию $Maj: \{0, 1\}^n \rightarrow \{0, 1\}$, которая выдает $1$, если не менее половины входных битов равны
    $1$. Докажите, что существует: а) схема б) монотонная схема в) монотонная формула полиномиального размера, вычисляющая функцию
    $Maj$.
\end{ptask}

\begin{ptask}{5}
    Докажите, что функция $Maj$ не может быть вычислена при помощи схем полиномиального размера константной глубины из гейтов
    $\land, \lor, \neg$.
\end{ptask}