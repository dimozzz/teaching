\documentclass[a4paper, 12pt]{article}
% math symbols
\usepackage{amssymb}
\usepackage{amsmath}
\usepackage{mathrsfs}
\usepackage{mathseries}


\usepackage[margin = 2cm]{geometry}

\tolerance = 1000
\emergencystretch = 0.74cm



\pagestyle{empty}
\parindent = 0mm

\renewcommand{\coursetitle}{DM/ML}
\setcounter{curtask}{1}

\setmathstyle{31.03}{}{}
\setcounter{curtask}{15}

\begin{document}

\begin{definition*}
    \deftext{Коммуникационный протокол} для функции $f\colon X \times Y \to Z$~--- это корневое двоичное
    дерево, которое описывает совместное вычисление Алисой и Бобом функции $f$. В этом дереве каждая
    внутренняя вершина $v$ помечена меткой $a$ или $b$, означающей очередь хода Алисы или Боба
    соответственно. Для каждой вершины, помеченной $a$, определена функция $g_v\colon X \to \{0, 1\}$, которая
    говорит Алисе, какой бит нужно послать, если вычисление находится в этой вершине. Аналогично, для
    каждой вершины $v$ с пометкой $b$ определена функция $h_v\colon Y \to \{0, 1\}$, которая определяет бит,
    который Боб должен отослать в этой вершине. Каждая внутренняя вершина имеет двух потомков, ребро к
    первому потомку помечено $0$, а ребро ко второму потомку помечено $1$. Каждый лист помечен значением
    из множества $Z$.

    Каждая пара входов $(x, y)$ определяет путь от корня до листа в описанном двоичном дереве
    естественным обрзом. Будем говорить, что коммуникационный протокол вычисляет функцию $f$, если для
    всех пар $(x, y) \in X \times Y$, этот путь заканчивается в листе с пометкой $f(x, y)$.

    \deftext{Коммуникационной сложностью} функции $f$ назовем наименьшую глубину протокола, вычисляющего
    функцию $f$, и будем ее обозначать $\DCC(f)$.
\end{definition*}

\libproblem{cc}{balance}
\libproblem{cc}{basic-lower}
\libproblem{cc}{median-easy}

\begin{definition*}
    Игры Карчмера--Вигдерсона. Рассмотрим функцию $f\colon \{0, 1\}^n \to \{0, 1\}$. Для $f$ можно
    рассмотреть следующую коммуникационную задачу $\KW_f$: Алиса получает число $x \in f^{-1}(1)$,
    Боб~--- $y \in f^{-1}(0)$. Их цель~--- найти хотя бы одну позицию $i$, в которой $x_i \ne y_i$. В
    случае, если таких битов несколько, то подойдет любой.
\end{definition*}

\libproblem{cc}{karchmer-wigderson-proof}
\libproblem{cc}{kw-parity-comb}
\libproblem{combinatorics}{binomial-distr-dist}


\breakline


\libproblem[9]{complexity}{formula-balancing}
\libproblem[10]{complexity}{decision-depth-bound}
\libproblem[14]{struct-complexity}{pcp-log-p}

\end{document}