\documentclass[12pt, fleqn, a4paper]{article}


\usepackage{amsmath}
\usepackage{amssymb}
\usepackage{amsfonts}
\usepackage{textcomp}
\usepackage{amsthm}
\usepackage{mathtools}
\usepackage{xspace}
\usepackage[classfont = bold]{complexity}
\usepackage{fullpage}
\usepackage[russian, english]{babel}
\usepackage[utf8]{inputenc}
\usepackage[
    sorting = ydnt,
    style = alphabetic,
    maxbibnames = 99,
    backend = biber
    ]{biblatex}
\addbibresource{main.bib}



\newtheorem{conjecture}{Conjecture}[section]
\theoremstyle{definition}
\newtheorem{theorem}{Theorem}[section]
\newtheorem*{theorem*}{Theorem}
\newtheorem{lemma}{Lemma}[section]
\newtheorem{corollary}{Corollary}[section]
\newtheorem{proposition}{Proposition}[section]
\newtheorem{fact}{Fact}[section]
\newtheorem{problem}{Problem}[section]
\newtheorem{exercise}{Exercise}[section]
\newtheorem{example}{Example}[section]
\newtheorem{definition}{Definition}[section]
\newtheorem{remark}{Remark}[section]
\newtheorem{algorithm}{Algorithm}[section]


\newcommand{\class}[1]{\mathbf{#1}}
\newcommand{\co}{\mathrm{co}}
\newcommand{\alg}[1]{\mathit{#1}}
\newcommand{\lang}[1]{\mathtt{#1}}


% classes (1)

\newcommand{\DTime}{\class{DTime}}
\newcommand{\RTime}{\class{RTime}}
\newcommand{\UTime}{\class{UTime}}
\newcommand{\NTime}{\class{NTime}}
\newcommand{\BPTime}{\class{BPTime}}


\renewcommand{\P}{\class{P}}
\newcommand{\ZPP}{\class{ZPP}}
\newcommand{\RP}{\class{RP}}
\newcommand{\coRP}{\co\class{RP}}
\newcommand{\UP}{\class{UP}}
\newcommand{\coUP}{\co\class{UP}}
\newcommand{\NP}{\class{NP}}
\newcommand{\coNP}{{\co}\class{NP}}
\newcommand{\BPP}{\class{BPP}}
\newcommand{\SigmaP}[1]{\Sigma^{#1}\class{P}}
\newcommand{\PH}{\class{PH}}
\newcommand{\PP}{\class{PP}}
\newcommand{\IP}{\class{IP}}
\newcommand{\OP}{\class{\oplus P}}


\newcommand{\EXP}{\class{EXP}}
\newcommand{\MIP}{\class{MIP}}
\newcommand{\NEXP}{\class{NEXP}}
\newcommand{\coNEXP}{{\co}\class{NEXP}}
\newcommand{\MAEXP}{\class{MA}_\class{EXP}}


% classes (2)

\newcommand{\Ppoly}{\class{P}/\class{poly}}
\newcommand{\NC}{\class{NC}}


\newcommand{\DSpace}{\class{DSpace}}
\newcommand{\NSpace}{\class{NSpace}}
\newcommand{\PSPACE}{\class{PSPACE}}

\newcommand{\EXPSPACE}{\class{EXPSPACE}}


% algorithms and proof systems


\newcommand{\DPLL}{\alg{DPLL}}
\newcommand{\OBDD}{\alg{OBDD}}
\newcommand{\pOBDD}{\pi\text{-}\alg{OBDD}}
\newcommand{\DPLLL}{\alg{DPLL}_{lin}}
\newcommand{\ResL}{\alg{Res}_{lin}}
\newcommand{\SemL}{\alg{Sem}_{lin}}



% languages


\newcommand{\SAT}{\lang{SAT}}
\newcommand{\GNI}{\lang{GNI}}
\newcommand{\MAJSAT}{\lang{MAJ}\text{-}\lang{SAT}}
\newcommand{\QBF}{\lang{QBF}}



% other

\newcommand{\poly}{\mathrm{poly}}
\newcommand{\Nat}{\mathbb{N}}
\newcommand{\bool}{\{0, 1\}}

\newcommand{\Img}{\mathop{\mathrm{Im}}}

\DeclareMathOperator*{\supp}{supp}
\DeclareMathOperator*{\Exp}{E}
\DeclareMathOperator*{\rk}{rk}




%%% Local Variables:
%%% mode: latex
%%% TeX-master: t
%%% End:


\begin{document}

	\setcounter{curtask}{9}

\mytitle{2 (на 3.10)}

\begin{task}
    Докажите, что множество всех рациональных чисел меньших $\pi$ разрешимо.
\end{task}

\begin{task}
    Существует ли алгоритм, проверяющий, работает ли данная программа
    полиномиальное время?
\end{task}

\begin{task}
    Приведите пример двух непересекающихся неперечислимых множеств.
\end{task}

\begin{task}
    Докажите, что для каждой вычислимой функции $f$ найдется
    псевдообратная вычислимая функция $g$. А именно, $g$ определена на
    множестве значений $f$, и для всех $x$ из области определения $f$
    выполняется $f(g(f(x))) = f(x)$.
\end{task}

\begin{task}
    Приведите пример неразрешимого множества $A \subseteq \Nat \times \Nat$,
    такого, что все его горизонтальные и вертикальные сечения
    разрешимы (т.е. для любого $x$ разрешимы $A \cap \{\{x\} \times \Nat\}$
    и $A \cap \{\Nat \times \{x\}\}$)
\end{task}

\begin{task}
    Докажите, что существует язык, который можно распознать с памятью $2^n$ ($n$~---
    длина слова), но нельзя с памятью $n$. (подсказка: диагонализация)
\end{task}

\end{document}



\setmathstyle{31.03}{}{}
\setcounter{curtask}{15}

\begin{document}

\begin{definition*}
    \deftext{Коммуникационный протокол} для функции $f\colon X \times Y \to Z$~--- это корневое двоичное
    дерево, которое описывает совместное вычисление Алисой и Бобом функции $f$. В этом дереве каждая
    внутренняя вершина $v$ помечена меткой $a$ или $b$, означающей очередь хода Алисы или Боба
    соответственно. Для каждой вершины, помеченной $a$, определена функция $g_v\colon X \to \{0, 1\}$, которая
    говорит Алисе, какой бит нужно послать, если вычисление находится в этой вершине. Аналогично, для
    каждой вершины $v$ с пометкой $b$ определена функция $h_v\colon Y \to \{0, 1\}$, которая определяет бит,
    который Боб должен отослать в этой вершине. Каждая внутренняя вершина имеет двух потомков, ребро к
    первому потомку помечено $0$, а ребро ко второму потомку помечено $1$. Каждый лист помечен значением
    из множества $Z$.

    Каждая пара входов $(x, y)$ определяет путь от корня до листа в описанном двоичном дереве
    естественным обрзом. Будем говорить, что коммуникационный протокол вычисляет функцию $f$, если для
    всех пар $(x, y) \in X \times Y$, этот путь заканчивается в листе с пометкой $f(x, y)$.

    \deftext{Коммуникационной сложностью} функции $f$ назовем наименьшую глубину протокола, вычисляющего
    функцию $f$, и будем ее обозначать $\DCC(f)$.
\end{definition*}

\libproblem{cc}{balance}
\libproblem{cc}{basic-lower}
\libproblem{cc}{median-easy}

\begin{definition*}
    Игры Карчмера--Вигдерсона. Рассмотрим функцию $f\colon \{0, 1\}^n \to \{0, 1\}$. Для $f$ можно
    рассмотреть следующую коммуникационную задачу $\KW_f$: Алиса получает число $x \in f^{-1}(1)$,
    Боб~--- $y \in f^{-1}(0)$. Их цель~--- найти хотя бы одну позицию $i$, в которой $x_i \ne y_i$. В
    случае, если таких битов несколько, то подойдет любой.
\end{definition*}

\libproblem{cc}{karchmer-wigderson-proof}
\libproblem{cc}{kw-parity-comb}
\libproblem{combinatorics}{binomial-distr-dist}


\breakline


\libproblem[9]{complexity}{formula-balancing}
\libproblem[10]{complexity}{decision-depth-bound}
\libproblem[14]{struct-complexity}{pcp-log-p}

\end{document}