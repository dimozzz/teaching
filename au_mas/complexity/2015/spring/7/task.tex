\setcounter{curtask}{28}

\mytitle{7 (на 5.04)}

\begin{task}
    Докажите, что если существует такая функция $f \in E$, что $H_{avg}(f)(n) \ge 2^{\epsilon n}$ при всех $n$, то $MA = NP$.
\end{task}

\begin{task}
	Let $G: \{0, 1\}^{*} \rightarrow \{0, 1\}^{*}$ be an $S(l)$-length candidate pseudorandom generator that fails to derandomize
    a particular $BPP$ algorithm $A$ on the average case. That is, letting $L \in BPP$ be the language such that $\Pr[A(x) = L(x)]
    \ge \frac{2}{3}$, it holds for every sufficiently large $n$, with probability at least $\frac{1}{n}$ over choice $x \gets \{0,
    1\}$, $\Pr[A(x, G(U_{l(n)}))] \le \frac{1}{2}$ (we let $l(n)$ be such that $S(l(n)) = m(n)$ where $m(n)$ denotes the length of
    random tape that used by $A$ on inputs of length $n$). Prove that there exists a probabilistic polynomial-time algorithm $D$
    that on input $1^{n}$ outputs a circuit $D_n$ such that with probability at least $\frac{1}{2n}$ over randomness of $D$,
    $|E[D_n(G(U_{l(n)}))] - E[D_n(U_{m(n)})]| \ge 0.1$.
\end{task}


\breakline


\begin{ptask}{19}
    Пусть $M[X, X]$~--- $0 / 1$-матрица, которая содержит перестановочную матрицу размера $|X|$ (т.е. ее перманент над
    $\mathbb{R}$ не ноль). б) Докажите при помощи этой техники, что $L(MOD_2) = \Omega(n^2)$.
\end{ptask}

\begin{ptask}{21}
    Пусть $f_1(x_{1 1}, \dots, x_{1 n_1}), f_2(x_{2 1}, \dots, x_{2 n_2}), \dots, f_m(x_{m 1}, \dots, x_{m n_m})$~--- произвольные
    булевы функции, зависящие от непересекающегося множества переменных. Докажите, что выполняется неравенство:\\
    $L(f_1(x_{1 1}, \dots, x_{1 n_1}) \oplus f_2(x_{2 1}, \dots, x_{2 n_2}) \oplus \dots \oplus f_m(x_{m 1}, \dots, x_{m n_m}))
    \ge \frac{1}{2} \sum\limits_{i} L(f_i)$, где $L(f)$~--- минимальное количество гейтов в формуле $\{\land, \lor, \neg\}$,
    вычисляющей $f$.
\end{ptask}

\begin{ptask}{26}
    Пусть $E_1: \{0, 1\}^n \rightarrow \Sigma^m$ и $E_2: \Sigma \rightarrow \{0, 1\}^k$~--- это два кода с локальными списочными
    декодерами. Декодер кода $E_1$ выдает список размера $l_1$ и обрабатывает $1 - \epsilon_1$ ошибок. Декодер для кода $E_2$
    выдает список размера $l_2$ и обрабатывает $\frac{1}{2} - \epsilon_2$ ошибок. Докажите, что у каскадного кода $E_1 \cdot E_2$
    существует локальный списочный декодер, который обрабатывает $\frac{1}{2} - \epsilon_1 \epsilon_2 l_2$ ошибок и выдает список
    размера $l_1 l_2$.
\end{ptask}

\begin{ptask}{27}
    а) Покажите, что существует полиномиальный от $n$ алгоритм $A$, который получает вход, распределенный согласно распределению
    $X$ с $H_{\infty}(X) \ge n^{100}$ и имеет оракульный доступ к функции $f: \{0, 1\}^n \rightarrow \{0, 1\}$, который
    удовлетворяет следующим свойствам:
    \begin{itemize}
        \item если $E[f(u_n)] \ge \frac{2}{3}$, то $A$ отвечает $1$ с вероятностью хотя бы $0.99$;
        \item если $E[f(u_n)] \le \frac{1}{3}$, то $A$ отвечает $0$ с вероятностью хотя бы $0.99$;
	\end{itemize}

    Такой алгоритм будем называть аппроксиматором функции.

    б) Покажите, что не существует аппроксиматора без доступа к случайным битам.

    в) Покажите, что если распределение $X$ находится на расстоянии более $\frac{1}{5}$ от каждого распределения $Y$ с $H(Y) \ge
    \frac{n}{2}$, то не существует аппроксиматора, вход которого распределен согласно $X$.
\end{ptask}
