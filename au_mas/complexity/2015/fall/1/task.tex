\setcounter{curtask}{1}

\mytitle{1 (на 08.09)}

\begin{task}
    Покажите, что любое бесконечное перечислимое множество можно записать в виде: ${a(1), a(2), a(3), \dots}$, где $a$~---
    вычислимая функция, все значения которой различны.
\end{task}

\begin{task}
    Докажите, что непустое подножество натуральных чисел разрешимо тогда и только тогда, когда оно есть множество значений всюду
    определенной неубывающей функции с натуральными аргументами и значениями.
\end{task}

\begin{task}
    Пусть $X$, $Y$~--- перечислимые множества. Докажите, что всегда найдутся такие перечислимые $X' \subseteq X$, $Y' \subseteq
    Y$, что $X' \cup Y' = X \cup Y$ и $X' \cap Y' = \emptyset$.
\end{task}

\begin{task}
    Покажите, что множество всех показателей $n$, для которых существует решение уравнения $x^n + y^n = z^n$ в целых положительных
    числах, перечислимо.
\end{task}

\begin{task}
    Является ли множество всюду останавливающихся алгоритмов перечислимым? А его дополнение?
\end{task}

\begin{task}
    Приведите пример неперечислимого множества, что его дополнение также является неперечислимым.
\end{task}

\begin{task}
    Существует ли алгоритм, проверяющий, работает ли данная программа полиномиальное время?
\end{task}

\begin{task}
    Покажите, что каждый язык, который принимается $k$-ленточной недетерминированной машиной Тьюринга за время $f(n)$, может быть
    принят на $2$-ленточной машине Тьюринга за время $O(f(n))$.
\end{task}