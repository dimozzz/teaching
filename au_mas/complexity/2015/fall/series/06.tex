\mytitle{6 (на 20.10)}


\begin{task}
	Докажите, что если $\NP \subseteq \DTime[n^{\log n}]$, то $\PH \subseteq \bigcup\limits_{k \ge 1} \DTime[n^{\log^k n}]$    
\end{task}

\begin{task}
	Докажите, что если язык $A$ сводится за полиномиальное время по Тьюрингу (оракульно) к $B \in \Sigma_i^P$, то $A \in
    \Sigma_{i + 1}^P$.    
\end{task}

\begin{task}
	Унарным назвается язык, все слова которого состоят из одного символа. Докажите, что если все унарные языки из $\NP$ лежат в
    $\P$, то $\EXP = \NEXP$. 
\end{task}

\begin{task}
	Постройте примеры полных задач относительно сведений по Карпу в классах $\EXP, \NEXP$ и классе $\NE = \bigcup\limits_{c > 0}
    \NTime[2^{cn}]$.
\end{task}




\breakline

\begin{ptask}{8}
    Покажите, что каждый язык, который принимается $k$-ленточной недетерминированной машиной Тьюринга за время $f(n)$, может быть
    принят на $2$-ленточной машине Тьюринга за время $O(f(n))$.
\end{ptask}

\begin{ptask}{23}
	Докажите, что любой перечислимый предикат арифметичен. 
\end{ptask}

\begin{ptask}{24}
	б) двустороннее ассоциативное исчисление, для которого вопрос о выводимости строки $x$ из строки $y$ является алгоритмически
    неразрешимым (подсказка: посмотрите а предыдущий пункт, что означает, что мы прошли по какому-то ребру назад?).
\end{ptask}


\begin{ptask}{25}
    Покажите, что:
    \begin{enumerate}[topsep = 0pt, itemsep = -1ex]
        \item [а)] для любой конечной (или перечислимой) сигнатуры множество тавтологий в этой сигнатуре перечислимо;
        \item [б)] если в сигнатуре есть достаточное количество функциональных и предикатных символов арности 1 и 2, то множество
		    тавтологий в этой сигнатуре неразрешимо.
    \end{enumerate}
\end{ptask}


\begin{ptask}{29}
    Докажите, что:
    \begin{enumerate}[topsep = 0pt, itemsep = -1ex]
        \item [в)] универсальное множество для $\Sigma_k$ не содержится в $\Pi_k$; (подсказка: рассмотрите диагональ универсального
		    множества, може ли она лежать в $\Sigma_k$ и $\Pi_k$ одновременно?)
        \item [г)] $\Sigma_k \subsetneq \Sigma_{k + 1}$.
    \end{enumerate}
\end{ptask}


\begin{ptask}{30}
	Пусть $T$~--- это множество номеров замкнутых формул в сигнатуре $\{+, \times, {=\}}$, которые истинны в естественной
    интерпретации на множестве натуральных чисел. Докажите, что:
    \begin{enumerate}[topsep = 0pt, itemsep = -1ex]
        \item [а)] для любого $P \in \Sigma_k$ выполняется $P \le_m T$;
        \item [б)] (теорема Тарского) $T$ не является арифметичным;
        \item [в)] (теорема Геделя о неполноте) Покажите, что $T$ не является перечислимым.
    \end{enumerate}
\end{ptask}


\begin{ptask}{31}
	Докажите, что:
    \begin{enumerate}[topsep = 0pt, itemsep = -1ex]
		\item [а)] $\DSpace[n^2] \subsetneq \DSpace[n^3]$ (пожсказка: используйте метод диагонализации. Как долго может работать
		    машина, которая использует $k$ ячеет памяти?);
        \item [б)] $\NSpace[n^2] \subsetneq \NSpace[n^3]$.
    \end{enumerate}
\end{ptask}