\documentclass[a4paper, 12pt]{article}
% math symbols
\usepackage{amssymb}
\usepackage{amsmath}
\usepackage{mathrsfs}
\usepackage{mathseries}


\usepackage[margin = 2cm]{geometry}

\tolerance = 1000
\emergencystretch = 0.74cm



\pagestyle{empty}
\parindent = 0mm

\renewcommand{\coursetitle}{DM/ML}
\setcounter{curtask}{1}

\setmathstyle{20.10}{}{}

\begin{document}

\setcounter{curtask}{32}

\libproblem{struct-complexity}{hierarchy-quasipoly}
\libproblem{struct-complexity}{hierarchy-oracle}
\libproblem{struct-complexity}{unary-p-np-exp-nexp}
\libproblem{struct-complexity}{ne-complete}

\breakline

\libproblem[8]{struct-complexity}{two-tapes}
\libproblem[23]{computability}{enumerable-ariphmetic}
\libproblem[24]{computability}{associative-calculus}

\dzcomment{
    Посмотрите на предыдущий пункт, что означает, что мы прошли по какому-то ребру назад?
}

\libproblem[25]{computability}{taut-enumerable}
\libproblem[29]{computability}{sigma-k-universal}
\libproblem[30]{computability}{tarsky-godel}
\libproblem[31]{struct-complexity}{space-hierarchy}

\dzcomment{
    Используйте метод диагонализации.
}

\end{document}