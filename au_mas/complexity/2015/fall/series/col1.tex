\documentclass[a4paper, 12pt]{article}
% math symbols
\usepackage{amssymb}
\usepackage{amsmath}
\usepackage{mathrsfs}
\usepackage{mathseries}


\usepackage[margin = 2cm]{geometry}

\tolerance = 1000
\emergencystretch = 0.74cm



\pagestyle{empty}
\parindent = 0mm

\renewcommand{\coursetitle}{DM/ML}
\setcounter{curtask}{1}

\setmathstyle{}{}{}

\begin{document}

\setcounter{curtask}{1}

\begin{itemize}
	\item Иван: 2, 9, 13, 18, 23, 25
	\item Людмила: 8, 12, 17, 20, 22, 25
\end{itemize}

\libproblem{computability}{enumerable-output}
\libproblem{computability}{enumerable-graphic}

\begin{definition*}
    Действительное число $\alpha$ будем называть вычислимым, если существует вычислимая функция $a$,
    которая по любому рациональному $\varepsilon > 0$ дает рациональное приближение к $\alpha$ с ошибкой
    не более $\varepsilon$, т.е. $|\alpha - a(\epsilon)| \le \epsilon$ для любого рационального $\epsilon
    > 0$.
\end{definition*}

\libproblem{computability}{subset-le}
\libproblem{computability}{computable-digits}
\libproblem{computability}{computable-number-approx}
\libproblem{computability}{computable-number-ariph}
\libproblem{computability}{computable-number-limit}
\libproblem{computability}{enumerable-number}
\libproblem{computability}{enumerable-number-comp}
\libproblem{computability}{computable-cuts}
\libproblem{computability}{universal-diagonal}
\libproblem{computability}{prime-div}
\libproblem{computability}{sup-points}
\libproblem{computability}{enum-uncomp-numbers}

\begin{definition*}
    Пусть фиксирована главная универсальная функция для класса вычислимых функций одного
    аргумента. Рассмотрим нумерацию вычислимых действительных чисел в соответствии с определением:
    номером числа $\alpha$ является любой номер любой функции, которая по рациональному $\varepsilon > 0$
    дает $\varepsilon$-приближение к $\alpha$.
\end{definition*}

\libproblem{computability}{sum-comp-numbers}
\libproblem{computability}{comp-numbers-zero}
\libproblem{computability}{alg-comp-digits}
\libproblem{computability}{i-print-succ-i}
\libproblem{computability}{reverse-kleene}
\libproblem{computability}{composition-kleene}
\libproblem{computability}{hierarchy-basic-prop}
\libproblem{computability}{set-injections}
\libproblem{computability}{set-surjections}
\libproblem{computability}{non-separable}
\libproblem{computability}{inf-non-separable}

\end{document}