\mytitle{}
\setcounter{curtask}{1}



\begin{itemize}
	\item Иван: 2, 9, 13, 18, 23, 25
	\item Людмила: 8, 12, 17, 20, 22, 25
\end{itemize}


\begin{task}
	Докажите, что перечислимое множество можно эквивалентно определить как множество чисел, которые могут появиться на выходе
    недетерминированной Машины Тьюринга (при фиксированном входе).
\end{task}


\begin{task}
	Пусть $F$ перечислимое множество пар натуральных чисел. Докажите, что существует вычислимая функция $f$, определенная на тех и
    только тех $x$, для которых найдется $y$, при котором $(x, y) \in F$, причем значение $f(x)$ является одним из таких $y$.
\end{task}

\begin{definition}
    Действительное число $\alpha$ называется вычислимым, если существует вычислимая функция $a$, которая по любому рациональному
    $\epsilon > 0$ дает рациональное приближение к $\alpha$ с ошибкой не более $\epsilon$, т.е. $|\alpha - a(\epsilon)| \le
    \epsilon$ для любого рационального $\epsilon > 0$.
\end{definition}


\begin{task}
    Докажите, что число $\alpha$ вычислимо тогда и только тогда, когда множество рациональных чисел, меньших $\alpha$, разрешимо.
\end{task}

\begin{task}
	Докажите, что число $\alpha$ вычислимо тогда и только тогда, когда последовательность знаков представляющей его десятичной
    (или двоичной) дроби вычислима.
\end{task}


\begin{task}
	Докажите, что число $\alpha$ вычислимо тогда и только тогда, когда существует вычислимая последовательность рациональных
    чисел, вычислимо сходящаяся к $\alpha$ (последнее означает, что можно алгоритмически указать $N$ по $\epsilon$ в стандартном
    $\epsilon-N$-определении сходимости).
\end{task}


\begin{task}
	Покажите, что сумма, произведение, разность и частное вычислимых действительных чисел вычислимы. Покажите, что корень
    многочлена с вычислимыми коэффициентами вычислим. 
\end{task}

\begin{task}
    Сформулируйте и докажите утверждение о том, что предел вычислимо сходящейся последовательности вычислимых действительных чисел
    вычислим.
\end{task}

\begin{task}
    Действительное число называют перечислимым снизу, если множество всех рациональных чисел, меньших $\alpha$, перечислимо
    (перечислимость сверху определяется аналогично). Докажите, что число перечислимо снизу тогда и только тогда, когда оно
    является пределом некоторой вычислимой возрастающей последовательности рациональных чисел.
\end{task}

\begin{task}
    Докажите, что действительное число вычислимо тогда и только тогда, когда оно перечислимо снизу и сверху.
\end{task}

\begin{task}
    Все сечения $U_n$ некоторой функции $U$ двух аргументов вычислимы. Следует ли отсюда, что функция $U$ вычислима?
\end{task}

\begin{task}
    Пусть $U$ перечислимое множество пар натуральных чисел, универсальное для класса всех перечислимых множеств натуральных
    чисел. Докажите, что его <<диагональное сечение>> $K = \{x \mid (x, x) \in U\}$ является перечислимым неразрешимым
    множеством.
\end{task}

\begin{task}
    Некоторое множество $S$ натуральных чисел разрешимо. Разложим все числа из $S$ на простые множители и составим множество $D$
    всех простых чисел, встречающихся в этих разложениях. Можно ли утверждать, что множество $D$ разрешимо?
\end{task}

\begin{task}
    Множество $U \subseteq \mathbb{N} \times \mathbb{N}$ разрешимо. Можно ли утверждать, что множество <<нижних точек>> множества
    $U$, то есть множество $V = \{(x, y) \mid (x, y) \in U \land ((x, z) \notin U \mbox{ для всех } z < y)\}$ разрешимо?
\end{task}

\begin{task}
    Покажите, что существуют перечислимые снизу, но не вычислимые действительные числа.
\end{task}


\begin{definition}
    Пусть фиксирована главная универсальная функция для класса вычислимых функций одного аргумента. Тогда возникает нумерация
    вычислимых действительных чисел в соответствии с определением: номером числа $\alpha$ является любой номер любой функции,
    которая по рациональному $\epsilon > 0$ дает $\epsilon$-приближение к $\alpha$.
\end{definition}


\begin{task}
    Покажите, что существует алгоритм, который по любым двум номерам двух вычислимых действительных чисел дает (некоторый) номер
    их суммы.
\end{task}


\begin{task}
	Покажите, что не существует алгоритма, который по любому номеру любого вычислимого действительного числа отвечает на вопрос,
    равно ли это число нулю.
\end{task}


\begin{task}
    Известно, что всякое вычислимое действительное число имеет вычислимое десятичное разложение. Покажите, что тем не менее нет
    алгоритма, который по любому номеру любого вычислимого действительного числа дает номер вычислимой функции, задающей его
    десятичное разложение.
\end{task}

\begin{task}
    Докажите, что в любом разумном (задающем главную нумерацию) языке программирования существует последовательность различных
    программ $p_0, p_1, p_2, \dots$ с таким свойством: программа $p_i$ печатает программу $p_{i + 1}$. 
\end{task}


\begin{task}
    Докажите, что существует пара программ $A$, $B$ таких, что $A$ печатает текст $B$ (обычным образом), а $B$ печатает текст $A$
    задом наперед.
\end{task}

\begin{task}
    Пусть $f$ и $g$  вычислимые всюду определенные функции. Докажите, что найдутся такие номера машин Тьюринга $m$ и  $n$ что
    алгоритм с номерои $f(n)$ ведет себя так же, как и алгоритм с номером $m$, а алгоритм с номером $g(m)$ ведет себя так же, как
    и алгоритм с номером $n$.
\end{task}


\begin{task}
    Проверьте, что множества $\Sigma_n$ и $\Pi_n$ замкнуты относительно объединения и пересечения и декартого произведения. 
\end{task}


\begin{task}
	Являются ли перечислимыми множество всех программ, вычисляющих инъективные функции. А коперечислимым?
\end{task}


\begin{task}
    Являются ли перечислимыми множество всех программ, вычисляющих сюръективные функции. А коперечислимым?
\end{task}


\begin{task}
    Докажите, что существуют непересекающиеся перечислимые множества $A$ и $B$, которые не могут быть отделены разрешимым
    множеством: не существует такого разрешимого $C$, что $A \subseteq C$ и $B \cap C = \emptyset$.
\end{task}

\begin{task}
    Докажите, что существует счетное число непересекающихся перечислимых множеств, никакие два из которых не отделить разрешимым.
\end{task}