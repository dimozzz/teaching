\mytitle{5 (на 13.10)}


\textbf{Определение.} $\Sigma_0 = \Pi_0$~--- множество разрешимых предикатов на множестве натуральных чисел. $\Sigma_{k + 1}$~---
это множество предикатов, которые представляются в виде $\exists y P(x,y)$, где $P \in \Pi_{k}$, а предикаты из $\Pi_{k + 1}$
представляются в виде $\forall y P(x, y)$, где $P \in \Sigma_{k}$. Последовательность $\Sigma_k$ (и $\Pi_k$) называется
арифметической иерархией. 

\begin{task}
    Покажите, что:
    \begin{enumerate}[topsep = 0pt, itemsep = -1ex]
        \item [а)] $\Sigma_1$~--- это множество перичислимых предикатов, а $\Pi_1$~--- коперечислимых;
        \item [б)] $Q \in \Sigma_k$ тогда и только тогда, когда $Q$ можно представить в виде: $Q(x) = \exists y_1 \forall y_2
		    \exists y_3 \dots P(x, y_1, y_2, \dots, y_n)$, где $P$~--- разрешимый предикат (соответственно $Q \in \Pi_k
            \Leftrightarrow Q(x) = \forall y_1 \exists y_2 \forall y_3 \dots P(x, y_1, y_2, \dots, y_n)$);
        \item [в)] $\Sigma_k \cup \Pi_k \subseteq \Sigma_{k + 1} \cap \Pi_{k + 1}$;
        \item [г)] каждый арифметичный предикат содержится в $\Sigma_k$ для некоторого $k$;
    	\item [д)] все предикаты из $\Sigma_k$ являются арифметичными.
    \end{enumerate}
\end{task}



\textbf{Определение.} Множество $A$ $m$-сводится к множеству $B$, если существует такая вычислимая всюду определенная функция
$f$, что  $x \in A \Leftrightarrow f(x) \in B$. Обозначение: $A \le_m B$.

\begin{task}
    Покажите, что:
    \begin{enumerate}[topsep = 0pt, itemsep = -1ex]
        \item [а)] если $A {\le}_{m} B$, $B$~--- разрешимо, то $A$~--- разрешимо;
        \item [б)] если $A {\le}_{m} B$, $B$~--- перечислимо, докажите, что $A$~--- перечислимо;
        \item [в)] $A {\le}_{m} B \Leftrightarrow \mathbb{N} \setminus A \le_m \mathbb{N} \setminus B$;
        \item [г)] $A {\le}_{m} B$, $B \in \Sigma_k$ докажите, что $A \in \Sigma_k$.
    \end{enumerate}
\end{task}


\begin{task}
    Докажите, что:
    \begin{enumerate}[topsep = 0pt, itemsep = -1ex]
        \item [а)] существует универсальное перечислимое множество. Т.е. такое перечислимое множество пар $U$, что для любого
		    перечислимого множества $A$ найдется элемент $a$, что $A = \{x \mid (a, x) \in U\}$. если $A {\le}_{m} B$, $B$~---
            разрешимо, то $A$~--- разрешимо;
        \item [б)] для всех $k \ge 1$ существует универсальное множество в $\Sigma_k$ и $\Pi_k$;
        \item [в)] универсальное множество для $\Sigma_k$ не содержится в $\Pi_k$;
        \item [г)] $\Sigma_k \subsetneq \Sigma_{k + 1}$.
    \end{enumerate}
\end{task}


\begin{task}
	Пусть $T$~--- это множество номеров замкнутых формул в сигнатуре $\{+, \times, =\}$, которые истинны в естественной
    интерпретации на множестве натуральных чисел. Докажите, что:
    \begin{enumerate}[topsep = 0pt, itemsep = -1ex]
        \item [а)] для любого $P \in \Sigma_k$ выполняется $P \le_m T$;
        \item [б)] (теорема Тарского) $T$ не является арифметичным;
        \item [в)] (теорема Геделя о неполноте) Покажите, что $T$ не является перечислимым.
    \end{enumerate}
\end{task}


\begin{task}
	Докажите, что:
    \begin{enumerate}[topsep = 0pt, itemsep = -1ex]
		\item [а)] $\DSpace[n^2] \subsetneq \DSpace[n^3]$;
        \item [б)] $\NSpace[n^2] \subsetneq \NSpace[n^3]$.
    \end{enumerate}
\end{task}


\breakline

\begin{ptask}{8}
    Покажите, что каждый язык, который принимается $k$-ленточной недетерминированной машиной Тьюринга за время $f(n)$, может быть
    принят на $2$-ленточной машине Тьюринга за время $O(f(n))$.
\end{ptask}

\begin{ptask}{23}
	Докажите, что любой перечислимый предикат арифметичен. 
\end{ptask}

\begin{ptask}{24}
	б) двустороннее ассоциативное исчисление, для которого вопрос о выводимости строки $x$ из строки $y$ является алгоритмически
    неразрешимым.
\end{ptask}


\begin{ptask}{25}
	а) Покажите, что для любой конечной (или перечислимой) сигнатуры множество тавтологий в этой сигнатуре перечислимо. б)
    Покажите, что если в сигнатуре есть достаточное количество функциональных и предикатных символов арности 1 и 2, то множество
    тавтологий в этой сигнатуре неразрешимо.
\end{ptask}