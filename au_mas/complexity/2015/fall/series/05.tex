\documentclass[a4paper, 12pt]{article}
% math symbols
\usepackage{amssymb}
\usepackage{amsmath}
\usepackage{mathrsfs}
\usepackage{mathseries}


\usepackage[margin = 2cm]{geometry}

\tolerance = 1000
\emergencystretch = 0.74cm



\pagestyle{empty}
\parindent = 0mm

\renewcommand{\coursetitle}{DM/ML}
\setcounter{curtask}{1}

\setmathstyle{13.10}{}{}

\begin{document}

\setcounter{curtask}{27}

\begin{definition*}
    $\Sigma_0 = \Pi_0$~--- множество разрешимых предикатов на множестве натуральных чисел.
    $\Sigma_{k + 1}$~--- это множество предикатов, которые представляются в виде $\exists y P(x, y)$, где
    $P \in \Pi_{k}$, а предикаты из $\Pi_{k + 1}$ представляются в виде $\forall y P(x, y)$, где $P \in
    \Sigma_{k}$. Последовательность $\Sigma_k$ (и $\Pi_k$) будем называть \deftext{арифметической иерархией}.
\end{definition*}

\libproblem{computability}{ariphmetic-hierarchy}

\begin{definition*}
    Множество $A$ $m$-сводится к множеству $B$, если существует такая вычислимая всюду определенная
    функция $f$, что $x \in A \Leftrightarrow f(x) \in B$. Обозначение: $A \le_m B$.
\end{definition*}

\libproblem{computability}{m-reduction}
\libproblem{computability}{sigma-k-universal}
\libproblem{computability}{tarsky-godel}
\libproblem{complexity}{space-hierarchy}


\breakline

\libproblem[8]{complexity}{two-tapes}
\libproblem[23]{computability}{enumerable-ariphmetic}
\libproblem[24]{computability}{associative-calculus}
\libproblem[25]{computability}{taut-enumerable}

\end{document}