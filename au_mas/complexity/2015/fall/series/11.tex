\mytitle{11 (на 01.12)}


\begin{task}
    Покажите, что $\AM = \AM_1$.
\end{task}


\begin{task}
    Покажите, что $\AM \subseteq \Pi_2^P$.
\end{task}


\breakline

\begin{ptask}{8}
    Покажите, что каждый язык, который принимается $k$-ленточной недетерминированной машиной Тьюринга за время $f(n)$, может быть
    принят на $2$-ленточной машине Тьюринга за время $O(f(n))$.
\end{ptask}



\begin{ptask}{35}
	Постройте примеры полных задач относительно сведений по Карпу в классах $\EXP, \NEXP$ и классе $\NE = \bigcup\limits_{c > 0}
    \NTime[2^{cn}]$.
\end{ptask}


\begin{ptask}{42}
	Обозначим UCYCLE множество всех неориентрованных графов, в которых есть цикл. Докажите, что UCYCLE принадлежит классу $\LL$. 
\end{ptask}


\begin{ptask}{46}
    $\BPL_H$~--- это класс языков, для которых существует вероятностная машина Тьюринга $M$, которая использует логарифмическую
    память, останавливается с вероятностью $1$, и для всех $x$ выполняется, что $\Pr[M(x) = L(x)] \ge \frac{2}{3}$. Покажите, что
    $\BPL_H \subseteq \P$.
\end{ptask}


\begin{ptask}{48}
    Существует вариант класса $\MA$ с односторонней ошибкой. $L \in \MA_1$, если существует такая полиномиальная вероятностная
    машина $V$ и полином $p$, что если $x \in L$, то найдется такая строка $y \in \{0, 1\}^{p(n)}$, что $\Pr[V(x, y) = 1] = 1$, а
    если $x \notin L$, то для любой строки $y \in \{0, 1\}^{p(n)}$ выполняется $\Pr[V(x, y) = 1] < \frac{1}{3}$. Покажите, что
    $\MA = \MA_1$.
\end{ptask}


\begin{ptask}{49}
    Покажите, что $\MA \subseteq \AM$.
\end{ptask}


\begin{ptask}{50}
    Покажите, что $\MA \subseteq \Sigma_2^P$.
\end{ptask}
