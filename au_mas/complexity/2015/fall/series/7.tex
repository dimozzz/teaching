\mytitle{7 (на 27.10)}


\begin{task}
	Докажите, что существует язык, для которого любой алгоритм, работающий время $O(n^2)$ решает его правильно на менее, чем на
    половине входов какой-то длины, но этот язык распознается алгоритмом, работающим время $O(n^3)$.
\end{task}

\begin{task}
	Докажите, что ${\NTime}[n] \neq \PSPACE$.
\end{task}

\begin{task}
	Приведите пример разрешимого языка в $\P/poly$, который не лежит в $\P$. 
\end{task}

\begin{task}
	Докажите, что $\DSpace[n] \neq \NP$
\end{task}


\breakline

\begin{ptask}{8}
    Покажите, что каждый язык, который принимается $k$-ленточной недетерминированной машиной Тьюринга за время $f(n)$, может быть
    принят на $2$-ленточной машине Тьюринга за время $O(f(n))$.
\end{ptask}

\begin{ptask}{23}
	Докажите, что любой перечислимый предикат арифметичен. 
\end{ptask}


\begin{ptask}{30}
	Пусть $T$~--- это множество номеров замкнутых формул в сигнатуре $\{+, \times, {=\}}$, которые истинны в естественной
    интерпретации на множестве натуральных чисел. Докажите, что:
    \begin{enumerate}[topsep = 0pt, itemsep = -1ex]
        \item [а)] для любого $P \in \Sigma_k$ выполняется $P \le_m T$;
        \item [б)] (теорема Тарского) $T$ не является арифметичным;
        \item [в)] (теорема Геделя о неполноте) Покажите, что $T$ не является перечислимым.
    \end{enumerate}
\end{ptask}

\begin{ptask}{32}
	Докажите, что если $\NP \subseteq \DTime[n^{\log n}]$, то $\PH \subseteq \bigcup\limits_{k \ge 1} \DTime[n^{\log^k n}]$    
\end{ptask}


\begin{ptask}{35}
	Постройте примеры полных задач относительно сведений по Карпу в классах $\EXP, \NEXP$ и классе $\NE = \bigcup\limits_{c > 0}
    \NTime[2^{cn}]$.
\end{ptask}
