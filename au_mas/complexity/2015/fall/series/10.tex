\mytitle{10 (на 24.11)}


\begin{task}
    $\BPL_H$~--- это класс языков, для которых существует вероятностная машина Тьюринга $M$, которая использует логарифмическую
    память, останавливается с вероятностью $1$, и для всех $x$ выполняется, что $\Pr[M(x) = L(x)] \ge \frac{2}{3}$. Покажите, что
    $\BPL_H \subseteq \P$.
\end{task}

\begin{task}
    $\BPL_H$~--- это класс языков, для которых существует вероятностная машина Тьюринга $M$, которая использует логарифмическую
    память, останавливается с вероятностью $1$, и для всех $x$ выполняется, что $\Pr[M(x) = L(x)] \ge \frac{2}{3}$. Покажите, что
    $\BPL_H \subseteq \P$.
\end{task}


\begin{definition}
    Язык $L \in \MA$, если  существует такая полиномиальная вероятностная машина $V$ и полином $p$, что если $x \in L$, то
    найдется такая строка $y \in \{0, 1\}^{p(n)}$, что $\Pr[V(x, y) = 1] \ge \frac{2}{3}$, а если $x \notin L$, то для любой
    строки $y \in \{0, 1\}^{p(n)}$ выполняется $\Pr[V(x, y) = 1] < \frac{1}{3}$. 
\end{definition}


\begin{task}
    Существует вариант класса $\MA$ с односторонней ошибкой. $L \in \MA_1$, если существует такая полиномиальная вероятностная
    машина $V$ и полином $p$, что если $x \in L$, то найдется такая строка $y \in \{0, 1\}^{p(n)}$, что $\Pr[V(x, y) = 1] = 1$, а
    если $x \notin L$, то для любой строки $y \in \{0, 1\}^{p(n)}$ выполняется $\Pr[V(x, y) = 1] < \frac{1}{3}$. Покажите, что
    $\MA = \MA_1$.
\end{task}



\begin{definition}
    Язык $L \in \AM$, если  существует такая детерминированная машина $V$ и полином $p$, что если $x \in L$, то
    $\Pr\limits_{z \gets \{0, 1\}^{p(n)}}[\exists y \in \{0, 1\}^{p(n)}: V(x, z, y) = 1] \ge \frac{2}{3}$, а если $x \notin L$, то 
	$\Pr\limits_{z \gets \{0, 1\}^{p(n)}}[\exists y \in \{0, 1\}^{p(n)}: V(x, z, y) = 1] < \frac{1}{3}$.
\end{definition}


\begin{task}
    Покажите, что $\MA \subseteq \AM$.
\end{task}


\begin{task}
    Покажите, что $\MA \subseteq \Sigma_2^P$.
\end{task}




\breakline

\begin{ptask}{8}
    Покажите, что каждый язык, который принимается $k$-ленточной недетерминированной машиной Тьюринга за время $f(n)$, может быть
    принят на $2$-ленточной машине Тьюринга за время $O(f(n))$.
\end{ptask}



\begin{ptask}{35}
	Постройте примеры полных задач относительно сведений по Карпу в классах $\EXP, \NEXP$ и классе $\NE = \bigcup\limits_{c > 0}
    \NTime[2^{cn}]$.
\end{ptask}


\begin{ptask}{41}
    Докажите, что если унарный язык $\NP$-полный, то $\P = \NP$.
\end{ptask}

\begin{ptask}{42}
	Обозначим UCYCLE множество всех неориентрованных графов, в которых есть цикл. Докажите, что UCYCLE принадлежит классу $\LL$. 
\end{ptask}

\begin{ptask}{44}
    Пусть $\ZPP$~--- это класс языков, которые принимаются вероятностной машиной Тьюринга без ошибки, математическое ожидание
    времени работы которых полиномиально. Докажите, что:
    \begin{enumerate}[topsep = 0pt, itemsep = -1ex]
        \item [а)] $L \in \ZPP$ тогда и только тогда, когда существует полиномиальная по времени вероятностная машина Тьюринга
			$M$, которая выдает $\{0, 1, ?\}$, что для всех $x \in \{0, 1\}^*$ с вероятностью $1$, $M(x) \in \{L(x), ?\}$ и 
            $\Pr[M(x) = {?}] \le \frac{1}{2}$;
        \item [б)] $\ZPP = \RP \cap \coRP$.
    \end{enumerate}
\end{ptask}

\begin{ptask}{45}
    $\BPL$~--- это класс языков, для которых существует вероятностная машина Тьюринга $M$, которая использует логарифмическую
    память, останавливается при всех последовательностях случайных битов и для всех $x$ выполняется, что $\Pr[M(x) = L(x)] \ge
    \frac{2}{3}$. Покажите, что $\BPL \subseteq \P$. 
\end{ptask}
