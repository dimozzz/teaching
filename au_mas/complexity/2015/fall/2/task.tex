\setcounter{curtask}{9}

\mytitle{2 (на 22.09)}


\begin{task}
	Приведите пример неразрешимого подмножества $\mathcal{N} \times \mathcal{N}$, такого что все его горизонтальные и вертикальные
    сечения (т.е. пересечения с $N \times \{x\}$ и с $\{x\} \times N$) разрешимы. 
\end{task}


\begin{task}
    Постройте пример двух перечислимых множеств, которые нельзя отделить никаким разрешимым (это значит, что не существует такого
    разрешимого множества, которое содержало бы первое перечислимое множество и не пересекалось бы со вторым).
\end{task}


\begin{task}
    а) Докажите, что существует \textit{универсальное} перечислимое множество, т.е. такое перечислимое подмножество $U \subseteq
    \mathcal{N} \times \mathcal{N}$, что для любого перечислимого подмножества $A \subseteq \mathcal{N}$ найдется такое $a \in
    \mathcal{N}$, что $A = \{x | (a, x) \in U\}$.
	б) Покажите, что универсального разрешимого множества не существует.
\end{task}

\begin{task}
	Покажите, что существует всюду определенная вычислимая функция $a(n)$, принимающая рациональные значения, что существует
    предел $\alpha = \lim\limits_{n \to \infty} a(n) \in \mathbb{R}$, но не существует алгоритма, который бы по рациональному
    числу $\epsilon$ выдал такой $n_0$, что при $n > n_0$ выполняется $|a(n) - \alpha| < \epsilon$. 
\end{task}



Мы называем алгоритмы $\mathcal{A}$ и $\mathcal{B}$ эквивалентными если 
\begin{itemize}
	\item $\forall x\  \mathcal{A}(x)$ останавливается $\Leftrightarrow \mathcal{B}(x)$ останавливается;
	\item $\forall x$ если $\mathcal{A}(x)$ останавливается, то и $\mathcal{A}(x) = \mathcal{B}(x)$.
\end{itemize} 

Такую же эквивалентность можно ввесли на множестве натуральных чисел $a \equiv b \Leftrightarrow {<}a{>} \sim {<}b{>}$. Множество
$S\subseteq \mathcal{N}$ называется инвариантным, если $\forall a \in S, b \in \mathbb{N} \setminus S, a \not\equiv b$.

\begin{task}
   (Теорема Успенского-Райса) Докажите, что если множество $S$ инвариантно и разрешимо, то либо $S = \emptyset$, либо $S = \mathcal{N}$. 
\end{task}

\begin{task}
   Покажите, что множество описаний машин Тьюринга, которые останавливаются на всех входах, является неперечислимым множеством и
   дополнение его тоже неперечислимо.
\end{task}

\begin{task}
	Покажите, что язык 2-SAT (выполнимых формул в 2-КНФ) лежит в классе $\P$. 
\end{task}


\breakline

\begin{ptask}{8}
    Покажите, что каждый язык, который принимается $k$-ленточной недетерминированной машиной Тьюринга за время $f(n)$, может быть
    принят на $2$-ленточной машине Тьюринга за время $O(f(n))$.
\end{ptask}