\setcounter{curtask}{21}

\mytitle{2 (на 06.10)}


\begin{task}
	Выведите теорему Успенского-Райса из теоремы о неподвижной точке.
\end{task}


Вычислимая функция $U(n,x)$ называется универсальной вычислимой функции для вычислимых функций одного аргумента, если для любой
другой вычислимой функции $f$ найдется такое число $m$, что $f(x) = U(m, x)$. 

Но не все универсальные вычислимые функции задают то, что мы понимаем под языками программирования. Требуется более сильное
свойство. Вычислимая функция $U(n, x)$ называется главной нумерацией, если для любой вычислимой функции $V(n, x)$ найдется всюду
определенная вычислимая функция $s$, что $V(n, x) = U(s(n), x)$.  

\begin{task}
	а) Покажите, что функция $U(n,x) = {<}n{>}(x)$ является главной нумерацией. б) Покажите, что если теорема о неподвижной точке
    верна в одной главной нумерации, то она верна и в любой другой.
\end{task}

\begin{task}
	Докажите, что любой перечислимый предикат арифметичен. 
\end{task}


Ассоциативным исчислением называется конечный набор правил вида $\{s_i \to t_i\}_{i \in I}$, где $s_i, t_i$~--- строчки. Говорят,
что строка $y$ выводится из строки $x$, если из строки $x$ можно получить строку $y$, заменяя несколько раз подстроку $s_i$ на
$t_i$. Ассоциативное исчисление называется двусторонним, если наряду с правилом $s \to t$ есть и правило $t \to s$.

\begin{task}
	Покажите, что существует такое а) обыкновенное б) двустороннее ассоциативное исчисление, для которого вопрос о выводимости
    строки $x$ из строки $y$ является алгоритмически неразрешимым. 
\end{task}

\begin{task}
	а) Покажите, что для любой конечной (или перечислимой) сигнатуры множество тавтологий в этой сигнатуре перечислимо. б)
    Покажите, что если в сигнатуре есть достаточное количество функциональных и предикатных символов арности 1 и 2, то множество
    тавтологий в этой сигнатуре неразрешимо.
\end{task}

\begin{task}
	Покажите, что язык, состоящий из выполнимых формул в КНФ, в которых каждый дизъюнкт является либо хорновским (дизъюнкт
    называется хорновским, если не более одной переменной входит в него без отрицания), либо состоит из двух литералов, является
    $\NP$-полным.
\end{task}


\breakline

\begin{ptask}{8}
    Покажите, что каждый язык, который принимается $k$-ленточной недетерминированной машиной Тьюринга за время $f(n)$, может быть
    принят на $2$-ленточной машине Тьюринга за время $O(f(n))$.
\end{ptask}

\begin{task}{18}
	Предикат, заданный на множестве натуральных чисел ($\mathcal{N} = \{0, 1, 2, \dots\}$) называется арифметичным, если он
    выражается с помощью формулы исчисления предикатов в сигнатуре $(+, \times, =)$ в естественной интерпретации на множестве
    натуральных чисел. Докажите, что следующие предикаты являются арифметичными: 
	а) $x < y$; б) $x = 0$; в) $x = 1$; г) $x = c$, где $c$~--- это некоторая натуральная константа; д) $a \bmod b = r$; е)
    $a$~--- это степень двойки; ж) $a$~--- это степень четверки. 
\end{task}


\begin{task}{19}
	а) Докажите, что для любого целого $k$ найдется сколь угодно большое $b$, что $b + 1, 2 b + 1, \dots, k b + 1$~--- попарно
    взаимно простые числа. б) Докажите, что для любой последовательности $x_0, x_1, \dots, x_n$ натуральных чисел можно найти
    такие числа $a$ и $b$, что $x_i = a \bmod b (i + 1) + 1$. в) Докажите, что предикат: $a$~--- степень шестерки арифметичен.    
\end{task}