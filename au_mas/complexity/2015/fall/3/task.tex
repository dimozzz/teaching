\setcounter{curtask}{16}

\mytitle{2 (на 29.09)}

\begin{task}
    а) Докажите, что для любой вычислимой функции $f$ существует всюду определенная вычислимая функция $g$, которая является
    $\equiv$-продолжением $f$, т.е. для всех $x$ для которых определено $f(x)$ выполняется $f(x) \equiv g(x)$.

    б) (Теорема Клини о неподвижной точке) $h$~--- всюду определенная вычислимая функция. Тогда $\exists m \in \mathbb{N}$, 
	что $m \equiv h(m)$. \textit{Подсказка: пусть $u(n) = {<n>} (n)$, а $u'(n)$~--- это $\equiv$-продолжение $u(n)$. Пусть $t(n) =
    h(u'(n))$, выберете $m = u'(\sharp t)$.}
\end{task}


\begin{task}
    Используя теорему Клини  а) докажите, что существует алгоритм, который на всех входах выводит свой номер; б) покажите, что
    существует алгоритм, который всюду применим и выдает $1$ на числе, которое является квадратом его номера, а на всех остальных
    входах выдает ноль; в) докажите, что существуют два различных алгоритма $\mathcal{A}$ и $\mathcal{B}$, что алгоритм
    $\mathcal{A}$ печатает $\sharp \mathcal{B}$, а алгоритм $\mathcal{B}$ печатает $\sharp \mathcal{A}$.
\end{task}


\begin{task}
	Предикат, заданный на множестве натуральных чисел ($\mathcal{N} = \{0, 1, 2, \dots\}$) называется арифметичным, если он
    выражается с помощью формулы исчисления предикатов в сигнатуре $(+, \times, =)$ в естественной интерпретации на множестве
    натуральных чисел. Докажите, что следующие предикаты являются арифметичными: 
	а) $x < y$; б) $x = 0$; в) $x = 1$; г) $x = c$, где $c$~--- это некоторая натуральная константа; д) $a \bmod b =r$; е) $a$~---
    это степень двойки; ж) $a$~--- это степень четверки. 
\end{task}


\begin{task}
	а) Докажите, что для любого целого $k$ найдется сколь угодно большое $b$, что $b + 1, 2 b + 1, \dots, k b + 1$~--- попарно
    взаимно простые числа. б) Докажите, что для любой последовательности $x_0, x_1, \dots, x_n$ натуральных чисел можно найти
    такие числа $a$ и $b$, что $x_i = a \bmod b (i + 1) + 1$. 
	в) Докажите, что предикат: $a$~--- степень шестерки арифметичен.    
\end{task}

\begin{task}
   Покажите, что если $\P = \NP$, то $\EXP = \NEXP$. 
\end{task}


\breakline

\begin{ptask}{8}
    Покажите, что каждый язык, который принимается $k$-ленточной недетерминированной машиной Тьюринга за время $f(n)$, может быть
    принят на $2$-ленточной машине Тьюринга за время $O(f(n))$.
\end{ptask}

\begin{ptask}{10}
    Постройте пример двух перечислимых множеств, которые нельзя отделить никаким разрешимым (это значит, что не существует такого
    разрешимого множества, которое содержало бы первое перечислимое множество и не пересекалось бы со вторым).
\end{ptask}

\begin{ptask}{15}
	Покажите, что язык $2$-SAT (выполнимых формул в $2$-КНФ) лежит в классе $\P$. 
\end{ptask}
