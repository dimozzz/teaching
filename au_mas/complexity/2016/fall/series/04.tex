\documentclass[a4paper, 12pt]{article}
% math symbols
\usepackage{amssymb}
\usepackage{amsmath}
\usepackage{mathrsfs}
\usepackage{mathseries}


\usepackage[margin = 2cm]{geometry}

\tolerance = 1000
\emergencystretch = 0.74cm



\pagestyle{empty}
\parindent = 0mm

\renewcommand{\coursetitle}{DM/ML}
\setcounter{curtask}{1}

%\setmathstyle{Апрель 9}{Теория информации}{2 курс}


\begin{document}

\setcounter{curtask}{22}

\libproblem{computability}{uspensky-rice-kleene}

\begin{definition*}
    Вычислимую функцую $U(n, x)$ будем называть \deftext{универсальной вычислимой функцией} для вычислимых
    функций одного аргумента, если для любой другой вычислимой функции $f$ найдется такое число $m$, что
    $f(x) = U(m, x)$.

    Не все универсальные вычислимые функции задают то, что мы понимаем под языками
    программирования. Требуется более сильное свойство. Вычислимую функцию $U(n, x)$ будем называть
    \deftext{главной нумерацией}, если для любой вычислимой функции $V(n, x)$ найдется такая всюду
    определенная вычислимая функция $s$, что $V(n, x) = U(s(n), x)$.
\end{definition*}

\libproblem{computability}{main-numbering-kleene}
\libproblem{computability}{enumerable-ariphmetic}

\begin{definition*}
    Ассоциативным исчислением будем называть конечный набор правил вида $\{s_i \to t_i\}_{i \in I}$, где
    $s_i, t_i$~--- строчки. Говорят, что строка $y$ выводится из строки $x$, если из строки $x$ можно
    получить строку $y$, заменяя несколько раз подстроку $s_i$ на $t_i$. Ассоциативное исчисление
    называется двусторонним, если наряду с правилом $s \to t$ есть и правило $t \to s$.
\end{definition*}

\libproblem{computability}{associative-calculus}
\libproblem{computability}{taut-enumerable}
\libproblem{complexity}{horn-sat-np}
\libproblem{complexity}{diagonalization-half}


\breakline

\libproblem[8]{complexity}{two-tapes}
\libproblem[18]{computability}{kleene-app}
\libproblem[21]{complexity}{p-np-exp-nexp}


\end{document}



%%% Local Variables:
%%% mode: latex
%%% TeX-master: t
%%% End:
