\documentclass[a4paper, 12pt]{article}
% math symbols
\usepackage{amssymb}
\usepackage{amsmath}
\usepackage{mathrsfs}
\usepackage{mathseries}


\usepackage[margin = 2cm]{geometry}

\tolerance = 1000
\emergencystretch = 0.74cm



\pagestyle{empty}
\parindent = 0mm

\renewcommand{\coursetitle}{DM/ML}
\setcounter{curtask}{1}

%\setmathstyle{Апрель 9}{Теория информации}{2 курс}


\begin{document}

\setcounter{curtask}{39}

\libproblem{complexity}{ntime-neq-pspace}
\libproblem{complexity}{dspace-neq-np}
\libproblem{complexity}{subgraph-isomorphism}
\libproblem{complexity}{ph-in-pspace}
\libproblem{complexity}{logspace-composition}
\libproblem{complexity}{unary-msfp}


\breakline

\libproblem[26]{computability}{taut-enumerable}
\libproblem[33]{complexity}{space-hierarchy}
\libproblem[34]{complexity}{hierarchy-quasipoly}
\libproblem[35]{complexity}{hierarchy-oracle}
\libproblem[36]{complexity}{unary-p-np-exp-nexp}
\libproblem[37]{complexity}{ne-complete}
\libproblem[38]{complexity}{prime-np}


\end{document}



%%% Local Variables:
%%% mode: latex
%%% TeX-master: t
%%% End:
