\documentclass[a4paper, 12pt]{article}
% math symbols
\usepackage{amssymb}
\usepackage{amsmath}
\usepackage{mathrsfs}
\usepackage{mathseries}


\usepackage[margin = 2cm]{geometry}

\tolerance = 1000
\emergencystretch = 0.74cm



\pagestyle{empty}
\parindent = 0mm

\renewcommand{\coursetitle}{DM/ML}
\setcounter{curtask}{1}

%\setmathstyle{Апрель 9}{Теория информации}{2 курс}


\begin{document}

\setcounter{curtask}{52}

\libproblem{complexity}{bpl-in-p-hard}

\begin{definition*}
    Язык $L \in \MA$, если  существует такая полиномиальная вероятностная машина $V$ и полином $p$, что:
    \begin{itemize}
        \item если $x \in L$, то найдется такая строка $y \in \{0, 1\}^{p(n)}$, что $\Pr[V(x, y) = 1] \ge
            \frac{2}{3}$;
        \item если $x \notin L$, то для любой строки $y \in \{0, 1\}^{p(n)}$ выполняется $\Pr[V(x, y) =
            1] < \frac{1}{3}$.
    \end{itemize}
\end{definition*}

\libproblem{complexity}{ma-perfect-compl}


\begin{definition*}
    Язык $L \in \AM$, если  существует такая детерминированная машина $V$ и полином $p$, что:
    \begin{itemize}
        \item если $x \in L$, то:
            $$
                \Pr\limits_{z \gets \{0, 1\}^{p(n)}}[\exists y \in \{0, 1\}^{p(n)}: V(x, z, y) = 1] \ge
                \frac{2}{3};
            $$
        \item если $x \notin L$, то
            $$
                \Pr\limits_{z \gets \{0, 1\}^{p(n)}}[\exists y \in \{0, 1\}^{p(n)}: V(x, z, y) = 1] <
                \frac{1}{3}.
            $$
            
    \end{itemize}
\end{definition*}

\libproblem{complexity}{ma-in-am}
\libproblem{complexity}{ma-in-sigma-2}


\breakline


\libproblem[47]{complexity}{unary-np-complete}
\libproblem[48]{complexity}{ucycle-logspace}
\libproblem[51]{complexity}{bpl-in-p}


\end{document}



%%% Local Variables:
%%% mode: latex
%%% TeX-master: t
%%% End:
