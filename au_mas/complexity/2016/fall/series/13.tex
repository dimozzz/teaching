\documentclass[a4paper, 12pt]{article}
% math symbols
\usepackage{amssymb}
\usepackage{amsmath}
\usepackage{mathrsfs}
\usepackage{mathseries}


\usepackage[margin = 2cm]{geometry}

\tolerance = 1000
\emergencystretch = 0.74cm



\pagestyle{empty}
\parindent = 0mm

\renewcommand{\coursetitle}{DM/ML}
\setcounter{curtask}{1}

%\setmathstyle{Апрель 9}{Теория информации}{2 курс}


\begin{document}

\setcounter{curtask}{63}

\libproblem{struct-complexity}{logspace-in-p}
\libproblem{struct-complexity}{nc-in-p}

\begin{definition*}
    Будем говорить, что язык $L \in \UP$, если существует такая недетерминированная машина Тьюринга $M$,
    что для любого $x$ выполнено $M(x) = L(x)$ и существует не более одной подсказки, которая принимается
    машиной $M$.
\end{definition*}

\libproblem{struct-complexity}{usat-p-np}


\breakline

\libproblem[52]{struct-complexity}{bpl-in-p-hard}
\libproblem[60]{struct-complexity}{pspace-collapse-ma}
\libproblem[61]{struct-complexity}{p-np-hard-function}
\libproblem[62]{struct-complexity}{permanent-oracle-rand}


\end{document}



%%% Local Variables:
%%% mode: latex
%%% TeX-master: t
%%% End:
