\documentclass[a4paper, 12pt]{article}
% math symbols
\usepackage{amssymb}
\usepackage{amsmath}
\usepackage{mathrsfs}
\usepackage{mathseries}


\usepackage[margin = 2cm]{geometry}

\tolerance = 1000
\emergencystretch = 0.74cm



\pagestyle{empty}
\parindent = 0mm

\renewcommand{\coursetitle}{DM/ML}
\setcounter{curtask}{1}

%\setmathstyle{Апрель 9}{Теория информации}{2 курс}


\begin{document}

\setcounter{curtask}{9}

\libproblem{computability}{kolmogorov-empty}
\libproblem{computability}{hor-ver-cuts}
\libproblem{computability}{non-separable}
\libproblem{computability}{universal-enumerable}
\libproblem{computability}{uncomputable-limit}

\begin{definition*}
    Назовем алгоритмы $\alg{A}$ и $\alg{B}$ \deftext{эквивалентными} если:
    \begin{itemtask}
        \item $\forall x\  \alg{A}(x)$ останавливается $\Leftrightarrow \alg{B}(x)$ останавливается;
        \item $\forall x$ если $\alg{A}(x)$ останавливается, то и $\alg{A}(x) = \alg{B}(x)$.
    \end{itemtask}

    Такую же эквивалентность можно ввести на множестве натуральных чисел $a \equiv b \Leftrightarrow
    \avg{a} \sim \avg{b}$. Множество $S \subseteq \mathcal{N}$ назовем \deftext{инвариантным}, если
    $\forall a \in S, b \in \mathbb{N} \setminus S, a \not\equiv b$.    
\end{definition*}


\libproblem{computability}{uspensky-rice}
\libproblem{complexity}{2-sat-p}
\libproblem{complexity}{horn-sat-p}

\breakline

\libproblem[8]{complexity}{two-tapes}


\end{document}



%%% Local Variables:
%%% mode: latex
%%% TeX-master: t
%%% End:
