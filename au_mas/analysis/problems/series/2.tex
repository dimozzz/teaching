\mytitle{2 (на 14.09)}

\begin{task}
    Докажите, что для каждой вычислимой функции $f$ найдется псевдообратная вычислимая функция $g$. А именно, $g$ определена на
    множестве значений $f$, и для всех $x$ из области определения $f$ выполняется $f(g(f(x))) = f(x)$.
\end{task}

\begin{task}
    Существует ли алгоритм, проверяющий, работает ли данная программа
    полиномиальное время? (т.е. на каждом входе алгоритм делает не более $p(|x|)$
    шагов, где $p$~--- полином, а $x$~--- вход алгоритма).
\end{task}

\begin{task}
    Приведите пример числа такого числа $r \in \mathbb{R}$, что множество $\{q \in
    \mathbb{Q} \mid q \le r\}$ не является перечислимым.
\end{task}


\breakline

\begin{ptask}{4}
    Докажите, что всякое бесконечное перечислимое множество содержит бесконечное разрешимое подмножество.
\end{ptask}

\begin{ptask}{5}
    Приведите пример неразрешимого подмножества $\mathbb{N} \times \mathbb{N}$, такого что все его горизонтальные и вертикальные
    сечения (т.е. пересечения с $\mathbb{N} \times \{x\} $ и с $\{x\} \times \mathbb{N}$) разрешимы.
\end{ptask}


\begin{ptask}{6}
    Приведите пример множества, которое а) не является перечислимым б) кроме того и его дополнение тоже не является перечислимым.
\end{ptask}

\begin{ptask}{7}
    Докажите, что непустое множество натуральных чисел разрешимо тогда и только тогда, когда оно есть множество значений всюду
    определённой неубывающей вычислимой функции с натуральными аргументами и значениями.
\end{ptask}
