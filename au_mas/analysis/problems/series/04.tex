\mytitle{4 (на 16.03)}

\begin{task}
    Даны функции $f, g: \{-1, 1\}^n \to \mathbb{R}$. Пусть Фурье спектр $f$ $\epsilon_1$-сосредоточен на $\mathcal{F}$ и
    $||f - g||_2^2 \le \epsilon_2$. Докажите, что Фурье спектр $g$ $2 (\epsilon_1 + \epsilon_2)$-сосредоточен на $\mathcal{F}$.
\end{task}


\begin{task}
    Пусть функция $f: \{-1, 1\}^n \to \{-1, 1\}$ считается схемой глубины $d$ и размера $s$. Докажите, что $\I[f] \le
    (\log(s))^{d - 1}$
\end{task}


\begin{theorem}[KKL]
    Для любой функции $f:\{-1, 1\}^n \to \{-1, 1\}$ верно: $\max\limits_{i} (\Inf_i[f]) \ge \Var[f] \cdot
    \Omega(\frac{\log(n)}{n})$.
\end{theorem}

\begin{task}(можно пользоваться KKL теоремой)
    Докажите, что если $f:\{-1, 1\}^n \to \{-1, 1\}$ транзитивно-симметрична и $\Var[f] \ge \Omega(1)$, то $\I[f] \ge \log(n)$. 
\end{task}


\begin{task}(сложная)
    Пусть $f:\{-1, 1\}^n \to \{-1, 1\}$ вычислима ДНФ размера $s$. Докажите, что:
   	\begin{enumerate}[topsep = 0pt, itemsep = -1ex]
        \item [а)] существует $S \subseteq [n]$, что $|S| \le \log(s) + O(1)$ и $\hat{f}(S) \ge \Omega(\frac{1}{s})$;
        \item [б)] функцию $f$ можно выучить (в классе ДНФ формул размера не более $s$) с ошибкой $\frac{1}{2} -
            \Omega(\frac{1}{s})$ за время $\poly(n, s)$.
	\end{enumerate}
\end{task}

\begin{task}
    Докажите, что если для множеств $S, T \subseteq [n]$ выполняется $|S| + |T| = n + 1$, тогда $\widehat{Maj_n}(S) =
    \widehat{Maj_n}(T)$.
\end{task}



\breakline

\begin{ptask}{21}
    Пусть функция $f: \{-1, 1\}^n \to \mathbb{R}$ и $\epsilon > 0$. Покажите, что функция $f$ $\epsilon$-сконцентрирована на таком
    семействе множеств $F \subseteq 2^{[n]}$, что $|F| \le \frac{\norm{f}_1^2}{\epsilon}$.
\end{ptask}


%%% Local Variables:
%%% mode: latex
%%% TeX-master: t
%%% End:
