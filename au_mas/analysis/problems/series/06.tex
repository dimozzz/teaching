\mytitle{6 (на 31.03)}


\begin{task}
    Пусть некоторое свойство $m$-битных строк имеет $\PCPP$ длины $\ell(m)$. Покажите, что для этого свойства существует $\PCPP$
    длины $poly(\ell(m))$, в котором проверяющий алгоритм делает 3 запроса и затем использует один из восьми возможных
    OR-предикатов: $v_{i_1} \lor v_{i_2} \lor v{i_3}, \lnot v_{i_1} \lor v_{i_2} \lor v_{i_3}, \dots, \lnot v_{i_1} \lor \lnot
    v_{i_2} \lor \lnot v_{i_3}$.
\end{task}

\begin{task}
    Покажите, что для случайной функции $f: \{-1, 1\}^n \to \{-1, 1\}$ выполняется $\Exp[\Inf_i^{(1 - \delta)}[f]] =
    \frac{(1 - \delta / 2)^n}{2 - \delta}$.
\end{task}

\begin{task}
    Покажите, что если функция $f:\{-1, 1\}^n \to \{-1, 1\}$ отлична от константы и зависит только от $k$ переменных, то
    $\Inf_i^{(1 - \delta)}[f] \ge (\frac{1}{2} - \frac{\delta}{2})^{k - 1} / k$ хотя бы для одного $i$.
\end{task}

\begin{task}
    Пусть для $f: \{-1, 1\} \to \mathbb{R}$ выполняется $\Inf_i[f] \le \epsilon$ для всех $i$. Покажите, что $f$ является
    $\sqrt{\epsilon}$-регулярной.
\end{task}

\begin{task}
    Покажите, что для всех четных $n$ существует функция $f: \{-1, 1\}^n \to \{-1, 1\}$, которая является
    $2^{-\frac{n}{2}}$-регулярной, но найдется $i$, что $\Inf_i[f] \ge \frac{1}{2}$.
\end{task}

\begin{task}
    Рассмотрим функцию $f: \{-1, 1\}^{n + 1} \to \{-1, 1\}$, котрая определена так: $f(x_0, x_1, \dots, x_n) = x_0 Maj_n(x_1, x_2,
    \dots, x_n)$.

    Покажите, что:
    \begin{enumerate}[topsep = 0pt, itemsep = -1ex]
        \item [а)] $\Inf_0^{1 - \delta} [f] = \Stab_{1 - \delta}[Maj_n]$ для всех $\delta \in (-1, 1)$;
%        \item [б)] $f$ не вляется $(\epsilon, \delta)$-квазислучайной, если $\epsilon < 1 - \sqrt{\delta}$;
		\item [б)] $f$ является $\frac{1}{\sqrt{n}}$-регулярной.
	\end{enumerate}
\end{task}


\breakline

\begin{ptask}{25}
    Пусть функция $f: \{-1, 1\}^n \to \{-1, 1\}$ считается схемой глубины $d$ и размера $s$. Докажите, что $\I[f] \le
    (\log(s))^{d - 1}$
\end{ptask}

\begin{ptask}{26}(можно пользоваться KKL теоремой)
    Докажите, что если $f:\{-1, 1\}^n \to \{-1, 1\}$ транзитивно-симметрична и $\Var[f] \ge \Omega(1)$, то $\I[f] \ge \log(n)$. 
\end{ptask}


\begin{ptask}{27}
    Пусть $f:\{-1, 1\}^n \to \{-1, 1\}$ вычислима ДНФ размера $s$. Докажите, что:
   	\begin{enumerate}[topsep = 0pt, itemsep = -1ex]
        \item [а)] существует $S \subseteq [n]$, что $|S| \le \log(s) + O(1)$ и $\hat{f}(S) \ge \Omega(\frac{1}{s})$;
        \item [б)] функцию $f$ можно выучить (в классе ДНФ формул размера не более $s$) с ошибкой $\frac{1}{2} -
            \Omega(\frac{1}{s})$ за время $\poly(n, s)$.
	\end{enumerate}
\end{ptask}

\begin{ptask}{28}
    Докажите, что если для множеств $S, T \subseteq [n]$ выполняется $|S| + |T| = n + 1$, тогда $\widehat{Maj_n}(S) =
    \widehat{Maj_n}(T)$.
\end{ptask}

\begin{ptask}{30}
    Докажите, что если существует $r$-запросовая $\PCPP$ с коэффициентом отказа $\lambda$ для некоторого множества, то по нему
    можно эффективно построить $3$-х запросовую $\PCPP$ с коэффициентом отказа $\lambda / (r 2^r)$, причем длина доказательства
    увеличивается на $r 2^rm$, где $m$~--- это размер описания тестирующего алгоритма.
\end{ptask}


\begin{ptask}{34}
    Дана булева схема $C$ с $n$ входами. Докажите, что для множества для множества выполняющих наборов схемы $C$ можно построить
    $3$-запросовую $\PCPP$ с длиной доказательства $2^{poly (|C|)}$.
\end{ptask}



%%% Local Variables:
%%% mode: latex
%%% TeX-master: t
%%% End:
