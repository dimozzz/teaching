\documentclass[a4paper, 12pt]{article}
% math symbols
\usepackage{amssymb}
\usepackage{amsmath}
\usepackage{mathrsfs}
\usepackage{mathseries}


\usepackage[margin = 2cm]{geometry}

\tolerance = 1000
\emergencystretch = 0.74cm



\pagestyle{empty}
\parindent = 0mm

\renewcommand{\coursetitle}{DM/ML}
\setcounter{curtask}{1}

\setmathstyle{}{}{}

\begin{document}

\setcounter{curtask}{17}

\libproblem{boolean-analysis}{decision-tree-spectrum}
\libproblem{boolean-analysis}{degree-to-active-var}
\libproblem{boolean-analysis}{degree-pr-non-zero}
\libproblem{boolean-analysis}{degree-inf-gap}
\libproblem{boolean-analysis}{1-norm-concentration}
\libproblem{boolean-analysis}{k-junta-learning}

\begin{definition*}
    \deftext{Матрица Уолша--Адамара} $H_k$~--- это вещественная матрица размера $2^k \times 2^k$, которую
    можно определить индуктивно: $H_0 \coloneqq 1$,
    $$
        H_{k + 1} \coloneqq \left(
            \begin{array}{cc}
              H_k & H_k \\
                H_k & -H_k
            \end{array}
        \right).
    $$
\end{definition*}

\libproblem{boolean-analysis}{walsh-hadamard-expansion}

\end{document}

%%% Local Variables:
%%% mode: latex
%%% TeX-master: t
%%% End:
