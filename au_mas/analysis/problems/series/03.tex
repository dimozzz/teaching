\begin{task}
    Докажите, что если функция $f: \{-1, 1\}^n \to \{-1, 1\}$ вычисляется деревом принятия решения размера $s$, то спектр $f$
    $\epsilon$-сосредоточен на степенях до $\log{\frac{s}{\epsilon}}$.
\end{task}


\begin{task}
	Пусть функция $f:\{-1, 1\}^n \to \{-1, 1\}$ имеет степень не более $k$. Докажите, что $f$ зависит от $k 2^{k - 1}$
    переменных.
\end{task}


\begin{task}
	Пусть функция $f: \{-1, 1\}^n \to \mathbb{R}$ отлична от нуля и имеет степень не более $k$. Докажите, что $\Pr[f(x) \neq 0]
    \ge 2^{-k}$.
\end{task}


\begin{task}
    Пусть функция $f: \{-1, 1\}^n \to \{-1, 1\}$ имеет степень не более $k$. Докажите, что $\Inf_i[f]$ либо $0$, либо как минимум
    $\frac{1}{2^{k - 1}}$.
\end{task}

\begin{task}
    Пусть функция $f: \{-1, 1\}^n \to \mathbb{R}$ и $\epsilon > 0$. Покажите, что функция $f$ $\epsilon$-сконцентрирована на таком
    семействе множеств $F \subseteq 2^{[n]}$, что $|F| \le \frac{\norm{f}_1}{\epsilon}$.
\end{task}

\begin{task}
    Покажите, что функции из класса $k$-junta могут быть выучены с ошибкой $0$ за время $poly(n, 2^k)$.
\end{task}


\begin{task}
	Матрица Уолша-Адамара $H_k$~--- это вещественная матрица размера $2^k \times 2^k$, которая определяется индуктивно: $H_0 = 1$,
    $H_{k + 1} = \left( \begin{array}{cc} H_k & H_k \\ H_k & -H_k \end{array} \right)$.

    \begin{enumerate}[topsep = 0pt, itemsep = -1ex]
        \item [а)] Проверьте, что различные столбцы матрицы $H_n$ ортогональны.
        \item [б)] Будем индексировать строки и столбцы матрицы $H_n$ битовыми строками длины $n$. Докажите, что клетка матрицы
    		$H_n$ с координатами $(x, y)$ содержит число $(-1)^{x \cdot y}$.
		\item [в)] Пусть функция $f: \mathbb{F}_2^n \to \mathbb{R}$ представлена в виде вектора из $\mathbb{R}^{2^n}$. Покажите,
			что $2^{-n}H_n f = \hat f$, где $\hat f$ проиндексирована характеристическими векторами множеств.
		\item [г)] Покажите, что можно вычислить $H_n f$, используя $n 2^n$ сложений и вычитаний.
		\item [д)] Проверьте, что $\hat {\hat f} = 2^{-n}f$.
	\end{enumerate}
\end{task}


%%% Local Variables:
%%% mode: latex
%%% TeX-master: t
%%% End:
