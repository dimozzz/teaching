\mytitle{9 (на 04.05)}

Пусть $B \in \mathbb{R}$. Назовем случайную величину $X$ $B$-reasonable если $E[X^4] \le B E[X^2]^2$.

\begin{task}
    Для любого $1 < b < B$ покажите, что существует случайная $b$-reasonable величина $X$, что случайная величина $1 + X$ не
    $B$-reasonable.
\end{task}


\begin{task}
    Покажите, что KKL теорема не верная для функций $f: \{-1, 1\} \to [-1, 1]$, даже если $\Var[f] \ge \Omega(1)$.
\end{task}

\begin{ptask}{25}
    Пусть $f: \{-1, 1\}^n \to \mathbb{R}$. Докажите, что $||f||_4 \le s^{\frac{1}{4}} ||f||_2$, где $s$~--- число ненулевых
    коэффициентов Фурье.
\end{ptask}

\begin{task}
    Пусть $f: \{-1, 1\}^n \to \mathbb{R}$ имеет степень не более $1$. $x_i$~--- независимые $3$-reasonable случайные
    величины и $E[x_i] = E[x_i^3] = 0$. Покажите, что $f(x)$ является $3$-reasonable.
\end{task}

\begin{task}
    Пусть $f: \{-1, 1\}^n \to \mathbb{R}$ и для любого $i \in [m]$ выполенено $Pr[f(x) = i] = \frac{1}{m}$. Пусть $0 \le \rho
    \le 1$ и $(x, y)$~--- $\rho$-скоррелированы. Покажите, что $\Pr[f(x) = f(y)] \le (\frac{1}{m})^{\frac{1 - \rho}{1 + \rho}}$.
\end{task}



\breakline

\begin{ptask}{25}
    Пусть функция $f: \{-1, 1\}^n \to \{-1, 1\}$ считается схемой глубины $d$ и размера $s$. Докажите, что $\I[f] \le
    (\log(s))^{d - 1}$
\end{ptask}



\begin{ptask}{27}
    Пусть $f:\{-1, 1\}^n \to \{-1, 1\}$ вычислима ДНФ размера $s$. Докажите, что:
   	\begin{enumerate}[topsep = 0pt, itemsep = -1ex]
        \item [а)] существует $S \subseteq [n]$, что $|S| \le \log(s) + O(1)$ и $\hat{f}(S) \ge \Omega(\frac{1}{s})$;
        \item [б)] функцию $f$ можно выучить (в классе ДНФ формул размера не более $s$) с ошибкой $\frac{1}{2} -
            \Omega(\frac{1}{s})$ за время $\poly(n, s)$.
	\end{enumerate}
\end{ptask}

\begin{ptask}{48}
    Для любого $k \in \mathbb{N}_{+}$ покажите, что существует $PTF$ степени не более $k$, что $\W^{\le k} \le
    \frac{1}{2^{k - 1}}$.
\end{ptask}

\begin{ptask}{49}
    Пусть $f: \{-1, 1\}^n \to \{-1, 1\}$ можно вычислить ДНФ размера $s$, докажите, что $f$ можно представить в виде $PTF$ с
    $O(n s^2)$ мономами.
\end{ptask}



%%% Local Variables:
%%% mode: latex
%%% TeX-master: t
%%% End:
