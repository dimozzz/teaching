\begin{task}
	Постройте $3$-запросовую $\PCPP$ систему для класса $\{w \mid IP_n(w) = 1\}$ с доказательствами длины $O(n)$, где $IP_n(x_1, x_2,
    \dots, x_n, y_1, y_2,\dots y_n) = -(1)^{\sum\limits_i x_i y_i}$.
\end{task}


\begin{task}
    Докажите, что если существует $r$-запросовая $\PCPP$ с коэффициентом отказа $\lambda$ для некоторого множества, то по нему
    можно эффективно построить $3$-х запросовую $\PCPP$ с коэффициентом отказа $\lambda / (r 2^r)$, причем длина доказательства
    увеличивается на $r 2^rm$, где $m$~--- это размер описания тестирующего алгоритма.
\end{task}

\begin{task}
    Пусть $\gamma \in \mathbb{F}_2^n$, $\Gamma$~--- это матрица из $\mathbb{F}^{n \times n}$, а векторы $x, y$ выбираются случайно
    и равномерно из $\mathbb{F}_2^n$ независимым образом. Покажите, что если $\Gamma = \gamma \gamma^T$, то
    $\Pr[(\gamma^T x) (\gamma^T y) = \Gamma \bullet (x y^T)] = 1$. А если $\Gamma \neq \gamma \gamma^T$, то
    $\Pr[(\gamma^T x) (\gamma^T y) = \Gamma \bullet (x y^T)] \le \frac{3}{4}$, где для двух матриц $A, B \in
    \mathbb{F}^{n \times n}$, $A \bullet B = \sum\limits_{i = 1}^n \sum\limits_{j = 1}^n A_{i, j} B_{i, j}$.
\end{task}


\begin{task}
	Предположим, что есть оракульный доступ к двум функциям $\ell: \mathbb{F}_2^n \to \mathbb{F}_2$ и
    $q: \mathbb{F}_2^{n \times n} \to \mathbb{F}_2$. Предложите $4$-запросовый тестирующий алгоритм, который обладает следующими
    свойствами:
    \begin{itemize}
        \item если $\ell = \chi_{\gamma}$ и $q = \chi_{\gamma \gamma^T}$, то тест принимает с вероятностью $1$;
        \item если тест принимает с вероятностью хотя бы $1 - \lambda \epsilon$, то существует некоторое $\gamma \in
			\mathbb{F}_2^n$, что $\ell$ является $\epsilon$-близким к $\chi_\gamma$, а $q$ является $\epsilon$-близким к
            $\chi_{\gamma \gamma^T}$.
    \end{itemize}
\end{task}


\begin{task}
    Пусть $L$~--- это список однородных полиномиальных равенств степени $2$ над переменными $w_1, w_2, \dots, w_n \in
    \mathbb{F}_2$. Пусть $\mathcal{L}$ состоит из $w \in \mathbb{F}_2^n$, которые выполняют все равенства из списка $L$. Покажите,
    что существует $4$-запросовая $\PCPP$ c длиной доказательства $2^n + 2^{n^2}$ для $L$.
\end{task}


\begin{task}
    Дана булева схема $C$ с $n$ входами. Докажите, что для множества для множества выполняющих наборов схемы $C$ можно построить
    $3$-запросовую $\PCPP$ с длиной доказательства $2^{poly (|C|)}$.
\end{task}


\breakline

\begin{ptask}{25}
    Пусть функция $f: \{-1, 1\}^n \to \{-1, 1\}$ считается схемой глубины $d$ и размера $s$. Докажите, что $\I[f] \le
    (\log(s))^{d - 1}$
\end{ptask}

\begin{ptask}{26}(можно пользоваться KKL теоремой)
    Докажите, что если $f:\{-1, 1\}^n \to \{-1, 1\}$ транзитивно-симметрична и $\Var[f] \ge \Omega(1)$, то $\I[f] \ge \log(n)$. 
\end{ptask}


\begin{ptask}{27}(сложная)
    Пусть $f:\{-1, 1\}^n \to \{-1, 1\}$ вычислима ДНФ размера $s$. Докажите, что:
   	\begin{enumerate}[topsep = 0pt, itemsep = -1ex]
        \item [а)] существует $S \subseteq [n]$, что $|S| \le \log(s) + O(1)$ и $\hat{f}(S) \ge \Omega(\frac{1}{s})$;
        \item [б)] функцию $f$ можно выучить (в классе ДНФ формул размера не более $s$) с ошибкой $\frac{1}{2} -
            \Omega(\frac{1}{s})$ за время $\poly(n, s)$.
	\end{enumerate}
\end{ptask}

\begin{ptask}{28}
    Докажите, что если для множеств $S, T \subseteq [n]$ выполняется $|S| + |T| = n + 1$, тогда $\widehat{Maj_n}(S) =
    \widehat{Maj_n}(T)$.
\end{ptask}



%%% Local Variables:
%%% mode: latex
%%% TeX-master: t
%%% End:
