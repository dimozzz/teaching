\begin{enumerate}
    \item Ряд и коэффициенты Фурье. Подсчет коэффициентов с помощью интерполяции. Линейное пространство функций, ортогональность
        $\chi_S$. Скалярное произведение и его свойства. BLR тест.
    \item $\Inf_i[f]$. Примеры вычисления $\Inf_i[f]$. Дискретная производная. Коэффициенты фурье для дискретной производной и 
        подсчет $\Inf_i[f]$ и $\I[f]$. $\rho$-коррелированные распределения, $\Stab_{\rho}[f]$. $T_{\rho}(x)$ и коэффициенты фурье
        для $\Stab_{\rho}$. Теорема Arrow.
    \item Алгоритм Голдрейха-Левина. Взвешивание корзин. PAC learning. Деревья разбора.
    \item Связь ширины ДНФ и степени до которой спектр $\epsilon$-сконцетрирован (простой вариант
        $\frac{2w}{\epsilon}$). Mansour's conjecture. $\delta$-подстановки. Мат. ожидание коэффициентов фурье (и квадратов
        коэффициентов). $\I[f] \le O(\log(DNF_{size}(f)))$.
    \item Swithing lemma (формулировка). Связь ширины ДНФ и степени до которой спектр $\epsilon$-сконцетрирован (сложный вариант
        $w \log(\frac{1}{\epsilon})$). Мат. ожидание $2^{DT(f_{(J|z)})}$. Оценка на число множеств, где сконцетрирован
        спектр. Переход к схемам константной глубины (идея).
    \item Формулировка FKN и KKL теорем. BLR + NAE. Оценка вероятности неудачи в теореме Arrow. Тест на диктатора (6 запросов и
        три запроса). Тест на семейство диктаторов. Отсутствие теста для четности (доказательство самостоятельно). PCPP
        системы. PCPP система для четности. Существование PCPP системы с доказательством размера $2^{2^{n}}$ для любого семейства
        функций (начало).
    \item Существование PCPP системы с доказательством размера $2^{2^{n}}$ для любого семейства функций
        (окончание). $\epsilon$-регулярные функции. Оценка коэффициентов Фурье для случайной функции. Связь первой нормы и
        $\epsilon$-регулярности. Связь четвертой нормы и $\epsilon$-регулярности. $(1 - \delta)$-стабильное влияние. $(\epsilon,
        \delta)$-стабильные влияния. Примеры. $(\epsilon, k)$-регулярные регулярные функции. Изменение среднего при подстановках в
        $(\epsilon, k)$-регулярную функцию.
    \item Псевдослучайность. Теорема Виолы.
    \item Majority?
    \item Гиперсжимаемость и KKL теорема.
\end{enumerate}