Lecture 7.

|\hat{f}(\emptyset) - p| \le 2 \sqrt{n} 2^{-n / 2}
|\hat{f}(S)| \le 2 \sqrt{n} 2^{-n / 2}


Опр. \epsilon-регулярная f.

||f||_{\infty}^{s} \le ||f||_1

Сл-е. ||f||_1 \le \epsilon => \epsilon-регулярная


Плотность вероятности. \epsilon-регулярная плотность.

Связь с четвертой спектральной нормой.


Очень силньное условие. Для монотонных оно означает Inf_i[f] \le \epsilon.
Для немонотонных плохо, так так влияние у случайной очень большое.

(1 - \delta)-стабильное влияние.
Утв. E[Inf_i^{(1 - \delta)}] = (1 - \delta / 2)^n / (2 - \delta) (на дом!)

(1 - \delta)-стабильное влияние через Фурье (стр. 57)


(\epsilon, \delta)-стабильные влияния.

Примеры Maj, \chi_S
нестабильная --- k-хунта


(\epsilon, k)-регулярные (регулярные до множеств размера k).

Оценки на изменение среднего при подстановках.