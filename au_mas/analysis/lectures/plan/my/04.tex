Lecture 4.


Если функция вычисляется маленькой схемой константной глубины, то это дает условия на спектр.
Был пример про DT.

Более того, мы получим алгоритм для выучивания функций (как следствие).


DNF_w(f) \le w => I[f] \le 2w

DNF_w(f) \le w => спектр сконцентрирован на степенях до 2 w / \epsilon (I[f] = sum k W^k)

Следствие ДНФ (КНФ) маленькое => есть узкое ДНФ, которое считает приближенную ф-цию. (см. домашнюю задачу)
Следствие мы еще и выучить все можем.


Если ширина w, то спектр сосредоточен O(log(w / \epsilon) / \epsilon)




Случайные подстановки.

\delta-множество.

E[\hat{f}|_{J|z}] = Pr[S \in J] \hat{f}(S) = \delta^{|S|} \hat{f}(S)

E[\hat{f}|_{J|z}^2] = ...


Следствие влияние Inf_i[f_{J | z}] = \delta Inf.


T. DNF_size < s => Inf[f] = O(log(s))



Switching lemma.

Фурье спектр 3 \epsilon  сконцентрирован на 3k / delta.

T 4.22