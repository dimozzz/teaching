Lecture 8-9.


НОРМИРОВКИ У PROBABILITY DENSITY


Полиномы над F_2. Заметим, что степень у xor сильно упала.

Существование и единственность представления.


f(x) = \sum c_s x^S,    c_s = \sum\limits_{supp(x) \in S} f(x)   (в интерполяц. полиноме x - 1 = x + 1)
Следствие коэф. при макс степени не ноль => f(x) равно единице в неч. числе точек.


deg_{F_2} (f) \le deg (f)  f:{-1, 1}^n \to {-1, 1}

Следствие про k-resilent
g = x_1 + ... + f


Теорема про (0, k)-regular
подстановка переменным кроме макс. монома. еще одна подстановка сильно меняет значение.





Алгоритм обучения k-хунта!





Конструкции.

1. bent-functions F_2^n \to {-1, 1}

f g (тензорно) --- bent

f \chi_s, f M --- bent

IP g(y) --- bent  IP(x, y) = \chi_y (x) 


2. small cardinality \eps-регулярное

16 (n / \eps)^2

enc --- линейная функция




Мат. ожидание по \eps density
E_\phi [f(x)] - E[f(x)] \le \eps ||f||_1 (спектральная)
(E_\phi [f(x)] = <\phi, f>)

E_\phi [f^2(x)] - E[f(x)]^2 \le \eps ||f||_2^2 (спектральная)

Алгоритм нахождения мат. ожидания


2. small cardinality (log(n)!) (\eps, k)-регулярное

\phi_A = \sum\limits_{\gamma \in perp(A)} X_{\gamma}

матрица Вандермонта
