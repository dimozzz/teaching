Lecture 10-11.


BLR тест. Можно ли за 0.99n случайный битов?

сделаем за n + \log(n / \eps)

y ~ \phi

\teta \le E[f(x) f(y) f(x + y)] = E[f(y) f * f(y)] \le \sqrt{E[f^2(y)]} \sqrt{E[f * f^2 (y)]}

E[f * f^2 (y)] \le E[f * f^2 ] + ||f * f|| \eps  (см. прошлую пару + то, что от умножения на \chi ничего не зависит)

E[f * f^2 ] + ||f * f|| \eps = \sum \hat{f}^4 + \eps \le  max{\hat{f}^2} \sum \hat{f}^2 + \eps

=> есть большой коэф


fooling polynomial
IP по IP --- крушение надежд

\Delta_y f(x) = f(x + y) + f(x) (над F_2)

\Delta понижает стпень

g(x) = f(x + y) + f(x + y') понижает степень


теорема Виолы.

I. E[f]^2 > \eps_d

в конце замена f'(x) = f(x + y_0) (там где лишняя свертка).





Линейные threshold.

Много представлений одно ф-ции, т.к. можно делать маленькие сдвиги.

Упр. можно выбрать a_i целыми.

обычно будем считать, что \sum a_i^2 = 1
нигде не равно 0!


Т. |S| \le 1, \hat{g}(S) = \hat{f}(S) => g = f

f(x) = sgn(l(x))

f(x) l(x) = |l(x)| > g(x) l(x)

E[f l] = \sum_{|S| \le 1} f(S) l(S)
