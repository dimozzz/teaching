Lecture 12.


Приближение через PTF. f : {-1, 1}^n \to R

Pr[T = T] = |\hat{f}(T)| / ||f||_1

кидаем s семплов и выдаем сумму знаков


Следствие. f : {-1, 1}^n \to {-1, 1} => f выражается PTF! И представляется Maj(Parity, \neg Parity, ...)
про выражение



Нормальное распределение. ЦПТ!!!

I[Maj] = \sum \hat{Maj}(i) = ... = E[|\sum x_i|]

E[Maj] = \int норм.




Коэффициенты

Берем производную, получаем ф-цию Half: \{-1, 1\}^n \to {0, 1}

Дернем T_{\rho} Half (1, 1, 1, 1, 1, 1) = ...

И как полином полином от \rho.




W[Maj] стремится сверху к ряду для arcsin(\rho)