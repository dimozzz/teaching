\documentclass[a4paper, 12pt]{article}
% math symbols
\usepackage{amssymb}
\usepackage{amsmath}
\usepackage{mathrsfs}
\usepackage{mathseries}


\usepackage[margin = 2cm]{geometry}

\tolerance = 1000
\emergencystretch = 0.74cm



\pagestyle{empty}
\parindent = 0mm

\renewcommand{\coursetitle}{DM/ML}
\setcounter{curtask}{1}

\setmathstyle{АУ}{Задание 2}{26.02}
\setcounter{curtask}{9}

\begin{document}

\libproblem{inf-theory}{standard-interval-alg}

\begin{definition*}
    Семейство распределений $\mathcal{D}_n$ будем называть \deftext{вычислимым за полиномиальное время},
    если функция распределения ($F_{\mathcal{D}_n}(x) \coloneqq \sum\limits_{y < x} \mathcal{D}_n(x)$,
    где $<$~--- лексикографический порядок) вычислима за полиномиальное время.

    Пусть $S$~--- это сэмплер для некоторого распределения. Если $S$, используя случайную строку $r$,
    выдает $x$, то будем считать, что сэмплер отображает множество строк $[0.r0, 0.r1)$ в $x$. Семейство
    распределений $\mathcal{D}_n$ будем называть \deftext{обратимым}, если сэмплер для этого семейства
    распределений обладает следующим свойством: прообраз каждой строки~--- это отрезок, концы которого
    вычисляются за полиномиальное от $|x|$ время.
\end{definition*}

\libproblem{avg-case-complexity}{computable-dist-inv}
\libproblem{avg-case-complexity}{heurbpp-joint-prob}
\libproblem{cryptography}{prime-not-owf}
\libproblem{cryptography}{cycles-in-owf}


\breakline

\libproblem[3]{cryptography}{owf-poly-image}


\end{document}



%%% Local Variables:
%%% mode: latex
%%% TeX-master: t
%%% End:
