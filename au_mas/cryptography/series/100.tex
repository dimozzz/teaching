\documentclass[a4paper, 12pt]{article}
% math symbols
\usepackage{amssymb}
\usepackage{amsmath}
\usepackage{mathrsfs}
\usepackage{mathseries}


\usepackage[margin = 2cm]{geometry}

\tolerance = 1000
\emergencystretch = 0.74cm



\pagestyle{empty}
\parindent = 0mm

\renewcommand{\coursetitle}{DM/ML}
\setcounter{curtask}{1}

\setmathstyle{АУ}{Зачет}{}

\begin{document}

\libproblem{cryptography}{owf-nonuniversal-hardcore}
\libproblem{cryptography}{stat-dist-comp-indist}
\libproblem{cryptography}{noninjective-prg}
\libproblem{cryptography}{prg-composition-comp}

\begin{definition*}
    Пусть $\alpha_n$~--- случайная величина со значениями в $\{0, 1\}^{p(n)}$, где $p$~---
    полином. Говорят, что $\alpha_n$ \deftext{псевдослучайна по Яо}, если для всех $0 \le i \le p(n) - 1$
    бит $(\alpha_n)_{i + 1}$ нельзя предсказать по префиксу $(\alpha_n)_{\le i}$, т.е. для любого
    полиномиально вероятностного алгоритма $A$ и для любого полинома $q$ выполнено:
    $$
        \Pr[A((\alpha_n)_{\le i}) = (\alpha_n)_{i + 1}] \le \frac{1}{2} + \frac{1}{q(n)}.
    $$
\end{definition*}

\libproblem{cryptography}{yao-comp-indist-uniform}
\libproblem{cryptography}{comp-indist-uniform-yao}

\begin{definition*}
    Полиномиально вычислимую функцию $f\colon \{0, 1\}^* \to \{0, 1\}^*$ будем называть распределенной
    односторонней, если существует такой полином $p$, что для всех полиномиальных по времени
    вероятностных алгоритмов $A$ и для всех достаточно больших $n$, статистическое расстояние между
    распределениями $(\distrib{U}_n, f(\distrib{U}_n))$ и $(A(1^n, f(\distrib{U}_n)), \distrib{U}_n)$
    больше, чем $1 / p(n)$.
\end{definition*}

\libproblem{cryptography}{weak-owf-distrib-owf}
\libproblem{cryptography}{one-bit-indist-mbits}
\libproblem{cryptography}{strong-owf-inv-density}
\libproblem{cryptography}{bit-commitment-to-owf}
\libproblem{cryptography}{owf-without-hardcore-bit}
\libproblem{cryptography}{public-key-one-bit-mbits}
\libproblem{cryptography}{prg-composition-prg}
\libproblem{avg-case-complexity}{avg-delta-notin-avgp}
\libproblem{avg-case-complexity}{avgp-notin-np}
\libproblem{avg-case-complexity}{bh-u-avgp-np-u}
\libproblem{avg-case-complexity}{np-pisamp-in-heurp}


\end{document}


%%% Local Variables:
%%% mode: latex
%%% TeX-master: t
%%% End:
