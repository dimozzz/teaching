\setcounter{curtask}{23}

\mytitle{5 (на 20.03)}

\begin{task}
    Докажите, что если существует $l(n)$-псевдослучайный генератор, то для любого
    многочлена $p(n)$ для всех достаточно больших $n$ для любой строки $\alpha \in
    \{0, 1\}^{l(n)}$ выполняется: $\Pr\limits_{x \gets U_n}[G(x) = \alpha] \le
    \frac{1}{p(n)}$.
\end{task}

\begin{task}
    Покажите, что если существуют односторонние функции, то существует и такая
    односторонняя, что ни один из битов входа не является трудным битом для нее.
\end{task}

\begin{task}
    Пусть $\alpha_n$, $\beta_n$~--- случайные величины со значениями в $\{0,
    1\}^{p(n)}$, где $p$~--- полином. Докажите, что если $\alpha_n$, $\beta_n$
    статистически неразличимы, то они и вычислительно неразличимы. 
\end{task}


\breakline

\begin{ptask}{12}
    Докажите, что функция $f(xy) = prime(x) + prime(y)$ не является односторонней,
    где $x$, $y$~--- бинарные строки длины $n$, а $prime(x)$~--- минимальное простое
    число, большее $x$.
\end{ptask}

\begin{ptask}{14}
    Зафиксируем язык $L$. Для каждого алгоритма найдется такое семейство
    распределений, что алгоритм работает либо не верно на этом языке, либо алгоритм
    работает не полиномиальное время. Докажите, что существует такое семейство
    распределений $D$, что $(L, D) \notin AvgP$.
\end{ptask}

\begin{ptask}{20}
    Докажите, что если $(BH, U^{BH}) \in Avg_{\frac{1}{n}}P$, то $(NP, U) \in AvgP$.
\end{ptask}

\begin{ptask}{21}
    Докажите, что если $(NP, PISamp) \subseteq Heur_{\frac{1}{n}}P$, то $(NP, PISamp)
    \subseteq HeurP$.
\end{ptask}

\begin{ptask}{22}
    Пусть $EXP \neq NEXP$, докажите что существует задача поиска $(L, D) \in
    \widetilde{(NP, PSamp)}$,
    которая детерминированно не сводится к задаче из класса $\widetilde{(NP, U)}$.
\end{ptask}
