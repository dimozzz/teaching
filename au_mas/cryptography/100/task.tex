\setcounter{curtask}{1}

\mytitle{}

\begin{task}
    Полиномиально вычислимый предикат $P$ называется универсальным трудным битом,
    если он является трудным битом для всех односторонних функции. Докажите, что если
    существуют односторонние функции, то существует НЕуниверсальный трудный бит.
\end{task}

\begin{task}
    Пусть существуют два полиномиально-вычислимых семейства распределений, которые
    вычислительно неразличимы, но различимы статистически. Докажите, что существуют
    односторонние функции.
\end{task}

\begin{task}
    а) Пусть существует односторонняя функция, докажите, что существует не инъективный
    псевдослучайный генератор.

    б) Пусть существует односторонняя функция, докажите, что существует инъективный
    псевдослучайный генератор.
\end{task}

\begin{task}
    Пусть $G$~--- псевдослучайный генератор, а $h$~--- полиномиально вычислимая
    функция. Докажите, что $G(h)$ и $h(G)$ являются псевдослучайными генераторами.
\end{task}

\begin{task}
    Пусть $\alpha_n$~--- случайная величина со значениями в $\{0, 1\}^{p(n)}$, где
    $p$~--- полином. Говорят, что $\alpha_n$ псевдослучайна по Яо, если для всех $0
    \le i \le p(n) - 1$ бит $(\alpha_n)_{i + 1}$ нельзя предсказать по префиксу
    $(\alpha_n)_{\le i}$ (т.е. для любого полиномиально вероятностного алгоритма $A$
    и для любого полинома $q$ $\Pr[A((\alpha_n)_{\le i}) = (\alpha_n)_{i + 1}] \le
    \frac{1}{2} + \frac{1}{q(n)}$).

    Докажите, что $\alpha_n$ псевдослучайная по Яо величина, вычислительно неотличима
    от $U_{p(n)}$.
\end{task}

\begin{task}
    См. предыдущую задачу.

    Докажите, что если случайная величина вычислительно неотличима от $U_{p(n)}$, то
    она псевдослучайна по Яо.
\end{task}



\begin{task}
    Полиномиально вычислимая функция $f: \{0, 1\}^* \to  \{0, 1\}^*$ называется
    распределенной односторонней, если существует такой полином $p$ и для всех
    полиномиальных по времени вероятностных алгоритмов $A$ и для всех достаточно
    больших $n$, статистическое расстояние между распределениями $(U_n, f(U_n))$ и
    $(A(1^n, f(U_n)), U_n)$ больше, чем $1 / p(n)$.
    
    Докажите, что а) если $f$ слабая односторонняя, то она распределенная
    односторонняя б) Докажите, что если существуют распределенные односторонние
    функции, то существуют и односторонние.
\end{task}

\begin{task}
    Пусть $A_n, B_n$~--- семейство распределений. Верно ли, что если для любого
    полиномиального семейства схем $C_n: \{0, 1\}^{n} \to \{0, 1\}$ и любого полинома
    $p$ выполнено свойство:
    $|\Pr\limits_{x \gets A_n}[C_n(x) = 1] - \Pr\limits_{x \gets B_n}[C_n(x) = 1]|
    \le \frac{1}{p(n)}$, то и для любого полиномиального семейства схем
    $C_n': \{0, 1\}^{n} \to \{0, 1\}^{n}$,
    $\forall a |\Pr\limits_{x \gets A_n}[C_n'(x) = a] - \Pr\limits_{x \gets B_n}[C_n'(x) = a]|
    \le \frac{1}{p(n)}$?
\end{task}

\begin{task}
    Пусть $f$~--- сильно односторонняя функция. Докажите, что для любого
    вероятностного полиномиального по времени алгоритма и любого положительного
    полинома $p$, множество $B = \{x \mid \Pr[A(f(x)) \in f^{-1}(f(x))] \ge
    \frac{1}{p(|x|)} \}$ имеет пренебрежимо малую плотность (т.е. последовательность
    $\frac{|B \cap \{0, 1\}^n|}{2^n}$ пренебрежимо мала).
\end{task}

\begin{task}
    Докажите, что если существует протокол привязки к биту, то существуют и
    односторонние функции.
\end{task}

\begin{task}
    Покажите, что если существуют односторонние функции, то существует и такая
    односторонняя, что ни один из битов входа не является трудным битом для нее.
\end{task}

\begin{task}
    Докажите, что если существует схема кодирования с открытым ключом для
    однобитового сообщения, то существует и схема кодирования с открытым ключем для
    произвольных сообщений полиномиальной длины.
\end{task}

\begin{task}
    Пусть $G$~--- псевдослучайный генератор. Докажите, что $G^{(p(n))}$ (многократная
    композиция)~--- псевдослучайный генератор.
\end{task}

\begin{task}
    Докажите, что $Avg_{\frac{1}{n^{100}}}P \notin AvgP$.
\end{task}

\begin{task}
    Пусть $NP \neq coNP$, докажите, что существует такой язык $L$, что $(L, U) \in
    AvgP$, но $L \notin NP$.
\end{task}

\begin{task}
    Докажите, что если $(BH, U^{BH}) \in Avg_{\frac{1}{n}}P$, то $(NP, U) \in AvgP$.
\end{task}

\begin{task}
    Докажите, что если $(NP, PISamp) \subseteq Heur_{\frac{1}{n}}P$, то $(NP, PISamp)
    \subseteq HeurP$.
\end{task}

