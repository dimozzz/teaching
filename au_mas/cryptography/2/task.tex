\setcounter{curtask}{9}

\mytitle{2 (на 26.02)}


\begin{task}
    Придумайте полиномиальный от $n$ алгоритм, который в интервале $[a, b]$, где $a,
    b$~--- двоично рациональные числа из не более, чем $n$ цифр, находит наибольший
    стандартный интервал (вида $[0.r0, 0.r1]$).
\end{task}

\begin{task}
    Семейство распределений $D_n$ называется вычислимым за полиномиальное время, если
    функция распределения ($F_{D_n}(x) = \sum\limits_{y < x}D_n(x)$, где $<$~---
    лексикографический порядок) вычислима за полиномиальное время. Пусть $S$~--- это
    сэмплер для некоторого распределения. Если $S$, используя случайную строку $r$,
    выдает $x$, то будем считать, что сэмплер отображает множество строк $[0.r0,
    0.r1)$ в $x$. Семейство распределений $D_n$ называется обратимым, если сэмплер
    для этого семейства распределений обладает следующим свойством: прообраз каждой
    строки~--- это отрезок, концы которого вычисляются за полиномиальное от $|x|$
    время. Докажите, что вычислимое распределение является: а) полиномиально
    моделируемым б) обратимым полиномиально моделируемым.
\end{task}

\begin{task}
    Докажите, что в определении $HeurBPP$ условие:
    $\Pr\limits_{x \gets D_n}[\Pr\limits_{r}[A(x, \delta, 1^n) \neq L(x)] >
    \frac{1}{4}] < \delta$ можно заменить на:
    $\Pr\limits_{x \gets D_n, r}[A(x, \delta, 1^n) \neq L(x)] <
    \delta$
\end{task}

\begin{task}
    Докажите, что функция $f(xy) = prime(x) + prime(y)$ не является односторонней,
    где $x$, $y$~--- бинарные строки длины $n$, а $prime(x)$~--- минимальное простое
    число, большее $x$.
\end{task}

\begin{task}
    Пусть $cyc_{f}(x)$~--- это минимальное $n$, что $f^{(n)}(x) = x$. Докажите, что
    среднее значение $cyc_{f}$ на строчках длины $n$ не может быть ограничена
    полиномом для слабой односторонней функции.
\end{task}

\breakline

\begin{ptask}{3}
    Докажите, что а) сильная б) слабая односторонняя функция не может иметь
    полиномиально ограниченный образ.
\end{ptask}