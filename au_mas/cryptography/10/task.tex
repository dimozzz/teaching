\setcounter{curtask}{36}

\mytitle{10 (на 15.05)}

\begin{task}
    Пусть существуют односторонние функции. Докажите, что существует несмещенный
    трудный бит (т.е. такой трудный бит $h$, что $\Pr\limits_{x \gets U_n}[h(x) = 1]
    = \frac{1}{2}$). (Подсказка: используйте явную конструкцию). 
\end{task}

\breakline

\begin{ptask}{12}
    Докажите, что функция $f(xy) = prime(x) + prime(y)$ не является односторонней,
    где $x$, $y$~--- бинарные строки длины $n$, а $prime(x)$~--- минимальное простое
    число, большее $x$.
\end{ptask}

\begin{ptask}{28}
    Докажите, что если существует односторонняя функция, то существует и
    неуниверсальная односторонняя функция.
\end{ptask}

\begin{ptask}{31}
    Пусть $G$~--- это $2n$-генератор. $G_0$ и $G_1$~--- это первая и вторая половины
    выхода. Пусть $s \in \{0, 1\}^n$, определим $f_s(x) =
    G_{s_1}(G_{s_2}(\dots))$. Является ли $f_s$ семейством псевдослучайных функций?
\end{ptask}


\begin{ptask}{33}
    Пусть существует протокол привязки в биту. Пусть Мерлин и Артур (Артур~---
    полиномиальный алгоритм) знают граф $G$, и Мерлин хочет убедить Артура в том, что
    данный граф раскрашивается в $3$ цвета. Покажите, что существует такой протокол,
    что с вероятностью $1 - \frac{1}{n}$ Боб может узнать является ли граф $G$ $3$
    раскрашиваемым, но после выполнения данного протокола Боб не сможет узнать
    раскраску графа с вероятностью $1 - \frac{1}{n}$.
\end{ptask}

\begin{ptask}{34}
    Докажите, что если существует протокол привязки к биту, то существуют и
    односторонние функции.
\end{ptask}

\begin{ptask}{35}
    Пусть $(G, E, D)$~--- протокол с секретным ключом. Противник интересуется
    результатом функции $f: \{0, 1\}^* \to \{0, 1\}^*$. Пусть противник применяет к
    шифру семейство схем полиномиального размера $C_n$. Докажите, что существует
    такое семейство схем полиномиального размера $F_n$, что для любой
    последовательности сообщений $x_n$ вероятность $C_n(E(d_n, x_n)) = f(x_n)$
    примерно равно вероятности того, что $F_n(r_n) = f(x_n)$, где случайная величина
    $r_n$ распределена равномерно.
\end{ptask}