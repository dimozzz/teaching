\setcounter{curtask}{1}

\begin{center}
    \textbf{Сложные задачи.}
\end{center}

Данные задачи не являются обязательными. Также они не будут разбираться на
занятиях. Каждая задача оценивается в $3$ обычных.

\begin{task}
    Является ли перечислимым множество всех программ, вычисляющим
    сюръективные функции? А его дополнение?
\end{task}

\begin{task}
    Пусть $f, g$~--- вычислимые всюду определенные функции. Докажите, что найдутся
	такие номера машин Тьюринга $n$ и $m$, что алгоритм с номером $f(n)$ ведется себя
    так же, как и алгоритм с номером $m$, а алгоритм с номером $g(m)$ так же, как и
    алгоритм с номером $n$.
\end{task}

\begin{task}
    Докажите, что существует счетное число не пересекающихся перечислимых множеств,
    никакие два из которых нельзя отделить разрешимым.
\end{task}

