\setcounter{curtask}{25}

\mytitle{5 (на 22.10)}

\begin{task}
    Пусть $L_1, L_2 \in NP$. Принадлежит ли объединение этих языков $NP$? А пересечение?
\end{task}

\begin{task}
    Хорновской формулой называется формула в КНФ, в которой в каждый дизъюнкт
    максимум одна переменная входит без отрицания. Покажите, что множество
    хорновских выполнимых формул содержится в классе $P$.
\end{task}

\begin{task}
    По формуле в $2$-КНФ построим ориентированный граф. Вершинами графа будет
    множество переменных и отрицаний переменных. Для каждого дизъюнкта $(l_1 \lor
    l_2)$ проведем в графе два ребра: из $\neg l_1$ в $l_2$ и из $\neg l_2$ в
    $l_1$. Докажите, что формула выполнима тогда и только тогда, когда для каждой
    переменной $x$ вершины $x$ и $\neg x$ находятся в разных компонентах сильной
    связности. Покажите, что язык формул в 2-КНФ принадлежит классу $P$.
\end{task}


\begin{task}
    Докажите, что язык планарных графов лежит в $P$.
\end{task}

\begin{task}
	Используя полноту языка $3-SAT$, докажите, что язык формул, где каждый дизъюнкт либо
    хорновский, либо состоит из двух литералов, $NP$-полный. (подсказка: попробуйте
    записать каждый дизъюнкт в виде нескольких, указанных в задаче).
\end{task}



\breakline


\begin{ptask}{23}
    Докажите, что если множество $A$ задается предикатом из класса $\Sigma_n$, то
    множество $A \times A$ также задается предикатом из класса $\Sigma_n$.
\end{ptask}

\begin{ptask}{24}
    Докажите, что дизъюнктное объединение двух множеств, задаваемых предикатом из
    класса $\Sigma_n$, задается предикатом из класса $\Sigma_{n + 1}$, а также
    предикатом из класса $\Pi_{n + 1}$.
\end{ptask}
