\setcounter{curtask}{16}

\mytitle{3 (на 1.10)}

\begin{task}
    Докажите, что существует пара таких различных программ $A, B$, что $A$ печатает
    текст $B$, а $B$ печатает текст $A$.
\end{task}

\begin{task}
    Докажите, что: {\it a)} существует универсальное перечислимое множество ---
    $U \subseteq \mathbb{N} \times \mathbb{N}$, что для любого
    перечислимого $A$ найдется такое $a \in \mathbb{N}$, что $A = \{x
    \mid (a, x) \in U\}$
    {\it b)} не существует универсального вычислимого множества
\end{task}


\begin{task}
    Пусть $U(n, x)$ главная нумерация. Рассмотрим функцию $f(n) = U(n, n)~mod~2$.
    
    а) Является ли $f$ вычислимой? А всюду определенной?

    б) Существует ли вычислимое всюду определенное продолжение функции $f$
    (т.е. такая функция $g$, которая совпадает с $f$ на области определения $f$)
    (подсказка: используйте метод диагонализации)?

    в) Докажите, что существуют перечислимые множества $A, B$, которые не могут быть
    отделены разрешимым множеством, т.е не существует такого разрешимого множества
    $C$, что $A \subseteq C$ и $B \cap C = \emptyset$
\end{task}


\begin{task}
    Покажите, что следующие предикаты являются арифметическими:

    а) $x$ взаимно просто с $y$

    б) $z$ НОК $x$ и $y$

    в) $z$ наибольший простой общий делитель $x$ и $y$
\end{task}

\begin{task}
    Покажите, что следующий предикат является арифметическим: $x, y, z$~--- члены
    геометрической прогрессии с простым знаменателем. 
\end{task}




\breakline


\begin{ptask}{6}
    Является ли множество всюду останавливающихся алгоритмов перечислимым? А его
    дополнение?
\end{ptask}

\begin{ptask}{10}
    Существует ли алгоритм, проверяющий, работает ли данная программа
    полиномиальное время? (т.е. на каждом входе алгоритм делает не более $p(|x|)$
    шагов, где $p$~--- полином, а $x$~--- вход алгоритма).
\end{ptask}

\begin{ptask}{15}
    Пусть $S$~--- разрешимое множество натуральных чисел. Разложим все
    числа из $S$ на простые множители, из данных простых составим
    множество $D$. Верно ли что $D$ разрешимо?
\end{ptask}