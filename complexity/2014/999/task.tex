\setcounter{curtask}{1}

\begin{center}
    \textbf{Часть I}
\end{center}

\begin{task}
    Докажите, что существует пара разлиных программ $A, B$ такая, что $A$ печатает текст $B$, а $B$ печатает текст $A$.
\end{task}

\begin{task} (простые множества Поста)
    Назовем множество {\it иммунным}, если оно бесконечно, но не
    содержит бесконечных перечислимых подмножеств. Перечислимое
    множество называется {\it простым}, если его дополнение иммунно.
    Докажите, что простые множества существуют.
\end{task}

\begin{task}
    Предъявите такое число $x \in \mathbb{R}$, что множество рациональных чисел меньших $x$~--- не является перечислимым.
\end{task}

\begin{task}
    Докажите, что существуют перечислимые множества $A, B$, которые не
    могут быть отделены разрешимым множеством, т.е не существует
    такого разрешимого множества $C$, что $A \subseteq C$ и $B \cap C
    = \emptyset$
\end{task}

\begin{task}
    Докажите, что не существует универсального вычислимого множества.
\end{task}

\begin{task}
    Является ли разрешимым множество программ, которые считают линейную по каждому агрументу функцию от своих переменных.
\end{task}




\begin{center}
    \textbf{Часть II}
\end{center}


\begin{task}
    Докажите включение $NP^{BPP}^{BPP} \subseteq \Sigma_3$
\end{task}

\begin{task}
    Докажите, что $P/poly \nsubseteq EXP$
\end{task}

\begin{task}
    Докажите следующие включения: $P \subseteq BPP \subseteq PH \subseteq EXP$.
\end{task}

\begin{task}
    Докажите, что существует язык, для которого любой алгоритм, работающий время
    $O(n^{\log(n)})$ решает его неправильно на всех входах какой-то длины, но
    этот язык распознается алгоритмом, работающим время $O(n^{2 \log(n)})$.
\end{task}

\begin{task}
    Докажите, что $BPL \subseteq P$ ($BPL$~--- класс языков, решаемых с ограниченной
    ошибкой на полиномиальной памяти).
\end{task}

\begin{task}
    Пусть $NP \subseteq DTime[n^{\log(n)}]$, докажите, что $PH \subseteq \bigcup\limits_{k}DTime[n^{\log^k(n)}]$.
\end{task}

\begin{task}
    Докажите, что язык $L = \{(\phi, 1^k) \mid$ функция, заданная формулой $\phi$, не может быть посчитана формулой размера $k$
    $\}$ лежит в $PH$.
\end{task}