\setcounter{curtask}{44}

\mytitle{8 (на 12.11)}

\begin{task}
    Приведите пример разрешимого языка из $P/poly \setminus P$.
\end{task}

\begin{task}
    Пусть $NP \subseteq DTime[n^{\log(n)}]$, докажите, что $PH \subseteq \bigcup\limits_{k}DTime[n^{\log^k(n)}]$
\end{task}

\begin{task}
    а) Сколько существует булевых функций от $n$ переменных?
    б) Сколько существует булевых схем от $n$ переменных размера $s$?
    в) Докажите, что существует булева функция от $n$ переменных, для подсчета
    которой необходима схема размером не менее $\frac{2^n}{100n}$
\end{task}


\begin{task}
    Докажите, что существует такая линейная функция $f: \{0, 1\}^{n}
    \rightarrow \{0, 1\}^n$, что ее схемная сложность не менее
    $\frac{n^2}{100 \log(n)}$.  
\end{task}

\begin{task}
    Пусть $P = NP$, докажите, что $P \notin SIZE[n^k]$, где $k$~--- фиксированная константа.
\end{task}

\begin{task}
    Докажите, что $NEXP^{NEXP^{NEXP^{NEXP}}} \notin SIZE[\frac{2^n}{100n}]$ (подсказка: см. доказательство теоремы Каннана).
\end{task}


\breakline


\begin{ptask}{35} (Алгоритм из разбора не имеет очевидной реализации за $O(\log(n))$ памяти)
    Докажите, что язык графов с циклом разрешим алгоритмом, который использует
    $O(\log(n))$ памяти, где $n$~--- число вершин графа.
\end{ptask}


\begin{task}
    Докажите, что существует язык, для которого любой алгоритм, работающий время
    $O(n^2)$ решает его правильно ровно на половине входов какой-то длины, но
    этот язык распознается алгоритмом, работающим время $O(n^3)$.
\end{task}

\begin{task}
    Докажите, что $PH \subseteq PSpace$.
\end{task}

\begin{task}
    Пусть функции $f, g: \{0, 1\}^* \rightarrow \{0, 1\}^*$ можно посчитать с
    использованием $O(\log(n))$ памяти (напомним, что память считается только на
    рабочих лентах, входная лента доступна только для чтения, а по выходной ленте
    головка машины Тьюринга движется только слева направо). Докажите, что функцию
    $f(g(x))$ можно также посчитать с использованием $O(\log(n))$ памяти.
\end{task}

\begin{task}
	Докажите, что если все унарные языки из $NP$ лежат в $P$, то $EXP = NEXP$.
\end{task}
