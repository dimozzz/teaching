\setcounter{curtask}{50}

\mytitle{9 (на 19.11)}

\begin{task}
    Приведите пример разрешимого языка из $P/poly$, но не лежащего в классе $BPP$.
\end{task}

\begin{task}
    Пусть $NP \subseteq BPP$. Докажите, что $NP = RP$.
\end{task}

\begin{task}
    Докажите, что $BPP = P^{BPP}$.
\end{task}


Пусть $ZPP$~--- это класс языков, которые принимаются вероятностно машиной Тьюринга
без ошибки, математическое ожидание времени работы, которых полиномиально.

\begin{task}
	Докажите, что $L \in ZPP$ тогда и только тогда, когда существует полиномиальная
    по времени вероятностная машина Тьюринга, которая выдает $\{0, 1, ?\}$
\end{task}



\breakline

\begin{task}
    Докажите, что $PH \subseteq PSpace$.
\end{task}

\begin{task}
    Пусть функции $f, g: \{0, 1\}^* \rightarrow \{0, 1\}^*$ можно посчитать с
    использованием $O(\log(n))$ памяти (напомним, что память считается только на
    рабочих лентах, входная лента доступна только для чтения, а по выходной ленте
    головка машины Тьюринга движется только слева направо). Докажите, что функцию
    $f(g(x))$ можно также посчитать с использованием $O(\log(n))$ памяти.
\end{task}

\begin{task}
	Докажите, что если все унарные языки из $NP$ лежат в $P$, то $EXP = NEXP$.
\end{task}

\begin{ptask}{45}
    Пусть $NP \subseteq DTime[n^{\log(n)}]$, докажите, что $PH \subseteq \bigcup\limits_{k}DTime[n^{\log^k(n)}]$
\end{ptask}


\begin{task}{48}
    Пусть $P = NP$, докажите, что $P \notin SIZE[n^k]$, где $k$~--- фиксированная константа.
\end{task}

\begin{task}{49}
    Докажите, что $NEXP^{NEXP^{NEXP^{NEXP}}} \notin SIZE[\frac{2^n}{100n}]$ (подсказка: см. доказательство теоремы Каннана).
\end{task}
