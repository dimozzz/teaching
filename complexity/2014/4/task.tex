\setcounter{curtask}{21}

\mytitle{4 (на 7.10)}


\begin{task}
    Докажите, что объединение и пересечение двух множеств, задаваемых предикатом из
    класса $\Sigma_n$, также задается предикатом из класса $\Sigma_n$.
\end{task}

\begin{task}
    Докажите, что если множество $A$ задается предикатом из класса $\Sigma_n$, то
    множество $A \times A$ также задается предикатом из класса $\Sigma_n$.
\end{task}

\begin{task}
    Докажите, что дизъюнктное объединение двух множеств, задаваемых предикатом из
    класса $\Sigma_n$, задается предикатом из класса $\Sigma_{n + 1}$, а также
    предикатом из класса $\Pi_{n + 1}$.
\end{task}

\begin{task} (простые множества Поста)
    Назовем множество {\it иммунным}, если оно бесконечно, но не
    содержит бесконечных перечислимых подмножеств. Перечислимое
    множество называется {\it простым}, если его дополнение иммунно.
    Докажите, что простые множества существуют. (подсказка: используйте
    диагонализацию).
\end{task}




\breakline

\begin{ptask}{16}
    Докажите, что существует пара таких различных программ $A, B$, что $A$ печатает
    текст $B$, а $B$ печатает текст $A$.
\end{ptask}

\begin{ptask}{17}
    Докажите, что: {\it a)} существует универсальное перечислимое множество ---
    $U \subseteq \mathbb{N} \times \mathbb{N}$, что для любого
    перечислимого $A$ найдется такое $a \in \mathbb{N}$, что $A = \{x
    \mid (a, x) \in U\}$
    {\it b)} не существует универсального вычислимого множества
\end{ptask}


\begin{ptask}{18}
    Пусть $U(n, x)$ главная нумерация. Рассмотрим функцию $f(n) = U(n, n)~mod~2$.
    
    а) Является ли $f$ вычислимой? А всюду определенной?

    б) Существует ли вычислимое всюду определенное продолжение функции $f$
    (т.е. такая функция $g$, которая совпадает с $f$ на области определения $f$)
    (подсказка: используйте метод диагонализации)?

    в) Докажите, что существуют перечислимые множества $A, B$, которые не могут быть
    отделены разрешимым множеством, т.е не существует такого разрешимого множества
    $C$, что $A \subseteq C$ и $B \cap C = \emptyset$
\end{ptask}


