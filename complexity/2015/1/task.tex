\setcounter{curtask}{1}

\mytitle{1 (на 12.02)}

\begin{task}
    Рассмотрим функцию $Maj: \{0, 1\}^n \rightarrow \{0, 1\}$, которая выдает $1$, если не менее половины входных битов равны
    $1$. Докажите, что существует: а) схема б) монотонная схема в) монотонная формула полиномиального размера, вычисляющая функцию
    $Maj$.
\end{task}

\begin{task}
    Докажите, что для любой симметрической булевой функции (симметрическая функция зависит только от числа единиц во входе)
    существует вычисляющая ее а) схема б) монотонная схема полиномиального размера.
\end{task}

\begin{task}
    Докажите, что любая формула в КНФ (ДНФ), которая вычисляет функцию а) $x_1 + x_2 + x_3 + \dots + x_n ~mod~ 2$; б) $Maj(x_1,
    \dots, x_n)$ имеет экспоненциальный размер.
\end{task}

\begin{task}
    Докажите, что существует формула от $\land, \lor, \neg$ размера $O(n^2)$, которая вычисляет функцию $x_1 + x_2 + x_3 + \dots
    + x_n ~mod~ 2$.
\end{task}

\begin{task}
    Докажите, что функция $Maj$ не может быть вычислена при помощи схем полиномиального размера константной глубины из гейтов
    $\land, \lor, \neg$.
\end{task}