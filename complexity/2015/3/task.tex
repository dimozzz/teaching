\setcounter{curtask}{9}

\mytitle{3 (на 24.03)}

\begin{task}
    Докажите, что у любой формулы размера $s$ существует эквивалентная формула глубины $O(\log(s))$.
\end{task}

\begin{task}
    Какие значения может принимать глубина дерева решений (decision tree) для функции $f: \{0, 1\}^n \rightarrow \{0, 1\}$, где
    все аргументы не являются фиктивными (т.е. для каждого номера $i$ найдется вход $x$, что $f(x) \neq f(x^{i})$).
\end{task}

\begin{task}
    Пусть $n = k^2$. Рассмотрим функцию $f: {0, 1}^n \rightarrow \{0, 1\}$, заданную следующим образом: вход разделен на блоки по
    $k$ битов, функция равно $1$ тогда и только тогда, когда существует блок в котором два последовательных бита равны единице, а
    остальные биты равны нулю. Оцените $s(f), bs(f), C(f), D(f)$.
\end{task}

\begin{task}
    Рассмотрим функцию $f = \bigvee\limits_{i = 1}^{n} x_i$. Докажите, что $R(f) = n$.
\end{task}

\begin{task}
    Докажите, что $\mathrm{PCP}(0, \log(n)) = \mathrm{P}$.
\end{task}

\begin{task}
    Докажите, что если $SAT \in \mathrm{PCP}(o(\log(n)), 1)$, то $P = NP$.
\end{task}


