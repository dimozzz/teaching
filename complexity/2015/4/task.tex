\setcounter{curtask}{15}

\mytitle{3 (на 31.03). Коммуникационная сложность.}


Алиса и Боб хотят совместно вычислить значение функции $f: X \times Y \rightarrow Z$, причем Алиса знает $x \in X$, а Боб знает $y
\in Y$. коммуникационным протоколом называется бинарное дерево, листья которого помечены элементами $Z$, а в остальных вершинах
стоят функции $a_i: X \rightarrow \{0, 1\}$ или $b_j: Y \rightarrow \{0, 1\}$. Вычисление по такому протоколу состоит в следующем:
начиная от корня движемся по дереву к тому потомку, куда указывает функция, стоящая в вершине. Например, если в вершине стоит
функция $a_i$ и ее значение равно $0$, то переходим к левому ребенку, иначе к правому. В листе, к которому мы приходим должно
стоять значение $f(x, y)$. За $C_L(f)$ будем обозначать минимальное число листьев в протоколе для функции $f$, а за $C(f)$~---
минимальную глубину.


\begin{task}
    Докажите, что $C(f) = O(\log(C_L(f)))$.
\end{task}

\begin{task}
    Каждая функция $f: X \times Y \rightarrow Z$ задает раскраску элементов матрицы $M[X, Y]$ в цвета из множества
    $Z$. Прямоугольником называется множество $X' \times Y'$, где $X' \subseteq X$ и $Y' \subseteq Y$. Прямоугольник называется
    одноцветным если все элементы $M[X', Y']$ покрашены в один цвет. Пусть $R(M)$~--- минимальное число непересекающихся одноцветных
    прямоугольников, которыми можно покрыть все элементы $M$.
    а) Докажите, что $C_L(f) \ge R(M)$. б) Докажите, что $R(M) \ge rk(M)$, если $Z$~--- некоторое поле.
    в) Докажите, что коммуникационная сложность функции $GT: \{0, 1\}^n \times \{0, 1\}^n \rightarrow \{0, 1\}$, которая равна $1$
    тогда и только тогда, когда $x > y$ (как натуральные числа в двоичной записи), не менее $n$.
\end{task}

\begin{task}
    Пусть у Алисы и Боба есть множества $X, Y \subseteq \{1, \dots, n\}$. Они хотят посчитать функцию $MED(X, Y)$, которая
    возвращает медиану мультимножества $X \cup Y$. Докажите, что для этого им достаточно: а) $O(n)$ б) $O(\log^2(n))$ битов коммуникации.
\end{task}



Игры Карчмера-Вигдерсона. Дана функция $f: \{0, 1\}^n \rightarrow \{0, 1\}$. Алиса получает $x \in f^{-1}(0)$, а Боб получает $y
\in f^{-1}(1)$. Им требуется вычислить какую-нибудь координату $i$, что $x_i \neq y_i$.

\begin{task}
    Докажите, что $C(f) = d(f)$ и $C_L(f) = L(f)$, где $d(f)$~--- минимальная глубина формулы, которая вычисляет $f$ в базисе
    $\{\land, \lor, \neg\}$, а $L(f)$~--- соответственно число листьев.
\end{task}

\begin{task}
    Пусть $M[X, X]$~--- $0 / 1$-матрица, которая содержит перестановочную матрицу размера $|X|$ (т.е. ее перманент над
    $\mathbb{R}$ не ноль). а) Докажите, что $R(M) \cdot T(M) \ge |X|^2$, где $T(M)$~--- число единиц в матрице. б) Докажите при
    помощи этой техники, что $L(MOD_2) = \Omega(n^2)$.
\end{task}


\begin{task}
    Пусть $S_t$~--- биномиальное распределение с $t$ сбалансированными монетами. Докажите, что для любого $\delta < 1$,
    $\sum\limits_{i = 0}^{t + \delta \sqrt{t}} |\Pr[S_t = i] - \Pr[S_{t + \delta \sqrt{t}} = i]| \le 20 \delta$.
\end{task}


\breakline

\begin{ptask}{9}
    Докажите, что у любой формулы размера $s$ существует эквивалентная формула глубины $O(\log(s))$.
\end{ptask}

\begin{ptask}{10}
    Какие значения может принимать глубина дерева решений (decision tree) для функции $f: \{0, 1\}^n \rightarrow \{0, 1\}$, где
    все аргументы не являются фиктивными (т.е. для каждого номера $i$ найдется вход $x$, что $f(x) \neq f(x^{i})$).
\end{ptask}

\begin{ptask}{14}
    Докажите, что если $SAT \in \mathrm{PCP}(o(\log(n)), 1)$, то $P = NP$.
\end{ptask}

