\setcounter{curtask}{33}

\mytitle{7 (на 14.11)}

\begin{task}
    Пусть $L_1, L_2 \in NP \cap coNP$. Докажите, что $L_1 \oplus L_2 \in NP \cap coNP$.
\end{task}

\begin{task}
	Докажите:
    а) $P \subseteq NP \cap coNP$
    б) $P = NP \Rightarrow NP = coNP$
\end{task}

\begin{task}
	Докажите, что если все унарные языки из $NP$ лежат в $P$, то $EXP = NEXP$
\end{task}

\begin{task}
    Докажите, что существует язык, для которого любой алгоритм, работающий время
    $O(n^2)$ решает его правильно ровно на половине входов какой-то длины, но
    этот язык распознается алгоритмом, работающим время $O(n^3)$.
\end{task}

\breakline

\begin{ptask}{29}
    а) Докажите, что число $n$ простое тогда и только тогда, когда для каждого
    простого делителя $q$ числа $n - 1$ существует $a \in {2, 3, \dots, n - 1}$ при котором
    $a^{n - 1} = 1~mod~n$, а $a^{\frac{n - 1}{q}} \ne 1~mod~n$
\end{ptask}

\begin{ptask}{30}
    Изменится ли класс $\widetilde{P}$, если из определения убрать условие
    ``$\widetilde{P} \subseteq \widetilde{NP}$''.
\end{ptask}

\begin{ptask}{31}
    Докажите $NP$ полноту следующей задачи:
    на вход подается пара графов $(G_1, G_2)$, необходимо определить, изоморфен ли
    граф $G_2$ подграфу графа $G_1$.
\end{ptask}