\setcounter{curtask}{7}

\mytitle{2 (на 21.02)}

\begin{task}
    Изменится ли класс $\widetilde{P}$, если из определения убрать условие
    ``$\widetilde{P} \subseteq \widetilde{NP}$''.
\end{task}

\begin{task}
    Докажите, что любую $k$-ленточную машину Тьюринга, работающую время $f(n)$ можно
    моделировать на $2$-х ленточной, работающей за время $f^2(n)$.
\end{task}

\begin{task}
	Докажите:
    а) $P \subseteq NP \cap coNP$
    б) $P = NP \Rightarrow NP = coNP$
\end{task}

\begin{task}
    Докажите $NP$ полноту следующей задачи:
    на вход подается пара графов $(G_1, G_2)$, необходимо определить, изоморфен ли
    граф $G_2$ подграфу графа $G_1$.
\end{task}

\breakline

\begin{ptask}{2}
    Определим класс $EXP = \bigcup\limits_{i}DTime[2^{n^i}]$. Докажите, что
    $NP \subseteq EXP$.
\end{ptask}

\begin{ptask}{3}
    Хорновской формулой называется формула в КНФ, в которой в каждый дизъюнкт
    максимум одна переменная входит без отрицания. Покажите, что множество
    хорновских выполнимых формул содержится в классе $P$.
\end{ptask}

\begin{ptask}{4}
    а) Докажите, что число $n$ простое тогда и только тогда, когда для каждого
    простого делителя $q$ числа $n - 1$ существует $a \in {2, 3, \dots, n - 1}$ при котором
    $a^{n - 1} = 1~mod~n$, а $a^{\frac{n - 1}{q}} \ne 1~mod~n$
    б) докажите, что язык простых чисел лежит в $NP$.
\end{ptask}