\setcounter{curtask}{23}

\mytitle{5 (на 20.03)}

\begin{task}
    Докажите, что существует язык, для которого любой алгоритм, работающий время
    $O(n^2)$ решает его правильно на менее, чем половине входов какой-то длины, но
    этот язык распознается алгоритмом, работающим время $O(n^3)$.
\end{task}

\begin{task}
    Докажите, что $DSpace[n] \ne NP$.
\end{task}

\begin{task}
    Докажите, что если язык $A$ сводится по Тьюрингу к языку $B \in \Sigma_{i}$, то
    $A \in \Sigma_{i + 1}$.
\end{task}

\breakline

\begin{ptask}{4}
    а) Докажите, что число $n$ простое тогда и только тогда, когда для каждого
    простого делителя $q$ числа $n - 1$ существует $a \in {2, 3, \dots, n - 1}$ при котором
    $a^{n - 1} = 1~mod~n$, а $a^{\frac{n - 1}{q}} \ne 1~mod~n$
    б) докажите, что язык простых чисел лежит в $NP$.
\end{ptask}

\begin{ptask}{8}
    Докажите, что любую $k$-ленточную машину Тьюринга, работающую время $f(n)$ можно
    моделировать на $2$-х ленточной, работающей за время $f^2(n)$.
\end{ptask}


\begin{ptask}{12}
	Докажите, что если существует унарный $NP$-полный язык, то $P = NP$.
\end{ptask}

\begin{ptask}{13}
    Докажите, что существует оракул $A$ и такой язык $L \in NP^A$, что $L$ не
    сводится за полиномиальное время к $3-SAT$ даже если машине выполняющей сведение
    разрешить пользоваться $A$.
\end{ptask}

\begin{ptask}{17}
    Докажите, что для любой функции $f(n) = \Omega(\log(n))$ выполняется:
    $NSpace[f(n)] \subseteq DSpace[f^2(n)]$
\end{ptask}

\begin{ptask}{18}
    Докажите, что существует функция $f(n)$ схемной сложности не менее $\frac{2^n}{100n}$
\end{ptask}

\begin{ptask}{19}
    Пусть $P = NP$, докажите, что существует такой язык $L \in EXP$, что
    $L \notin SIZE[\frac{2^n}{100n}]$. (Подсказка: использовать теорему $PH \notin SIZE[n^k]$)
\end{ptask}

\begin{ptask}{20}
    Приведите пример:
    а) неразрешимого языка из $P/poly$
    б) разрешимого языка из $P/poly \setminus P$
\end{ptask}

\begin{ptask}{21}
	Опишите $NC$ схему для:
    а) умножения двух матриц над конечным полем
    б) возведения матрицы с степень $n$ над конечным полем.
\end{ptask}