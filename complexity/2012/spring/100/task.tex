\setcounter{curtask}{1}

\mytitle{}

\begin{task}
    Докажите включение $NP^{BPP} \subseteq \Sigma_3$
\end{task}

\begin{task}
    Докажите, что $P/poly \nsubseteq EXP$
\end{task}

\begin{task}
    Докажите, что существует язык, для которого любой алгоритм, работающий время
    $O(n^2)$ решает его правильно ровно на половине входов какой-то длины, но
    этот язык распознается алгоритмом, работающим время $O(n^3)$.
\end{task}

\begin{task}
    Докажите, что задача умножения двух полиномов одинаковой степени над конечным
    полем лежит в $NC$.
\end{task}

\breakline

\begin{task}
    Докажите, что $BPL \subseteq P$ ($BPL$~--- класс языков, решаемых с ограниченной
    ошибкой на полиномиальной памяти).
\end{task}

\begin{task}
    Пусть $P^{A} = NP^{A}$. Докажите, что $PH^{A} = P^{A}$.
\end{task}

\begin{task}
    Докажите, что $NP \ne DSpace[n]$
\end{task}