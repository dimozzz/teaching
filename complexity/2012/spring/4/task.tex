\setcounter{curtask}{17}

\mytitle{4 (на 13.03)}

\begin{task}
    Докажите, что для любой функции $f(n) = \Omega(\log(n))$ выполняется:
    $NSpace[f(n)] \subseteq DSpace[f^2(n)]$
\end{task}

\begin{task}
    Докажите, что существует функция $f(n)$ схемной сложности не менее $\frac{2^n}{100n}$
\end{task}

\begin{task}
    Пусть $P = NP$, докажите, что существует такой язык $L \in EXP$, что
    $L \notin SIZE[\frac{2^n}{100n}]$. (Подсказка: использовать теорему $PH \notin SIZE[n^k]$)
\end{task}

\begin{task}
    Приведите пример:
    а) неразрешимого языка из $P/poly$
    б) разрешимого языка из $P/poly \setminus P$
\end{task}

\begin{task}
	Опишите $NC$ схему для:
    а) умножения двух матриц над конечным полем
    б) возведения матрицы с степень $n$ над конечным полем.
\end{task}

\breakline

\begin{ptask}{4}
    а) Докажите, что число $n$ простое тогда и только тогда, когда для каждого
    простого делителя $q$ числа $n - 1$ существует $a \in {2, 3, \dots, n - 1}$ при котором
    $a^{n - 1} = 1~mod~n$, а $a^{\frac{n - 1}{q}} \ne 1~mod~n$
    б) докажите, что язык простых чисел лежит в $NP$.
\end{ptask}

\begin{ptask}{8}
    Докажите, что любую $k$-ленточную машину Тьюринга, работающую время $f(n)$ можно
    моделировать на $2$-х ленточной, работающей за время $f^2(n)$.
\end{ptask}

\begin{ptask}{10}
    Докажите $NP$ полноту следующей задачи:
    на вход подается пара графов $(G_1, G_2)$, необходимо определить, изоморфен ли
    граф $G_2$ подграфу графа $G_1$.
\end{ptask}

\begin{ptask}{11}
	Докажите, что если все унарные языки из $NP$ лежат в $P$, то $EXP = NEXP$
\end{ptask}

\begin{ptask}{12}
	Докажите, что если существует унарный $NP$-полный язык, то $P = NP$.
\end{ptask}

\begin{ptask}{13}
    Докажите, что существует оракул $A$ и такой язык $L \in NP^A$, что $L$ не
    сводится за полиномиальное время к $3-SAT$ даже если машине выполняющей сведение
    разрешить пользоваться $A$.
\end{ptask}

\begin{ptask}{14}
    Докажите, что если $\Sigma_i = \Pi_i$, то $PH = \Sigma_i$
\end{ptask}

\begin{ptask}{15}
	Докажите, что язык формул, где каждый клоз либо хорновский, либо состоит из двух
    литералов, $NP$-полный.
\end{ptask}