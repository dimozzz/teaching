\setcounter{curtask}{12}

\mytitle{3 (на 18.04)}


\begin{task}
    Опишите графы Кэли для групп: а) $G = \mathbb{Z}_6$, $S = \{3\}$; 
	б) $G = \mathbb{Z}_6, S = \{2, -2\}$; в) $G$~--- группа  
	симметрий квадрата, $S$~--- это множество из двух симметрий: относительно вертикальной оси
	и относительно главной диагонали; г) $G = S_n$, $S = A_n$.
\end{task}

\begin{task}
    Чему равны собственные числа матрицы смежности цикла.
\end{task}

\begin{task}
    Пусть $A$ матрица смежности неогриентированного графа, $B$ диагональная матрица
    $b_{ii}$ равно степени вершины $i$. Докажите, что все алгербраические дополнения
    матрицы $B - A$ одинаковы.
\end{task}


\breakline

\begin{ptask}{5}
    Рассмотрим случайный двудольный граф, в котором вершины разбиты на две доли: $L$
	и $R$ по $n$ вершин в каждой доле. Для каждой вершины левой доли выбирается
	независимо случайным образом $d$ соседей из правой доли (кратные ребра
	разрешены). Докажите, что для всех $\epsilon > \frac{1}{d}$ найдется такое число
	$\alpha>0$, что с большой вероятностью выполняется следующее свойство: для
    каждого множества $S \subseteq L$ размера не больше $\frac{\alpha n}{d}$
    выполняется $|\Gamma(S)| \ge |S|(1 - \epsilon)d$.
\end{ptask}

\begin{ptask}{6}
    Докажите, что хроматическое число алгебраического $(n,d,\alpha)$-экспандера
    больше, чем $\frac{1}{\alpha}$.
\end{ptask}

\begin{ptask}{7}
    Пусть $G$~--- это алгебраический $(n,d,\alpha)$-экспандер. Пусть 
	$k\le \frac{1}{\alpha}$ и $n$ делится на $k$. Докажите, что если покрасить
    вершины  в $k$ цветов так, чтобы каждый цвет использовался ровно $\frac{n}{k}$
	раз, то найдется хотя бы одна вершина, среди соседей которой встречаются все $k$
    цветов.
\end{ptask}


\begin{ptask}{9}(*)
	Пусть $G$~--- это алгебраический $(n,D, 1-\varepsilon)$-экспандер, а $H$~--- это
    алгебраический $(D, d, 1-\delta)$-экспандер. Докажите, что их зиг-заг
    произведение является $(nD, d^2, 1 - \epsilon \delta^2)$-экспандером.
\end{ptask}

\begin{ptask}{10}(*)
	Докажите, что в любом связном недвудольном $d$-регулярном графе второе по модулю
	собственное число не превосходит а) $d - \Theta(1 / n^c)$ для некоторого $c$;
	б) $(d - \Theta(1 / n^2))$.
\end{ptask}

\begin{ptask}{11}
	В вершине связного $d$-регулярного графа сидит пингвин, который в каждый момент
	времени перепрыгивает в случайную соседнюю вершину, пока не окажется в
    фиксированной вершине $v$. Покажите, что среднее число его прыжков полиномиально
    относительно от числа вершин в графе. (Подсказка: см. предыдущую задачу).
\end{ptask}