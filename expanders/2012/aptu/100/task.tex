\setcounter{curtask}{1}

\mytitle{}

\begin{task}
    Пусть $G$~--- $(n, d, \alpha)$-экспандер. Рассмотрим двудольный граф: левая доля
    состоит из ребер $G$, правая доля из вершин $G$. Соединим вершины из левой и
    правой доли, если в исходном графе соответствующие вершина и ребро были
    инцидентны. Докажите, что построеный граф обладает свойством вершинного расширения.
\end{task}

\begin{task}
    Докажите, что для достаточно больших $n$ и $k \le n$ существует двудольный граф
    $(L, R, E)$, у которого левая доля $L$ состоит из $n$ вершин, а правая $R$ из
    $O(k\log(n))$, степень всех вершин с левой доле равна $d = O(\log(n))$, и
    выполнено следующее свойство для всех $A \subseteq L, |A| \le k$:
    $\Gamma(A) \ge \frac{7}{8}d|A|$
\end{task}

\begin{task}(*)(Лемма Динур)
    Пусть $G$~--- $(n, d, \alpha)$-экспандер и $B \subseteq E$. Выберем случайное
    ребро из $B$ и случайно один из его концов. Затем делается $i$ случайные шагов по
    графу. Покажите, что вероятность того, что последнее ребро в данном пути
    принадлежит $B$ не более $\frac{|B|}{|E|} + \alpha^i$.
\end{task}

