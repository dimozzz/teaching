\setcounter{curtask}{15}

\mytitle{4 (на 25.04)}

\begin{task}
    Пуст дан код Рида-Соломона, который исправляет $e$ ошибок. Покажите, что есть
	алгоритм, который исправляет $p$ пропусков и $k$ ошибок, если $p / 2 + k \le e$.
    Пропуск~--- это отсутствие символа, а не искажение его.
\end{task}

\begin{task}(The Hat Problem)
	The Hat Problem involves $n$ people in a room, each of whom is given a
    black/white hat chosen uniformly at random (and independent of the choices of all
    other people). Each person can see the hat color of all other people, but not
    their own. Each person is asked if (s)he wishes to guess their own hat
    color. They can either guess, or abstain. Each person makes their choice without
    knowledge of what the other people are doing. They either win collectively, or
    lose collectively. They win if all the people who don't abstain guess their hat
    color correctly {\em and} at least one person does not abstain. They lose if all
    people abstain, or if some person guesses their color incorrectly.  Your goal
    below is to come up with a strategy that will allow the $n$ people to win, with
    pretty high probability. The problem involves some careful modelling, and some
    knowledge of Hamming codes!

	a) Lets say that a directed graph $G$ is a subgraph of the $n$-dimensional
    hypercube if its vertex set is $\{0, 1\}^n$ and if $u \to v$ is an edge in $G$,
    then $u$ and $v$ differ in at most one coordinate. Let $K(G)$ be the number of
    vertices of $G$ with in-degree at least one, and out-degree zero.  Show that the
    probability of winning the hat problem equals the maximum, over directed
    subgraphs $G$ of the $n$-dimensional hypercube, of $K(G) / 2^n$.

	b) Using the fact that the out-degree of any vertex is at most $n$, show that
	$K(G)/2^n$ is at most $\frac{n}{n+1}$ for any directed subgraph $G$ of the 
	$n$-dimensional hypercube.

	c) Show that if $n = 2^{\ell} - 1$, then there exists a directed subgraph $G$ of
    the $n$-dimensional hypercube with $K(G) / 2^n = \frac{n}{n + 1}$. (This is where
    the Hamming code comes in.)    
\end{task}

\breakline

\begin{ptask}{5}
    Рассмотрим случайный двудольный граф, в котором вершины разбиты на две доли: $L$
	и $R$ по $n$ вершин в каждой доле. Для каждой вершины левой доли выбирается
	независимо случайным образом $d$ соседей из правой доли (кратные ребра
	разрешены). Докажите, что для всех $\epsilon > \frac{1}{d}$ найдется такое число
	$\alpha > 0$, что с большой вероятностью выполняется следующее свойство: для
    каждого множества $S \subseteq L$ размера не больше $\frac{\alpha n}{d}$
    выполняется $|\Gamma(S)| \ge |S|(1 - \epsilon)d$.
\end{ptask}

\begin{ptask}{7}
    Пусть $G$~--- это алгебраический $(n,d,\alpha)$-экспандер. Пусть 
	$k\le \frac{1}{\alpha}$ и $n$ делится на $k$. Докажите, что если покрасить
    вершины  в $k$ цветов так, чтобы каждый цвет использовался ровно $\frac{n}{k}$
	раз, то найдется хотя бы одна вершина, среди соседей которой встречаются все $k$
    цветов.
\end{ptask}


\begin{ptask}{9}(*)
	Пусть $G$~--- это алгебраический $(n,D, 1-\varepsilon)$-экспандер, а $H$~--- это
    алгебраический $(D, d, 1-\delta)$-экспандер. Докажите, что их зиг-заг
    произведение является $(nD, d^2, 1 - \frac{\epsilon \delta}{8})$-экспандером.
\end{ptask}

\begin{ptask}{10}(*)
	Докажите, что в любом связном недвудольном $d$-регулярном графе второе по модулю
	собственное число не превосходит а) $d - \Theta(1 / n^c)$ для некоторого $c$;
	б) $(d - \Theta(1 / n^2))$.
\end{ptask}

\begin{ptask}{11}
	В вершине связного $d$-регулярного графа сидит пингвин, который в каждый момент
	времени перепрыгивает в случайную соседнюю вершину, пока не окажется в
    фиксированной вершине $v$. Покажите, что среднее число его прыжков полиномиально
    относительно от числа вершин в графе. (Подсказка: см. предыдущую задачу).
\end{ptask}

\begin{ptask}{12}
    Опишите графы Кэли для групп: а) $G = \mathbb{Z}_6$, $S = \{3\}$; 
	б) $G = \mathbb{Z}_6, S = \{2, -2\}$; в) $G$~--- группа  
	симметрий квадрата, $S$~--- это множество из двух симметрий: относительно вертикальной оси
	и относительно главной диагонали; г) $G = S_n$, $S = A_n$.
\end{ptask}

\begin{ptask}{14}
    Пусть $A$ матрица смежности неогриентированного графа, $B$ диагональная матрица
    $b_{ii}$ равно степени вершины $i$. Докажите, что все алгербраические дополнения
    матрицы $B - A$ одинаковы.
\end{ptask}