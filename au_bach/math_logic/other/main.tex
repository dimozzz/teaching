\documentclass[a4paper, 12pt]{article}
% math symbols
\usepackage{amssymb}
\usepackage{amsmath}
\usepackage{mathrsfs}
\usepackage{mathseries}
\usepackage{tabularx}
\usepackage{multirow}


\usepackage[margin = 2cm]{geometry}

\tolerance = 1000
\emergencystretch = 0.74cm

\newcolumntype{?}[1]{!{\vrule width #1}}
\newcolumntype{Y}{>{\centering\arraybackslash}X}

\makeatletter
\def\hlinewd#1{%
\noalign{\ifnum0=`}\fi\hrule \@height #1 %
\futurelet\reserved@a\@xhline} 

\renewcommand{\coursetitle}{ML}

\newcommand{\separ}{~\vdash~}


\pagestyle{empty}
\parindent = 0mm
\setlength{\extrarowheight}{20pt}

\begin{document}


	\Huge


    \begin{tabularx}{\textwidth}{?{2pt}Y|Y?{2pt}}
		\hlinewd{2pt}
		Аксиомы & Правило сечения \\
		\hline
		$\frac{}{\Gamma, A \separ A, \Delta}$ & $\frac{\Gamma \separ \Delta, A ~~~ A, \Gamma \separ \Pi}{\Gamma \separ
	        \Delta, \Pi}$ \\ [20pt]  
        \hlinewd{2pt}
        <<Левые>> правила & <<Правые>> правила \\
        \hline
        $\frac{\Gamma, A, B \separ \Delta}{\Gamma, A \land B \separ \Delta}$ & $\frac{\Gamma \separ A, B, \Delta}{\Gamma
            \separ A \lor B, \Delta}$ \\ [20pt]
        \hline
		$\frac{\Gamma, A \separ \Delta ~~~ \Gamma, B \separ \Delta}{\Gamma, A \lor B \separ \Delta}$ & $\frac{\Gamma \separ
            A, \Delta ~~~ \Gamma \separ B, \Delta}{\Gamma \separ A \land B, \Delta}$\\ [20pt]
        \hline
		$\frac{\Gamma \separ A, \Delta ~~~ \Gamma, B \separ \Delta}{\Gamma, A \to B \separ \Delta}$ & $\frac{\Gamma, A
            \separ B, \Delta}{\Gamma \separ  A \to B, \Delta}$\\ [20pt]
        \hline
		$\frac{\Gamma \separ A, \Delta}{\Gamma, \neg A \separ \Delta}$ & $\frac{\Gamma, A \separ \Delta}{\Gamma \separ
            \neg A, \Delta}$ \\ [20pt]
        \hline
        \hlinewd{2pt}
        $\frac{\Gamma, A(t \diagup x), \forall x A \separ \Delta}{\Gamma, \forall x A \separ \Delta}$ &
            $\frac{\Gamma \separ A(y \diagup x), \Delta}{\Gamma \separ \forall x A, \Delta}$)\\ [20pt] 
        \hline
	    $\frac{\Gamma, A(y \diagup x) \separ \Delta}{\Gamma, \exists x A \separ \Delta}$)& 
			$\frac{\Gamma \separ A(t \diagup x), \exists x A, \Delta}{\Gamma \separ \exists x A, \Delta}$\\ [20pt]
        \hlinewd{2pt}
	\end{tabularx}

    \normalsize
    
    \paragraph{Примечания:}
    \begin{enumerate}
        \item $A(t \diagup x)$ обозначает, что в формуле $A$ переменная $x$ заменяется на терм $t$, при
            этом для каждого вхождения переменной $x$ никакие переменные терма $t$ не должны попасть в
            область действия кванторов по одноименным переменным (в формуле $A$). Например, для формулы
            $\forall y~P(x, y)$ вместо $x$ нельзя подставить $f(y)$.
        \item $A(y \diagup x)$ означает, что в формуле $A$ мы заменили все вхождения $x$ на переменную
            $y$, при этом переменная $y$ должна быть свежей то есть не входить ни в $A$, ни в другие
            формулы из секвенции.
    \end{enumerate}
    

\end{document}