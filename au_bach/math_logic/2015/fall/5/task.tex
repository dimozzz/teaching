\setcounter{curtask}{23}

\mytitle{5 (на 06.10)}

\newcommand{\dom}[2]{\left[\frac{#1}{#2}\right]}


\begin{task}
    Используя теорему Клини а) докажите, что существует алгоритм, который на всех входах выводит свой номер;
    б) докажите, что существует алгоритм, который на всех входах выводит квадрат своего номера.
\end{task}

\begin{task}
    Используя теорему Клини а) покажите, что существует алгоритм, который всюду останавливается и выдает $1$ на числе, которое
    является квадратом его номера, а на всех остальных входах выдает ноль; б) докажите, что существуют два различных алгоритма
    $\mathcal{A}$ и $\mathcal{B}$, что алгоритм $\mathcal{A}$ печатает $\sharp \mathcal{B}$, а алгоритм $\mathcal{B}$ печатает
    $\sharp \mathcal{A}$. 
\end{task}

\begin{task}
	Докажите, что для любой вычислимой функции $f$ в любой главной нумерации (главной универсальной функции) $V(n, x)$ существует бесконечное число номеров $n$,
	что для любого $x$ выполнено, что $V(n, x) = f(x)$ (при чем $V(n, x)$ не определенно тогда и только тогда, когда $f(x)$ не определена).
\end{task}

\begin{task}
   Покажите, что существуют универсальная вычислимая функция, которая не является главной.
\end{task}


\breakline

\begin{ptask}{18} (простые множества Поста)
    Назовем множество {\it иммунным}, если оно бесконечно, но не содержит бесконечных перечислимых подмножеств. Перечислимое
    множество называется {\it простым}, если его дополнение иммунно. Докажите, что простые множества существуют.
\end{ptask}

\begin{ptask}{21}
	Задача Поста состоит в следующем: есть доминошки $n$ видов $\dom{s_1}{t_1}, \dom{s_n}{t_n}$, $s_i$ и $t_i$~--- конечные
    строки, есть неограниченный запас доминошек каждого вида, доминошки переворачивать нельзя. Требуется определить, можно ли
    составить несколько доминошек так, чтобы в верхней и нижней их половине читалась одна и та же строка, такие последовательности
    доминошек будем называть согласованными. Докажите, что задача Поста алгоритмически неразрешима.
\end{ptask}

\begin{ptask}{22}
	В алфавите есть буквы $R$ и $S$. Для каждого слова разрешается вычеркивать или дописывать в произвольные места подслова $RRR$
    и $SS$. Также можно заменять подслово $SRS$ на $RR$ и наоборот. Придумайте алгоритм, который по двум словам в этом алфавите
    проверит, можно ли по этим правилам одно получить из другого.
\end{ptask}
