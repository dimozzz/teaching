\setcounter{curtask}{27}

\mytitle{6 (на 12.10)}

\newcommand{\dom}[2]{\left[\frac{#1}{#2}\right]}

\begin{task}
    Пусть $g(x_1, \dots, x_k) = y_0$, где $y_0 = \min \{y \mid f(x_1, \dots, x_k, y) = 0\}$. Покажите, что при вычислимой не всюду
    определенной $f$, $g$ может быть невычислимой.
\end{task}

\begin{task}
    \begin{enumerate}[topsep = 0pt, itemsep = -1ex]
        \item[а)] Покажите, что функция $U(n, x) = {<}n{>}(x)$ является главной нумерацией.
        \item[б)] Покажите, что если теорема о неподвижной точке верна в одной главной нумерации, то она верна и в любой другой.
	\end{enumerate}
\end{task}

\begin{task}
    Докажите, что следующие предикаты являются арифметичными:
    \begin{enumerate}[topsep = 0pt, itemsep = -1ex]
        \item[а)] $x < y$;
        \item[б)] $x = 0$;
        \item[в)] $x = 1$;
        \item[г)] $x = c$, где $c$~--- константа;
        \item[д)] $x~mod~b = r$;
        \item[е)] $a$~--- степень двойки;
        \item[ж)] $a$~--- степень четверки.
	\end{enumerate}
\end{task}

\begin{task}
    Покажите, что следующий предикат является арифметическим: $x, y, z$~--- члены
    геометрической прогрессии с простым знаменателем. 
\end{task}

\begin{task}
    Является ли перечислимым множество всех программ, вычисляющим сюръективные функции? А его дополнение?
\end{task}




\breakline

\begin{ptask}{18} (простые множества Поста)
    Назовем множество {\it иммунным}, если оно бесконечно, но не содержит бесконечных перечислимых подмножеств. Перечислимое
    множество называется {\it простым}, если его дополнение иммунно. Докажите, что простые множества существуют.
\end{ptask}

\begin{ptask}{21}
	Задача Поста состоит в следующем: есть доминошки $n$ видов $\dom{s_1}{t_1}, \dom{s_n}{t_n}$, $s_i$ и $t_i$~--- конечные
    строки, есть неограниченный запас доминошек каждого вида, доминошки переворачивать нельзя. Требуется определить, можно ли
    составить несколько доминошек так, чтобы в верхней и нижней их половине читалась одна и та же строка, такие последовательности
    доминошек будем называть согласованными. Докажите, что задача Поста алгоритмически неразрешима.
\end{ptask}

\begin{ptask}{26}
   Покажите, что существуют универсальная вычислимая функция, которая не является главной.
\end{ptask}
