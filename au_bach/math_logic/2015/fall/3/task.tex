\setcounter{curtask}{11}

\mytitle{3 (на 21.09)}

\begin{task}
    Докажите, что множество всех рациональных чисел меньших $\pi$ разрешимо.
\end{task}

\begin{task}
    Пусть $S$~--- разрешимое множество натуральных чисел. Разложим все числа из $S$ на простые множители, из данных простых
    составим множество $D$. Верно ли что $D$ разрешимо?
\end{task}

\begin{task}
    Докажите, что существуют перечислимые множества $A, B$, которые не могут быть отделены разрешимым множеством, т.е не
	существует такого разрешимого множества $C$, что $A \subseteq C$ и $B \cap C = \emptyset$.
\end{task}


\begin{task}
    Покажите, что множество описаний машин Тьюринга, которые останавливаются на всех входах, является неперечислимым множеством и
    дополнение его тоже неперечислимо.
\end{task}


\begin{task}
	Опишите машины Тьюринга решающие следующие задачи (и докажите их корректность):
	\begin{enumerate}
		\item является ли строка полиндромом;
		\item дана строка из $0$ и $1$, проверить, что число единиц в ней делится на $3$;
		\item дано число $a$ в двоичной записи, вывести $a - 1$, если $a > 0$, а иначе вывести $0$.
	\end{enumerate}
\end{task}

\begin{task}
	Напишите программы с конечным числом переменных решающие следующие задачи:
	\begin{enumerate}
		\item даны числа $a$ и $b$, нужно найти $a \cdot b$;
		\item даны числа $a$ и $b$, нужно найти $a^b$;
		\item даны числа $a$ и $b$, нужно найти остаток и частное от деления $a$ на $b$;
		\item дано число $p$, выяснить простое ли оно;
		\item дано число $n$ нужно найти $n$-ое простое число.
	\end{enumerate}
\end{task}


\breakline

\begin{ptask}{9}
    Существует ли алгоритм, проверяющий, работает ли данная программа полиномиальное время? (т.е. на каждом входе алгоритм делает
    не более $p(|x|)$ шагов, где $p$~--- полином, а $x$~--- вход алгоритма).
\end{ptask}
