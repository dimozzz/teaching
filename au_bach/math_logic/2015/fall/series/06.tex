\mytitle{6 (на 12.10)}

\newcommand{\dom}[2]{\left[\frac{#1}{#2}\right]}

\begin{task}
    Пусть $g(x_1, \dots, x_k) = y_0$, где $y_0 = \min \{y \mid f(x_1, \dots, x_k, y) = 0\}$. Покажите, что при вычислимой не всюду
    определенной $f$, $g$ может быть невычислимой.
\end{task}

\begin{task}
	Пусть $H = \{(n, x) \mid {<}n{>}(x) \mbox{ останавливается}\}$. Покажите, что $H \in \Sigma_1$ и любое множество из $\Sigma_1$
    $m$-сводится к $H$.
\end{task}

\begin{task}
	Покажите, что множество номеров алгоритмов, которые не останавливаются ни на одном входе
    \begin{enumerate}[topsep = 0pt, itemsep = -1ex]
        \item [а)] лежит в классе $\Pi_1$;
        \item [б)] любое другое множество из $\Pi_1$ $m$-сводится к этому множеству;
    	\item [в)] покажите, что это множество не лежит в $\Sigma_1$. 
    \end{enumerate}
\end{task}

\begin{task}
    Является ли перечислимым множество всех программ, вычисляющим сюръективные функции? А его дополнение?
\end{task}


\begin{task}
	Обозначим через $K(x)$ минимальное такое число $n$, что алгоритм с номером $n$ (номер алгоритма~--- это номер его текста, при
    этом строчки упорядочиваются сначала по длине, потом по алфавиту) на входе $0$ входе печатает $x$ и останавливается. Докажите,
    что $K(x)$ не является вычислимой функцией.
\end{task}

\begin{task}
	Пусть предикат $A(n, x)$ обладает таким свойством: для любого разрешимого предиката $R(x)$ найдется такое натуральное число
    $r$, что $A(r, x) = R(x)$ для всех $x$. Покажите, что предикат $A$ не разрешим.
\end{task}



\breakline

\begin{ptask}{18} (простые множества Поста)
    Назовем множество {\it иммунным}, если оно бесконечно, но не содержит бесконечных перечислимых подмножеств. Перечислимое
    множество называется {\it простым}, если его дополнение иммунно. Докажите, что простые множества существуют.
\end{ptask}

\begin{ptask}{21}
	Задача Поста состоит в следующем: есть доминошки $n$ видов $\dom{s_1}{t_1}, \dom{s_n}{t_n}$, $s_i$ и $t_i$~--- конечные
    строки, есть неограниченный запас доминошек каждого вида, доминошки переворачивать нельзя. Требуется определить, можно ли
    составить несколько доминошек так, чтобы в верхней и нижней их половине читалась одна и та же строка, такие последовательности
    доминошек будем называть согласованными. Докажите, что задача Поста алгоритмически неразрешима.
\end{ptask}

\begin{ptask}{26}
   Покажите, что существуют универсальная вычислимая функция, которая не является главной.
\end{ptask}
