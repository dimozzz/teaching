\documentclass[a4paper, 12pt]{article}
% math symbols
\usepackage{amssymb}
\usepackage{amsmath}
\usepackage{mathrsfs}
\usepackage{mathseries}


\usepackage[margin = 2cm]{geometry}

\tolerance = 1000
\emergencystretch = 0.74cm



\pagestyle{empty}
\parindent = 0mm

\renewcommand{\coursetitle}{DM/ML}
\setcounter{curtask}{1}

\setmathstyle{06.12}{Задание 13. Пиьменное}{АУ}
\setcounter{curtask}{65}

\begin{document}

\task{
    Будем говорить, что замкнутая формула $\varphi$ семантически следует из теории $T$, если любая модель
    теории $T$ является моделью формулы $\varphi$, обозначение: $T \models \varphi$. Покажите, что
    $\varphi$ семантически следует из $T$ тогда и только тогда, когда $\varphi$ выводима из $T$. Коротко:
    $T \models \varphi \iff T \vdash \varphi$.
}

\libproblem{math-logic}{finite-models-to-infin}


\task{
    В алгебре вам доказывали, что если $K$~--- некоторое поле, а многочлен $f \in K[x]$ неприводим, то
    существует $K'$ надполе поля $K$, в котором многочлен $f$ имеет корень (в качестве поля $K'$ можно
    взять $\faktor{K[x]}{\avg{F}}$, это кольцо является полем как фактор-кольцо по максимальному
    идеалу). С помощью теоремы о компактности покажите, что для всякого поля $K$ существует его надполе
    $K'$ такое, что каждый неконстантный многочлен с коэффициентами из $K$ имеет корень в $K'$.
}

\task{
    С помощью теоремы о компактности докажите, что любой частичный порядок на множестве можно продолжить
    до линейного порядка (т.е. до порядка, в котором любые два элемента сравнимы).
}

\task{
    Предъявите алгоритм, который по всякой формуле $\varphi$ сигнатуры $\sigma$ выдаст $\Sigma_2$-формулу
    $\psi$ сигнатуры $\sigma$ с добавленными предикатными символами, что формула $\varphi$ общезначима
    тогда и только тогда, когда формула $\psi$ общезначима.
}



\breakline

\task[60]{
    Пусть сигнатура содержит только одноместные предикатные символы. Покажите, что:
    \begin{enumcyr}
        \item всякая выполнимая формула, содержащая $n$ предикатных символов, выполнима и в
            интерпретации, в носителе которой не более $2^n$ элементов;
        \item существует алгоритм, проверяющий выполнимость таких формул.
    \end{enumcyr}
}

\libproblem[64]{math-logic}{q-less-two-models-cont}


\end{document}