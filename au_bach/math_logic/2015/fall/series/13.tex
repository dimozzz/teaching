\mytitle{13 (на 06.12, в письменном виде)}

\begin{task}
	Будем говорить, что замкнутая формула $\phi$ семантически следует из теории $T$, если любая модель теории $T$
    является моделью формулы $\phi$, обозначение: $T\models \phi$. Покажите, что $\phi$ семантически следует из $T$
    тогда и только тогда, когда $\phi$ выводима из $T$. Коротко: $T \models \phi \iff T \vdash \phi$. 
\end{task}


\begin{task}
    Пусть теория $T$ имеет модель со сколь угодно большим носителем. Докажите, что $T$ имеет модель с бесконечным
    носителем.
\end{task}


\begin{task}
    В алгебре вам доказывали, что если $K$~--- некоторое поле, а многочлен $f \in K[x]$ неприводим, то существует $K'$
    надполе поля $K$, в котором многочлен $f$ имеет корень (в качестве поля $K'$ можно взять $K[x] / {<}f{>}$, это
    кольцо является полем как фактор-кольцо по максимальному идеалу). С помощью теоремы о компактности покажите, что для
    всякого поля $K$ существует его надполе $K'$ такое, что каждый неконстантный многочлен с коэффициентами из $K$ имеет
    корень в $K'$.
\end{task}

\begin{task}
    С помощью теоремы о компактности докажите, что любой частичный порядок на множестве можно продолжить до линейного
    порядка (т.е. до порядка, в котором любые два элемента сравнимы).
\end{task}


\begin{task}
	Предъявите алгоритм, который по всякой формуле $\phi$ сигнатуры $\sigma$ выдаст $\Sigma_2$-формулу $\psi$ сигнатуры
    $\sigma$ с добавленными предикатными символами, что формула $\phi$ общезначима тогда и только тогда, когда формула
    $\psi$ общезначима.
\end{task}


\breakline


\begin{ptask}{59}
	Пусть сигнатура содержит только одноместные предикатные символы. Покажите, что:
    \begin{enumerate}[topsep = 0pt, itemsep = -1ex]
        \item [а)] всякая выполнимая формула, содержащая $n$ предикатных символов, выполнима и в интерпретации, в
			носителе которой не более $2^n$ элементов;
        \item [б)] существует алгоритм, проверяющий выполнимость таких формул.
    \end{enumerate}
\end{ptask}


\begin{ptask}{63}
    Построите две неизоморфные интерпретации теории $Th(\mathbb{Q}, <, =)$ (плотный линейный порядок без первого и
    последнего элемента) мощности континуум.
\end{ptask}
