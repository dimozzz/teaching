\mytitle{7 (на 19.10)}

\newcommand{\dom}[2]{\left[\frac{#1}{#2}\right]}

\begin{task}
	Покажите, что универсальный предикат для класса одноместных разрешимых предикатов не является разрешимым.
\end{task}


\begin{task}
    Докажите, что следующие функции являются примитивно рекурсивными:
    \begin{enumerate}[topsep = 0pt, itemsep = -1ex]
        \item [а)] $x^y$;
        \item [б)] $x!$;
        \item [в)] Покажите, что усеченное вычитание $x \dot{-} y$, которое равняется $x - y$, если $x \ge y$ и нулю иначе,
			является примитивно рекурсивным; 
    	\item [г)] $min(x, y)$;
        \item [д)] $max(x, y)$;
    \end{enumerate}
\end{task}


\vspace{0.5cm}

Множество $S \subseteq \mathcal{N}^k$ называется примитивно рекурсивным, если его характеристическая функция примитивно
рекурсивна.

\begin{task}	
    \begin{enumerate}[topsep = 0pt, itemsep = -1ex]
        \item [а)] Покажите, что множество $S \subseteq \mathcal{N}^k$ примитивно рекурсивное, тогда и только тогда, когда оно
			есть множество нулей некоторой примитивно-рекурсивной функции;
        \item [б)] покажите, что объединение, пересечение и дополнение примитивно рекурсивных множеств является примитивно
			рекурсивным.
        \item [в)] покажите, что предикат $x = y$ примитивно рекурсивен;
        \item [г)] покажите, что предикат $x > y$ примитивно рекурсивен.
    \end{enumerate}
\end{task}


\begin{task}
    Пусть отношение $R(x, y)$ задает примитивно рекурсивное множество (т.е. множество $\{(x, y) \mid R(x, y) = 1\}$), докажите,
    что отношения $S(x, z) = \exists (y \le z) R(x, y)$ и $T(x, z) = \forall (y \le z) R(x, y)$ также задают примитивно
    рекурсивные множества.
\end{task}

\begin{task}
    Докажите, что существует такое подмножество натуральных чисел, что его симметрическая разность с любым перечислимым множеством
    имеет бесконечный размер.
\end{task}



\breakline

\begin{ptask}{21}
	Задача Поста состоит в следующем: есть доминошки $n$ видов $\dom{s_1}{t_1}, \dom{s_n}{t_n}$, $s_i$ и $t_i$~--- конечные
    строки, есть неограниченный запас доминошек каждого вида, доминошки переворачивать нельзя. Требуется определить, можно ли
    составить несколько доминошек так, чтобы в верхней и нижней их половине читалась одна и та же строка, такие последовательности
    доминошек будем называть согласованными. Докажите, что задача Поста алгоритмически неразрешима.
\end{ptask}


\begin{ptask}{31}
	Обозначим через $K(x)$ минимальное такое число $n$, что алгоритм с номером $n$ (номер алгоритма~--- это номер его текста, при
    этом строчки упорядочиваются сначала по длине, потом по алфавиту) на входе $0$ входе печатает $x$ и останавливается. Докажите,
    что $K(x)$ не является вычислимой функцией.
\end{ptask}

\begin{ptask}{32}
	Пусть предикат $A(n, x)$ обладает таким свойством: для любого разрешимого предиката $R(x)$ найдется такое натуральное число
    $r$, что $A(r, x) = R(x)$ для всех $x$. Покажите, что предикат $A$ не разрешим.
\end{ptask}
