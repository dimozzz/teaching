\documentclass[a4paper, 12pt]{article}
% math symbols
\usepackage{amssymb}
\usepackage{amsmath}
\usepackage{mathrsfs}
\usepackage{mathseries}


\usepackage[margin = 2cm]{geometry}

\tolerance = 1000
\emergencystretch = 0.74cm



\pagestyle{empty}
\parindent = 0mm

\renewcommand{\coursetitle}{DM/ML}
\setcounter{curtask}{1}

\setmathstyle{19.10}{Задание 7}{АУ}
\setcounter{curtask}{33}

\begin{document}

\task{
    Покажите, что универсальный предикат для класса одноместных разрешимых предикатов не является
    разрешимым.
}

\libproblem{computability}{primitive-recursion-ex}

\begin{definition*}
    Множество $S \subseteq \mathbb{N}^k$ будем называть \deftext{примитивно рекурсивным}, если его
    характеристическая функция примитивно рекурсивна.
\end{definition*}

\libproblem{computability}{primitive-recursion-set-ex}
\libproblem{computability}{primitive-recursion-set-proj}
\libproblem{computability}{inf-non-computable}

\breakline

\libproblem[21]{computability}{post-string-eq}
\libproblem[26]{computability}{non-main-numbering}

\task[31]{
    Обозначим через $K(x)$ минимальное такое число $n$, что алгоритм с номером $n$ (номер алгоритма~---
    это номер его текста, при этом строчки упорядочиваются сначала по длине, потом по алфавиту) на входе
    $0$ входе печатает $x$ и останавливается. Докажите, что $K(x)$ не является вычислимой функцией.
}

\task[32]{
    Пусть предикат $A(n, x)$ обладает таким свойством: для любого разрешимого предиката $R(x)$ найдется
    такое натуральное число $r$, что $A(r, x) = R(x)$ для всех $x$. Покажите, что предикат $A$ не
    разрешим.
}

\end{document}