\documentclass[a4paper, 12pt]{article}
% math symbols
\usepackage{amssymb}
\usepackage{amsmath}
\usepackage{mathrsfs}
\usepackage{mathseries}


\usepackage[margin = 2cm]{geometry}

\tolerance = 1000
\emergencystretch = 0.74cm



\pagestyle{empty}
\parindent = 0mm

\renewcommand{\coursetitle}{DM/ML}
\setcounter{curtask}{1}

\setmathstyle{}{Большое задание}{АУ}

\begin{document}

\libproblem{math-logic}{q-less-two-models-cont}
\libproblem{math-logic}{finite-models-to-infin}
\libproblem{math-logic}{field-extension-compactness}
\libproblem{math-logic}{rcf-nonstandard-point}
\libproblem{set-theory}{partial-to-full-order}
\libproblem{math-logic}{inf-to-countable-model}
\libproblem{math-logic}{z-less-fin-ax}
\libproblem{math-logic}{n-less-fin-ax}

\begin{definition*}
    Пусть есть вполне упорядоченное (в.у.) множество $P$. $\vec{x} \coloneqq \{y < x \mid x \in P\}$~---
    будем называть \deftext{начальным отрезком}.
\end{definition*}

\libproblem{math-logic}{initial-segment-noniso}

\task{
    Докажите, что в любом линейном пространстве есть басиз.
}

\breakline

\task{
    Докажите, что:
	\begin{enumcyr}
        \item для любого множества $A$ верно: $A \notin A$;
        \item класс ординалов не является множеством.
    \end{enumcyr}
}

\libproblem*{math-logic}{acf-char-0-big-char}

\end{document}