\documentclass[a4paper, 12pt]{article}
% math symbols
\usepackage{amssymb}
\usepackage{amsmath}
\usepackage{mathrsfs}
\usepackage{mathseries}


\usepackage[margin = 2cm]{geometry}

\tolerance = 1000
\emergencystretch = 0.74cm



\pagestyle{empty}
\parindent = 0mm

\renewcommand{\coursetitle}{DM/ML}
\setcounter{curtask}{1}

\setmathstyle{14.12}{Задание 14}{АУ}
\setcounter{curtask}{70}

\begin{document}


\task{
    Приведите пример конечно аксиоматизируемой, но неразрешимой теории. Указание: используйте неразрешимость
    ассоциативного исчисления.
}

\libproblem{set-theory}{inf-bin-zero-order}
\libproblem{set-theory}{polynomial-lex-order}

\breakline

\task[60]{
    Пусть сигнатура содержит только одноместные предикатные символы. Покажите, что:
    \begin{enumcyr}
        \item всякая выполнимая формула, содержащая $n$ предикатных символов, выполнима и в
            интерпретации, в носителе которой не более $2^n$ элементов;
        \item существует алгоритм, проверяющий выполнимость таких формул.
    \end{enumcyr}
}

\task[67]{
    В алгебре вам доказывали, что если $K$~--- некоторое поле, а многочлен $f \in K[x]$ неприводим, то
    существует $K'$ надполе поля $K$, в котором многочлен $f$ имеет корень (в качестве поля $K'$ можно
    взять $\faktor{K[x]}{\avg{F}}$, это кольцо является полем как фактор-кольцо по максимальному
    идеалу). С помощью теоремы о компактности покажите, что для всякого поля $K$ существует его надполе
    $K'$ такое, что каждый неконстантный многочлен с коэффициентами из $K$ имеет корень в $K'$.
}


\end{document}