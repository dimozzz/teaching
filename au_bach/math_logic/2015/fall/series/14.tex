\mytitle{14 (на 14.12)}

\begin{task}
	Приведите пример конечно аксиоматизируемой, но неразрешимой теории. Указание: используйте неразрешимость
    ассоциативного исчисления.
\end{task}

\begin{task}
	Рассмотрим множество невозрастающих последовательностей натуральных чисел, в которых все члены, начиная с
    некоторого, равны нулю. Введем в нем порядок: сначала сравниваем первые члены, при равенстве вторые члены и
    т.д. Покажите, что так получится вполне упорядоченное множество.
\end{task}

\begin{task}
	Рассмотрим множество всех многочленов от одной переменной $x$, коэффициенты которого натуральные числа. Введем такой
    порядок многочлен $P(x)$ больше многочлена $Q(x)$, если для всех достаточно больших $x$ выполняется $P(x) >
    Q(x)$. Покажите, что так получится вполне упорядоченное множество.
\end{task}


\breakline

\begin{ptask}{51}
    Будет ли интерпретация $(\mathbb{N}, =, <)$ элементарно эквивалентна: $(\mathbb{N} + \mathbb{Z}, =, <)$. А будут ли эти
    интерпретации изоморфны?
\end{ptask}

\begin{ptask}{53}
    \begin{enumerate}[topsep = 0pt, itemsep = -1ex]
        \item [а)] Покажите, что естественные интерпретации $(=, +, *, 0, 1)$ для всех алгебраически замкнутых полей
			характеристики $0$ являются элементарно эквивалентными.
        \item [б)] Для двух алгебраически замкнутых полей $k_1$ и $k_2$ характеристики $0$ выполняется, что $k_1$ является
		    надполем поля $k_2$. Покажите, что естественная интерпретация $(=, +, *, 0, 1)$ в поле $k_1$ является элементарным
            расширением естественной интерпретации $(=, +, *, 0, 1)$ в поле $k_2$.
        \item [в)] Докажите теорему Гильберта о нулях: всякая система полиномиальных уравнений с коэффициентами в алгебраически
			замкнутом поле характеристики ноль, имеющее решение в расширении поля, имеет решение и в самом поле.
        \item [г)] Докажите переформулировку теоремы Гильберта о нулях: если система полиномиальных уравнений
        	$\bigwedge\limits_{i = 1}^k P_i(x_1, x_2, \dots, x_n) = 0$ не имеет решения в некотором алгебраически замкнутом поле
            характеристики $0$, то найдутся такие многочлены $Q_1(x_1, \dots, x_n), \dots, Q_k(x_1, \dots, x_n)$, что
            $\sum\limits_i Q_i P_1 = 1$.
    \end{enumerate}
\end{ptask}

\begin{ptask}{58}
	Заменим 11-ую аксиому $A \lor \lnot A$ на $\lnot \lnot A \to A$. Покажите, что множество выводимых формул не изменится.
\end{ptask}


\begin{ptask}{59}
	Пусть сигнатура содержит только одноместные предикатные символы. Покажите, что:
    \begin{enumerate}[topsep = 0pt, itemsep = -1ex]
        \item [а)] всякая выполнимая формула, содержащая $n$ предикатных символов, выполнима и в интерпретации, в
			носителе которой не более $2^n$ элементов;
        \item [б)] существует алгоритм, проверяющий выполнимость таких формул.
    \end{enumerate}
\end{ptask}

\begin{ptask}{66}
    В алгебре вам доказывали, что если $K$~--- некоторое поле, а многочлен $f \in K[x]$ неприводим, то существует $K'$
    надполе поля $K$, в котором многочлен $f$ имеет корень (в качестве поля $K'$ можно взять $K[x] / {<}f{>}$, это
    кольцо является полем как фактор-кольцо по максимальному идеалу). С помощью теоремы о компактности покажите, что для
    всякого поля $K$ существует его надполе $K'$ такое, что каждый неконстантный многочлен с коэффициентами из $K$ имеет
    корень в $K'$.
\end{ptask}

