\documentclass[a4paper, 12pt]{article}
% math symbols
\usepackage{amssymb}
\usepackage{amsmath}
\usepackage{mathrsfs}
\usepackage{mathseries}


\usepackage[margin = 2cm]{geometry}

\tolerance = 1000
\emergencystretch = 0.74cm



\pagestyle{empty}
\parindent = 0mm

\renewcommand{\coursetitle}{DM/ML}
\setcounter{curtask}{1}

\setmathstyle{30.11}{Задание 12. Пиьменное}{АУ}
\setcounter{curtask}{60}

\begin{document}

\task{
    Пусть сигнатура содержит только одноместные предикатные символы. Покажите, что:
    \begin{enumcyr}
        \item всякая выполнимая формула, содержащая $n$ предикатных символов, выполнима и в
            интерпретации, в носителе которой не более $2^n$ элементов;
        \item существует алгоритм, проверяющий выполнимость таких формул.
    \end{enumcyr}
}

\task{
    Покажите, что:
    \begin{enumcyr}
        \item если формула $A(x)$ выводима, то выводима и формула $\forall x A(x)$;
        \item множество выводимых формул не изменится, если мы добавим правило обобщения
            $\frac{A(x)}{\forall x A(x)}$, а правила Бернайса заменим на две новые аксиомы:
            \begin{itemtask}
                \item $(\forall x (A \to B)) \to (A \to \forall x B)$;
                \item $(\exists x (B \to A)) \to (\exists x B \to A)$,
            \end{itemtask}
             в этих аксиомах $x$ не входит в $A$ свободно.
    \end{enumcyr}
}

\task{
    Покажите, что:
    \begin{enumcyr}
        \item если $A \to B$ выводима, то и $\exists x A \to \exists x B$ выводима;
        \item формула $\forall x A(x)\to \forall y A(y)$ выводима.
    \end{enumcyr}
}

\task{
    Приведите пример формулы, которая истинна во всех интерпретациях с конечным носителем, но не является
    общезначимой.
}

\begin{definition*}
    Пусть $I$~--- интерпретация. \deftext{Теорией} $\Theory(I)$ будем называть множество замкнутых
    формул, которые истины в интерпретации $I$.
\end{definition*}


\libproblem{math-logic}{q-less-two-models-cont}

\breakline

\libproblem[52]{math-logic}{n-equiv-n-plus-z}

\task[54]{
    \begin{enumcyr}
        \item Покажите, что естественные интерпретации $(=, +, *, 0, 1)$ для всех алгебраически замкнутых
            полей характеристики $0$ являются элементарно эквивалентными.
        \item Для двух алгебраически замкнутых полей $k_1$ и $k_2$ характеристики $0$ выполняется, что
            $k_1$ является надполем поля $k_2$. Покажите, что естественная интерпретация $(=, +, *, 0,
            1)$ в поле $k_1$ является элементарным расширением естественной интерпретации $(=, +, *, 0,
            1)$ в поле $k_2$.
        \item Докажите теорему Гильберта о нулях: всякая система полиномиальных уравнений с
            коэффициентами в алгебраически замкнутом поле характеристики ноль, имеющее решение в
            расширении поля, имеет решение и в самом поле.
        \item Докажите переформулировку теоремы Гильберта о нулях: если система полиномиальных уравнений
            $\bigwedge\limits_{i = 1}^k P_i(x_1, x_2, \dots, x_n) = 0$ не имеет решения в некотором
            алгебраически замкнутом поле характеристики $0$, то найдутся такие многочлены $Q_1(x_1,
            \dots, x_n), \dots, Q_k(x_1, \dots, x_n)$, что $\sum\limits_i Q_i P_1 = 1$.
    \end{enumcyr}
}

\task[59]{
    Заменим 11-ую аксиому $A \lor \lnot A$ на $\lnot \lnot A \to A$. Покажите, что множество выводимых
    формул не изменится.
}


\end{document}