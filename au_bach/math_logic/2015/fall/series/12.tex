\mytitle{12 (на 30.11, в письменном виде)}

\begin{task}
	Пусть сигнатура содержит только одноместные предикатные символы. Покажите, что:
    \begin{enumerate}[topsep = 0pt, itemsep = -1ex]
        \item [а)] всякая выполнимая формула, содержащая $n$ предикатных символов, выполнима и в интерпретации, в носителе которой
    		не более $2^n$ элементов;
        \item [б)] существует алгоритм, проверяющий выполнимость таких формул.
    \end{enumerate}
\end{task}


\begin{task}
    Покажите, что:
    \begin{enumerate}[topsep = 0pt, itemsep = -1ex]
        \item [а)] если формула $A(x)$ выводима, то выводима и формула $\forall x A(x)$;
        \item [б)] множество выводимых формул не изменится, если мы добавим правило обобщения $\frac{A(x)}{\forall x A(x)}$, а
    		правила Бернайса заменим на две новые аксиомы:\\ $(\forall x (A \to B)) \to (A \to \forall x B)$ и
			$(\exists x (B \to A)) \to (\exists x B \to A)$, в этих аксиомах $x$ не входит в $A$ свободно.
    \end{enumerate}
\end{task}

\begin{task}
    Покажите, что:
    \begin{enumerate}[topsep = 0pt, itemsep = -1ex]
        \item [а)] если $A \to B$ выводима, то и $\exists x A \to \exists x B$ выводима;
        \item [б)] формула $\forall x A(x)\to \forall y A(y)$ выводима.
    \end{enumerate}
\end{task}


\begin{task}
	Приведите пример формулы, которая истинна во всех интерпретациях с конечным носителем, но не является общезначимой.
\end{task}


Пусть $I$~--- интерпретация. Теорией $Th(I)$ называется множество замкнутых формул, которые истины в интерпретации $I$.

\begin{task}
    Построите две неизоморфные интерпретации теории $Th(\mathbb{Q}, <, =)$ (плотный линейный порядок без первого и последнего
    элемента) мощности континуум.
\end{task}


\breakline

\begin{ptask}{51}
    Будет ли интерпретация $(\mathbb{N}, =, <)$ элементарно эквивалентна: $(\mathbb{N} + \mathbb{Z}, =, <)$. А будут ли эти
    интерпретации изоморфны?
\end{ptask}

\begin{ptask}{53}
    \begin{enumerate}[topsep = 0pt, itemsep = -1ex]
        \item [а)] Покажите, что естественные интерпретации $(=, +, *, 0, 1)$ для всех алгебраически замкнутых полей
			характеристики $0$ являются элементарно эквивалентными.
        \item [б)] Для двух алгебраически замкнутых полей $k_1$ и $k_2$ характеристики $0$ выполняется, что $k_1$ является
		    надполем поля $k_2$. Покажите, что естественная интерпретация $(=, +, *, 0, 1)$ в поле $k_1$ является элементарным
            расширением естественной интерпретации $(=, +, *, 0, 1)$ в поле $k_2$.
        \item [в)] Докажите теорему Гильберта о нулях: всякая система полиномиальных уравнений с коэффициентами в алгебраически
			замкнутом поле характеристики ноль, имеющее решение в расширении поля, имеет решение и в самом поле.
        \item [г)] Докажите переформулировку теоремы Гильберта о нулях: если система полиномиальных уравнений
        	$\bigwedge\limits_{i = 1}^k P_i(x_1, x_2, \dots, x_n) = 0$ не имеет решения в некотором алгебраически замкнутом поле
            характеристики $0$, то найдутся такие многочлены $Q_1(x_1, \dots, x_n), \dots, Q_k(x_1, \dots, x_n)$, что
            $\sum\limits_i Q_i P_1 = 1$.
    \end{enumerate}
\end{ptask}

\begin{ptask}{58}
	Заменим 11-ую аксиому $A \lor \lnot A$ на $\lnot \lnot A \to A$. Покажите, что множество выводимых формул не изменится.
\end{ptask}

