\documentclass[a4paper, 12pt]{article}
% math symbols
\usepackage{amssymb}
\usepackage{amsmath}
\usepackage{mathrsfs}
\usepackage{mathseries}


\usepackage[margin = 2cm]{geometry}

\tolerance = 1000
\emergencystretch = 0.74cm



\pagestyle{empty}
\parindent = 0mm

\renewcommand{\coursetitle}{DM/ML}
\setcounter{curtask}{1}

\setmathstyle{26.10}{Задание 8}{АУ}
\setcounter{curtask}{38}

\begin{document}

\task{
    Докажите, что существует такое множество $S \subseteq \mathbb{N}$, что для любого перечислимого
    множества $A$ множества $A \cap S$ и $A \setminus S$ имеют бесконечный размер.
}

\begin{definition*}
    \deftext{Общерекурсивная функция}~--- частично рекурсивная функция, определенная для всех значений.
\end{definition*}

\task{
    Пусть $f$~--- общерекурсивная. Докажите (не пользуясь вычислительной эквивалентностью с машинами
    Тьюринга), что если изменить значение в конечном числе точек, то получится общерекурсивная функция.
}

\libproblem{computability}{primitive-recursion-inv}
\libproblem{computability}{primitive-recursion-rec-def}


\task{
    Предъявите:
    \begin{enumcyr}
        \item $2$;
        \item $3$;
        \item бесконечное количество
    \end{enumcyr}
    таких упорядоченных счетных множеств, что никакие два из них не изоморфны.
}


\breakline

\libproblem[21]{computability}{post-string-eq}

\end{document}