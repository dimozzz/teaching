\mytitle{8 (на 26.10)}

\newcommand{\dom}[2]{\left[\frac{#1}{#2}\right]}


\begin{task}
    Докажите, что существует такое множество $S \subseteq \mathbb{N}$, что для любого перечислимого множества $A$ множества $A
    \cap S$ и $A \setminus S$ имеют бесконечный размер.
\end{task}

Общерекурсивная функция~--- частично рекурсивная функция, определенная для всех значений.

\begin{task}
    Пусть $f$~--- общерекурсивная. Докажите (не пользуясь вычислительной эквивалентностью с машинами Тьюринга), что если изменить
    значение в конечном числе точек, то получится общерекурсивная функция.
\end{task}

\begin{task}
    Покажите, что функция обратная к примитивно рекурсивной биекции $f: \mathbb{N} \rightarrow \mathbb{N}$ может не быть
    примитивно рекурсивной.
\end{task}

\begin{task}
    Покажите, что для любой одноместной примитивно рекурсивной функции $h$ и для любой трехместной примитивно рекурсивной функции
    $g$ рекурсивное определение:

    $f(x, 0) = h(x)$

    $f(x, i + 1) = g(x, i, f(2 x, i))$

    задает примитивно рекурсивную функцию.
\end{task}

\begin{task}
	Предъявите: а) $2$ б) $3$ в) бесконечное количество таких упорядоченных счетных множеств, что никакие два из них не изоморфны.
\end{task}

\breakline

\begin{ptask}{21}
	Задача Поста состоит в следующем: есть доминошки $n$ видов $\dom{s_1}{t_1}, \dom{s_n}{t_n}$, $s_i$ и $t_i$~--- конечные
    строки, есть неограниченный запас доминошек каждого вида, доминошки переворачивать нельзя. Требуется определить, можно ли
    составить несколько доминошек так, чтобы в верхней и нижней их половине читалась одна и та же строка, такие последовательности
    доминошек будем называть согласованными. Докажите, что задача Поста алгоритмически неразрешима.
\end{ptask}