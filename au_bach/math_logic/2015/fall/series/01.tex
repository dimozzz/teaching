\mytitle{1 (на 07.09)}

\begin{task}
	Не ссылаясь на теорему Ферма, покажите, что множество всех показателей $n$, для которых существует решение уравнения $x^n +
    y^n = z^n$ в целых положительных числах, перечислимо. (Как теперь известно, это множество содержит лишь числа $1$ и $2$.
\end{task}


\begin{task}
    Диофантовым называется уравнение, имеющее вид $P(x_1, \dots, x_n) = 0$, где $P$~--- многочлен с целыми
    коэффициентами. Докажите, что множество диофантовых уравнений, имеющих целые решения, перечислимо. (Оно неразрешимо: в этом
    состоит известный результат Ю. В. Матиясевича, явившийся решением знаменитой «10-й проблемы Гильберта»).
\end{task}


\begin{task}
	а) Докажите, что объединение и пересечение перечислимых множеств перечислимо.

    б) Докажите, что декартово произведение перечислимых множеств перечислимо. 
\end{task}


\begin{task}
    Докажите, что всякое бесконечное перечислимое множество содержит бесконечное разрешимое подмножество.
\end{task}

\begin{task}
    Приведите пример неразрешимого подмножества $\mathbb{N} \times \mathbb{N}$, такого что все его горизонтальные и вертикальные
    сечения (т.е. пересечения с $\mathbb{N} \times \{x\} $ и с $\{x\} \times \mathbb{N}$) разрешимы.
\end{task}


\begin{task}
    Приведите пример множества, которое а) не является перечислимым б) кроме того и его дополнение тоже не является перечислимым.
\end{task}

\begin{task}
    Докажите, что непустое множество натуральных чисел разрешимо тогда и только тогда, когда оно есть множество значений всюду
    определённой неубывающей вычислимой функции с натуральными аргументами и значениями.
\end{task}