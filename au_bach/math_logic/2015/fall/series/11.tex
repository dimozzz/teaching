\mytitle{11 (на 23.11)}


Пусть $I_1$ и $I_2$~--- две интерпретации одной сигнатуры. Мы говорим, что $I_1$ является расширением $I_2$, если $M_1$ (носитель
$I_1$) является надмножеством $M_2$ (носителя $I_2$), все функциональный и предикатные символы согласованы на $M_2$. Говорят, что
это расширение является элементарным, если любая (не обязательно замкнутая формула) на любой оценке свободных переменных,
принимающих значения в $M_2$ одновременно истинна и ложна в интерпретациях $I_1$ и $I_2$.

\begin{task}
    \begin{enumerate}[topsep = 0pt, itemsep = -1ex]
        \item [а)] Покажите, что естественные интерпретации $(=, +, *, 0, 1)$ для всех алгебраически замкнутых полей
			характеристики $0$ являются элементарно эквивалентными.
        \item [б)] Для двух алгебраически замкнутых полей $k_1$ и $k_2$ характеристики $0$ выполняется, что $k_1$ является
		    надполем поля $k_2$. Покажите, что естественная интерпретация $(=, +, *, 0, 1)$ в поле $k_1$ является элементарным
            расширением естественной интерпретации $(=, +, *, 0, 1)$ в поле $k_2$.
        \item [в)] Докажите теорему Гильберта о нулях: всякая система полиномиальных уравнений с коэффициентами в алгебраически
			замкнутом поле характеристики ноль, имеющее решение в расширении поля, имеет решение и в самом поле.
        \item [г)] Докажите переформулировку теоремы Гильберта о нулях: если система полиномиальных уравнений
        	$\bigwedge\limits_{i = 1}^k P_i(x_1, x_2, \dots, x_n) = 0$ не имеет решения в некотором алгебраически замкнутом поле
            характеристики $0$, то найдутся такие многочлены $Q_1(x_1, \dots, x_n), \dots, Q_k(x_1, \dots, x_n)$, что
            $\sum\limits_i Q_i P_1 = 1$.
    \end{enumerate}
\end{task}

\begin{task}
	Покажите (в случае пропозициональных формул), что если $F_1, F_2, \dots, F_n \vdash F$, то формула $(\bigwedge\limits_{i = 1}^n
    F_i) \to F$ является тавтологией.
\end{task}

\begin{task}
	Покажите, что для любых пропозициональных формул $A, B$ и $C$ формула $(A \to B) \to ((B \to C) \to (A \to C))$ является
    выводимой.
\end{task}

\begin{task}
	Покажите, что если формула $\phi$ является выводимой, то и формула, которая получится при подстановке другой формулы вместо
    переменной формулы $\phi$, тоже будет выводимой.
\end{task}

\begin{task}
	Покажите, что следующие формулы являются выводимыми:
    \begin{enumerate}[topsep = 0pt, itemsep = -1ex]
        \item [а)] $A \to \lnot \lnot A$ и $\lnot \lnot \lnot A \to \lnot A$;
        \item [б)] $((A \lor B) \to C) \to ((A \to C) \land (B \to C))$ и $((A \to C) \land (B \to C)) \to ((A \lor B) \to C)$;
        \item [в)] $((A \land C) \lor (B \land C)) \to ((A \lor B) \land C)$ и $((A \lor B) \land C) \to ((A \land C) \lor (B
			\land C))$;
        \item [г)] $((A \lor C) \land (B \lor C)) \to ((A \land B) \lor C)$ и $((A \land B) \lor C) \to ((A \lor C) \lor
            (B \lor C))$;
        \item [д)] $(\lnot B \to \lnot A) \to (A \to B)$.
    \end{enumerate}
\end{task}

\begin{task}
	Заменим 11-ую аксиому $A \lor \lnot A$ на $\lnot \lnot A \to A$. Покажите, что множество выводимых формул не изменится.
\end{task}


\breakline

\begin{ptask}{51}
    Будет ли интерпретация $(\mathbb{N}, =, <)$ элементарно эквивалентна: $(\mathbb{N} + \mathbb{Z}, =, <)$. А будут ли эти
    интерпретации изоморфны?
\end{ptask}

\begin{ptask}{52}
    Будет ли интерпретация $(\mathbb{Q}, =, <)$ элементарно эквивалентна:
    \begin{enumerate}[topsep = 0pt, itemsep = -1ex]
        \item [а)] $(\mathbb{Q} + \mathbb{Q}, =, <)$;
        \item [б)] $(\mathbb{Q} + \mathbb{R}, =, <)$.
    \end{enumerate}
\end{ptask}