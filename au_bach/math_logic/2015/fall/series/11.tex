\documentclass[12pt, fleqn, a4paper]{article}


\usepackage{amsmath}
\usepackage{amssymb}
\usepackage{amsfonts}
\usepackage{textcomp}
\usepackage{amsthm}
\usepackage{mathtools}
\usepackage{xspace}
\usepackage[classfont = bold]{complexity}
\usepackage{fullpage}
\usepackage[russian, english]{babel}
\usepackage[utf8]{inputenc}
\usepackage[
    sorting = ydnt,
    style = alphabetic,
    maxbibnames = 99,
    backend = biber
    ]{biblatex}
\addbibresource{main.bib}



\newtheorem{conjecture}{Conjecture}[section]
\theoremstyle{definition}
\newtheorem{theorem}{Theorem}[section]
\newtheorem*{theorem*}{Theorem}
\newtheorem{lemma}{Lemma}[section]
\newtheorem{corollary}{Corollary}[section]
\newtheorem{proposition}{Proposition}[section]
\newtheorem{fact}{Fact}[section]
\newtheorem{problem}{Problem}[section]
\newtheorem{exercise}{Exercise}[section]
\newtheorem{example}{Example}[section]
\newtheorem{definition}{Definition}[section]
\newtheorem{remark}{Remark}[section]
\newtheorem{algorithm}{Algorithm}[section]


\newcommand{\class}[1]{\mathbf{#1}}
\newcommand{\co}{\mathrm{co}}
\newcommand{\alg}[1]{\mathit{#1}}
\newcommand{\lang}[1]{\mathtt{#1}}


% classes (1)

\newcommand{\DTime}{\class{DTime}}
\newcommand{\RTime}{\class{RTime}}
\newcommand{\UTime}{\class{UTime}}
\newcommand{\NTime}{\class{NTime}}
\newcommand{\BPTime}{\class{BPTime}}


\renewcommand{\P}{\class{P}}
\newcommand{\ZPP}{\class{ZPP}}
\newcommand{\RP}{\class{RP}}
\newcommand{\coRP}{\co\class{RP}}
\newcommand{\UP}{\class{UP}}
\newcommand{\coUP}{\co\class{UP}}
\newcommand{\NP}{\class{NP}}
\newcommand{\coNP}{{\co}\class{NP}}
\newcommand{\BPP}{\class{BPP}}
\newcommand{\SigmaP}[1]{\Sigma^{#1}\class{P}}
\newcommand{\PH}{\class{PH}}
\newcommand{\PP}{\class{PP}}
\newcommand{\IP}{\class{IP}}
\newcommand{\OP}{\class{\oplus P}}


\newcommand{\EXP}{\class{EXP}}
\newcommand{\MIP}{\class{MIP}}
\newcommand{\NEXP}{\class{NEXP}}
\newcommand{\coNEXP}{{\co}\class{NEXP}}
\newcommand{\MAEXP}{\class{MA}_\class{EXP}}


% classes (2)

\newcommand{\Ppoly}{\class{P}/\class{poly}}
\newcommand{\NC}{\class{NC}}


\newcommand{\DSpace}{\class{DSpace}}
\newcommand{\NSpace}{\class{NSpace}}
\newcommand{\PSPACE}{\class{PSPACE}}

\newcommand{\EXPSPACE}{\class{EXPSPACE}}


% algorithms and proof systems


\newcommand{\DPLL}{\alg{DPLL}}
\newcommand{\OBDD}{\alg{OBDD}}
\newcommand{\pOBDD}{\pi\text{-}\alg{OBDD}}
\newcommand{\DPLLL}{\alg{DPLL}_{lin}}
\newcommand{\ResL}{\alg{Res}_{lin}}
\newcommand{\SemL}{\alg{Sem}_{lin}}



% languages


\newcommand{\SAT}{\lang{SAT}}
\newcommand{\GNI}{\lang{GNI}}
\newcommand{\MAJSAT}{\lang{MAJ}\text{-}\lang{SAT}}
\newcommand{\QBF}{\lang{QBF}}



% other

\newcommand{\poly}{\mathrm{poly}}
\newcommand{\Nat}{\mathbb{N}}
\newcommand{\bool}{\{0, 1\}}

\newcommand{\Img}{\mathop{\mathrm{Im}}}

\DeclareMathOperator*{\supp}{supp}
\DeclareMathOperator*{\Exp}{E}
\DeclareMathOperator*{\rk}{rk}




%%% Local Variables:
%%% mode: latex
%%% TeX-master: t
%%% End:


\begin{document}

	\setcounter{curtask}{9}

\mytitle{2 (на 3.10)}

\begin{task}
    Докажите, что множество всех рациональных чисел меньших $\pi$ разрешимо.
\end{task}

\begin{task}
    Существует ли алгоритм, проверяющий, работает ли данная программа
    полиномиальное время?
\end{task}

\begin{task}
    Приведите пример двух непересекающихся неперечислимых множеств.
\end{task}

\begin{task}
    Докажите, что для каждой вычислимой функции $f$ найдется
    псевдообратная вычислимая функция $g$. А именно, $g$ определена на
    множестве значений $f$, и для всех $x$ из области определения $f$
    выполняется $f(g(f(x))) = f(x)$.
\end{task}

\begin{task}
    Приведите пример неразрешимого множества $A \subseteq \Nat \times \Nat$,
    такого, что все его горизонтальные и вертикальные сечения
    разрешимы (т.е. для любого $x$ разрешимы $A \cap \{\{x\} \times \Nat\}$
    и $A \cap \{\Nat \times \{x\}\}$)
\end{task}

\begin{task}
    Докажите, что существует язык, который можно распознать с памятью $2^n$ ($n$~---
    длина слова), но нельзя с памятью $n$. (подсказка: диагонализация)
\end{task}

\end{document}



\setmathstyle{25.11}{Задание 11}{АУ}
\setcounter{curtask}{54}

\begin{document}

\begin{definition*}
    Пусть $I_1$ и $I_2$~--- две интерпретации одной сигнатуры. Будем говорить, что $I_1$
    \deftext{является расширением} $I_2$, если $M_1$ (носитель $I_1$) является надмножеством $M_2$
    (носителя $I_2$), все функциональный и предикатные символы согласованы на $M_2$. Будем говорить, что
    это расширение является \deftext{элементарным}, если любая (не обязательно замкнутая формула) на
    любой оценке свободных переменных, принимающих значения в $M_2$ одновременно истинна или одновременно
    ложна в интерпретациях $I_1$ и $I_2$.
\end{definition*}


\task{
    \begin{enumcyr}
        \item Покажите, что естественные интерпретации $(=, +, *, 0, 1)$ для всех алгебраически замкнутых
            полей характеристики $0$ являются элементарно эквивалентными.
        \item Для двух алгебраически замкнутых полей $k_1$ и $k_2$ характеристики $0$ выполняется, что
            $k_1$ является надполем поля $k_2$. Покажите, что естественная интерпретация $(=, +, *, 0,
            1)$ в поле $k_1$ является элементарным расширением естественной интерпретации $(=, +, *, 0,
            1)$ в поле $k_2$.
        \item Докажите теорему Гильберта о нулях: всякая система полиномиальных уравнений с
            коэффициентами в алгебраически замкнутом поле характеристики ноль, имеющее решение в
            расширении поля, имеет решение и в самом поле.
        \item Докажите переформулировку теоремы Гильберта о нулях: если система полиномиальных уравнений
            $\bigwedge\limits_{i = 1}^k P_i(x_1, x_2, \dots, x_n) = 0$ не имеет решения в некотором
            алгебраически замкнутом поле характеристики $0$, то найдутся такие многочлены $Q_1(x_1,
            \dots, x_n), \dots, Q_k(x_1, \dots, x_n)$, что $\sum\limits_i Q_i P_1 = 1$.
    \end{enumcyr}
}

\task{
    Покажите (в случае пропозициональных формул), что если $F_1, F_2, \dots, F_n \vdash F$, то формула
    $\left(\bigwedge\limits_{i = 1}^n F_i\right) \to F$ является тавтологией.
}

\task{
    Покажите, что для любых пропозициональных формул $A, B$ и $C$ формула $(A \to B) \to ((B \to C) \to
    (A \to C))$ является выводимой.
}


\task{
    Покажите, что если формула $\varphi$ является выводимой, то и формула, которая получится при
    подстановке другой формулы вместо переменной формулы $\varphi$, тоже будет выводимой.
}


\task{
    Покажите, что следующие формулы являются выводимыми:
    \begin{enumcyr}
        \item $A \to \lnot \lnot A$ и $\lnot \lnot \lnot A \to \lnot A$;
        \item $((A \lor B) \to C) \to ((A \to C) \land (B \to C))$ и $((A \to C) \land (B \to C)) \to
            ((A \lor B) \to C)$;
        \item $((A \land C) \lor (B \land C)) \to ((A \lor B) \land C)$ и $((A \lor B) \land C) \to ((A
            \land C) \lor (B \land C))$;
        \item $((A \lor C) \land (B \lor C)) \to ((A \land B) \lor C)$ и $((A \land B) \lor C) \to ((A
            \lor C) \lor (B \lor C))$;
        \item $(\lnot B \to \lnot A) \to (A \to B)$.
    \end{enumcyr}
}


\task{
    Заменим 11-ую аксиому $A \lor \lnot A$ на $\lnot \lnot A \to A$. Покажите, что множество выводимых
    формул не изменится.
}

\breakline

\libproblem[52]{math-logic}{n-equiv-n-plus-z}
\libproblem[53]{math-logic}{q-equiv-q-plus-r}

\end{document}