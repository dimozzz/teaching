\documentclass[a4paper, 12pt]{article}
% math symbols
\usepackage{amssymb}
\usepackage{amsmath}
\usepackage{mathrsfs}
\usepackage{mathseries}


\usepackage[margin = 2cm]{geometry}

\tolerance = 1000
\emergencystretch = 0.74cm



\pagestyle{empty}
\parindent = 0mm

\renewcommand{\coursetitle}{DM/ML}
\setcounter{curtask}{1}

\setmathstyle{25.11}{Задание 11}{АУ}
\setcounter{curtask}{54}

\begin{document}

\begin{definition*}
    Пусть $I_1$ и $I_2$~--- две интерпретации одной сигнатуры. Будем говорить, что $I_1$
    \deftext{является расширением} $I_2$, если $M_1$ (носитель $I_1$) является надмножеством $M_2$
    (носителя $I_2$), все функциональный и предикатные символы согласованы на $M_2$. Будем говорить, что
    это расширение является \deftext{элементарным}, если любая (не обязательно замкнутая формула) на
    любой оценке свободных переменных, принимающих значения в $M_2$ одновременно истинна или одновременно
    ложна в интерпретациях $I_1$ и $I_2$.
\end{definition*}


\task{
    \begin{enumcyr}
        \item Покажите, что естественные интерпретации $(=, +, *, 0, 1)$ для всех алгебраически замкнутых
            полей характеристики $0$ являются элементарно эквивалентными.
        \item Для двух алгебраически замкнутых полей $k_1$ и $k_2$ характеристики $0$ выполняется, что
            $k_1$ является надполем поля $k_2$. Покажите, что естественная интерпретация $(=, +, *, 0,
            1)$ в поле $k_1$ является элементарным расширением естественной интерпретации $(=, +, *, 0,
            1)$ в поле $k_2$.
        \item Докажите теорему Гильберта о нулях: всякая система полиномиальных уравнений с
            коэффициентами в алгебраически замкнутом поле характеристики ноль, имеющее решение в
            расширении поля, имеет решение и в самом поле.
        \item Докажите переформулировку теоремы Гильберта о нулях: если система полиномиальных уравнений
            $\bigwedge\limits_{i = 1}^k P_i(x_1, x_2, \dots, x_n) = 0$ не имеет решения в некотором
            алгебраически замкнутом поле характеристики $0$, то найдутся такие многочлены $Q_1(x_1,
            \dots, x_n), \dots, Q_k(x_1, \dots, x_n)$, что $\sum\limits_i Q_i P_1 = 1$.
    \end{enumcyr}
}

\task{
    Покажите (в случае пропозициональных формул), что если $F_1, F_2, \dots, F_n \vdash F$, то формула
    $\left(\bigwedge\limits_{i = 1}^n F_i\right) \to F$ является тавтологией.
}

\task{
    Покажите, что для любых пропозициональных формул $A, B$ и $C$ формула $(A \to B) \to ((B \to C) \to
    (A \to C))$ является выводимой.
}


\task{
    Покажите, что если формула $\varphi$ является выводимой, то и формула, которая получится при
    подстановке другой формулы вместо переменной формулы $\varphi$, тоже будет выводимой.
}


\task{
    Покажите, что следующие формулы являются выводимыми:
    \begin{enumcyr}
        \item $A \to \lnot \lnot A$ и $\lnot \lnot \lnot A \to \lnot A$;
        \item $((A \lor B) \to C) \to ((A \to C) \land (B \to C))$ и $((A \to C) \land (B \to C)) \to
            ((A \lor B) \to C)$;
        \item $((A \land C) \lor (B \land C)) \to ((A \lor B) \land C)$ и $((A \lor B) \land C) \to ((A
            \land C) \lor (B \land C))$;
        \item $((A \lor C) \land (B \lor C)) \to ((A \land B) \lor C)$ и $((A \land B) \lor C) \to ((A
            \lor C) \lor (B \lor C))$;
        \item $(\lnot B \to \lnot A) \to (A \to B)$.
    \end{enumcyr}
}


\task{
    Заменим 11-ую аксиому $A \lor \lnot A$ на $\lnot \lnot A \to A$. Покажите, что множество выводимых
    формул не изменится.
}

\breakline

\libproblem[52]{math-logic}{n-equiv-n-plus-z}
\libproblem[53]{math-logic}{q-equiv-q-plus-r}

\end{document}