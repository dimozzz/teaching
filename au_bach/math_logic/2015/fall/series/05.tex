\documentclass[a4paper, 12pt]{article}
% math symbols
\usepackage{amssymb}
\usepackage{amsmath}
\usepackage{mathrsfs}
\usepackage{mathseries}


\usepackage[margin = 2cm]{geometry}

\tolerance = 1000
\emergencystretch = 0.74cm



\pagestyle{empty}
\parindent = 0mm

\renewcommand{\coursetitle}{DM/ML}
\setcounter{curtask}{1}

\setmathstyle{06.10}{Задание 5}{АУ}
\setcounter{curtask}{23}

\begin{document}

\task{
    Используя теорему Клини докажите, что:
    \begin{enumcyr}
        \item докажите, что существует алгоритм, который на всех входах выводит свой номер;
        \item докажите, что существует алгоритм, который на всех входах выводит квадрат своего номера.
    \end{enumcyr}
}


\libproblem{computability}{kleene-app}
\libproblem{computability}{main-numbering-func-inf}
\libproblem{computability}{non-main-numbering}

\breakline

\libproblem[18]{computability}{post-simple-set}
\libproblem[21]{computability}{post-string-eq}
\libproblem[22]{discrete-math}{string-rsr-alg}

\end{document}