\mytitle{(большое)}

Пусть $I$~--- интерпретация. $Th(I)$~--- множество формул верной в
данной интерпретации.


\begin{task}
    Построите две неизоморфные интерпретации теории $Th(\mathbb{Q}, <, =)$ (плотный линейный порядок без первого и последнего
    элемента) мощности континуум.
\end{task}


\begin{task}
    Пусть $T$ имеет интерпретацию со сколь угодно большим носителем. Докажите, что $T$ имеет интерпретацию с бесконечным
	носителем.
\end{task}

\begin{task}
    Докажите, что для всякого поля $K$ существует его расширение $K'$ такое, что каждый многочлен с коэффициентами из $K$ имеет
    корень в $K'$ (можно пользоваться тем, что для конкретного многочлена такое расширение существует).
\end{task}

\begin{task}
  	Существует ли интерпретация $RCF$ (Real Closed Field~--- поле, в котором у любого числа есть квадратный корень, и у любого
    полинома нечетной степени есть корень), которая содержит $\mathbb{R}$ и такую точку $r$, что $r$ больше любого натурального
    числа $n$.
\end{task}


\begin{task}
    Докажите, что любой частичный порядок продолжается до линейного.
\end{task}

\begin{task}
    Докажите, что если у теории $T$, над языком, в котором не более, чем счетное число функциональных и предикатных символов, есть
    интерпретация с бесконечным носителем, то у нее есть интерпретация и со счетным носителем.
\end{task}




\begin{task}
    Будет ли теория $Th((Z, <, =))$ конечно аксиоматизируемой.
\end{task}

\begin{task}
    Будет ли теория $Th((N, <, =))$ конечно аксиоматизируемой.
\end{task}


\breakline


Пусть есть вполне упорядоченное (в.у.) множество $P$. $\vec{x} = \{y < x \mid x \in P\}$~--- называется начальным отрезком. 

\begin{task}
    Доказать, что любое в.у. множество не изоморфно никакому своему начальному отрезку.
\end{task}

\begin{task}
    Докажите, что в любом линейном пространстве есть басиз.
\end{task}

\begin{task}
    Докажите, что:
	\begin{enumerate}[topsep = 0pt, itemsep = -1ex]
        \item [а)] для любого множества $A$ верно: $A \notin A$;
        \item [б)] класс ординалов не является множеством.
    \end{enumerate}
\end{task}


\breakline

Сложные

\begin{task}
    Докажите, что если формула $\phi$ верна в алгебраически замкнутом поле в характеристикой $0$, то найдется $p_o$, что для
    любого $p > p_o$ $\phi$ будет верна в любом алгебраически замкнутом поле с характеристикой $p$.
\end{task}
