\mytitle{9 (на 09.11)}

\begin{task}
    Докажите, что:
    \begin{enumerate}[topsep = 0pt, itemsep = -1ex]
        \item [а)] множество $\mathbb{Q}$ со стандартным порядком изоморфно множеству $\mathbb{Q}_{+}$ (множество положительных
		    рациональных чисел) со стандартным порядком (т. е. существует биекция, которая сохраняет порядок);
        \item [б)] счетное множество $M$, на котором задан плотный порядок (т.е. между любыми двумя элементами есть еще один
			элемент) и в котором нет минимального и максимального элемента, изоморфно множеству $\mathbb{Q}$ со стандартным
            порядком;
        \item [в)] любая замкнутая формула логики первого порядка истинна в интерпретации $(M, <)$ (где $M$~--- счетное множество
			без минимального и максимального элемента, а порядок $<$ плотный) тогда и только тогда, когда она истинна в
            интерпретации $(\mathbb{Q}, <)$.
    \end{enumerate}
\end{task}

\begin{task}
    Покажите, что в интерпретации $(\mathbb{Z}, =, <)$ предикат $y = x + 1$ невыразим при помощи бескванторной формулы.
\end{task}

\begin{task}
    Выразим ли предикат $x = 0$ в интерпретации $(\mathbb{N}, =, <)$ а) бескванторной формулой; б) любой формулой.
\end{task}

\begin{task}
    Можно ли в данной интерпретации провести элиминацию кванторов $(\mathbb{Q}, =, +)$? Если нет, то можно ли добавить какой-нибудь
    выразимый предикат так, чтобы с новым предикатом элиминация квантором стала возможной.
\end{task}

\begin{task}
    Можно ли в данной интерпретации провести элиминацию кванторов $(\mathbb{Q}, =, S)$, где $S$~--- прибавление единицы? Если нет,
    то можно ли добавить какой-нибудь выразимый предикат так, чтобы с новым предикатом элиминация кванторов стала возможной.
\end{task}
