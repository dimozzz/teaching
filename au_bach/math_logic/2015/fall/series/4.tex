\setcounter{curtask}{17}

\mytitle{4 (на 28.09)}

\newcommand{\dom}[2]{\left[\frac{#1}{#2}\right]}

\begin{task}
    Существует ли алгоритм, проверяющий, что данная программа считает полиномиально вычислимую функцию. (т.е. такую функцию, для
    которой существует алгоритм, вычисляющий ее, который работает полиномиальное время).
\end{task}

\begin{task} (простые множества Поста)
    Назовем множество {\it иммунным}, если оно бесконечно, но не содержит бесконечных перечислимых подмножеств. Перечислимое
    множество называется {\it простым}, если его дополнение иммунно. Докажите, что простые множества существуют.
\end{task}

\begin{task}
    Докажите, что существует: 
    \begin{enumerate}
        \setlength\itemsep{0cm}
        \item[(а)] три;
        \item[(б)] счетное число не пересекающихся перечислимых множеств, никакие два из которых нельзя отделить разрешимым.
    \end{enumerate}
\end{task}

\begin{task}
    Является ли перечислимым множество всех программ, вычисляющих инъективные функции. А его дополнение?
\end{task}

\begin{task}
	Задача Поста состоит в следующем: есть доминошки $n$ видов $\dom{s_1}{t_1}, \dom{s_n}{t_n}$, $s_i$ и $t_i$~--- конечные
    строки, есть неограниченный запас доминошек каждого вида, доминошки переворачивать нельзя. Требуется определить, можно ли
    составить несколько доминошек так, чтобы в верхней и нижней их половине читалась одна и та же строка, такие последовательности
    доминошек будем называть согласованными. Докажите, что задача Поста алгоритмически неразрешима.
\end{task}

\begin{task}
	В алфавите есть буквы $R$ и $S$. Для каждого слова разрешается вычеркивать или дописывать в произвольные места подслова $RRR$
    и $SS$. Также можно заменять подслово $SRS$ на $RR$ и наоборот. Придумайте алгоритм, который по двум словам в этом алфавите
    проверит, можно ли по этим правилам одно получить из другого.
\end{task}


\breakline


\begin{ptask}{14}
    Покажите, что множество описаний машин Тьюринга, которые останавливаются на всех входах, является неперечислимым множеством и
    дополнение его тоже неперечислимо.
\end{ptask}

\begin{ptask}{16}
	Напишите программы с конечным числом переменных решающие следующие задачи:
	\begin{enumerate}
		\item даны числа $a$ и $b$, нужно найти $a \cdot b$;
		\item даны числа $a$ и $b$, нужно найти $a^b$;
		\item даны числа $a$ и $b$, нужно найти остаток и частное от деления $a$ на $b$;
		\item дано число $p$, выяснить простое ли оно;
		\item дано число $n$ нужно найти $n$-ое простое число.
	\end{enumerate}
\end{ptask}


%%% Local Variables:
%%% mode: latex
%%% TeX-master: t
%%% End:
