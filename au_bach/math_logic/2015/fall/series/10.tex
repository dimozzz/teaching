\mytitle{9 (на 09.11)}

\begin{task}
    Можно ли в данной интерпретации провести элиминацию кванторов $(\mathbb{N}, =, S)$, где $S$~--- прибавление единицы? Если нет,
    то можно ли добавить какой-нибудь выразимый предикат так, чтобы с новым предикатом элиминация кванторов стала возможной.
\end{task}

\begin{task}
    Пусть $T$ теория следующего языка: $\{<, R, B\}$, где $R$ (red) и $B$ (blue)
    унарные предикаты.
    
	$T$ содержит все аксиомы плотного линейного порядка без первого и последнего элемента, а также:
	\[ \forall xy \exists zw (x < z < w < y \; \wedge \; R(z) \; \wedge \; B(w)) \]
	\[ \forall x \; (R(x)\; \vee \; B(x)) \]
	\[ \forall x \; (R(x) \leftrightarrow \neg B(x). \]
    
	Докажите, что любые интерпретации данной теории на счетном множестве изоморфны.
\end{task}


Две интерпретации одной сигнатуры называются элементарно
эквивалентными, если каждая замкнутая формула в первой интерпретации
верна тогда и только тогда, когда она верна во второй.

\begin{task}
    Будет ли интерпретация $(\mathbb{N}, =, <)$ элементарно эквивалентна: $(\mathbb{N} + \mathbb{N}, =, <)$. (Две копии
    нат. чисел, все элементы из второй копии больше элементов из первой).
\end{task}

\begin{task}
    Будет ли интерпретация $(\mathbb{N}, =, <)$ элементарно эквивалентна: $(\mathbb{N} + \mathbb{Z}, =, <)$.
\end{task}


\breakline


\begin{ptask}{46}
    Можно ли в данной интерпретации провести элиминацию кванторов $(\mathbb{Q}, =, +)$? Если нет, то можно ли добавить какой-нибудь
    выразимый предикат так, чтобы с новым предикатом элиминация квантором стала возможной.
\end{ptask}

\begin{ptask}{47}
    Можно ли в данной интерпретации провести элиминацию кванторов $(\mathbb{Q}, =, S)$, где $S$~--- прибавление единицы? Если нет,
    то можно ли добавить какой-нибудь выразимый предикат так, чтобы с новым предикатом элиминация кванторов стала возможной.
\end{ptask}
