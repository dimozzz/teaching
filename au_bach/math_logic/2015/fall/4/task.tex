\setcounter{curtask}{17}

\mytitle{4 (на 28.09)}

\begin{task}
    Существует ли алгоритм, проверяющий, что данная программа считает полиномиально вычислимую функцию. (т.е. такую функцию, для
    которой существует алгоритм, вычисляющий ее, который работает полиномиальное время).
\end{task}

\begin{task} (простые множества Поста)
    Назовем множество {\it иммунным}, если оно бесконечно, но не содержит бесконечных перечислимых подмножеств. Перечислимое
    множество называется {\it простым}, если его дополнение иммунно. Докажите, что простые множества существуют.
\end{task}

\begin{task}
    Докажите, что существует: а) три б) счетное число не пересекающихся перечислимых множеств, никакие два из которых нельзя
    отделить разрешимым. 
\end{task}

\begin{task}
    Является ли перечислимым множество всех программ, вычисляющих инъективные функции. А его дополнение?
\end{task}



\breakline


\begin{ptask}{14}
    Покажите, что множество описаний машин Тьюринга, которые останавливаются на всех входах, является неперечислимым множеством и
    дополнение его тоже неперечислимо.
\end{ptask}

\begin{ptask}{16}
	Напишите программы с конечным числом переменных решающие следующие задачи:
	\begin{enumerate}
		\item даны числа $a$ и $b$, нужно найти $a \cdot b$;
		\item даны числа $a$ и $b$, нужно найти $a^b$;
		\item даны числа $a$ и $b$, нужно найти остаток и частное от деления $a$ на $b$;
		\item дано число $p$, выяснить простое ли оно;
		\item дано число $n$ нужно найти $n$-ое простое число.
	\end{enumerate}
\end{ptask}