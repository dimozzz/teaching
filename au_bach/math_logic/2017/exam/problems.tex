% Простые вопросы 2
\z Может ли подмножество разрешимого множества быть неразрешимым?

\z Обязательно ли объединение двух неразрешимых множеств неразрешимо?

\z Множества $A, B \subseteq \mathbb{N}$ разрешимы. Обязательно ли разрешимо множество $A + B = \{x + y
\mid x \in A, y \in B\}$? 

\z Докажите, что для любого перечислимого множества пар $U$ найдётся такая вычислимая функция $f$, что
график $f$ содержится в $U$, а область определения $f$ совпадает с проекцией $U$ на первую координату. 

\z Докажите, что образ и прообраз перечислимого множества относительно вычислимой функции перечислимы.


\z Cуществует ли вычислимая функция, совпадающая с любой другой хотя бы на одном аргументе (в частности,
обе могут быть не определены)?

\z Покажите, что каждое разрешимое множество $m$-сводится к любому множеству, кроме (возможно) 
пустого и $\mathbb{N}$.

\z Приведите пример универсального множества для $\Sigma_1$.

\z Приведите пример универсального множества для $\Pi_1$.

\hrule


\z Докажите, что не существует универсальной тотально вычислимой функции, т.е. такой всюду определённой
функции $U:\mathbb{N} \times \mathbb{N} \to \mathbb{N}$, что для любой всюду определённой вычислимой
функции $f:\mathbb{N} \to \mathbb{N}$ найдётся такое $n$, что при всех $x$ выполнено $U(n, x) = f(x)$.

\z Докажите, что множество номеров алгоритмов, которые останавливаются на входе $0$, но не останавливаются
на входе $1$, не является перечислимым и дополнение тоже не является перечислимым.

\z Покажите, что множество $W = \{n \mid {<}n{>}(n)~\text{останавливается}\}$ является $m$-полным множеством
в классе перечислимых множеств, т.е. все остальные перечислимые множества $m$-сводятся к $W$.

\z Даны два пересекающихся перечислимых множества $X$ и $Y$. Докажите, что найдутся непересекающиеся
перечислимые множества $X' \subset X$ и $Y' \subset Y$ , что $X' \cup Y' = X \cup Y$.

\z Докажите, что не существует алгоритма, который по программе $M$ определил бы, является
ли последовательность $M(1) ,M(2), M(3), \dots$ периодической с некоторого места 
(мы считаем, что если программа не останавливается, то ее ответ $\perp$).

\z Покажите, что существует всюду определенная вычислимая функция $a(n)$, принимающая рациональные
значения, что существует предел $\alpha = \lim\limits_{n \to \infty} a(n) \in \mathbb{R}$, но не
существует алгоритма, который бы по рациональному числу $\epsilon$ выдал такой $n_0$, что при $n > n_0$
выполняется $|a(n) - \alpha| < \epsilon$.

\z Пусть $F$ перечислимое множество пар натуральных чисел. Докажите, что существует вычислимая функция
$f$, определенная на тех и только тех $x$, для которых найдется $y$, при котором $(x, y) \in F$, причем
значение $f(x)$ является одним из таких $y$.

\z Пусть $U$ перечислимое множество пар натуральных чисел, универсальное для класса всех перечислимых
множеств натуральных чисел. Докажите, что его <<диагональное сечение>> $K = \{x \mid (x, x) \in U\}$
является перечислимым неразрешимым множеством.

\z Множество $U \subseteq \mathbb{N} \times \mathbb{N}$ разрешимо. Можно ли утверждать, что множество
<<нижних точек>> множества $U$, то есть множество $V = \{ (x, y) \mid (x, y) \in U \land ((x, z) \notin
U~\text{для всех}~z < y )\}$ разрешимо?


% Клини 1
\hrule

\z Докажите, что существует пара программ $A$, $B$ таких, что $A$ печатает текст $B$ (обычным образом), а
$B$ печатает текст $A$ задом наперед.

\z Пусть $f$ и $g$  вычислимые всюду определенные функции. Докажите, что найдутся такие номера машин
Тьюринга $m$ и  $n$ что алгоритм с номерои $f(n)$ ведет себя так же, как и алгоритм с номером $m$, а
алгоритм с номером $g(m)$ ведет себя так же, как и алгоритм с номером $n$.

\z Докажите, что в любой главной нумерации есть два алгоритма, номера которых отличаются на единицу,
вычисляющие одну и ту же функцию.

\z Докажите, что для любого натурального $k$ найдутся $k$ разных алгоритмов $A_1, A_2, \dots A_k$, что
$A_1$ на входе $0$ печатает текст $A_2$, $A_2$ печатает текст $A_3$, \dots, $A_k$ печатает текст $A_1$.

\z Докажите, что для любой вычислимой функции $g$ найдётся $n$, такое что при любом $x$ выполнено
${<}n{>}(x) = n + g(x)$.


% Задача 1
\hrule

\z Рассмотрим следующие множества номеров машин Тьюринга:
\begin{itemtask}
    \item $A_0$: номера машин, которые останавливаются на собственном номере;
    \item $A_1$: номера машин, которые останавливаются на любом входе;
    \item $A_2$: номера машин, которые не останавливаются ни на одном входе;
    \item $A_3$: номера машин, которые останавливаются на бесконечном множестве входов;
    \item $A_4$: номера машин, которые останавливаются на всех числах из некоторой арифметической
        прогрессии;
    \item $A_5$: номера машин, которые вычисляют некоторую биекцию $f:\mathbb{N} \to \mathbb{N}$.
\end{itemtask}
Пусть $d$~--- это день Вашего рождения по модулю $6$, $m$~--- это месяц Вашего рождения по модулю
$6$. Если $d = m$, то $m := m - 1$.
\begin{enumcyr}
    \item Какие из множеств $A_d$ и $A_m$ перечислимы?
    \item Какие из множеств $A_d$ и $A_m$ коперечислимы (являются дополнением перечислимого)?
    \item $m$-cводится ли $A_d$ к $A_m$? А $A_m$ к $A_d$?
\end{enumcyr}