Двухместная функция $U(n, x)$ называется универсальной для класса функций $\mathfrak{F}$, если $U \in \mathfrak{F}$ и для
любой одноместной функции $f \in \mathfrak{F}$ найдется такое $n$, что $f(x) = U(n, x)$.

Пусть $U$~--- универсальная функция для класса вычислимых функций. Будем говорить, что $U$ задает нумерацию функций в
следующем смысле: $f_n(x) = U(n, x)$. Нумерация, заданная функцией $U(n, x)$ называется главной, если для любой вычислимой
функции $V(n, x)$ существует такая вычислимая, всюду определенная функция $s: \mathbb{N} \to \mathbb{N}$, что $V(n, x) =
U(s(n), x)$.