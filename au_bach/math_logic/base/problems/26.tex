Используя теорему Клини доказите, что:
\begin{enumcyr}
    \item существует алгоритм, который всюду останавливается и выдает $1$ на числе, которое является квадратом его номера, а
	    на всех остальных входах выдает ноль;
    \item существуют два различных алгоритма $\mathcal{A}$ и $\mathcal{B}$, что алгоритм $\mathcal{A}$ печатает $\sharp
	    \mathcal{B}$, а алгоритм $\mathcal{B}$ печатает $\sharp \mathcal{A}$.
\end{enumcyr}