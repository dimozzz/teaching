\setmathstyle{АУ}{Серия 7. 30.03}{2 курс}

\begin{task}
    Рассмотрим задачу $\MaxkSAT$, в которой ко формуле в $3$-КНФ необходимо найти максимальное число клозов, которые можно
    одновременно удовлетворить. Придумайте полиномиальный вероятностный алгоритм, который по $3$-КНФ формуле ``в среднем''
    (мат. ожидание) выдает $\frac{7}{8}$ приближение задачи $\MaxkSAT$.
\end{task}

\begin{task}
	Придумайте ``в среднем'' (мат. ожидание) полиномиальный вероятностный алгоритм, который по $3$-КНФ формуле выдает
    $\frac{7}{8}$ приближение задачи $\MaxkSAT$.
\end{task}

\begin{task}
    Докажите, что если $\NP \subseteq \BPP$, то $\NP = \RP$.
\end{task}


\begin{task}
    Пусть $\ZPP$~--- это класс языков, которые принимаются вероятностной машиной Тьюринга без ошибки, математическое ожидание
    времени работы которых полиномиально. Докажите, что:
    \begin{enumerate}[topsep = 0pt, itemsep = -1ex]
        \item [а)] $L \in \ZPP$ тогда и только тогда, когда существует полиномиальная по времени вероятностная машина Тьюринга
			$M$, которая выдает $\{0, 1, ?\}$, что для всех $x \in \{0, 1\}^*$ с вероятностью $1$, $M(x) \in \{L(x), ?\}$ и 
            $\Pr[M(x) = {?}] \le \frac{1}{2}$;
        \item [б)] $\ZPP = \RP \cap \coRP$.
    \end{enumerate}
\end{task}

\begin{task}
    $\BPL$~--- это класс языков, для которых существует вероятностная машина Тьюринга $M$, которая использует логарифмическую
    память, останавливается при всех последовательностях случайных битов и для всех $x$ выполняется, что $\Pr[M(x) = L(x)] \ge
    \frac{2}{3}$. Покажите, что $\BPL \subseteq \P$. 
\end{task}


\begin{task}(подсказка: понизьте ошибку)
	Докажите, что $\MA \subseteq \AM$.
\end{task}



\breakline

\begin{ptask}{26}(подсказка: $\NEXP^{\NP} vs. \NEXP$)
    Докажите, что если $\P = \NP$, то существует язык из $\EXP$, схемная сложность которого не меньше $\frac{2^n}{10 n}$.
\end{ptask}

\begin{ptask}{37} (подсказка: представьте формулу, как дерево и найдите ``среднюю'' вершину)
    Покажите, что язык можно разрешить булевой формулой размера $s$ тогда и только тогда, когда этот язык можно разрешить булевой
    схемой глубина $O(\log(s))$.
\end{ptask}


