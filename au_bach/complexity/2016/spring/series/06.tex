\setmathstyle{АУ}{Серия 6. 23.03}{2 курс}

\begin{task}
    Докажите, что $\NC^1 \subseteq \Ll$.
\end{task}

\begin{task}
    Докажите, что задача $\lang{CircuitEval}$ $\P$-полная.
\end{task}

\begin{task}
    Пусть $L$~--- $\P$-полный язык. Докажите, что $L \in \Ll \Leftrightarrow \Ll = \P$.
\end{task}

\begin{task}
    Пусть $L$~--- $\P$-полный язык. Докажите, что $L \in \NC \Leftrightarrow \NC = \P$.
\end{task}

\begin{task}
    Докажите, что $\NC^1 \neq \PSpace$.
\end{task}

\begin{task} (подсказка: представьте формулу, как дерево и найдите ``среднюю'' вершину)
    Покажите, что язык можно разрешить булевой формулой размера $s$ тогда и только тогда, когда этот язык можно разрешить булевой
    схемой глубина $O(\log(s))$.
\end{task}

\breakline

\begin{ptask}{31}
	Обозначим $\lang{UCYCLE}$ множество всех неориентрованных графов, в которых есть цикл. Докажите, что $\lang{UCYCLE}$
    принадлежит классу $\Ll$. 
\end{ptask}


\begin{ptask}{24}
    Докажите, что:
    \begin{enumerate}[topsep = 0pt, itemsep = -1ex]
        \item [а)] задача проверки графа на сильную связность лежит в $\NL$;
        \item [б)] задача проверки графа на сильную связность является полной в классе $\NL$ (относительно сведений по Карпу,
            использующих логарифмическую память).
	\end{enumerate}
\end{ptask}

\begin{ptask}{26}(подсказка: $\NEXP^{\NP} vs. \NEXP$)
    Докажите, что если $\P = \NP$, то существует язык из $\EXP$, схемная сложность которого не меньше $\frac{2^n}{10 n}$.
\end{ptask}

