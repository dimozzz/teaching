\documentclass[a4paper, 12pt]{article}
% math symbols
\usepackage{amssymb}
\usepackage{amsmath}
\usepackage{mathrsfs}
\usepackage{mathseries}


\usepackage[margin = 2cm]{geometry}

\tolerance = 1000
\emergencystretch = 0.74cm



\pagestyle{empty}
\parindent = 0mm

\renewcommand{\coursetitle}{DM/ML}
\setcounter{curtask}{1}

\setmathstyle{АУ}{Серия 3. 02.03}{2 курс}
\setcounter{curtask}{17}


\begin{document}

\libproblem{struct-complexity}{hierarchy-oracle}

\begin{definition*}
    $\PSPACE$~--- класс языков, разрешимых на ДМТ с использованием полиномиальной памяти.

    $\DTIME[f(n)], \NTIME[f(n)]$~--- классы языков, разрешимых на ДМТ и НМТ соответственно за
    $\bigO{f(n)}$ шагов на словах длины $n$.
\end{definition*}

\libproblem{struct-complexity}{ph-in-pspace}
\libproblem{struct-complexity}{ph-collapse-relativization}
\libproblem{struct-complexity}{ne-complete}
\libproblem{struct-complexity}{hierarchy-quasipoly}

\dzcomment{
    \textit{Подсказка:} вспомните задачу $\P = \NP \Rightarrow \EXP = \NEXP$.
}

\libproblem{struct-complexity}{baker-gill-solovay-easy}

\breakline

\libproblem[10]{struct-complexity}{prime-np}
\libproblem[16]{struct-complexity}{p-exp-time-hierarchy}

\dzcomment{
    \textit{Подсказка:} вспомните прошлый семестр.
    
    \textit{Подсказка:} а можно ли придумать язык, чтобы обмануть конкретный алгоритм?
}


\end{document}