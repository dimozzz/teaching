\mytitle{3 (на 02.03)}


\begin{task}
	Докажите, что если язык $A$ сводится за полиномиальное время по Тьюрингу (оракульно) к $B \in \Sigma_i^P$, то $A \in
    \Sigma_{i + 1}^P$.    
\end{task}


$\PSpace$~--- класс языков, разрешимых на ДМТ с использованием полиномиальной памяти.

\begin{task}
    Докажите, что $\PH \subseteq \PSpace$.
\end{task}

\begin{task}
    Пусть $\P^{A} = \NP^{A}$. Докажите, что $\PH^{A} = \P^{A}$.
\end{task}

$\DTime[f(n)]~~(\NTime[f(n)]) $~--- класс языков, разрешимых на ДМТ(НМТ) за $O(f(n))$ шагов на словах длины $n$.

\begin{task}
	Постройте примеры полных задач относительно сведений по Карпу в классах:
    \begin{enumerate}[topsep = 0pt, itemsep = -1ex]
        \item [а)] $\EXP, \NEXP$;
        \item [б)] $\class{NE} = \bigcup\limits_{c > 0} \NTime[2^{cn}]$.
	\end{enumerate}
\end{task}

\begin{task}(подсказка: вспомните задачу $\P = \NP \Rightarrow \EXP = \NEXP$)
    Пусть $\NP \subseteq \DTime[n^{\log(n)}]$, докажите, что $\PH \subseteq \bigcup\limits_{k} \DTime[n^{\log^k(n)}]$.
\end{task}

\begin{task}
    Докажите, что существует такой язык $L$, что $\P^{L} = \NP^{L}$.
\end{task}




\breakline


\begin{ptask}{10}
    Докажите, что:
   	\begin{enumerate}[topsep = 0pt, itemsep = -1ex]
        \item [а)] что число $n$ простое тогда и только тогда, когда для каждого простого делителя $q$ числа $n - 1$ существует $a
            \in {2, 3, \dots, n - 1}$ при котором $a^{n - 1} = 1~mod~n$, а $a^{\frac{n - 1}{q}} \ne 1~mod~n$;
        \item [б)] язык простых чисел лежит в $\NP$.
	\end{enumerate}
\end{ptask}


\begin{ptask}{16} (подсказка: вспомните прошлый семестр, подсказка: а можно ли придумать язык, чтобы обмануть конкретный алгоритм)
    Докажите, что $\P \neq \EXP$.
\end{ptask}

