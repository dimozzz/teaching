\mytitle{1 (на 24.02)}

\begin{task}
    Докажите, что:
   	\begin{enumerate}[topsep = 0pt, itemsep = -1ex]
        \item [а)] что число $n$ простое тогда и только тогда, когда для каждого простого делителя $q$ числа $n - 1$ существует $a
            \in {2, 3, \dots, n - 1}$ при котором $a^{n - 1} = 1~mod~n$, а $a^{\frac{n - 1}{q}} \ne 1~mod~n$;
        \item [б)] язык простых чисел лежит в $\NP$.
	\end{enumerate}
\end{task}


\begin{task}
    Докажите $\NP$-полноту следующих задач:
    \begin{enumerate}[topsep = 0pt, itemsep = -1ex]
        \item [а)] на вход подается пара графов $(G_1, G_2)$, необходимо определить, изоморфен ли граф $G_2$ подграфу графа $G_1$
            (подсказка для одного из решений, вершины графа $G_1$ кодируют подстановку для группы переменных из булевой формулы);
        \item [б)] на вход подается граф $G_1$ и число $k \le |G|$, необходимо определить, есть ли в графе $G$ клика размера $k$;
        \item [в)] на вход подается граф $G_1$ и число $k \le |G|$, необходимо определить, существует такое ли $V \subseteq G$,
            что $|V| \le k$ и все ребра графа $G$ инцидентны хотя бы одной вершине из множества $V$.
	\end{enumerate}
\end{task}

\vspace{0.5cm}

$\EXP$~--- класс языков, разрешимых на ДМТ за время $2^{\poly(n)}$. $\NEXP$~--- класс языков, разрешимых на НМТ за время
$2^{\poly(n)}$.

Пусть $\mathbf{C}$~--- класс языков, тогда ${\co}\mathbf{C} = \{L \mid \overline{L} \in \mathbf{C}\}$, где $\overline{L}$~---
дополнение языка.

\begin{task}
    Покажите, что:
    \begin{enumerate}[topsep = 0pt, itemsep = -1ex]
        \item [а)] $\P \subseteq \NP \cap \coNP$;
        \item [б)] $\NP \subseteq \EXP$.
	\end{enumerate}
\end{task}

\begin{task}
   Покажите, что если $\P = \NP$, то $\EXP = \NEXP$. 
\end{task}

\begin{task}
	Докажите, что язык $GNI$ (пар неизоморфных подграфов) лежит в $\P^{\NP}$.
\end{task}

\begin{task}
    Пусть существует $\NP$-полный унарный язык (все слова которого, состоят только из одного символа). Докажите, что $\P = \NP$.
\end{task}


\begin{task} (подсказка: вспомните прошлый семестр)
    Докажите, что $\P \neq \EXP$.
\end{task}

