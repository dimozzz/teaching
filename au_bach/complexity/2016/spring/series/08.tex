\setmathstyle{АУ}{Серия 8. 06.04}{2 курс}

\begin{task}
    Покажите, что:
	\begin{enumerate}[topsep = 0pt, itemsep = -1ex]
        \item [а)] если $\BPTime[f(n)] = \BPTime[g(n)]$, то $\BPTime[f(h(n))] = \BPTime[g(h(n))]$, где $f, g, h$~---
			конструктивные по времени, $f(n), g(n) \ge \log n$, $h(n) \ge n$~--- возрастающая функция;
        \item [б)] $\DTime[f(n)] \subseteq \BPTime[f(n)] \subseteq \DTime[2^{O(f(n))}]$;
        \item [в)] $\BPP \subseteq \BPTime[n^{\log n}] \subsetneq \BPTime[2^n]$.
    \end{enumerate}    
\end{task}


\begin{task}
    Определим язык $\lang{QNR} = \{(y, m) \mid \text{$y$ не является квадратичным вычетом по модулю $m$}\}$, докажите, что
    $\lang{QNR} \in \IP$.
\end{task}

\begin{task}
    $\BPL_H$~--- это класс языков, для которых существует вероятностная машина Тьюринга $M$, которая использует логарифмическую
    память, останавливается с вероятностью $1$, и для всех $x$ выполняется, что $\Pr[M(x) = L(x)] \ge \frac{2}{3}$. Покажите, что
    $\BPL_H \subseteq \P$.
\end{task}

\begin{task}
    Докажите, что $\BPP = \BPP^{\BPP}$.
\end{task}

\begin{task}
    Докажите, что $\BPP/\class{poly} \subseteq \Ppoly$ ($\BPP/\class{poly}$~--- класс языков, которые разрешаются вероятностными
    (есть специальные гейты, куда подаются случайные биты) схемами полиномиального размера).
\end{task}





\breakline

\begin{ptask}{26}(подсказка: $\NEXP^{\NP} vs. \NEXP$)
    Докажите, что если $\P = \NP$, то существует язык из $\EXP$, схемная сложность которого не меньше $\frac{2^n}{10 n}$.
\end{ptask}

\begin{ptask}{37} (подсказка: представьте формулу, как дерево и найдите ``среднюю'' вершину)
    Покажите, что язык можно разрешить булевой формулой размера $s$ тогда и только тогда, когда этот язык можно разрешить булевой
    схемой глубина $O(\log(s))$.
\end{ptask}

\begin{ptask}{40}
    Докажите, что если $\NP \subseteq \BPP$, то $\NP = \RP$.
\end{ptask}

\begin{ptask}{43}(подсказка: понизьте ошибку)
	Докажите, что $\MA \subseteq \AM$.
\end{ptask}
