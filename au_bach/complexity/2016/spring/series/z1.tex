\setcounter{curtask}{1}

\begin{task}
    Изменится ли класс $\widetilde{P}$, если из определения убрать условие ``$\widetilde{P} \subseteq \widetilde{NP}$''. Если да,
    от приведите пример задачи на которой они разделяются.
\end{task}

\begin{task}
    Докажите, что язык $L = \{(\phi, 1^k) \mid$ функция, заданная формулой $\phi$, не может быть посчитана формулой размера $k$ $\}$
    лежит в $PH$.
\end{task}


\begin{task}
    Докажите, что существует такая линейная функция $f: \{0, 1\}^{n} \to \{0, 1\}^{n}$, что ее схемная сложность не менее
    $\frac{n^2}{100 \log(n)}$.  
\end{task}


\begin{task}
    Пусть $\P = \NP$, докажите, что $\P \notin \Size[n^{100}]$.
\end{task}

\begin{task}
	Докажите, что $\SAT$ не является $\NP$-полной задачей относительно сведений, сохраняющих размер входа.
\end{task}

\begin{task}
    Рассмотрим язык $\class{EO}\text{-}3\text{-}\SAT$, который состоит из таких булевых формул в $3$-КНФ, что существует такой
    выполняющий набор, что в каждом клозе выполнен равно один литерал. Докажите, что $\class{EO}\text{-}3\text{-}\SAT$ является
    $\NP$~полным.
\end{task}

\begin{task}
	Пусть $2\text{-}\SAT$ является $\P$-полной. Докажите, что $\Ll = \P$.
\end{task}
