\mytitle{9 (на 15.04)}

\begin{task}
    Покажите, что:
    \begin{enumerate}[topsep = 0pt, itemsep = -1ex]
        \item [а)] задача вычисления числа полных паросочетаний в двудольном графе
            сводится к вычислению перманента;
        \item [б)] четность числа полных паросочетаний в двудольном графе можно
            узнать за полиномиальное время;
        \item [в)] если граф представляет собой шахматную доску с выбитыми клетками
            (вершины~--- клетки, ребра соединяют соседние клетки), то существует
            полиномиальный алгоритм, который считает число полных паросочетаний
            (подсказка: иногда вес ребра удобно взять комплексным).
    \end{enumerate}    
\end{task}

\begin{task}(повторяем лекцию, $\IP = \PSpace$ пользоваться нельзя)
    Докажите, что:
        \begin{enumerate}[topsep = 0pt, itemsep = -1ex]
        \item [а)] $\class{AMA} \subseteq \PSpace$;
        \item [б)] $\class{AMA} \subseteq \PSpace$ с приватными случайными битами.
    \end{enumerate}
\end{task}

\begin{task}
    Существует вариант класса $\MA$ с односторонней ошибкой. $L \in \MA_1$, если существует такая полиномиальная вероятностная
    машина $V$ и полином $p$, что если $x \in L$, то найдется такая строка $y \in \{0, 1\}^{p(n)}$, что $\Pr[V(x, y) = 1] = 1$, а
    если $x \notin L$, то для любой строки $y \in \{0, 1\}^{p(n)}$ выполняется $\Pr[V(x, y) = 1] < \frac{1}{3}$. Покажите, что
    $\MA = \MA_1$.
\end{task}

\begin{task}
    Покажите, что $\MA \subseteq \Sigma_2^P$.
\end{task}


\breakline

\begin{ptask}{26}(подсказка: $\NEXP^{\NP} vs. \NEXP$)
    Докажите, что если $\P = \NP$, то существует язык из $\EXP$, схемная сложность которого не меньше $\frac{2^n}{10 n}$.
\end{ptask}

\begin{ptask}{40}
    Докажите, что если $\NP \subseteq \BPP$, то $\NP = \RP$.
\end{ptask}

\begin{ptask}{44}
    Покажите, что:
	\begin{enumerate}[topsep = 0pt, itemsep = -1ex]
        \item [а)] если $\BPTime[f(n)] = \BPTime[g(n)]$, то $\BPTime[f(h(n))] = \BPTime[g(h(n))]$, где $f, g, h$~---
			конструктивные по времени, $f(n), g(n) \ge \log n$, $h(n) \ge n$~--- возрастающая функция;
        \item [б)] $\DTime[f(n)] \subseteq \BPTime[f(n)] \subseteq \DTime[2^{O(f(n))}]$;
        \item [в)] $\BPP \subseteq \BPTime[n^{\log n}] \subsetneq \BPTime[2^n]$.
    \end{enumerate}
\end{ptask}

\begin{ptask}{45}
    Определим язык $\lang{QNR} = \{(y, m) \mid \text{$y$ не является квадратичным вычетом по модулю $m$}\}$, докажите, что
    $\lang{QNR} \in \IP$.
\end{ptask}

\begin{ptask}{46}
    $\BPL_H$~--- это класс языков, для которых существует вероятностная машина Тьюринга $M$, которая использует логарифмическую
    память, останавливается с вероятностью $1$, и для всех $x$ выполняется, что $\Pr[M(x) = L(x)] \ge \frac{2}{3}$. Покажите, что
    $\BPL_H \subseteq \P$.
\end{ptask}
