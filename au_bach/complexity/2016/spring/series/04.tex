\mytitle{4 (на 09.03)}


$\Ll (\NL)$~--- класс языков для которых существует ДМТ (НМТ), которая использует $O(\log(n))$ памяти.

\begin{task}
    Докажите, что:
    \begin{enumerate}[topsep = 0pt, itemsep = -1ex]
        \item [а)] $\Ll \subseteq \P$;
        \item [б)] если $\SAT \in \Ll$, то $\NP \subseteq \Ll$.
	\end{enumerate}
\end{task}

\begin{task}
    Докажите, что:
    \begin{enumerate}[topsep = 0pt, itemsep = -1ex]
        \item [а)] задача проверки графа на сильную связность лежит в $\NL$;
        \item [б)] задача проверки графа на сильную связность является полной в классе $\NL$ (относительно сведений по Карпу,
            использующих логарифмическую память).
	\end{enumerate}
\end{task}


\begin{task}
	Приведите пример языка из $\Ppoly$, который не лежит в $\P$. 
\end{task}

\begin{task}(подсказка: $\NEXP^{\NEXP} vs. \NEXP$)
    Докажите, что если $\P = \NP$, то существует язык из $\EXP$, схемная сложность которого не меньше $\frac{2^n}{10 n}$.
\end{task}

\begin{task}
	Докажите, что существует язык, для которого любой алгоритм, работающий время $O(n^2)$ решает его правильно на менее, чем на
    половине входов какой-то длины, но этот язык распознается алгоритмом, работающим время $O(n^3)$.
\end{task}



\breakline


\begin{ptask}{10}
    Докажите, что:
   	\begin{enumerate}[topsep = 0pt, itemsep = -1ex]
        \item [а)] что число $n$ простое тогда и только тогда, когда для каждого простого делителя $q$ числа $n - 1$ существует $a
            \in {2, 3, \dots, n - 1}$ при котором $a^{n - 1} = 1~mod~n$, а $a^{\frac{n - 1}{q}} \ne 1~mod~n$;
        \item [б)] язык простых чисел лежит в $\NP$.
	\end{enumerate}
\end{ptask}


\begin{ptask}{20}
	Постройте примеры полных задач относительно сведений по Карпу в классах:
    \begin{enumerate}[topsep = 0pt, itemsep = -1ex]
        \item [а)] $\EXP, \NEXP$;
        \item [б)] $\class{NE} = \bigcup\limits_{c > 0} \NTime[2^{cn}]$.
	\end{enumerate}
\end{ptask}

\begin{ptask}{21}(подсказка: вспомните задачу $\P = \NP \Rightarrow \EXP = \NEXP$)
    Пусть $\NP \subseteq \DTime[n^{\log(n)}]$, докажите, что $\PH \subseteq \bigcup\limits_{k} \DTime[n^{\log^k(n)}]$.
\end{ptask}
