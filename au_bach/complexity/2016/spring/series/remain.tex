\setmathstyle{АУ}{Остаток}{2 курс}


\begin{ptask}{44}
    Покажите, что:
	\begin{enumerate}[topsep = 0pt, itemsep = -1ex]
        \item [в)] $\BPP \subseteq \BPTime[n^{\log n}] \subsetneq \BPTime[2^n]$.
    \end{enumerate}
\end{ptask}

\begin{ptask}{45}
    Определим язык $\lang{QNR} = \{(y, m) \mid \text{$y$ не является квадратичным вычетом по модулю $m$}\}$, докажите, что
    $\lang{QNR} \in \IP$.
\end{ptask}


\begin{ptask}{49}
    Покажите, что:
    \begin{enumerate}[topsep = 0pt, itemsep = -1ex]
        \item [в)] если граф представляет собой шахматную доску с выбитыми клетками
            (вершины~--- клетки, ребра соединяют соседние клетки), то существует
            полиномиальный алгоритм, который считает число полных паросочетаний
            (подсказка: иногда вес ребра удобно взять комплексным).
    \end{enumerate}
\end{ptask}


\begin{ptask}{54}
    Докажите, что: 
    \begin{enumerate}[topsep = 0pt, itemsep = -1ex]
        \item [а)] (снято, но если вдруг решили, то +3 задачи) язык простых чисел лежит в классе $\UP$.
    \end{enumerate}
\end{ptask}

\begin{ptask}{55}
    Покажите, что существует такой оракул $A$ и язык $L \in \NP^A$, что $L$ не
    сводится по Тьюрингу к $3\SAT$, даже если сведение может использовать оракул $A$.
\end{ptask}

\begin{ptask}{57}
    Покажите, что $\AM = \AM_1$.
\end{ptask}

\begin{ptask}{59}
    Покажите, что если $\PSpace \subseteq \Ppoly$, то $\PSpace = \MA$ (подсказака: используйте $\IP = \PSpace$).
\end{ptask}

\begin{ptask}{65}
    Докажите, что $\MAM = \AM$ (и $\MAM_1 = \AM_1$, данный факт можно использовать в задаче 57).
\end{ptask}

\begin{ptask}{66}
    Покажите, что $\AM \subseteq \Pi_2$.
\end{ptask}

\begin{ptask}{67}
    Пусть есть оракул, который считает перманент матрицы $n \times n$ над полем $\mathbb{F}$ верно для доли матриц
    $1 - \frac{1}{3n}$. Пусть $|\mathbb{F}| > 3n$). Докажите, что используя этот оракул можно построить вероятностный
    полиномиальный по времени алгоритм, который для каждой матрицы с большой вероятностью находит ее перманент.
\end{ptask}


\begin{ptask}{69}
    Пусть $\lang{GI}$~--- $\NP$-полный язык. Докажите, что:
    \begin{enumerate}[topsep = 0pt, itemsep = -1ex]
        \item [а)] $\coNP \in \AM$;
        \item [б)] $\Sigma_2 \in \MAM$
        \item [в)] $\PH = \Sigma_2 \cap \Pi_2$.
    \end{enumerate}
\end{ptask}
