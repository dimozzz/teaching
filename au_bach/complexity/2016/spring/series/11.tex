\mytitle{11 (на 27.04)}


\begin{task}
    \begin{enumerate}[topsep = 0pt, itemsep = -1ex]
        \item [а)] Постройте граф со следующими выделенными вершинами: $T, t_1, t_2, r$, со следующим свойством: в любой
            правильной раскраске графа в три цвета вершина $r$ покрашена в тот же цвет, что и $T$, тогда и только тогда,
            когда хотя бы одна из вершин $t_1, t_2$ покрашена в тот же цвет, что и $T$.
        \item [б)] Постройте граф со следующими выделенными вершинами: $T, t_1, t_2, t_3, r$, со следующим свойством: в любой
            правильной раскраске графа в три цвета вершина $r$ покрашена в тот же цвет, что и $T$, тогда и только тогда, когда
            хотя бы одна из вершин $t_1, t_2, t_3$ покрашена в тот же цвет, что и $T$.
        \item [в)] (подсказка: создайте в графе треугольник с вершинами: $True, False, Base$) Докажите, что язык графов,
            которые можно раскрасить в три цвета, $\NP$-полон.
    \end{enumerate}
\end{task}

\begin{task}
    Покажите, что $\AM = \AM_1$
\end{task}

\begin{task}
    Докажите, что:
    \begin{enumerate}[topsep = 0pt, itemsep = -1ex]
        \item [а)] $\P = \PCP(0, \log(n))$;
        \item [б)] $\NP = \PCP(0, \poly(n))$.
    \end{enumerate}
\end{task}


\begin{task}
    Покажите, что если $\PSpace \subseteq \Ppoly$, то $\PSpace = \MA$ (подсказака: используйте $\IP = \PSpace$).
\end{task}

\breakline

\begin{ptask}{26}(подсказка: $\NEXP^{\NP} vs. \NEXP$)
    Докажите, что если $\P = \NP$, то существует язык из $\EXP$, схемная сложность которого не меньше $\frac{2^n}{10 n}$.
\end{ptask}


\begin{ptask}{44}
    Покажите, что:
	\begin{enumerate}[topsep = 0pt, itemsep = -1ex]
        \item [а)] если $\BPTime[f(n)] = \BPTime[g(n)]$, то $\BPTime[f(h(n))] = \BPTime[g(h(n))]$, где $f, g, h$~---
			конструктивные по времени, $f(n), g(n) \ge \log n$, $h(n) \ge n$~--- возрастающая функция;
        \item [б)] $\DTime[f(n)] \subseteq \BPTime[f(n)] \subseteq \DTime[2^{O(f(n))}]$;
        \item [в)] $\BPP \subseteq \BPTime[n^{\log n}] \subsetneq \BPTime[2^n]$.
    \end{enumerate}
\end{ptask}

\begin{ptask}{45}
    Определим язык $\lang{QNR} = \{(y, m) \mid \text{$y$ не является квадратичным вычетом по модулю $m$}\}$, докажите, что
    $\lang{QNR} \in \IP$.
\end{ptask}

\begin{ptask}{46}
    $\BPL_H$~--- это класс языков, для которых существует вероятностная машина Тьюринга $M$, которая использует логарифмическую
    память, останавливается с вероятностью $1$, и для всех $x$ выполняется, что $\Pr[M(x) = L(x)] \ge \frac{2}{3}$. Покажите, что
    $\BPL_H \subseteq \P$.
\end{ptask}

\begin{ptask}{49}
    Покажите, что:
    \begin{enumerate}[topsep = 0pt, itemsep = -1ex]
        \item [в)] если граф представляет собой шахматную доску с выбитыми клетками
            (вершины~--- клетки, ребра соединяют соседние клетки), то существует
            полиномиальный алгоритм, который считает число полных паросочетаний
            (подсказка: иногда вес ребра удобно взять комплексным).
    \end{enumerate}
\end{ptask}


\begin{ptask}{56}
    Докажите, что язык булевых формул с ровно одним выполняющим набором ($\USAT$):
    \begin{enumerate}[topsep = 0pt, itemsep = -1ex]
        \item [а)] $\coNP$-трудным;
        \item [б)] лежит в $\P^{\NP}$.
    \end{enumerate}
\end{ptask}

\begin{ptask}{57}
    Докажите, что: 
    \begin{enumerate}[topsep = 0pt, itemsep = -1ex]
        \item [а)] язык простых чисел лежит в классе $\UP$;
        \item [б)] если $\USAT \in \UP$, то $\NP = \coNP$.
    \end{enumerate}
\end{ptask}

\begin{ptask}{58}
    Покажите, что существует такой оракул $A$ и язык $L \in \NP^A$, что $L$ не
    сводится по Тьюрингу к $3\SAT$, даже если сведение может использовать оракул $A$.
\end{ptask}