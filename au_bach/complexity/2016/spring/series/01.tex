\mytitle{1 (на 10.02)}

Языком будем называть подмножество множества конечных булевых слов. Язык $L \subseteq \bool^*$ разрешим алгоритмом $A$, если $A(x) = L(x)$.

\begin{task}
	Придумайте систему доказательств для языка алгоритмов, которые останавливаются
    хотя бы на одном входе.
\end{task}

\begin{task}
    Известно, что произведение матриц размера $n \times n$ можно посчитать за
    $O(n^{\omega})$, где $\omega =  2.37...$. Придумайте доказательство того, что 
    произведение двух матриц размера $n \times n$ не ноль, которое можно проверить за
    $O(n^2)$.
\end{task}

\begin{task}
    Граф задан матрицей смежности. Как доказать, что он не двудольный? Доказательство
    должно проверяться за $O(V)$, где $V$~--- число вершин в графе.
\end{task}

\begin{task}
    Хорновской формулой называется формула в КНФ, в которой в каждый дизъюнкт максимум одна переменная входит без
    отрицания. Предъявите полиномиальный алгоритм для определения выполнимости хорновских формул.
\end{task}

\begin{task}
	Рассмотрим язык выполнимых формул, где каждый клоз либо хорновский, либо состоит из двух литералов. Пусть у вас есть алгоритм
    $A$, который разрешает данный язык за полиномиальное время. Предъявите алгоритм, который разрешает любую КНФ формулу за
    полиномиальное время.
\end{task}

\begin{task}
    Пусть функции $f, g: \{0, 1\}^* \rightarrow \{0, 1\}^*$ можно посчитать с использованием $O(\log(n))$ памяти (память считается
    только на рабочих лентах, входная лента доступна только для чтения, а по выходной ленте головка машины Тьюринга движется
    только слева направо). Докажите, что функцию $f(g(x))$ можно также посчитать с использованием $O(\log(n))$ памяти.
\end{task}
