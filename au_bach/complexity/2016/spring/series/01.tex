\documentclass[a4paper, 12pt]{article}
% math symbols
\usepackage{amssymb}
\usepackage{amsmath}
\usepackage{mathrsfs}
\usepackage{mathseries}


\usepackage[margin = 2cm]{geometry}

\tolerance = 1000
\emergencystretch = 0.74cm



\pagestyle{empty}
\parindent = 0mm

\renewcommand{\coursetitle}{DM/ML}
\setcounter{curtask}{1}

\setmathstyle{АУ}{Серия на 17.02}{2 курс}


\begin{document}


\begin{definition*}
    Языком будем называть подмножество множества конечных булевых слов. Язык $L \subseteq \{0, 1\}^*$
    разрешим алгоритмом $A$, если $A(x) = L(x)$.
\end{definition*}

\task{
    Придумайте систему доказательств для языка алгоритмов, которые останавливаются
    хотя бы на одном входе.
}

\task{
    Известно, что произведение матриц размера $n \times n$ можно посчитать за $\bigO{n^{\omega}}$, где
    $\omega = 2.37...$. Придумайте доказательство того, что произведение двух матриц размера $n \times n$
    не ноль, которое можно проверить за $\bigO{n^2}$ (будем считать, что произвольный элемент матрицы
    можно записать на рабочую ленту МТ за $\bigO{\log n}$).
}

\task{
    Граф задан матрицей смежности. Как доказать, что он не двудольный? Доказательство должно проверяться
    за $\bigO{V \log V}$, где $V$~--- число вершин в графе (будем считать, что произвольный элемент
    матрицы можно записать на рабочую ленту МТ за $\bigO{\log |V|}$).
}

\libproblem{struct-complexity}{horn-sat-p}

\task{
    Рассмотрим язык выполнимых формул в КНФ, где каждый клоз либо хорновский, либо состоит из двух
    литералов. Пусть у вас есть алгоритм $A$, который разрешает данный язык за полиномиальное
    время. Предъявите алгоритм, который разрешает любую КНФ формулу за полиномиальное время.
}

\libproblem{struct-complexity}{logspace-composition}
\libproblem{struct-complexity}{search-to-decision-ham}

\task{
    Пусть $L_1, L_2 \in \NP$. Принадлежит ли объединение этих языков $\NP$? А пересечение?
}

\libproblem{struct-complexity}{2-sat-p}


\end{document}