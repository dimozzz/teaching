Докажите, что:
\begin{enumcyr}
    \item что число $n$ простое тогда и только тогда, когда для каждого простого делителя $q$ числа $n -
    	1$ существует $a \in {2, 3, \dots, n - 1}$ при котором $a^{n - 1} = 1~mod~n$, а
	    $a^{\frac{n - 1}{q}} \ne 1~mod~n$;
    \item язык простых чисел лежит в $\NP$.
\end{enumcyr}
