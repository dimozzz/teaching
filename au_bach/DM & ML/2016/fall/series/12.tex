\documentclass[a4paper, 12pt]{article}
% math symbols
\usepackage{amssymb}
\usepackage{amsmath}
\usepackage{mathrsfs}
\usepackage{mathseries}


\usepackage[margin = 2cm]{geometry}

\tolerance = 1000
\emergencystretch = 0.74cm



\pagestyle{empty}
\parindent = 0mm

\renewcommand{\coursetitle}{DM/ML}
\setcounter{curtask}{1}

\setmathstyle{}{}{}
\setcounter{curtask}{95}

\begin{document}

\begin{definition*}
    Назовем \deftext{вероятностной булевой схемой} такую схему, часть входов которой называются
    случайными битами. Пусть схема $C$ имеет $n + m$ входов, первые $n$ входов мы будем понимать как
    непосредственно входы, оставшиеся $m$ входов как случайные биты. Будем говорить, что схема $C$
    вычисляет функцию $f\colon \{0, 1\}^n \to \{0, 1\}$ с ограниченной ошибкой, если для каждого $x \in
    \{0, 1\}^n$ выполняется $\Pr\limits_{r}[f(x) = C(x, r)] \ge \frac{2}{3}$, где вероятность берется по
    случайной строке $r$, которая принимает все значения из множества $\{0, 1\}^m$ с равными
    вероятностями.
\end{definition*}

\libproblem{complexity}{prob-ckt-to-det-ckt}
\libproblem{combinatorics}{vasya-illness}
\libproblem{probabilistic}{variable-eq-0-var}
\libproblem{discrete-math}{ind-set-naive-max-size}
\libproblem{boolean-analysis}{unit-combination-upper}

\begin{definition*}
    Каждому $a \in \{0, 1\}^n$ сопоставим линейную функцию $f_a\colon \field_2^n \to \field_2$,
    определяемую следующим образом:
    $$
    f_a(x_1 x_2 \dots x_n) \coloneqq \sum\limits_{i = 1}^n a_i x_i.
    $$

    \deftext{Кодом Уолша--Адамара} строки $a \in \{0, 1\}^n$ назовем таблицу значений функции $f_a$ и
    обозначим $\funccplx{WH}(a)$. Заметим, что длина строки $\funccplx{WH}(a)$ равняется $2^n$. 
\end{definition*}

\libproblem{error-correcting}{wh-local-correction}

\breakline


\libproblem[75]{discrete-math}{connectivity-dec-tree}
\libproblem[81]{complexity}{m-clauses-2-3-sat}
\libproblem[91]{inf-theory}{family-set-size-freq}
\libproblem[94]{combinatorics}{independ-subsets-basic}

\end{document}



%%% Local Variables:
%%% mode: latex
%%% TeX-master: t
%%% End:
