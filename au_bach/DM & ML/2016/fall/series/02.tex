\documentclass[a4paper, 12pt]{article}
% math symbols
\usepackage{amssymb}
\usepackage{amsmath}
\usepackage{mathrsfs}
\usepackage{mathseries}


\usepackage[margin = 2cm]{geometry}

\tolerance = 1000
\emergencystretch = 0.74cm



\pagestyle{empty}
\parindent = 0mm

\renewcommand{\coursetitle}{DM/ML}
\setcounter{curtask}{1}

\setmathstyle{}{}{}

\begin{document}

\libproblem{discrete-math}{mon-func-mon-formula}
\libproblem{discrete-math}{peirce-arrow}

\begin{definition*}
    Рассмотрим пропозициональные формулы, которые используют константу $1$, конъюнкцию $\land$ и сумму по
    модулю два $\oplus$ (приоритет $\land$ выше, чем $\oplus$). \deftext{Мономом} будем называть
    константу $1$ и конъюнкцию нескольких переменных. \deftext{Многочленом Жегалкина} назовем формулу
    вида $m_1 \oplus m_2 \oplus \dots \oplus m_k$, где $m_i$~--- различные мономы, $k \ge 0$. Пример:
    $x_1 x_2 \oplus x_2 \oplus 1$.
\end{definition*}

\libproblem{discrete-math}{polynomial-representation}

\begin{definition*}
    Булеву функцию будем называть \deftext{самодвойственной}, если выполняется равенство $f(1 - x_1, 1 -
    x_2, \dots, 1 - x_n) = 1 - f(x_1, \dots, x_n)$. Булеву функцию будем называть \deftext{линейной},
    если она имеет вид $f(x) = a_0 + a_1 x_1 + a_2 x_2 + \dots + a_nx_n \bmod 2$, где $a_i \in \{0,
    1\}$.
\end{definition*}

\libproblem{discrete-math}{post-th-basis}
\libproblem{math-logic}{propositional-common-vars}
\libproblem{complexity}{cnf-dnf-size-easy}
\libproblem{discrete-math}{and-or-swap-eq}


\end{document}



%%% Local Variables:
%%% mode: latex
%%% TeX-master: t
%%% End:
