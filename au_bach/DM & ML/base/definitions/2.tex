Булева функция называется самодвойственной,	если выполняется равенство $f(1 - x_1, 1 - x_2, \dots, 1 - x_n) = 1 - f(x_1,
\dots, x_n)$. Булева функция называется линейной, если она имеет вид $f(x) = a_0 + a_1 x_1 + a_2 x_2 + \dots + a_nx_n \bmod
2$, где $a_i \in \{0, 1\}$.