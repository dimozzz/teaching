Рассмотрим пропозициональные формулы, которые используют константу $1$, конъюнкцию $\land$ и сумму по модулю два $\oplus$
(приоритет $\land$ выше, чем $\oplus$). Мономом будем называть константу $1$ и конъюнкцию нескольких переменных. Многочленом
Жегалкина называется формула вида $m_1 \oplus m_2 \oplus \dots \oplus m_k$, где $m_i$~--- различные мономы, $k \ge
0$. Пример: $x_1 x_2 \oplus x_2 \oplus 1$.