Доминирующее множество в графе~--- это такое множество, что для каждой вершины, либо она сама лежит в этом множестве, либо
она соединена ребром с вершиной из этого множества. В графе $G$ минимальная степень вершины равняется $d > 1$. Докажите, что
в $G$ есть доминирующее множество  размера не больше $n \frac{1 + \ln(d + 1)}{d + 1}$. Подсказка: рассмотрите случайное
подмножество вершин, в которое каждая вершина включается с вероятностью $p = \frac{\ln(d + 1)}{d + 1}$.