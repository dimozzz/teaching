Назовем вероятностной булевой схемой такую схему, часть входов которой называются случайными битами. Пусть схема $C$ имеет $n
+ m$ входов, первые $n$ входов мы будем понимать как непосредственно входы, оставшиеся $m$ входов как случайные биты. Будем
говорить, что схема $C$ вычисляет функцию $f: \{0, 1\}^n \to \{0, 1\}$ с ограниченной ошибкой, если для каждого $x \in
\{0,1\}^n$ выполняется $\Pr\limits_{r}[f(x) = C(x, r)] \ge \frac{2}{3}$, где вероятность берется по случайной строке $r$, которая
принимает все значения из множества $\{0,1\}^m$ с равными вероятностями. Пусть функция $f:\{0, 1\}^n \to \{0, 1\}$
вычисляется вероятностной схемой $C$ размера $s$ с ограниченной ошибкой. Покажите, что:
\begin{enumcyr}
    \item для каждого многочлена $p(n)$ найдется такая вероятностная схема $C'$ с $n + m'$ входами, размер которой
	    полиномиален относительно $s n$, что при всех $x \in \{0, 1\}^n$ выполняется $\Pr\limits_{r}[f(x) = C(x, r)] \ge 1 -
        2^{-p(n)}$, где вероятность берется по случайной строке $r$, которая принимает все значения из множества $\{0,
        1\}^{m'}$ с равными вероятностями;
    \item найдется обычная схема c $n$ входами, размер которой полиномиален относительно $s n$, что для всех $x \in \{0,
	    1\}^n$ выполняется $f(x) = C(x)$.
\end{enumcyr}
