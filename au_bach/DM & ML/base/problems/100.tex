Коды Уолша-Адамара.
\begin{enumcyr}
    \item Каждому $a \in \{0, 1\}^n$ соответствует линейная функция $f_a: \{0, 1\}^n \to \{0, 1\}$, определяемая так:
	    $f_a(x_1 x_2 \dots x_n) = \sum\limits_{i = 1}^n a_i x_i \bmod 2$. Кодом Уолша-Адамара строки $a \in \{0, 1\}^n$
        называется таблица значений функции $f_a$ и обозначается $\lang{WH}(a)$, нетрудно понять, что длина строки
        $\lang{WH}(a)$ равняется $2^n$. Проверьте, что для двух различных строк $a, b \in \{0, 1\}^n$ их коды $\lang{WH}(a)$
        и $\lang{WH}(b)$ отличаются ровно в половине позиций.
	\item Предположим, что у нас есть оракульный доступ к строке $Z$ (это значит, что можно делать запросы к строке $Z$, за
	    один запрос можно узнать один бит строки $Z$), которая отличается от $\lang{WH}(a)$ не более, чем в доле $\frac{1}{4}
        - \epsilon$ позиций, где $\epsilon$~--- это некоторая константа, причем строка $a \in \{0, 1\}^n$ нам
        неизвестна. Придумайте вероятностный алгоритм, который для всех $x\in \{0,1\}^n$ вычислит $f_a(x)$ с вероятностью как
        минимимум $\frac{9}{10}$, причем этот алгоритм может должен делать лишь константное число запросов к строке $Z$ и
        работать полиномиальное от $n$ время.
\end{enumcyr}