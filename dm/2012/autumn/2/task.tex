\setcounter{curtask}{9}

\mytitle{2 (на 2.10)}

\begin{task}
    Сколько существует чисел от $1$ до $16500$ не делящихся ни на 3, ни на 5, ни на
    11?
\end{task}

\begin{task}
    Сколько различных последовательностей можно составить из букв слова ``одновременно''.
\end{task}

\begin{task}
    Сколько способов выбрать из $30$ человек команду из 6 человек и одного капитана.
\end{task}

\begin{task}
    На полке стоит $n$ книг, сколькими способами можно расставить между ними $k$ перегородок?
\end{task}

\begin{task}
    Сколькими способами можно собрать ожерелье из $n$ различных бусинок?
\end{task}

\begin{task}
    Имеется доска $n \times n$, в клетке $1 \times 1$ находится лучник. За один шаг
    лучник может переместиться либо влево, либо вниз (ось $y$ направлена вниз) на
    одну клетку. Сколькими способами лучник может добраться в клетку $n \times n$:
    
    а)без дополнительных условий

    б)не наступая на центральную клетку
\end{task}

\breakline

\begin{ptask}{4}
    Пусть формула $\phi \rightarrow \psi$~--- тавтология. Докажите,
    что найдется такая формула $\tau$, содержащая только общие для
    $\phi$ и $\psi$ переменные, что $\phi \rightarrow \tau$ и
    $\tau \rightarrow \psi$ будут тавтологиями.
\end{ptask}

