\setcounter{curtask}{30}

\mytitle{6 (на 6.11)}


\begin{task}
    Докажите, что если вершины неориентированнго графа имеют степень не более $k$,
    то их можно покрасить в $k + 1$ цвет так, чтобы концы любого ребра были разных цветов.
\end{task}

\begin{task}
    Докажите, что если вершины неориентированнго графа имеют степень не более $k$,
    то их можно покрасить в $\lceil \frac{k}{2} \rceil + 1$ цвет так, чтобы из
    каждой вершины исходило не более одного ребра в вершину того же цвета.
\end{task}

\begin{task}
    Сколько может быть ребер в неориентированном графе на $2n$ вершинах без циклов
    длины $3$.
\end{task}

\begin{task}
    В классе поровну мальчиков и девочек. Каждый мальчик дружит с четным числом
    девочек. Докажите, что можно выбрать группу из мальчиков так, чтобы каждая
    девочка дружила с четным числом мальчиков из этой группы.
\end{task}

\begin{task}
    В связном графе на каждом ребре написали положительное число. Весом остовного
    дерева назовем сумму весов, входящих в него ребер. Докажите, что осточное дерево,
    на котором достигается минимум суммы чисел совпадает с одним из остовных
    деревьев, на котором достигается минимум квадратов чисел.
\end{task}