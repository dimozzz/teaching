\setcounter{curtask}{1}

\mytitle{}

\begin{task}
    Докажите, что каждое множество, состоящее из $n$ отличных от нуля вещественых
    чисел, содержит подмножество $A$ мощности строго большей, чем $\frac{n}{3}$, в
    котором нет троек $a_1, a_2, a_3$, удовлетворяющих условию $a_1 + a_2 = a_3$.
\end{task}

\begin{task}
    Чему равно число корневых бинарных деревьев с $n$ листьями.
\end{task}

\begin{task}
    Беспорядком называется перестановка без неподвижных точек. Пусть $\pi_n$~---
    число беспорядков на $n$ элементах. Найдите предел $\lim\limits_{n \to \infty}
    \frac{\pi_n}{n!}$.
\end{task}

\begin{task}
    Докажите, что из любого двусвязного графа, степени всех вершин которого больше
    двух, можно удалить вершину так, чтобы граф остался двусвязным.
\end{task}

\begin{task}
    Докажите, что из любых 18 человек либо 4 попарно знакомых, либо 4 попарно незнакомых.
\end{task}

\begin{task}
    Докажите, что в $2k$-регулярном графе есть $2$-регулярный подграф на всех вершинах.
\end{task}

\begin{task}
    В шеренгу стоит $nm + 1$ человек, докажите, что найдется либо $m + 1$ человек,
    стоящие по росту слева направо, либо $n + 1$ справа налево.
\end{task}

\begin{task}
    Докажите, что есть любые два нечетных цикла в графе имеют общую вершину, то
    хроматическое число графа не более $5$.
\end{task}