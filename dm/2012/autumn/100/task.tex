\setcounter{curtask}{1}

\mytitle{}

\begin{task}
    Сколькими способами можно выбрать из полной колоды ($52$ карты) шесть карт так,
    чтобы среди них были карты всех четырех мастей.
\end{task}

\begin{task}
    Докажите, что количество разбиений числа $n$ на слагаемые равно количеству
    разбиений числа $2n$ на $n$ слагаемых (порядок слагаемых не учитывается).
\end{task}

\begin{task}
    Сколько необходимо бинарных функций от ровно $n$ переменных, если каждая
    переменная должна быть существенная (т.е. никакую нельзя выкинуть).
\end{task}

\begin{task}
    Докажите, что среди $9$-ти людей есть либо три попарно знакомых, либо $4$ попарно
    незнакомых.
\end{task}

\begin{task}
    Чему равно число корневых бинарных деревьев с $n$ листьями.
\end{task}

\begin{task}
    Беспорядком называется перестановка без неподвижных точек. Пусть $\pi_n$~---
    число беспорядков на $n$ элементах. Найдите предел $\lim\limits_{n \to \infty}
    \frac{\pi_n}{n!}$.
\end{task}