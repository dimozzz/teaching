\setcounter{curtask}{31}

\mytitle{6 (на 17.10)}

\begin{task}
	Рассмотрим множество матриц $\mathcal{A} \subseteq \{-1, 0, 1\}^{n \times m}$,
	которые обладают следующими свойствами:
    \begin{itemize}
		\item в любом столбце не более двух ненулевых элементов;
    	\item существует такое разбиение множества строк на классы $X, Y$, что
		    для любого столбца $j$, в котором ровно два ненулевых элемента верно:
            $\sum\limits_{i \in X}{a_{i,j}} = \sum\limits_{i \in Y}{a_{i,j}}$.
    \end{itemize}

    Докажите, что:
    
    а) данное множество замкнуто относительно перехода к подматрице;
    
    б) если в любом столбце матрицы $A \in \mathcal{A}$ два ненулевых элемента, то
	$det(A) = 0$;
    
    в) если $A \in \mathcal{A}$, то $A$~--- тотально унимодулярна.
\end{task}


\begin{task}
    Пусть дан граф $G = (V, E)$, $|V| = n$, $|E| = m$, $x \in
    \mathbb{R}^m$. Рассмотрим следующую задачу линейного программирования.
    $\sum_{e \in E} x_e \rightarrow max$, $\forall e \in E ~~ x_e \ge 0$,
    $\forall v \in V ~~ \sum_{e, v \in \delta(e)} x_e \le 1$, где $\delta(e)$~---
    множество концов ребра $e$.

    Докажите, что если граф $G$ двудольный, то матрица данной задачи является
    тотально унимодулярной.
\end{task}

\begin{task}
    Предъявите такую несовместную задачу линейного программирования, что двойственная
    задача будет также несовместна.
\end{task}

\begin{task}
    Пусть $G = (V, E)$~--- двудольный граф. Реберное покрытие $G$~--- это такое
    множество $X \subseteq E$, что любая вершина $v \in V$ инцдидентна хотя бы одному
    ребру из множества $X$.

  	а) Опишите задачу линейного программирования для нахождения минимального
    реберного покрытия.

    б) Докажите, что если граф регулярный, то в минимальном покрытие не более
    $\frac{|V|}{2}$ ребер.

    в) Постройте двойственную задачу линейного программирования и изучите ее
    комбинаторный смысл.
\end{task}



\breakline
