\setcounter{curtask}{1}

%\mytitle{Контрольная работа}

\begin{center}
	{\bf Контрольная работа}.
\end{center}


\noindent{\bf Правила}.
Каждая задача контрольной может быть либо решена, либо не решена. За каждую решенную
задачу вы получаете указанное число баллов.  Если Вы заработали $A$ баллов за работу
в течение семестра и $B$ баллов за контрольную работу, то ваш итоговый балл за
практику $\min \{A + B, 80\}$.

Вы можете самостоятельно выбирать, какие задачи решать, исходя из того, сколько вам
нужно набрать баллов. Задачи следует записать и послать Д.О. Соколову по электронной
почте sokolov.dmt@gmail.com не позже среды 4-го декабря. В четверг 5-го декабря на
занятии можно будет получить результаты и исправить ошибки.

\breakline

\begin{task} ($\mathbf{15}$)
	Вещественное число $\alpha$ называется вычислимым, если существует алгоритм,
    который по рациональному числу $\epsilon$ выдает рациональное
    $\epsilon$-приближение к числу $\alpha$.  Предъявите такое число $x \in
    \mathbb{R}$, что: $x$ не является вычислимым, но при этом множество рациональных
    чисел меньших $x$ является перечислимым.
\end{task}

\begin{task} ($\mathbf{10}$)
	Некоторое множество $S$ натуральных чисел разрешимо. Разложим все числа из $S$ на
    простые множители и составим множество $D$ всех простых чисел, встречающихся в
    этих разложениях. Можно ли утверждать, что множество $D$ разрешимо?
\end{task}

\begin{task}($\mathbf{10}$)
    Существует ли алгоритм, проверяющий, работает ли данная программа полиномиальное
    от длины входа время?
\end{task}

\begin{task}($\mathbf{5}$)
    Докажите, что если $\NP \neq \coNP$, то $\P \neq \NP$.
\end{task}


\begin{task} ($\mathbf{5}$)
	Докажите $\NP$-полноту языка, который состоит из пар графов $G_1,G_2$, что у
    графа $G_2$ есть подграф, изоморфный $G_1$.
\end{task}

\begin{task}($\mathbf{5}$)
    Докажите, что $\PCP(\log(n), 0) = \P$.
\end{task}

\begin{task}($\mathbf{5}$)
	Докажите, что $\PSPACE \subseteq \EXP$, где $\EXP$~--- это множество языков,
    которые распознаются за время $2^{p(n)}$, где $p$~--- некоторый полином.
\end{task}

\begin{task}($\mathbf{15}$)
    Предъявите пример такой несовместной задачи линейного программирования, что
    двойственная задача также является несовместной.
\end{task}

\begin{task}
    Пусть дан граф $G = (V, E)$, $|V| = n$, $|E| = m$, $x \in
    \mathbb{R}^m$. Рассмотрим следующую задачу линейного программирования.
    $\sum_{e \in E} x_e \rightarrow max$, $\forall e \in E ~~ x_e \ge 0$,
    $\forall v \in V ~~ \sum_{e, v \in \delta(e)} x_e \le 1$, где $\delta(e)$~---
    множество концов ребра $e$.

    а) ($\mathbf{10}$) Докажите, что если граф $G$ двудольный, то оптимум достигается в вершине с
    целочисленными координатами.

    б) ($\mathbf{5}$) Предъявите пример, когда $G$ не является двудольным и максимум не целочисленный.
\end{task}

\begin{task}($\mathbf{15}$)
    Докажите, что существует такая линейная функция $f: \{0, 1\}^{n}
    \rightarrow \{0, 1\}^n$, что ее схемная сложность не менее
    $\frac{n^2}{100 \log(n)}$.  
\end{task}

\begin{task}
    Зафиксируем некоторый граф $G$ на $n$ вершинах. Пусть у Алисы и Боба есть
    множества вершин $X$ и  $Y$ соответственно, при этом множество $X$ является
    кликой (между любыми двумя вершинами есть ребро), а множество $Y$ является
    независимым множеством (никакие две вершины не соединены ребром). Они хотят
    посчитать функцию $CIS_G(X, Y)$, которая возвращает размер пересечения множеств
    $X$ и $Y$. Докажите, что для этого им достаточно:

    а) $O(n)$; ($\mathbf{5}$)

    б) $O(\log^2(n))$ битов коммуникации. ($\mathbf{25}$)
\end{task}

\begin{task} ($\mathbf{30}$)
    Пусть у Алисы и Боба есть множества $X, Y \subseteq \{1, \dots, n\}$. Они хотят
    посчитать функцию $MED(X, Y)$, которая возвращает медиану мультимножества $X \cup
    Y$. Докажите, что для этого им достаточно $O(\log(n))$ битов коммуникации.
\end{task}

\begin{task} ($\mathbf{10}$)
    Рассмотрим функцию $f(x_1, \dots, x_n) = x_1 \oplus x_2 \oplus \dots \oplus
    x_n$. Докажите, что $D(f) = n$ (детерминированная запросовая сложность).
\end{task}

\begin{task} ($\mathbf{20}$)
    Докажите, что для достаточно больших $n$ $R(Maj(x_1, x_2, \dots, x_n)) \le n -
    \frac{1}{4}$ (вероятностная запросовая сложность).
\end{task}


\begin{task} ($\mathbf{10}$)
	Пусть пара случайных величин $X, Y$  совместно распределена на некотором конечном
    множестве. Докажите, что $H(X, Y) \le H(X) + H(Y)$.
\end{task}


\begin{task} ($\mathbf{15}$)
    Коды БЧХ (коды Боуза–Чоудхури–Хоквингема) позволяют исправлять некоторое
    фиксированное число ошибок в двоичном кодовом слове. Чтобы описать кодовые слова
    БЧХ, мы рассмотрим поле $\mathbb{F}_n$ из $n = 2^m$ элементов и многочлены $P(z)
    = a_0 + a_1 z + a_2 z^2 + \dots + a_{n - d} z^{n - d}$ над этим полем. Кодовыми
    словами будут таблицы значений таких многочленов во всех элементах поля (таким
    образом, длина кодового слова равна $n$). Мы ограничимся рассмотрением только
    таких многочленов, которые в каждой точке поля принимают значение $0$ или
    $1$. Все такие последовательности битов длины $n$ и образуют код БЧХ.
    Докажите, что расстояние данного кода не менее $d$
\end{task}

\begin{task} ($\mathbf{5}$)
	Покажите, что если код Рида-Соломона имеет расстояние $d$, то он исправляет $e$
    ошибок и $p$ пропусков, если $2e + p < d$. (Пропуск~--- это отсутствие символа, а
    ошибка - это искажение).
\end{task}