\setcounter{curtask}{13}

\mytitle{5 (на 26.09)}



\begin{task}
    Приведите пример такой несовместной задачи линейного программирования, что
    двойственная задача также несовмесина.
\end{task}


\begin{task}
    а) Пусть задан полиэдр $P = \{x \mid Ax \le b\}$. Сформулируйте и докажите
    алгебраический критерий того, что точка $x_0$ является вершиной.
\end{task}


Паросочетанием в графе $G$ называется такое подмножество ребер, что никакие два
ребра из него не имеют общих концов.

\begin{task}
	а) Матрица $A \in R_{+}^{n \times n}$ называется бистохастической, если сумма
    элементов в любой строке и любом столбце $A$ равна $1$. Целая (элементы целые)
    бистохастическая матрица называется перестановочной. Докажите, что
    перестановочные матрицы и только они являются вершинами политопа бистохастических
    матриц.

    б) Пусть есть $n$ работников и $n$ работ, известны значения $a_{ij}$~---
    стоимость выполнения $i$-ым работником $j$-ой работы. Каждый работник может
    выполнять не более одной работы. Предложите полиномиальный алгоритм поиска
    минимальной стоимости выполнения всех работ. 
\end{task}



Матрица $A$ называется тотально унимодулярной, если определитель любой ее квадратной
подматрицы равен одномй из трех значений: $-1$, $0$, $1$.