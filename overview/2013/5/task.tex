\setcounter{curtask}{25}

\mytitle{5 (на 10.10)}

\begin{task}
	Рассмотрим полиэдр $P = \{x \mid Ax \le b, x \ge 0\}$. Матрица $A$ имеет размер
    $m$ на $n$ и ее строки линейно независимы. Доказать, что число вершин этого
    полиэдра не превосходит числа сочетаний из $n$ по $m$.    
\end{task}

\begin{task}
   Докажите, что если полиэдр $P = \{x \mid Ax \le b\} \subseteq \mathbb{R}^n$ не пуст, то
   он имеет вершину тогда и только тогда, когда ранг матрицы $A$ равен $n$.
\end{task}

\begin{task}
    Докажите, что любая точка выпуклого многогранника в $\mathbb{R}^n$ есть выпуклая
    комбинация не более, чем $n + 1$ вершины.
\end{task}

\begin{task}
    Пусть задан полиэдр $P = \{x \mid Ax \le b\}$. Для точки $x_0$ рассмотрим строки
    матрицы $A$, на которых достигается равенство в системе неравенств, задающей
    полиэдр (назовем получившуюся матрицу $B$). Докажите, что $x_0$ является вершиной
    $P$ тогда и только тогда, когда $x_0 \in \mathcal{P}$, $rank(B) = n$.
\end{task}


Матрица $A$ называется тотально унимодулярной, если определитель любой ее квадратной
подматрицы равен одному из трех значений: $-1$, $0$, $1$.

\begin{task}
    Рассмотрим полиэдр $P = \{x \mid Ax = b, x \ge 0\}$, где $b$~--- вектор с
    целочисленными координатами, а матрица $A$~--- тотально унимодулярна. Докажите,
    что все вершины данного полиэдра имеют целочисленные координаты.
\end{task}


\begin{task}
    Пусть дан граф $G = (V, E)$, $|V| = n$, $|E| = m$, $x \in
    \mathbb{R}^m$. Рассмотрим следующую задачу линейного программирования.
    $\sum_{e \in E} x_e \rightarrow max$, $\forall e \in E ~~ x_e \ge 0$,
    $\forall v \in V ~~ \sum_{e, v \in \delta(e)} x_e \le 1$, где $\delta(e)$~---
    множество концов ребра $e$.

    а) Какой ``физический'' смысл у данной задачи? А если вектор $x$ имеет
    целочисленные координаты?

    б) Докажите, что если граф $G$ двудольный, то оптимум достигается в вершине с
    целочисленными координатами.
\end{task}



\breakline

\begin{ptask}{17}
    Пусть $LINEQ$~--- язык выполнимых систем рациональных линейный уравнений. $LINEQ$
    состоит из пар $(A, b)$, где $A$~--- матрица $m \times n$, а $b$~--- такой
    рациональный вектор размерности $m$, что система $Ax = b$ имеет
    решения. Докажите, что язык $LINEQ$ лежит в классе $NP$.
\end{ptask}

\begin{ptask}{20}
    Хорновской формулой называется формула в КНФ, в которой в каждый дизъюнкт
    максимум одна переменная входит без отрицания. Покажите, что множество
    хорновских выполнимых формул содержится в классе $P$.
\end{ptask}

\begin{ptask}{24}
	Докажите, что если существует унарный $NP$-полный язык, то $P = NP$.    
\end{ptask}
