\setcounter{curtask}{21}

\mytitle{4 (на 22.10)}


Две интерпретации одной сигнатуры называются элементарно
эквивалентными, если каждая замкнутая формула в первой интерпретации
верна тогда и только тогда, когда она верна во второй.

\begin{task}
    Будет ли интерпретация $(\mathbb{N}, =, <)$ элементарно
    эквивалентна: $(\mathbb{N} + \mathbb{N}, =, <)$. (Две копии нат. чисел, все
    элементы из второй копии больше элементов из первой).
\end{task}

\begin{task}
    Будет ли интерпретация $(\mathbb{Q}, =, <)$ элементарно
    эквивалентна:
    а) $(\mathbb{Q} + \mathbb{Q}, =, <)$
    б) $(\mathbb{Q} + \mathbb{R}, =, <)$
\end{task}

\begin{task}
    Будет ли интерпретация $(\mathbb{N}, =, <)$ элементарно
    эквивалентна: $(\mathbb{N} + \mathbb{Z}, =, <)$
\end{task}

\begin{task}
    а) Докажите, что в интерпретации $(Q, =, <, +,$ рациональные
    константы) допустима элиминация кванторов.
    б) Докажите, что интерпретации $(Q, =, <, +,$ рациональные
    константы) и $(\mathbb{R}, =, <, +,$ рациональные константы)
    элементарно эквивалентны.
    в) Пусть единичный квадрат разрезан на несколько меньших
    квадратов. Докажите, что все они имеют рациональные стороны.
\end{task}

\begin{task}
    Построите две неизоморфные модели теории $Th(\mathbb{Q}, <, =)$
    (плотный линейный порядок без первого и последнего элемента)
    мощности континуум.
\end{task}

\begin{task}
    Предъявите не менее $2^{|\mathbb{R}|}$ неизоморфных моделей плотного линейного
    порядка континуальной мощности.
\end{task}

\breakline

\begin{ptask}{8}
    Докажите, что $\mathbb{R}^{\mathbb{N}} \le \mathbb{R}$
\end{ptask}

\begin{ptask}{9}
    Докажите, что $\mathbb{N}^{\mathbb{R}} \le 2^{\mathbb{R}}$
\end{ptask}


\begin{ptask}{10} (полнота резолюций)
    Пусть $\neg F$~--- противоречивая формула в КНФ, докажите что у нее есть вывод в 
    резолюциях. (подсказка: индукция по числу переменных).
\end{ptask}

\begin{ptask}{13}
    Предъявите теорию спектр которой:
    б) все числа не являющиеся степенями простого
    в) состоит из квадратов чисел
\end{ptask}

\begin{ptask}{19}
    $(\mathbb{Q}, =, S)$, где $S$~--- прибавление единицы.
\end{ptask}

\begin{ptask}{20}
    $(\mathbb{N}, =, S)$
\end{ptask}