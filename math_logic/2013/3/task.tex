\setcounter{curtask}{14}

\mytitle{3 (на 8.10)}


\begin{task}
    Докажите, что предикат $x = 2$ невыразим в множестве целых чисел с
    предикатами равенства и $x$ делит $y$.
\end{task}

\begin{task}
    Докажите, что предикат $y = x + 2011$ невыразим в интерпретации
    $(\mathbb{Z}, =, x \mapsto x^2)$.
\end{task}

\begin{task}
    Выразим ли предикат $x = 0$ в интерпретации $(\mathbb{N}, =, <)$
    а) бескванторной формулой
    б) любой формулой
\end{task}

В следующих задачах требуется описать множество выразимых предикатом в
данной интерпретации. Обычно требуется доказать, что это множество
совпадает с множеством бескванторных формул. Иногда такое доказать не
получится, тогда необходимо добавить выразимый предикат (выразимый с
квантором) и доказать, что выразимые~--- это бескванторные с новым
предикатом.

\begin{task}
    $(M, =)$, где $M$~--- призвольное бесконечное множество.
\end{task}

\begin{task}
    $(\mathbb{Q}, =, +)$
\end{task}

\begin{task}
    $(\mathbb{Q}, =, S)$, где $S$~--- прибавление единицы.
\end{task}

\begin{task}
    $(\mathbb{N}, =, S)$
\end{task}


\breakline


\begin{ptask}{1}
    Докажите, что не существует биекции между $\mathbb{N}$ и $\mathbb{R}$.
\end{ptask}

\begin{ptask}{2}
    Докажите, что $\mathbb{N}^c \le \mathbb{N}$.
\end{ptask}

\begin{ptask}{3}
    Докажите, что:
    а) $\mathbb{R} \le [0, 1]$
    б) $\mathbb{R} \le 2^{\mathbb{N}}$
    в) $\mathbb{R} \times \mathbb{R} \le \mathbb{R}$ (подсказка:
	    придумать явную биекцию квадрата на одну из сторон)
\end{ptask}

\begin{ptask}{3}
    Приведите пример формулы длины $n$ такой, что ее минимальный
    размер в КНФ $\Omega(2^n)$.
\end{ptask}

\begin{ptask}{4} (Корректность)
    Если $\neg F$ (записанная в КНФ) имеет вывод в резолюциях, то
    формула $F$ является тавтологией.
\end{ptask}

\begin{ptask}{5} (Теорема о компактности)
    Пусть $\Gamma$ множество формул. $\Gamma$ непротиворечиво тогда и
    только тогда, когда любое конечное подмножество $\Gamma$ непротиворечиво.
\end{ptask}


\begin{ptask}{8}
    Докажите, что $\mathbb{R}^{\mathbb{N}} \le \mathbb{R}$
\end{ptask}

\begin{ptask}{9}
    Докажите, что $\mathbb{N}^{\mathbb{R}} \le 2^{\mathbb{R}}$
\end{ptask}


\begin{ptask}{10} (полнота резолюций)
    Пусть $\neg F$~--- противоречивая формула в КНФ, докажите что у нее есть вывод в 
    резолюциях. (подсказка: индукция по числу переменных).
\end{ptask}

\begin{ptask}{13}
    Предъявите теорию спектр которой:
    б) все числа не являющиеся степенями простого
    в) состоит из квадратов чисел
\end{ptask}