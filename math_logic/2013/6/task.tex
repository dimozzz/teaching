\setcounter{curtask}{31}

\mytitle{6 (на 19.11)}

\begin{task}
    Пусть $T$ имеет сколь угодно большую конечную модель. Докажите, что $T$ имеет
    бесконечную модель.
\end{task}

\begin{task}(резолюция для исчисления предикатов)
    Система доказывает противоречивость замкнутой формулы следующим
    образом: формула приводится в предваренную нормальную форму, затем
    проводится скулемизация (избавляемся от кванторов $\exists$). Получается формула
    $\forall x_1 x_2 \dots \phi(x_1, \dots, x_n)$. Для формулы $\phi$ делается
    резолюционный вывод. Докажите корректность и полноту данного метода.
\end{task}


Пусть $I$~--- интерпретация. $Th(I)$~--- множество формул верной в
данной интерпретации.

\begin{task}
    Будет ли теория $Th((Z, <, =))$ конечно аксиоматизируемой.
\end{task}

\begin{task}
    Будет ли теория $Th((N, <, =))$ конечно аксиоматизируемой.
\end{task}


\breakline

\begin{ptask}{24}
    в) Пусть единичный квадрат разрезан на несколько меньших
    квадратов. Докажите, что все они имеют рациональные стороны.
\end{ptask}

\begin{ptask}{26}
    Предъявите не менее $2^{|\mathbb{R}|}$ неизоморфных моделей плотного линейного
    порядка континуальной мощности.

    (пункт 0: $\mathbb{Q} + \cdot + \mathbb{R}$ не изоморфно $\mathbb{Q} +
    \mathbb{R}$).
\end{ptask}

\begin{ptask}{27}
    Предъявите теорию $T$ у которой нет счетной модели, но есть бесконечная модель.
\end{ptask}

\begin{ptask}{28}
    Докажите, что если формула $\phi$ верна в алгебраически замкнутом
    поле в характеристикой 0, то найдется $p_o$, что для любого $p >
    p_o$ $\phi$ будет верна в алгебраически замкнутом поле с
    характеристикой $p$.
\end{ptask}