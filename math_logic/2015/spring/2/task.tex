\setcounter{curtask}{6}

\mytitle{2 (на 17.03)}

Пусть есть вполне упорядоченное (в.у.) множество $P$.
$\vec{x} = \{y < x \mid x \in P\}$~--- называется начальным отрезком.

\begin{task}
    Доказать, что любое в.у. множество не изоморфно никакому своему начальному отрезку.
\end{task}

\begin{task}
    Докажите, что в любом линейном пространстве есть басиз.
\end{task}

\begin{task}
    Докажите, что класс ординалов не является множеством.
\end{task}

\begin{task}
    Арифметика кардиналов.
    1. $\aleph_{\alpha} + \aleph_{\beta} = \aleph_{\alpha}  \aleph_{\beta} =
    	\max(\aleph_{\alpha}, \aleph_{\beta})$
    2. Если $\alpha \le \beta$, то $\aleph_{\alpha}^{\aleph_{\beta}} \le 2^{\aleph_{\beta}}$
\end{task}

\begin{task}
    Все ли подмножества вещественных чисел измеримы относительно меры Лебега?
\end{task}

\begin{task}
    Алиса и Боб играют в игру. Они загадали некоторой предикат $P$ после чего, они
    составляют бесконечное слово $\omega$ по очереди, называя биты. Алиса ходит
    первой, и побеждает, если $P(w) = 1$.Верно ли, что для любого предиката $P$
    существует выигрышная стратегия для одного из игроков?
\end{task}

\begin{task}
    Будет ли теория групп, где порядок каждого элемента равен 2:
    а) $\lambda$-категоричной для любого $\lambda \ge \aleph_0$?
    б) полной?
\end{task}

\begin{task}
    Пусть $L = \{s\}$. $T$ --- $L$-теория, которая говорит, что $s$~--- биекция без
	циклов (т.е $s^{(n)}(x) != x$). Для каких $\lambda$ данная теория будет
    $\lambda$-категоричной.
\end{task}


\breakline

Назовем функцию $f:M^{n} \to M^{m}$ выразимой, есть ее график выразим.

\begin{ptask}{3}
	Докажите, что:
    а) если $f, g$ выразимы, то $f(g)$ также выразима.
    б) если $f:M^{n} \to M$ выразима, то образ выразим.
    б) если $f:M^{n} \to M$ выразимая биекция, то $f^{-1}$ выразима.
\end{ptask}

\begin{ptask}{4}
    Будет ли теория $Th(\mathbb{Z}, <, =)$ (дискретный линейный порядок
    без первого и последнего элемента)
    а) категоричной в счетной мощности?
    б) полной?
\end{ptask}

\begin{ptask}{5}
    Докажите полноту DLO (Dense Linear Order, плотный линейный порядок без концов).
\end{ptask}
