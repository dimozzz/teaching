\setcounter{curtask}{1}

\mytitle{}

\paragraph{Часть 1.}


\begin{task}
    Арифметика кардиналов. Докажите, что:
    
    1. $\aleph_{\alpha} + \aleph_{\beta} = \aleph_{\alpha}  \aleph_{\beta} =
    	\max(\aleph_{\alpha}, \aleph_{\beta})$
     
    2. Если $\alpha \le \beta$, то $\aleph_{\alpha}^{\aleph_{\beta}} \le 2^{\aleph_{\beta}}$
\end{task}

\begin{task}
	Алиса и Боб играют в игру. Они загадали предикат $P$, после чего они составляют
    бесконечное число $\omega$, по очереди называя биты. Алиса ходит первой и
    побеждает, если $P(\omega) = 1$. Верно ли, что для любого предиката $P$
    существует выигрышная стратегия?
\end{task}

\begin{task}
    Бесконечное число волшебников стоит в ряд. Волшебники пронумерованы начиная с
    нуля. На голове каждого волшебника надета шляпа черного или белого
    цвета. Волшебники видят шляпы всех с большими номерами. Начиная с волшебника под
    номером $0$, они по очереди называют цвет (если названный цвет не совпадает с
    цветом собственной шляпы, тогда волшебнику отрубают голову). Волшебник слышит
    всех предыдущих. Предложите стратегию, в которой погибает конечное число
    волшебников.
\end{task}

\begin{task}
    Верно ли, что любое множество изоморфно некоторому ординалу?
\end{task}

\breakline

\paragraph{Часть 2.}

\begin{task}
    Пусть $T$~--- полная теория счетного языка, и $T$~---
    $\omega$-стабильна. Докажите, что для любой модели $M \models T$, и любого
    множества $A \subseteq M$, изолированные типы плотны в $S_n^M(A)$. 
\end{task}

\begin{task}
    Предъявите контрпример к ommiting type theorem, если разрешить иметь несчетный
    язык.
\end{task}

\begin{task}
    Пусть $A \subseteq M$, $a, b \in M^{*}$, $t_p^M(a, b / A)$~---
    изолирован. Докажите, что тип $t_p^M(a / (A \cup {b})$~--- изолирован.
\end{task}

\begin{task}
     Пусть $A \subseteq M$, $a, b \in M^{*}$. Докажите, что $t_p^M(a, b / A)$
     изолирован тогда и только тогда, когда $t_p^M(b / A)$ и $t_p^M(a / (A \cup
     {b}))$ изолированы.
\end{task}

\begin{task}
    Докажите, что $|S_n(ACF_p)| = \aleph_0$.

    a) $p > 0$.

    б) $p = 0$.
\end{task}

\begin{task}{19}
    Пусть $T$~--- полная теория счетного языка, и $T$ не $\lambda$-стабильна для
    некоторого $\lambda$.

	а) Докажите, что есть модель $M \models T$, множество $B \in M, |B| = \lambda$
    и такая $L_B$-формула $\phi$, что она входит более, чем в $\lambda$ различных
    полных типов.

    б) Докажите, что есть такая $L_B$-формула $\psi$, что $\phi \land \psi$ и $\phi
    \land \not \psi$ входят более, чем в $\lambda$ различных полных типов.


    в) Докажите, что $T$ не $\omega$-стабильна.
\end{task}


\breakline

\paragraph{Часть 3.}

\begin{task}
    Определим игру $P_{\omega}$, как $EF_{\omega}$, но первый игрок обязан выбирать
    точки из первой модели, а второй из второй. Докажите, что второй игрок имеет
    выигрышную стратегию тогда и только тогда, когда первая модель вкладывается во
    вторую. 
\end{task}