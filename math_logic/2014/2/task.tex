\setcounter{curtask}{6}

\mytitle{2 (на 12.02)}

\begin{task}
    Докажите, что теория алгебраически замкнутых полей фиксированной характеристики
    $\lambda$-категорична для любого $\lambda > \aleph_0$
\end{task}

\begin{task}
    Будет ли теория групп, где порядок каждого элемента равен 2:
    а) $\lambda$-категоричной для любого $\lambda \ge \aleph_0$?
    б) полной?
\end{task}

\begin{task}
    Пусть $L = \{s\}$. $T$ --- $L$-теория, которая говорит, что $s$~--- биекция без
	циклов (т.е $s^{(n)}(x) != x$). Для каких $\lambda$ данная теория будет
    $\lambda$-категоричной.
\end{task}

EF --- (Ehrenfeucht-Fraisse).

\begin{task}
    Определим игру $P_{\omega}$, как $EF_{\omega}$, но первый игрок обязан выбирать
    точки из первой модели, а второй из второй.
	а) докажите, что второй игрок имеет выигрышную стратегию тогда и только тогда,
    когда первая модель вкладывается во вторую.
    б) а что если, первый игрок выбирает на четных шагах из первой структуры, а на
    нечетных из второй?
\end{task}

\begin{task}
    Пусть $L = \{E\}$, где $E$~--- бинарное отношение. $T$~--- $L$-теория,
    утверждающая, что $E$~--- отношение эквивалентности и бесконечным числом классов.
    а) запишите аксиомы теории $T$.
    б) Сколько неизоморфных моделей у теории $T$ мощности: $\aleph_0$? $\aleph_1$?
    $\aleph_2$? $\aleph_{\omega_1}$?
\end{task}


\breakline

Назовем функцию $f:M^{n} \to M^{m}$ выразимой, есть ее график выразим.

\begin{ptask}{3}
	Докажите, что:
    а) если $f, g$ выразимы, то $f(g)$ также выразима.
    б) если $f:M^{n} \to M$ выразима, то образ выразим.
    б) если $f:M^{n} \to M$ выразимая биекция, то $f^{-1}$ выразима.
\end{ptask}

\begin{ptask}{4}
    Будет ли теория $Th(\mathbb{Z}, <, =)$ (дискретный линейный порядок
    без первого и последнего элемента)
    а) категоричной в счетной мощности?
    б) полной?
\end{ptask}
