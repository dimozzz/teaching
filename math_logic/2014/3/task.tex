\setcounter{curtask}{11}

\mytitle{3 (на 26.03)}

\begin{task}
    Пусть $M = (\mathbb{Q}, <)$,
    $p = \{\phi(v) \mid M \models \phi(\frac{1}{2})\}$.
    Реализуется ли тип $p$ в модели $M$, если да, то конечно ли число точек
    реализации? 
\end{task}

\begin{task}
    Приведите пример типа и модели, в которой данный тип не реализуется.
\end{task}

\begin{task}
    Предъявите счетную модель над счетным языком такую, что найдется несчетное число
    полных $1$-типов.
\end{task}

\begin{task}
    Пусть $\mathbb{Z}$ кольцо целых чисел. Покажите, что существует такое
    элементарное расширение $\mathbb{Z}$, что в нем есть нестандартное простое 
    число.
\end{task}

\breakline 

\begin{ptask}{6}
    Докажите, что теория алгебраически замкнутых полей фиксированной характеристики
    $\lambda$-категорична для любого $\lambda > \aleph_0$
\end{ptask}

\begin{ptask}{7}
    Будет ли теория групп, где порядок каждого элемента равен 2:
    б) полной?
\end{ptask}

\begin{ptask}{9}
    Определим игру $P_{\omega}$, как $EF_{\omega}$, но первый игрок обязан выбирать
    точки из первой модели, а второй из второй.
	а) докажите, что второй игрок имеет выигрышную стратегию тогда и только тогда,
    когда первая модель вкладывается во вторую.
    б) а что если, первый игрок выбирает на четных шагах из первой структуры, а на
    нечетных из второй?
\end{ptask}

\begin{ptask}{10}
    Пусть $L = \{E\}$, где $E$~--- бинарное отношение. $T$~--- $L$-теория,
    утверждающая, что $E$~--- отношение эквивалентности и бесконечным числом классов.
    а) запишите аксиомы теории $T$.
    б) Сколько неизоморфных моделей у теории $T$ мощности: $\aleph_0$? $\aleph_1$?
    $\aleph_2$? $\aleph_{\omega_1}$?
\end{ptask}