\setcounter{curtask}{14}

\mytitle{3 (на 6.10)}


\begin{task}
    Пусть $S = {S_{\alpha} \subseteq 2^{\mathbb{N}}}$, для любых $\alpha, \beta$ либо
    $S_{\alpha} \subset S_{\beta}$, либо $S_{\beta} \subset S_{\alpha}$. Может ли
    быть верно неравенство $\mathbb{N} < S$?
\end{task}

\begin{task}
    Придумайте теорию:
    а) без бесконечных моделей
    б) без конечных моделей
\end{task}

Спектром теории (формулы) называется множество всех конечных размеров моделей данной
теории.

\begin{task}
    Предъявите теорию спектр которой:
    а) все числа
    б) все четные числа
\end{task}

\begin{task}
    Предъявите теорию спектр которой все степени простых чисел.
\end{task}

\begin{task}
    $(M, =)$, где $M$~--- призвольное бесконечное множество. Докажите, что в данной
    интерпретации для любой формулы существует эквивалентная бескванторная формула.
\end{task}

\begin{task}
    Приведите пример трех неизоморфных линейных порядка на счетном множестве.
\end{task}