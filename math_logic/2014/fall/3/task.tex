\setcounter{curtask}{15}

\mytitle{3 (на 1.10)}

\begin{task}
    Какова мощность множества всех непрерывных функций с вещественными аргументами и
    вещественными значениями?
\end{task}

\begin{task}
    Какова мощность множества всех монотонных функций с вещественными аргументами и
    вещественными значениями?
\end{task}

\begin{task}
    Пусть $S = {S_{\alpha} \subseteq 2^{\mathbb{N}}}$, для любых $\alpha, \beta$ либо
    $S_{\alpha} \subset S_{\beta}$, либо $S_{\beta} \subset S_{\alpha}$. Докажите,
    что может быть верно неравенство $\mathbb{N} < S$. (подсказка: придумайте
    адаптивную кодировку (т.е. кодировка символа зависит от предыдущих символов)
    вещественных чисел). 
\end{task}

\begin{task}
    Приведите пример трех неизоморфных линейных порядка на счетном множестве.
\end{task}

\begin{task}
	Доказательство противоречивости множества дизъюнктов в системе Cutting
    Planes. Представляет собой последовательность линейных неравенств:
    $C_1 \ge c_1, C_2 \ge c_2, \dots, C_{m - 1} \ge c_{m - 1}, 0 \ge 1$.

    $j$-ое неравенство получено по одному из следующих правил:
    \begin{itemize}
        \item Аксиомы: $x \ge 0$, $1 - x \ge 0$,
	    \item $(x_1 \lor \neg x_2 \lor \dots)$~--- исходный дизъюнкт, тогда
    		$C_j = x_1 + (1 - x_2) + \dots, c_j = 1$,
        \item $С_j = C_i + C_k, c_j = c_i + c_k$, где $i, k < j$,
		\item $C_j = k C_i, c_j = k c_i$, где $i < j$, $k \in \mathbb{N}$,
        \item Пусть $C_i = k a_1 x_1 + k a_2 x_2 + \dots$, тогда $C_j = a_1 x_1 +
    		a_2 x_2 + \dots, c_j =  \lceil c_i \rceil$
        
    \end{itemize}

    Докажите, что противоречие можно получить тогда и только тогда, когда исходное
    множество дизъюнктов является противоречивым.
\end{task}

\begin{task}
    Добавим к исчислению высказываний правило подстановки. Оно разрешает заменить в
    исходной формуле все переменные на произвольные формулы (вхождение одной
    переменной заменяются на одинаковые формулы). Докажите, что класс выводимых
    формул не изменится, но лемма о дедукции перестанет быть верной. 
\end{task}