\setcounter{curtask}{15}

\mytitle{3 (на 1.10)}


\begin{task}
    Пусть $S = {S_{\alpha} \subseteq 2^{\mathbb{N}}}$, для любых $\alpha, \beta$ либо
    $S_{\alpha} \subset S_{\beta}$, либо $S_{\beta} \subset S_{\alpha}$. Докажите,
    что может быть верно неравенство $\mathbb{N} < S$. (подсказка: придумайте
    адаптивную кодировку (т.е. кодировка символа зависит от предыдущих символов)
    вещественных чисел). 
\end{task}

\begin{task}
    Приведите пример трех неизоморфных линейных порядка на счетном множестве.
\end{task}

\begin{task}
	Доказательство противоречивости множества дизъюнктов в системе Cutting
    Planes. Представляет собой последовательность линейных неравенств:
    $C_1 \ge c_1, C_2 \ge c_2, \dots, C_{m - 1} \ge c_{m - 1}, 0 \ge 1$.

    $j$-ое неравенство получено по одному из следующих правил:
    \begin{itemize}
        \item Аксиомы: $x \ge 0$, $1 - x \ge 0$,
	    \item $(x_1 \lor \neg x_2 \lor \dots)$~--- исходный дизъюнкт, тогда
    		$C_j = x_1 + (1 - x_2) + \dots, c_j = 1$,
        \item $С_j = C_i + C_k, c_j = c_i + c_k$, где $i, k < j$,
		\item $C_j = k C_i, c_j = k c_i$, где $i < j$, $k \in \mathbb{N}$,
        \item Пусть $C_i = k a_1 x_1 + k a_2 x_2 + \dots$, тогда $C_j = a_1 x_1 +
    		a_2 x_2 + \dots, c_j =  \lceil c_i \rceil$
        
    \end{itemize}

    Докажите, что противоречие можно получить тогда и только тогда, когда исходное
    множество дизъюнктов является противоречивым.
\end{task}


Назовем подмножество $\mathbb{N}^k$ арифметическим (неформальное определение),
если его характеристическую функцию можно задачи при помощи уравнений в натуральных
числах (можно использовать значки $(+, \times, =)$) связанных логическими связками ($\land,
\lor, \dots$) а также кванторами.

Пример (множество четных чисел) (вход функции $x$): 
$\exists y x = y + y$

Пример (множество кратных $4$) (вход функции $x$): 
$\exists y \exists z ((x = y + y) \land (y = z + z))$


\begin{task}
    Докажите, что следующие множества являются арифметичными:
    {\it a)} $x < y$ (такое подмножество $\mathbb{N}^2$, что первый элемент пары
	    меньше второго)
    {\it б)} $x = 0$
    {\it в)} $x = 1$
    {\it г)} $x = c$, где $c$~--- константа
    {\it д)} $x~mod~b = r$
    {\it е)} $x$~--- степень двойки
    {\it ж)} $x$~--- степень четверки
\end{task}

\begin{task}
    Докажите, что:
    {\it a)} для любого $k$ найдется сколь угодно большое $b$, что
    $b + 1, 2b + 1, 3b + 1, \dots, kb + 1$~--- взаимно простые числа.
    {\it б)} для любой последовательности натуральный чисел $x_0, x_1,
    \dots$ найдутся таки $a$ и $b$, что $x_i = a_i~ mod ~ b(i + 1) +
    1$.
    {\it в)} множество: $x$~--- степень шестерки арифметично.
\end{task}


