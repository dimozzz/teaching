\setcounter{curtask}{28}

\mytitle{5 (на 12.11)}


\begin{task}
    Будет ли интерпретация $(\mathbb{Q}, =, <)$ элементарно
    эквивалентна:
    а) $(\mathbb{Q} + \mathbb{Q}, =, <)$
    б) $(\mathbb{Q} + \mathbb{R}, =, <)$
\end{task}

\begin{task}
    Будет ли интерпретация $(\mathbb{N}, =, <)$ элементарно
    эквивалентна: $(\mathbb{N} + \mathbb{Z}, =, <)$
\end{task}

\begin{task}
    а) Докажите, что в интерпретации $(Q, =, <, +,$ рациональные
    константы) допустима элиминация кванторов.
    б) Докажите, что интерпретации $(Q, =, <, +,$ рациональные
    константы) и $(\mathbb{R}, =, <, +,$ рациональные константы)
    элементарно эквивалентны.
    в) Пусть единичный квадрат разрезан на несколько меньших
    квадратов. Докажите, что все они имеют рациональные стороны.
\end{task}


В следующих задачах требуется описать множество выразимых предикатом в
данной интерпретации. Обычно требуется доказать, что это множество
совпадает с множеством бескванторных формул. Иногда такое доказать не
получится, тогда необходимо добавить выразимый предикат (выразимый с
квантором) и доказать, что выразимые~--- это бескванторные с новым
предикатом.

\begin{task}
    $(M, =)$, где $M$~--- призвольное бесконечное множество.
\end{task}

\begin{task}
    $(\mathbb{Q}, =, +)$
\end{task}

\begin{task}
    $(\mathbb{Q}, =, S)$, где $S$~--- прибавление единицы.
\end{task}

\begin{task}
    $(\mathbb{N}, =, S)$
\end{task}





\breakline

\begin{ptask}{20}
    Добавим к исчислению высказываний правило подстановки. Оно разрешает заменить в
    исходной формуле все переменные на произвольные формулы (вхождение одной
    переменной заменяются на одинаковые формулы). Докажите, что класс выводимых
    формул не изменится, но лемма о дедукции перестанет быть верной. 
\end{ptask}


\begin{ptask}{23}
    Предъявите теорию спектр которой:
    а) все степени простых чисел
    б) все числа не являющиеся степенями простого
    в) состоит из квадратов чисел
\end{ptask}
