\setcounter{curtask}{36}

\mytitle{6 (на 26.11)}

\begin{task}
    Пусть $T$ имеет сколь угодно большую конечную модель. Докажите, что $T$ имеет
    бесконечную модель.
\end{task}

\begin{task}
    Предъявите теорию $T$ у которой нет счетной модели, но есть бесконечная модель.
\end{task}

\begin{task}
    Докажите, что если формула $\phi$ верна в алгебраически замкнутом
    поле в характеристикой $0$, то найдется $p_o$, что для любого $p >
    p_o$ $\phi$ будет верна в алгебраически замкнутом поле с
    характеристикой $p$.
\end{task}


Пусть $I$~--- интерпретация. $Th(I)$~--- множество формул верной в
данной интерпретации.

\begin{task}
    Будет ли теория $Th((Z, <, =))$ конечно аксиоматизируемой.
\end{task}

\begin{task}
    Будет ли теория $Th((N, <, =))$ конечно аксиоматизируемой.
\end{task}


\breakline


\begin{ptask}{20}
    Добавим к исчислению высказываний правило подстановки. Оно разрешает заменить в
    исходной формуле все переменные на произвольные формулы (вхождение одной
    переменной заменяются на одинаковые формулы). Докажите, что класс выводимых
    формул не изменится, но лемма о дедукции перестанет быть верной. 
\end{ptask}


\begin{ptask}{23}
    Предъявите теорию спектр которой:
    а) все степени простых чисел
    б) все числа не являющиеся степенями простого
    в) состоит из квадратов чисел
\end{ptask}

\begin{ptask}{31}
    а) Докажите, что в интерпретации $(Q, =, <, +,$ рациональные
    константы) допустима элиминация кванторов.
    б) Докажите, что интерпретации $(Q, =, <, +,$ рациональные
    константы) и $(\mathbb{R}, =, <, +,$ рациональные константы)
    элементарно эквивалентны.
    в) Пусть единичный квадрат разрезан на несколько меньших
    квадратов. Докажите, что все они имеют рациональные стороны.
\end{ptask}
