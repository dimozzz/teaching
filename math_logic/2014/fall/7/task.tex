\setcounter{curtask}{41}

\mytitle{7 (на 10.12)}


\begin{task}
    Докажите, что теория $Th(\mathbb{Z}, S, =, 0)$ не является
    конечно-аксиоматизируемой.
\end{task}

\begin{task}
  	Существует ли модель $RCF$ (Real Closed Field), которая содержит $\mathbb{R}$ и
    такую точку $r$, что $r$ больше любого натурального числа $n$.
\end{task}

\breakline



\begin{ptask}{20}
    Добавим к исчислению высказываний правило подстановки. Оно разрешает заменить в
    исходной формуле все переменные на произвольные формулы (вхождение одной
    переменной заменяются на одинаковые формулы). Докажите, что класс выводимых
    формул не изменится, но лемма о дедукции перестанет быть верной. 
\end{ptask}

\begin{ptask}{40}
    Будет ли теория $Th((N, <, =))$ конечно аксиоматизируемой.
\end{ptask}
