\setcounter{curtask}{1}

\mytitle{1 (на 10.09)}

\begin{task}
    Докажите, что не существует биекции между $\mathbb{N}$ и $\mathbb{R}$.
\end{task}

Будет говорить, что множество $A$ не больше множества $B$, если
существует биективное отображение из $A$ в подмножество $B$, и строго
меньше, если не существует биекции из $B$ в $A$. Под степенью будем
подразумевать декартову степень.

\begin{task}
    Докажите, что $\mathbb{N}^c \le \mathbb{N}$.
\end{task}

\begin{task}
    Докажите, что:
    а) $\mathbb{R} \le [0, 1]$
    б) $\mathbb{R} \le 2^{\mathbb{N}}$
    в) $\mathbb{R} \times \mathbb{R} \le \mathbb{R}$ (подсказка:
	    придумать явную биекцию квадрата на одну из сторон)
\end{task}

\begin{task}
    Приведите пример формулы длины $n$ такой, что ее минимальный
    размер в КНФ $\Omega(2^n)$.
\end{task}

\vspace{0.5cm}
Методом резолюций называется метод доказательства противоречивости множества
дизъюнктов. Данный метод содержит единственное правило вывода: $\frac{x \vee \alpha \
\ \ \neg x \vee \beta}{\alpha \vee \beta}$. Множество дизъюнктов считается
противоречивым (имеет вывод в резолюции), если из него можно вывести пустой дизъюнкт.

\begin{task} (Корректность)
    Если $\neg F$ (записанная в КНФ) имеет вывод в резолюциях, то
    формула $F$ является тавтологией.
\end{task}

\begin{task}
    Пусть формула $\phi \rightarrow \psi$~--- тавтология. Докажите,
    что найдется такая формула $\tau$, содержащая только общие для
    $\phi$ и $\psi$ переменные, что $\phi \rightarrow \tau$ и
    $\tau \rightarrow \psi$ будут тавтологиями.
\end{task}

\begin{task} (Теорема о компактности)
    Пусть $\Gamma$ множество формул. $\Gamma$ непротиворечиво тогда и
    только тогда, когда любое конечное подмножество $\Gamma$ непротиворечиво.
\end{task}
