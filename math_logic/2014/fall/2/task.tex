\setcounter{curtask}{8}

\mytitle{2 (на 17.09)}

Будем говорить, что множества $A$ и $B$ равны, если $A \le B$ и $B \le A$.

\begin{task}
    Докажите, что $[0, 1] = (0, 1) = [0, 1) = \mathbb{R}$
\end{task}

\begin{task}
    Докажите, что $\mathbb{R}^{\mathbb{N}} \le \mathbb{R}$
\end{task}

\begin{task}
    Докажите, что $\mathbb{N}^{\mathbb{R}} \le 2^{\mathbb{R}}$
\end{task}

\begin{task} (полнота резолюций)
    Пусть $\neg F$~--- противоречивая формула в КНФ, докажите что у нее есть вывод в 
    резолюциях. (подсказка: индукция по числу переменных).
\end{task}

\begin{task}
    Какое минимальное количество булевых функций от $n~ (n > 1)$
    переменных составляют полный базис?
\end{task}

\begin{task}(полином Жегалкина)
    Докажите, что любая булева функция от булевых переменных
    представляется в виде полинома над $\mathbb{F}_2$ (сложение~---
    $xor$, а умножение~--- конъюнкция).
\end{task}


Пусть $f(x_1, \dots, x_n)$~--- булева функция от булевых аргументов.
$f$ называется:
\begin{itemize}
	\item сохраняющей ноль, если $f(0, 0, \dots, 0) = 0$;
	\item сохраняющей единицу, если $f(1, 1, \dots, 1) = 1$;
	\item самодвойственной, если $f(x_1, x_2, \dots, x_n) = \neg
		f(\neg x_1, \neg x_2, \dots, \neg x_n)$;
    \item монотонной, если
		$\forall i~~ f(x_1, x_2, \dots, x_{i - 1}, 1, x_{i + 1},
        \dots,  x_n) \ge f(x_1, x_2, \dots, x_{i - 1}, 0, x_{i + 1},
        \dots,  x_n)$;
    \item линейной, если $f(x_1, x_2, \dots, x_n) = a_0 \oplus a_1x_1
		\oplus \dots \oplus a_nx_n$, где $a_i$~--- булевы константы.
\end{itemize}

\begin{task} (критерий Поста)
	Пусть $F = {f_1, \dots, f_k}$~--- набор булевых функций от $n$
    переменных. Будем говорить, что $F$ принадлежит классу функций,
    если все функции из множества $F$ принадлежат данному классу.
    Докажите, что:

	а) если $F$ принадлежит одному из описанных выше классов
    (сохраняющие ноль, ...), то любые
    композиции функций из $F$ принадлежат тому же классу.
    
    Пусть теперь $F$ не принадлежит ни одному из перечисленных
    классов.
    
    б) постройте константы и отрицание из композиций функций из $F$
    (указание: использовать не сохраняющие $0/1$ и не самодвойственную)

    в) постройте конъюнкцию из композиций функций из $F$ и докажите,
    что набор $F$ является базисом булевых функций от $n$ аргументов.

\end{task}



\breakline


\begin{ptask}{1}
    Докажите, что не существует биекции между $\mathbb{N}$ и $\mathbb{R}$.
\end{ptask}

Будет говорить, что множество $A$ не больше множества $B$, если
существует биективное отображение из $A$ в подмножество $B$, и строго
меньше, если не существует биекции из $B$ в $A$. Под степенью будем
подразумевать декартову степень.

\begin{ptask}{2}
    Докажите, что $\mathbb{N}^c \le \mathbb{N}$.
\end{ptask}

\begin{ptask}{3}
    Докажите, что:
    а) $\mathbb{R} \le [0, 1]$
    б) $\mathbb{R} \le 2^{\mathbb{N}}$
    в) $\mathbb{R} \times \mathbb{R} \le \mathbb{R}$ (подсказка:
	    придумать явную биекцию квадрата на одну из сторон)
\end{ptask}

\begin{ptask}{4}
    Приведите пример формулы длины $n$ такой, что ее минимальный
    размер в КНФ $\Omega(2^n)$.
\end{ptask}


\begin{ptask}{5}
    Если $\neg F$ (записанная в КНФ) имеет вывод в резолюциях, то
    формула $F$ является тавтологией.
\end{ptask}

\begin{ptask}{6}
    Пусть формула $\phi \rightarrow \psi$~--- тавтология. Докажите,
    что найдется такая формула $\tau$, содержащая только общие для
    $\phi$ и $\psi$ переменные, что $\phi \rightarrow \tau$ и
    $\tau \rightarrow \psi$ будут тавтологиями.
\end{ptask}

\begin{ptask}{7}
    Пусть $\Gamma$ множество формул. $\Gamma$ непротиворечиво тогда и
    только тогда, когда любое конечное подмножество $\Gamma$ непротиворечиво.
\end{ptask}