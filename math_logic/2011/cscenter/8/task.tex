\setcounter{curtask}{38}

\mytitle{8 (на 10.11)}

\begin{task}
    Приведите к предваренной нормальной форме формулу:
    $\forall x A(x) \rightarrow \forall x B(x)$
\end{task}

\begin{task}
    Докажите, что предикат $x = 2$ невыразим в множестве целых чисел с
    предикатами равенства и $x$ делит $y$.
\end{task}

\begin{task}
    Докажите, что предикат $y = x + 2011$ невыразим в интерпретации
    $(\mathbb{Z}, =, x \mapsto x^2)$.
\end{task}

Можно ли провести элиминацию кванторов в следующих теориях?

\begin{task}
    $(M, =)$, где $M$~--- произвольное бесконечное множество.
\end{task}

\begin{task}
    $(\mathbb{Q}, =, +)$
\end{task}

\begin{task}
    $(Z, =, <, +, 0, 1)$ (подсказка: рассмотреть предикат четности)
\end{task}

\breakline

\begin{ptask}{33}
    Докажите, что если у формулы существует резолюционный вывод, то
    существует и вывод размера $O(2^n)$, где $n$~--- количество переменных.
\end{ptask}

\begin{ptask}{34}
    Приведите пример формулы длины $n$ такой, что ее минимальный
    размер в КНФ $\Omega(2^n)$.
\end{ptask}

\begin{ptask}{36}
    б) докажите, что любая функция над конечным полем представляется в
	виде полинома.
\end{ptask}

\begin{ptask}{37} (критерий Поста)
	Пусть $F = {f_1, \dots, f_k}$~--- набор булевых функций от $n$
    переменных. Будем говорить, что $F$ принадлежит классу функций,
    если все функции из множества $F$ принадлежат данному классу.
    
    Пусть теперь $F$ не принадлежит ни одному из перечисленных
    классов.

    Докажите, что:
    б) постройте константы и отрицание из композиций функций из $F$
    (указание: использовать не сохраняющие $0/1$ и не самодвойственную)

    в) постройте конъюнкцию из композиций функций из $F$ и докажите,
    что набор $F$ является базисом булевых функций от $n$ аргументов
    (указание: использовать полином Жегалкина).

\end{ptask}