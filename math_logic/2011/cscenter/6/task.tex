\setcounter{curtask}{30}

\mytitle{6 (на 23.10)}

\begin{task}
    Является ли функция $\alpha_k{x}$ (функция Аккермана) примитивно
    рекурсивной для фиксированного $k$?
\end{task}

\begin{task}
	Является ли множество всех программ считающих инъективные функции
    перечислимым/коперечислимым?    
\end{task}

\breakline

\begin{ptask}{19}
    Докажите, что:
    {\it в)} найдется $x \in \mathbb{N}$ такой, что $\{x\} = \{y
    \mid (x, y) \in U\}$
\end{ptask}

\begin{ptask}{20}
    Докажите, что существуют вычислимые не главные нумерации.
\end{ptask}

\begin{ptask}{23} ({\it простые множества Поста})
    Назовем множество {\it иммунным}, если оно бесконечно, но не
    содержит бесконечных перечислимых подмножеств. Перечислимое
    множество называется {\it простым}, если его дополнение иммунно.
    Докажите, что простые множества существуют.
\end{ptask}

\begin{ptask}{24} ({\it машина Минского})
\end{ptask}

\begin{ptask}{25}
    Пусть $g(x_1, \dots, x_k) = y_0$, где $y_0 = \min \{y \mid f(x_1,
      \dots, x_k, y) = 0\}$. Покажите, что при вычислимой не всюду
      определенной $f$, $g$ может быть невычислимой.
\end{ptask}

\begin{ptask}{26}
    Покажите, что функция обратная к примитивно рекурсивной биекции
    $f: \mathbb{N} \rightarrow \mathbb{N}$
    может не быть примитивно рекурсивной.
\end{ptask}

\begin{ptask}{27}
	Пусть $g(x_1, \dots, x_k, y)$~--- примитивно рекурсивная функция.
    Докажите, что функция $f(x_1, \dots, x_k, y, z) =
	\begin{cases}
		\sum\limits_{i = 0}^{z} g(x_1, \dots, x_k, y + i),~~ y \le z  \\
		0,~~ y > z
	\end{cases}$
\end{ptask}