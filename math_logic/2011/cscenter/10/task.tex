\setcounter{curtask}{47}

\mytitle{10 (на 24.11)}

\begin{task}
    Пусть у теории $T$ с не более, чем счетной сигнатурой есть
	бесконечная модель, докажите, что у теории $T$ есть счетная
    модель.
\end{task}

\begin{task}
    Приведите пример совместной теории:
    а) без конечных моделей
    б) наименьшая модель, которой имеет мощность не менее
    $2^{\mathbb{N}}$. (подсказка: рассмотреть несчетную сигнатуру,
    для подобных сигнатур все доказанные результаты остаются верными)
\end{task}

(Теорема о компактности)
	Пусть $\Gamma$ множество замкнутых формул. $\Gamma$
    непротиворечиво тогда и только тогда, когда любое конечное
    подмножество $\Gamma$ непротиворечиво.
    
\begin{task}
    Докажите, что для всякого поля $K$ существует его расширение $K'$
    такое, что каждый многочлен с коэффициентами из $K$ имеет корень
    в $K'$ (можно пользоваться тем, что для конкретного многочлена
    такое расширение существует) (подсказка: использовать теорему о
	компактности).
\end{task}

\breakline

\begin{ptask}{34}
    Приведите пример формулы длины $n$ такой, что ее минимальный
    размер в КНФ $\Omega(2^n)$.
\end{ptask}

\begin{ptask}{37} (критерий Поста)
	Пусть $F = {f_1, \dots, f_k}$~--- набор булевых функций от $n$
    переменных. Будем говорить, что $F$ принадлежит классу функций,
    если все функции из множества $F$ принадлежат данному классу.
    
    Пусть теперь $F$ не принадлежит ни одному из перечисленных
    классов.

    в) постройте конъюнкцию из композиций функций из $F$ и докажите,
    что набор $F$ является базисом булевых функций от $n$ аргументов
    (указание: использовать полином Жегалкина).

\end{ptask}

\begin{ptask}{40}
    Докажите, что предикат $y = x + 2011$ невыразим в интерпретации
    $(\mathbb{Z}, =, x \mapsto x^2)$.
\end{ptask}


\begin{ptask}{43}
    $(Z, =, <, +, 0, 1)$ (подсказка: рассмотреть предикат четности)
\end{ptask}

\begin{ptask}{45}
	$\mathbb{Z} + \mathbb{Z}$~--- это две копии целых чисел, причем
    все числа из второй копии больше чисел из первой. Докажите, что
    $(\mathbb{Z}, <, =)$ элементарно эквивалентна $(\mathbb{Z} +
    \mathbb{Z}, <, =)$.
\end{ptask}

\begin{ptask}{46}
    Будет ли интерпретация $(\mathbb{N}, =, <)$ элементарно
    эквивалентна:
    а) $(\mathbb{N} + \mathbb{N}, =, <)$
    б) $(\mathbb{N} + \mathbb{Z}, =, <)$
\end{ptask}