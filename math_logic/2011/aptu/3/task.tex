\setcounter{curtask}{10}

\mytitle{3 (на 15.10)}

\begin{task}
    Докажите, что предикат $x = 2$ невыразим в множестве целых чисел с
    предикатами равенства и $x$ делит $y$.
\end{task}

\begin{task}
    Докажите, что предикат $y = x + 2011$ невыразим в интерпретации
    $(\mathbb{Z}, =, x \mapsto x^2)$.
\end{task}

Две интерпретации одной сигнатуры называются элементарно
эквивалентными, если каждая замкнутая формула в первой интерпретации
верна тогда и только тогда, когда она верна во второй.

\begin{task}
	$\mathbb{Z} + \mathbb{Z}$~--- это две копии целых чисел, причем
    все числа из второй копии больше чисел из первой. Докажите, что
    $(\mathbb{Z}, <, =)$ элементарно эквивалентна $(\mathbb{Z} +
    \mathbb{Z}, <, =)$.
\end{task}

\begin{task}
    Будет ли интерпретация $(\mathbb{N}, =, <)$ элементарно
    эквивалентна:
    а) $(\mathbb{N} + \mathbb{N}, =, <)$
    б) $(\mathbb{N} + \mathbb{Z}, =, <)$
\end{task}

\begin{task}
    а) Докажите, что в интерпретации $(Q, =, <, +,$ рациональные
    константы) допустима элиминация кванторов.
    б) Докажите, что интерпретации $(Q, =, <, +,$ рациональные
    константы) и $(\mathbb{R}, =, <, +,$ рациональные константы)
    элементарно эквивалентны.
    в) Пусть единичный квадрат разрезан на несколько меньших
    квадратов. Докажите, что все они имеют рациональные стороны.
\end{task}


\breakline

\begin{ptask}{5}
    Пусть формула $\phi \rightarrow \psi$~--- тавтология. Докажите,
    что найдется такая формула $\tau$, содержащая только общие для
    $\phi$ и $\psi$ переменные, что $\phi \rightarrow \tau$ и
    $\tau \rightarrow \psi$ будут тавтологиями.
\end{ptask}

\begin{ptask}{9}
    Докажите, что:
    {\it a)} для любого $k$ найдется сколь угодно большое $b$, что
    $b + 1, 2b + 1, 3b + 1, \dots, kb + 1$~--- взаимно простые числа.
    {\it б)} для любой последовательности натуральный чисел $x_0, x_1,
    \dots$ найдутся таки $a$ и $b$, что $x_i = a_i~ mod ~ b(i + 1) +
    1$.
    {\it в)} предикат: $a$~--- степень шестерки арифметичен.
\end{ptask}