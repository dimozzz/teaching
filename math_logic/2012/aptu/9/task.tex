\setcounter{curtask}{27}

\mytitle{9 (на 25.04)}

\begin{task}
    Докажите, что в ommiting types theorem можно опустить счетное число типов.
\end{task}

\breakline


\begin{ptask}{12}
    Выразимо ли $C$ в поле рациональных функций над $C$?
\end{ptask}

\begin{ptask}{17}
    Определим игру $P_{\omega}$, как $EF_{\omega}$, но первый игрок обязан выбирать
    точки из первой модели, а второй из второй.
	а) докажите, что второй игрок имеет выигрышную стратегию тогда и только тогда,
    когда первая модель вкладывается во вторую.
    б) а что если, первый игрок выбирает на четных шагах из первой структуры, а на
    нечетных из второй?
\end{ptask}

\begin{ptask}{19}(Overspill)
    Пусть $M$ нестандартная модель арифметики Пеано. Пусть
    $M \models \phi(n, \overline{w})$, для любого $n < \omega$, тогда найдется такое
    бесконечное $c$, что $M \models \phi(c, \overline{w})$
\end{ptask}

\begin{ptask}{24}
    Пусть $T$ теория $(\mathbb{Z}, s)$, где $s = s + 1$. Какие типы входят в
    $S_n(T)$? А какие изолированы?
\end{ptask}

\begin{ptask}{25}
    Пусть $T$ теория $(\mathbb{Z}, s, <)$, где $s = s + 1$. Какие типы входят в
    $S_n(T)$? А какие изолированы?
\end{ptask}

\begin{ptask}{26}
    Пусть $A \subseteq B$, $\phi$~--- $L_A$ формула. Докажите, что если $\phi$
    изолирует $t_p^M(a/B)$, то $\phi$ изолирует $t_p^M(a/A)$.
\end{ptask}