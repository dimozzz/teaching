\setcounter{curtask}{1}

\mytitle{1 (на 10.02?)}

Пусть есть вполне упорядоченное (в.у.) множество $P$.
$\vec{x} = \{y < x \mid x \in P\}$~--- называется начальным отрезком.

\begin{task}
    Доказать, что любое в.у. множество не изоморфно никакому своему начальному отрезку.
\end{task}

\begin{task}
    Докажите, что в любом линейном пространстве есть басиз.
\end{task}

\begin{task}
    Докажите, что класс ординалов не является множеством.
\end{task}

\begin{task}
    Арифметика кардиналов.
    1. $\aleph_{\alpha} + \aleph_{\beta} = \aleph_{\alpha}  \aleph_{\beta} =
    	\max(\aleph_{\alpha}, \aleph_{\beta})$
    2. Если $\alpha \le \beta$, то $\aleph_{\alpha}^{\aleph_{\beta}} \le 2^{\aleph_{\beta}}$
\end{task}

\begin{task}
    Все ли подмножества вещественных чисел измеримы относительно меры Лебега?
\end{task}

\begin{task}
    Алиса и Боб играют в игру. Они загадали некоторой предикат $P$ после чего, они
    составляют бесконечное слово $\omega$ по очереди, называя биты. Алиса ходит
    первой, и побеждает, если $P(w) = 1$.Верно ли, что для любого предиката $P$
    существует выигрышная стратегия для одного из игроков?
\end{task}