\setcounter{curtask}{13}

\mytitle{4 (на 14.02)}

\begin{task}
    Докажите, что теория алгебраически замкнутых полей фиксированной характеристики
    $\lambda$-категорична для любого $\lambda > \aleph_0$
\end{task}

\begin{task}
    Докажите полноту DLO (Dense Linear Order, плотный линейный порядок без концов).
\end{task}

\begin{task}
    Будет ли теория групп, где порядок каждого элемента равен 2:
    а) $\lambda$-категоричной для любого $\lambda \ge \aleph_0$?
    б) полной?
\end{task}

\begin{task}
    Пусть $T$ имеет сколь угодно большую конечную модель. Докажите, что $T$ имеет
    бесконечную модель.
\end{task}

\begin{task}
    Пусть $L = \{s\}$. $T$ --- $L$-теория, которая говорит, что $s$~--- биекция без
	циклов (т.е $s^{(n)}(x) != x$). Для каких $\lambda$ данная теория будет
    $\lambda$-категоричной.
\end{task}

\breakline


\begin{ptask}{9}
  	Существует ли модель $RCF$, которая содержит $\mathbb{R}$ и такую точку $r$, что
    $r$ больше любого натурального числа $n$.
\end{ptask}

\begin{ptask}{12}
    Выразимо ли $C$ в поле рациональных функций над $C$?
\end{ptask}
