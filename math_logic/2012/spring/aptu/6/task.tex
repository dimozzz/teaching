\setcounter{curtask}{17}

\mytitle{6 (на 04.04)}

EF --- (Ehrenfeucht-Fraisse).

\begin{task}
    Определим игру $P_{\omega}$, как $EF_{\omega}$, но первый игрок обязан выбирать
    точки из первой модели, а второй из второй.
	а) докажите, что второй игрок имеет выигрышную стратегию тогда и только тогда,
    когда первая модель вкладывается во вторую.
    б) а что если, первый игрок выбирает на четных шагах из первой структуры, а на
    нечетных из второй?
\end{task}

\begin{task}
    Пусть $L = \{E\}$, где $E$~--- бинарное отношение. $T$~--- $L$-теория,
    утверждающая, что $E$~--- отношение эквивалентности и бесконечным числом классов.
    а) запишите аксиомы теории $T$.
    б) Сколько неизоморфных моделей у теории $T$ мощности: $\aleph_0$? $\aleph_1$?
    $\aleph_2$? $\aleph_{\omega_1}$?
\end{task}

\begin{task}(Overspill)
    Пусть $M$ нестандартная модель арифметики Пеано. Пусть
    $M \models \phi(n, \overline{w})$, для любого $n < \omega$, тогда найдется такое
    бесконечное $c$, что $M \models \phi(c, \overline{w})$
\end{task}

\breakline


\begin{ptask}{9}
  	Существует ли модель $RCF$, которая содержит $\mathbb{R}$ и такую точку $r$, что
    $r$ больше любого натурального числа $n$.
\end{ptask}

\begin{ptask}{12}
    Выразимо ли $C$ в поле рациональных функций над $C$?
\end{ptask}

\begin{ptask}{13}
    Докажите, что теория алгебраически замкнутых полей фиксированной характеристики
    $\lambda$-категорична для любого $\lambda > \aleph_0$
\end{ptask}

\begin{ptask}{16}
    Пусть $L = \{s\}$. $T$ --- $L$-теория, которая говорит, что $s$~--- биекция без
	циклов (т.е $s^{(n)}(x) != x$). Для каких $\lambda$ данная теория будет
    $\lambda$-категоричной.
\end{ptask}