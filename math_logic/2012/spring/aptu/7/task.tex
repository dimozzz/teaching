\setcounter{curtask}{20}

\mytitle{7 (на 11.04)}

\begin{task}
    Пусть $M = (\mathbb{Q}, <)$, $p = \{\phi(v) \mid M \models \phi(\frac{1}{2})\}$.
    Реализуется ли тип $p$ в модели $M$, если да, то конечно ли числа точек реализации?
\end{task}

\begin{task}
    Приведите пример типа и модели, в которой данный тип не реализуется.
\end{task}

\begin{task}
    Предъявите счетную модель над счетным языком такую, что найдется несчетное число
    полных $1$-типов.
\end{task}

\begin{task}
    Пусть $\mathbb{Z}$ кольцо целых чисел. Покажите, что существует такое
    элементарное расширение $\mathbb{Z}$, что в нем есть нестандартное простое
    число.
\end{task}

\breakline


\begin{ptask}{12}
    Выразимо ли $C$ в поле рациональных функций над $C$?
\end{ptask}

\begin{ptask}{13}
    Докажите, что теория алгебраически замкнутых полей фиксированной характеристики
    $\lambda$-категорична для любого $\lambda > \aleph_0$
\end{ptask}

\begin{ptask}{17}
    Определим игру $P_{\omega}$, как $EF_{\omega}$, но первый игрок обязан выбирать
    точки из первой модели, а второй из второй.
	а) докажите, что второй игрок имеет выигрышную стратегию тогда и только тогда,
    когда первая модель вкладывается во вторую.
    б) а что если, первый игрок выбирает на четных шагах из первой структуры, а на
    нечетных из второй?
\end{ptask}

\begin{ptask}{19}(Overspill)
    Пусть $M$ нестандартная модель арифметики Пеано. Пусть
    $M \models \phi(n, \overline{w})$, для любого $n < \omega$, тогда найдется такое
    бесконечное $c$, что $M \models \phi(c, \overline{w})$
\end{ptask}
