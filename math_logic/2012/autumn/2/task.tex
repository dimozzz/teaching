\setcounter{curtask}{7}

\mytitle{2 (на 19.09)}

\begin{task}
    Докажите, что следующая формула является тавтологией для любой
    формулы $A$: $\neg\neg A \rightarrow A$
\end{task}

\begin{task}
    Докажите, что для любых формул $A$, $B$, $C$ верна формула:
    \begin{center}
        $(A \rightarrow B) \rightarrow ((B \rightarrow C) \rightarrow (A
	    \rightarrow C))$
    \end{center}
\end{task}

\begin{task} (Правило сечения)
    Докажите, что если: $\Gamma_1 \mapsto A$ и $\Gamma_2, A \mapsto
    B$, то $\Gamma_1, \Gamma_2 \mapsto B$
\end{task}

\begin{task} (Теорема о компактности)
    Пусть $\Gamma$ множество формул. $\Gamma$ непротиворечиво тогда и
    только тогда, когда любое конечное подмножество $\Gamma$ непротиворечиво.
\end{task}

\vspace{0.5cm}
Методом резолюций называется метод доказательства противоречивости
множества дизъюнктов. Данный метод содержит единственное правило
вывода: $\frac{x \vee \alpha \ \ \ \neg x \vee \beta}{\alpha \vee
  \beta}$. Множество дизъюнктов считается противоречивым (имеет вывод
в резолюции), если из него можно вывести пустой дизъюнкт.

\begin{task} (Корректность)
    Если $\neg F$ (записанная в КНФ) имеет вывод в резолюциях, то
    формула $F$ является тавтологией.
\end{task}

\begin{task}
    Пусть формула $\phi \rightarrow \psi$~--- тавтология. Докажите,
    что найдется такая формула $\tau$, содержащая только общие для
    $\phi$ и $\psi$ переменные, что $\phi \rightarrow \tau$ и
    $\tau \rightarrow \psi$ будут тавтологиями.
\end{task}

Пусть $f(x_1, \dots, x_n)$~--- булева функция от булевых аргументов.
$f$ называется:
\begin{itemize}
	\item сохраняющей ноль, если $f(0, 0, \dots, 0) = 0$;
	\item сохраняющей единицу, если $f(1, 1, \dots, 1) = 1$;
	\item самодвойственной, если $f(x_1, x_2, \dots, x_n) = \neg
		f(\neg x_1, \neg x_2, \dots, \neg x_n)$;
    \item монотонной, если
		$\forall i~~ f(x_1, x_2, \dots, x_{i - 1}, 1, x_{i + 1},
        \dots,  x_n) \ge f(x_1, x_2, \dots, x_{i - 1}, 0, x_{i + 1},
        \dots,  x_n)$;
    \item линейной, если $f(x_1, x_2, \dots, x_n) = a_0 \oplus a_1x_1
		\oplus \dots \oplus a_nx_n$, где $a_i$~--- булевы константы.
\end{itemize}

\begin{task} (критерий Поста)
	Пусть $F = {f_1, \dots, f_k}$~--- набор булевых функций от $n$
    переменных. Будем говорить, что $F$ принадлежит классу функций,
    если все функции из множества $F$ принадлежат данному классу.
    Докажите, что:

	а) если $F$ принадлежит одному из описанных выше классов
    (сохраняющие ноль, ...), то любые
    композиции функций из $F$ принадлежат тому же классу.
    
    Пусть теперь $F$ не принадлежит ни одному из перечисленных
    классов.
    
    б) постройте константы и отрицание из композиций функций из $F$
    (указание: использовать не сохраняющие $0/1$ и не самодвойственную)

    в) постройте конъюнкцию из композиций функций из $F$ и докажите,
    что набор $F$ является базисом булевых функций от $n$ аргументов.

\end{task}