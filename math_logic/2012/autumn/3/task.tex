\setcounter{curtask}{7}

\mytitle{3 (на 3.10)}

\begin{task} (используйте аналогию с деревьями)
    Докажите: {\it a)} если у формулы есть резолюционный вывод, то
    есть вывод длиной не более $O(2^n)$ {\it б)} полноту системы
    резолюций.
\end{task}

Предикат, заданный на множестве натуральных чисел называется
арифметичным, если он выражается при помощи формулы исчисления
предикатов в сигнатуре $(=, +, \times)$ в естественной интерпретации
на множестве натуральных чисел.

\begin{task}
    Докажите, что следующие предикаты являются арифметичными:
    {\it a)} $x < y$
    {\it б)} $x = 0$
    {\it в)} $x = 1$
    {\it г)} $x = c$, где $c$~--- константа
    {\it д)} $x~mod~b = r$
    {\it е)} $a$~--- степень двойки
    {\it ж)} $a$~--- степень четверки
\end{task}

\begin{task}
    Приведите к предваренной нормальной форме формулу:
    $\forall x A(x) \rightarrow \forall x B(x)$
\end{task}

\begin{task}
    Докажите, что:
    {\it a)} для любого $k$ найдется сколь угодно большое $b$, что
    $b + 1, 2b + 1, 3b + 1, \dots, kb + 1$~--- взаимно простые числа.
    {\it б)} для любой последовательности натуральный чисел $x_0, x_1,
    \dots$ найдутся таки $a$ и $b$, что $x_i = a_i~ mod ~ b(i + 1) +
    1$.
    {\it в)} предикат: $a$~--- степень шестерки арифметичен.
\end{task}